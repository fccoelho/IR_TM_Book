\chapter{Basic Information Retrieval}

    
    

    
    \section{Inverted Indices and Boolean
Search}\label{inverted-indices-and-boolean-search}

In this chapter, we will learn how to manually construct an inverted
index in pure Python. To help us in this task we shall make use of the
\href{http://nltk.org}{NLTK} natural language processing library.

    \begin{Verbatim}[commandchars=\\\{\}]
{\color{incolor}In [{\color{incolor}1}]:} \PY{k+kn}{import} \PY{n+nn}{nltk}
        \PY{k+kn}{import} \PY{n+nn}{os}
\end{Verbatim}

    \subsection{Choosing a collection of
Documents}\label{choosing-a-collection-of-documents}

Collections of documents, also know as \emph{corpora} or \emph{corpus}
(in sigular form), are the starting point of any information retrieval
task.

Em seguida vamos importar mais coisas necessárias para o nosso trabalho.
Note que estamos baixando a obra completa de Machado de Assis, com a
qual iremos alimentar nosso índice.

    \begin{Verbatim}[commandchars=\\\{\}]
{\color{incolor}In [{\color{incolor}2}]:} \PY{k+kn}{from} \PY{n+nn}{nltk}\PY{n+nn}{.}\PY{n+nn}{corpus} \PY{k}{import} \PY{n}{machado}\PY{p}{,} \PY{n}{mac\PYZus{}morpho}
        \PY{k+kn}{from} \PY{n+nn}{nltk}\PY{n+nn}{.}\PY{n+nn}{tokenize} \PY{k}{import} \PY{n}{WordPunctTokenizer}
        \PY{k+kn}{from} \PY{n+nn}{nltk}\PY{n+nn}{.}\PY{n+nn}{corpus} \PY{k}{import} \PY{n}{stopwords}
        \PY{k+kn}{import} \PY{n+nn}{string}
        \PY{k+kn}{from} \PY{n+nn}{collections} \PY{k}{import} \PY{n}{defaultdict}
        \PY{k+kn}{from} \PY{n+nn}{nltk}\PY{n+nn}{.}\PY{n+nn}{stem}\PY{n+nn}{.}\PY{n+nn}{snowball} \PY{k}{import} \PY{n}{PortugueseStemmer}
\end{Verbatim}

    Vamos também baixar o banco de \emph{stopwords} do NLTK. Stop words são
um conjunto de palavras que normalmente carregam baixo conteúdo
semântico e portanto não são alvo de buscas.

    \begin{Verbatim}[commandchars=\\\{\}]
{\color{incolor}In [{\color{incolor}3}]:} \PY{n}{nltk}\PY{o}{.}\PY{n}{download}\PY{p}{(}\PY{l+s+s1}{\PYZsq{}}\PY{l+s+s1}{stopwords}\PY{l+s+s1}{\PYZsq{}}\PY{p}{)}
        \PY{n}{nltk}\PY{o}{.}\PY{n}{download}\PY{p}{(}\PY{l+s+s1}{\PYZsq{}}\PY{l+s+s1}{machado}\PY{l+s+s1}{\PYZsq{}}\PY{p}{)}
\end{Verbatim}

    \begin{Verbatim}[commandchars=\\\{\}]
[nltk\_data] Downloading package stopwords to
[nltk\_data]     /home/fccoelho/nltk\_data{\ldots}
[nltk\_data]   Package stopwords is already up-to-date!
[nltk\_data] Downloading package machado to /home/fccoelho/nltk\_data{\ldots}
[nltk\_data]   Package machado is already up-to-date!

    \end{Verbatim}

            \begin{Verbatim}[commandchars=\\\{\}]
{\color{outcolor}Out[{\color{outcolor}3}]:} True
\end{Verbatim}
        
    \begin{Verbatim}[commandchars=\\\{\}]
{\color{incolor}In [{\color{incolor}7}]:} \PY{n}{textos} \PY{o}{=} \PY{p}{[}\PY{n}{machado}\PY{o}{.}\PY{n}{raw}\PY{p}{(}\PY{n+nb}{id}\PY{p}{)} \PY{k}{for} \PY{n+nb}{id} \PY{o+ow}{in} \PY{n}{machado}\PY{o}{.}\PY{n}{fileids}\PY{p}{(}\PY{p}{)}\PY{p}{]}
\end{Verbatim}

    \begin{Verbatim}[commandchars=\\\{\}]
{\color{incolor}In [{\color{incolor}9}]:} \PY{n}{textos}\PY{p}{[}\PY{l+m+mi}{0}\PY{p}{]}\PY{p}{[}\PY{p}{:}\PY{l+m+mi}{80}\PY{p}{]}
\end{Verbatim}

            \begin{Verbatim}[commandchars=\\\{\}]
{\color{outcolor}Out[{\color{outcolor}9}]:} 'Conto, Contos Fluminenses, 1870\textbackslash{}n\textbackslash{}nContos Fluminenses\textbackslash{}n\textbackslash{}nTexto-fonte:\textbackslash{}n\textbackslash{}nObra Completa'
\end{Verbatim}
        
    Lendo o texto \emph{puro} dos livros de Machado:

    \begin{Verbatim}[commandchars=\\\{\}]
{\color{incolor}In [{\color{incolor}10}]:} \PY{c+c1}{\PYZsh{}textos = [machado.raw(id) for id in machado.fileids()]}
         \PY{c+c1}{\PYZsh{}len(textos)}
\end{Verbatim}

    \begin{Verbatim}[commandchars=\\\{\}]
{\color{incolor}In [{\color{incolor} }]:} 
\end{Verbatim}

    Carregando a lista de stopwords em lingua portuguesa para limpeza dos
textos. Note que é preciso trazer as palavras para \emph{UTF-8} antes de
usá-las.

    \begin{Verbatim}[commandchars=\\\{\}]
{\color{incolor}In [{\color{incolor}11}]:} \PY{n}{swu} \PY{o}{=} \PY{n}{stopwords}\PY{o}{.}\PY{n}{words}\PY{p}{(}\PY{l+s+s1}{\PYZsq{}}\PY{l+s+s1}{portuguese}\PY{l+s+s1}{\PYZsq{}}\PY{p}{)} \PY{o}{+} \PY{n+nb}{list} \PY{p}{(}\PY{n}{string}\PY{o}{.}\PY{n}{punctuation}\PY{p}{)}
         \PY{c+c1}{\PYZsh{}swu = [word.decode(\PYZsq{}utf8\PYZsq{}) for word in sw]}
\end{Verbatim}

    \begin{Verbatim}[commandchars=\\\{\}]
{\color{incolor}In [{\color{incolor}13}]:} \PY{n}{stopwords}\PY{o}{.}\PY{n}{words}\PY{p}{(}\PY{l+s+s1}{\PYZsq{}}\PY{l+s+s1}{portuguese}\PY{l+s+s1}{\PYZsq{}}\PY{p}{)}\PY{p}{[}\PY{p}{:}\PY{l+m+mi}{10}\PY{p}{]}
         \PY{c+c1}{\PYZsh{}list (string.punctuation)}
\end{Verbatim}

            \begin{Verbatim}[commandchars=\\\{\}]
{\color{outcolor}Out[{\color{outcolor}13}]:} ['de', 'a', 'o', 'que', 'e', 'do', 'da', 'em', 'um', 'para']
\end{Verbatim}
        
    Um outro ingrediente essencial é um stemmer para a nossa língua. O
Stemmer reduz as palavras a uma abreviação que se aproxima da ``raiz''
da palavra.

    \begin{Verbatim}[commandchars=\\\{\}]
{\color{incolor}In [{\color{incolor}18}]:} \PY{n}{stemmer} \PY{o}{=} \PY{n}{PortugueseStemmer}\PY{p}{(}\PY{p}{)}
\end{Verbatim}

    \begin{Verbatim}[commandchars=\\\{\}]
{\color{incolor}In [{\color{incolor}19}]:} \PY{n}{WordPunctTokenizer}\PY{p}{(}\PY{p}{)}\PY{o}{.}\PY{n}{tokenize}\PY{p}{(}\PY{n}{textos}\PY{p}{[}\PY{l+m+mi}{0}\PY{p}{]}\PY{p}{)}\PY{p}{[}\PY{p}{:}\PY{l+m+mi}{10}\PY{p}{]}
\end{Verbatim}

            \begin{Verbatim}[commandchars=\\\{\}]
{\color{outcolor}Out[{\color{outcolor}19}]:} ['Conto',
          ',',
          'Contos',
          'Fluminenses',
          ',',
          '1870',
          'Contos',
          'Fluminenses',
          'Texto',
          '-']
\end{Verbatim}
        
    \subsection{Preparando o Texto}\label{preparando-o-texto}

Na célula abaixo, vamos normalizar os nossos textos trazendo todas as
palavras para caixa baixa e abreviando-as de forma a deixar apenas as
suas raízes. Neste passo, removeremos também as \emph{stopwords}. Tenha
paciência, esta análise vai levar algum tempo\ldots{}

    \begin{Verbatim}[commandchars=\\\{\}]
{\color{incolor}In [{\color{incolor}20}]:} \PY{n}{textos\PYZus{}limpos} \PY{o}{=} \PY{p}{[}\PY{p}{]}
         \PY{k}{for} \PY{n}{texto} \PY{o+ow}{in} \PY{n}{textos}\PY{p}{:}
             \PY{n}{tlimpo} \PY{o}{=} \PY{p}{[}\PY{n}{stemmer}\PY{o}{.}\PY{n}{stem}\PY{p}{(}\PY{n}{token}\PY{o}{.}\PY{n}{lower}\PY{p}{(}\PY{p}{)}\PY{p}{)} \PY{k}{for} \PY{n}{token} \PY{o+ow}{in} \PY{n}{WordPunctTokenizer}\PY{p}{(}\PY{p}{)}\PY{o}{.}\PY{n}{tokenize}\PY{p}{(}\PY{n}{texto}\PY{p}{)} \PY{k}{if} \PY{n}{token} \PY{o+ow}{not} \PY{o+ow}{in} \PY{n}{swu}\PY{p}{]}
             \PY{n}{textos\PYZus{}limpos}\PY{o}{.}\PY{n}{append}\PY{p}{(}\PY{n}{tlimpo}\PY{p}{)}
\end{Verbatim}

    Vejamos uma amostra do resultado:

    \begin{Verbatim}[commandchars=\\\{\}]
{\color{incolor}In [{\color{incolor}21}]:} \PY{n}{textos\PYZus{}limpos}\PY{p}{[}\PY{l+m+mi}{0}\PY{p}{]}\PY{p}{[}\PY{l+m+mi}{150}\PY{p}{:}\PY{l+m+mi}{160}\PY{p}{]}
\end{Verbatim}

            \begin{Verbatim}[commandchars=\\\{\}]
{\color{outcolor}Out[{\color{outcolor}21}]:} ['uma', 'tal', 'miss', 'doll', 'dev', 'ter', 'poet', 'tennyson', 'cor', 'ler']
\end{Verbatim}
        
    \subsection{Construindo um Índice
Invertido}\label{construindo-um-uxedndice-invertido}

De posse da nossa lista de termos \emph{normalizados}, podemos agora
construir o nosso índice invertido.

    \begin{Verbatim}[commandchars=\\\{\}]
{\color{incolor}In [{\color{incolor}22}]:} \PY{n}{indice} \PY{o}{=} \PY{n}{defaultdict}\PY{p}{(}\PY{k}{lambda}\PY{p}{:}\PY{n+nb}{set}\PY{p}{(}\PY{p}{[}\PY{p}{]}\PY{p}{)}\PY{p}{)}
         \PY{k}{for} \PY{n}{tid}\PY{p}{,}\PY{n}{t} \PY{o+ow}{in} \PY{n+nb}{enumerate}\PY{p}{(}\PY{n}{textos\PYZus{}limpos}\PY{p}{)}\PY{p}{:}
             \PY{k}{for} \PY{n}{term} \PY{o+ow}{in} \PY{n}{t}\PY{p}{:}
                 \PY{n}{indice}\PY{p}{[}\PY{n}{term}\PY{p}{]}\PY{o}{.}\PY{n}{add}\PY{p}{(}\PY{n}{tid}\PY{p}{)}
\end{Verbatim}

    Podemos verificar a estrutura interna do nosso índice, procurando por
uma palavra:

    \begin{Verbatim}[commandchars=\\\{\}]
{\color{incolor}In [{\color{incolor}23}]:} \PY{n}{indice}\PY{p}{[}\PY{n}{stemmer}\PY{o}{.}\PY{n}{stem}\PY{p}{(}\PY{l+s+s2}{\PYZdq{}}\PY{l+s+s2}{Dollar}\PY{l+s+s2}{\PYZdq{}}\PY{p}{)}\PY{p}{]}
\end{Verbatim}

            \begin{Verbatim}[commandchars=\\\{\}]
{\color{outcolor}Out[{\color{outcolor}23}]:} \{0, 3\}
\end{Verbatim}
        
    \begin{Verbatim}[commandchars=\\\{\}]
{\color{incolor}In [{\color{incolor}24}]:} \PY{n+nb}{print}\PY{p}{(}\PY{n}{stemmer}\PY{o}{.}\PY{n}{stem}\PY{p}{(}\PY{l+s+s1}{\PYZsq{}}\PY{l+s+s1}{Salarial}\PY{l+s+s1}{\PYZsq{}}\PY{p}{)}\PY{p}{)}
\end{Verbatim}

    \begin{Verbatim}[commandchars=\\\{\}]
salarial

    \end{Verbatim}

    Vamos ver o contexto em que a palavra \emph{Salário} ocorre em um dos
textos

    \begin{Verbatim}[commandchars=\\\{\}]
{\color{incolor}In [{\color{incolor}25}]:} \PY{n}{nltk}\PY{o}{.}\PY{n}{Text}\PY{p}{(}\PY{n}{WordPunctTokenizer}\PY{p}{(}\PY{p}{)}\PY{o}{.}\PY{n}{tokenize}\PY{p}{(}\PY{n}{textos}\PY{p}{[}\PY{l+m+mi}{182}\PY{p}{]}\PY{p}{)}\PY{p}{)}\PY{o}{.}\PY{n}{concordance}\PY{p}{(}\PY{l+s+s2}{\PYZdq{}}\PY{l+s+s2}{Salário}\PY{l+s+s2}{\PYZdq{}}\PY{p}{)}
\end{Verbatim}

    \begin{Verbatim}[commandchars=\\\{\}]
Displaying 2 of 2 matches:
operários que com esse acréscimo de salário proporcionariam às suas famílias ma
s 2 horas da sesta é equivalente ao salário de meio dia , em tais casos abonado

    \end{Verbatim}

    \begin{Verbatim}[commandchars=\\\{\}]
{\color{incolor}In [{\color{incolor}26}]:} \PY{k}{def} \PY{n+nf}{busca}\PY{p}{(}\PY{n}{consulta}\PY{p}{)}\PY{p}{:}
             \PY{l+s+sd}{\PYZdq{}\PYZdq{}\PYZdq{}}
         \PY{l+s+sd}{    A construção de uma função de busca é deixada com exercício ao leitor}
         \PY{l+s+sd}{    \PYZdq{}\PYZdq{}\PYZdq{}}
             \PY{k}{pass}
\end{Verbatim}

    Mas já podemos utilizar nosso índice diretamente com alguns termos e
verificar como o mesmo é eficiente.

    \begin{Verbatim}[commandchars=\\\{\}]
{\color{incolor}In [{\color{incolor}28}]:} \PY{o}{\PYZpc{}}\PY{k}{time}
         \PY{n}{results} \PY{o}{=} \PY{n}{indice}\PY{p}{[}\PY{n}{stemmer}\PY{o}{.}\PY{n}{stem}\PY{p}{(}\PY{l+s+s1}{\PYZsq{}}\PY{l+s+s1}{nacional}\PY{l+s+s1}{\PYZsq{}}\PY{p}{)}\PY{p}{]}\PY{o}{\PYZam{}}\PY{n}{indice}\PY{p}{[}\PY{n}{stemmer}\PY{o}{.}\PY{n}{stem}\PY{p}{(}\PY{l+s+s1}{\PYZsq{}}\PY{l+s+s1}{perdi}\PY{l+s+s1}{\PYZsq{}}\PY{p}{)}\PY{p}{]} \PY{o}{\PYZhy{}} \PY{n}{indice}\PY{p}{[}\PY{n}{stemmer}\PY{o}{.}\PY{n}{stem}\PY{p}{(}\PY{l+s+s1}{\PYZsq{}}\PY{l+s+s1}{campo}\PY{l+s+s1}{\PYZsq{}}\PY{p}{)}\PY{p}{]}
         \PY{n+nb}{list}\PY{p}{(}\PY{n}{results}\PY{p}{)}\PY{p}{[}\PY{p}{:}\PY{l+m+mi}{10}\PY{p}{]}
\end{Verbatim}

    \begin{Verbatim}[commandchars=\\\{\}]
CPU times: user 0 ns, sys: 0 ns, total: 0 ns
Wall time: 6.2 µs

    \end{Verbatim}

            \begin{Verbatim}[commandchars=\\\{\}]
{\color{outcolor}Out[{\color{outcolor}28}]:} [164, 69, 122, 137, 138, 171, 155, 141, 73, 219]
\end{Verbatim}
        
    Para um exame mais preciso do tempo de execução da nossa consulta,
podemos usar a mágica do \emph{\%\%timeit}

    \begin{Verbatim}[commandchars=\\\{\}]
{\color{incolor}In [{\color{incolor}29}]:} \PY{o}{\PYZpc{}\PYZpc{}}\PY{k}{timeit}
         results = indice[stemmer.stem(\PYZsq{}nacional\PYZsq{})]\PYZam{}indice[stemmer.stem(\PYZsq{}perdi\PYZsq{})] \PYZhy{} indice[stemmer.stem(\PYZsq{}campo\PYZsq{})]
         \PYZsh{}results
\end{Verbatim}

    \begin{Verbatim}[commandchars=\\\{\}]
10000 loops, best of 3: 75.7 µs per loop

    \end{Verbatim}

    \begin{Verbatim}[commandchars=\\\{\}]
{\color{incolor}In [{\color{incolor} }]:} 
\end{Verbatim}


    % Add a bibliography block to the postdoc
    
    
    
    