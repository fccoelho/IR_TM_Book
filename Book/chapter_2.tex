\chapter{Language Models}

    
    

    
    \section{Modelos de Linguagem}\label{modelos-de-linguagem}

Modelos de Linguagem envolvem a estimação de probabilidades condicionais
de ocorrência de um termo dado um contexto.
\[P_{bi}(t_1t_2t_3t_4) = P(t_1)P(t_2|t_1)P(t_3|t_1t_2)P(t_4|t_1t_2t_3)\]

    \begin{Verbatim}[commandchars=\\\{\}]
{\color{incolor}In [{\color{incolor}7}]:} \PY{k+kn}{import} \PY{n+nn}{nltk}
        \PY{k+kn}{import} \PY{n+nn}{string}
        \PY{k+kn}{from} \PY{n+nn}{nltk}\PY{n+nn}{.}\PY{n+nn}{corpus} \PY{k}{import} \PY{n}{machado}
        \PY{k+kn}{from} \PY{n+nn}{nltk}\PY{n+nn}{.}\PY{n+nn}{corpus} \PY{k}{import} \PY{n}{stopwords}
        \PY{o}{\PYZpc{}}\PY{k}{pylab} inline
\end{Verbatim}

    \begin{Verbatim}[commandchars=\\\{\}]
Populating the interactive namespace from numpy and matplotlib

    \end{Verbatim}

    \subsubsection{Modelo Unigrama (bag of
words)}\label{modelo-unigrama-bag-of-words}

neste modelo, a ordem das palavras não importa.
\[P_{bi}(t_1t_2t_3t_4) = P(t_1)P(t_2)P(t_3)P(t_4)\]

    \begin{Verbatim}[commandchars=\\\{\}]
{\color{incolor}In [{\color{incolor}8}]:} \PY{n}{swu} \PY{o}{=} \PY{n}{stopwords}\PY{o}{.}\PY{n}{words}\PY{p}{(}\PY{l+s+s1}{\PYZsq{}}\PY{l+s+s1}{portuguese}\PY{l+s+s1}{\PYZsq{}}\PY{p}{)} \PY{o}{+} \PY{n+nb}{list} \PY{p}{(}\PY{n}{string}\PY{o}{.}\PY{n}{punctuation}\PY{p}{)}
        \PY{n}{wordgen} \PY{o}{=} \PY{p}{(}\PY{n}{word} \PY{k}{for} \PY{n}{word} \PY{o+ow}{in} \PY{n}{machado}\PY{o}{.}\PY{n}{words}\PY{p}{(}\PY{p}{)} \PY{k}{if} \PY{n}{word}\PY{o}{.}\PY{n}{lower}\PY{p}{(}\PY{p}{)} \PY{o+ow}{not} \PY{o+ow}{in} \PY{n}{swu}\PY{p}{)}
        \PY{n}{fd} \PY{o}{=} \PY{n}{nltk}\PY{o}{.}\PY{n}{FreqDist}\PY{p}{(}\PY{n}{wordgen}\PY{p}{)}
        \PY{n}{fd}\PY{o}{.}\PY{n}{most\PYZus{}common}\PY{p}{(}\PY{l+m+mi}{20}\PY{p}{)}
\end{Verbatim}

            \begin{Verbatim}[commandchars=\\\{\}]
{\color{outcolor}Out[{\color{outcolor}8}]:} [('\textbackslash{}x97', 25386),
         ('é', 22286),
         ('--', 9643),
         ('{\ldots}', 8180),
         ('disse', 6444),
         ('ser', 4831),
         ('casa', 4821),
         ('tempo', 4339),
         ('ainda', 4226),
         ('coisa', 3999),
         ('dia', 3959),
         ('olhos', 3936),
         ('tudo', 3879),
         ('lo', 3726),
         ('outro', 3652),
         ('nada', 3648),
         ('tão', 3592),
         ('outra', 3553),
         ('D', 3495),
         ('dois', 3494)]
\end{Verbatim}
        
    \begin{Verbatim}[commandchars=\\\{\}]
{\color{incolor}In [{\color{incolor}9}]:} \PY{n}{x}\PY{p}{,}\PY{n}{y} \PY{o}{=} \PY{n+nb}{zip}\PY{p}{(}\PY{o}{*}\PY{n}{fd}\PY{o}{.}\PY{n}{most\PYZus{}common}\PY{p}{(}\PY{l+m+mi}{20}\PY{p}{)}\PY{p}{)}
\end{Verbatim}

    \begin{Verbatim}[commandchars=\\\{\}]
{\color{incolor}In [{\color{incolor}10}]:} \PY{n}{barh}\PY{p}{(}\PY{n}{bottom}\PY{o}{=}\PY{n+nb}{range}\PY{p}{(}\PY{n+nb}{len}\PY{p}{(}\PY{n}{x}\PY{p}{)}\PY{p}{)}\PY{p}{,} \PY{n}{width}\PY{o}{=}\PY{n}{y}\PY{p}{)}\PY{p}{;}
         \PY{n}{ax} \PY{o}{=} \PY{n}{gca}\PY{p}{(}\PY{p}{)}
         \PY{n}{ax}\PY{o}{.}\PY{n}{set}\PY{p}{(}\PY{n}{yticks}\PY{o}{=}\PY{n+nb}{range}\PY{p}{(}\PY{n+nb}{len}\PY{p}{(}\PY{n}{x}\PY{p}{)}\PY{p}{)}\PY{p}{,}\PY{n}{yticklabels}\PY{o}{=}\PY{n}{x}\PY{p}{)}\PY{p}{;}
         \PY{n}{xlabel}\PY{p}{(}\PY{l+s+s1}{\PYZsq{}}\PY{l+s+s1}{frequência}\PY{l+s+s1}{\PYZsq{}}\PY{p}{)}\PY{p}{;}
\end{Verbatim}

    \begin{center}
    \adjustimage{max size={0.9\linewidth}{0.9\paperheight}}{Language Models_files/Language Models_5_0.png}
    \end{center}
    { \hspace*{\fill} \\}
    
    \subsubsection{Bigramas}\label{bigramas}

Uma alternativa ao modelo bag-of-words onde as probabilidades são
independentes, é o modelo bigrama. onde calculamos a probabilidade de
ocorrência de uma palavra condicionada à ocorrência da anterior.
\[P_{bi}(t_1t_2t_3t_4) = P(t_1)P(t_2|t_1)P(t_3|t_2)P(t_4|t_3)\]
nltk.ConditionalFreqDist é um estimador de probabilidades condicionais.
Dada uma lista de bigramas, para cada palavra no vocabulário, ele
calculará uma distribuição de frequências para a próxima palavra.

    \begin{Verbatim}[commandchars=\\\{\}]
{\color{incolor}In [{\color{incolor}11}]:} \PY{n}{wordgen} \PY{o}{=} \PY{p}{(}\PY{n}{word} \PY{k}{for} \PY{n}{word} \PY{o+ow}{in} \PY{n}{machado}\PY{o}{.}\PY{n}{words}\PY{p}{(}\PY{p}{)} \PY{p}{)}\PY{c+c1}{\PYZsh{}if word.lower() not in swu)}
         \PY{n}{machado\PYZus{}2gram} \PY{o}{=} \PY{n}{nltk}\PY{o}{.}\PY{n}{ConditionalFreqDist}\PY{p}{(}\PY{n}{nltk}\PY{o}{.}\PY{n}{bigrams}\PY{p}{(}\PY{n}{wordgen}\PY{p}{)}\PY{p}{)}
\end{Verbatim}

    \emph{conditions()} retorna as palavras para as quais temos
distribuições condicionais construídas.

    \begin{Verbatim}[commandchars=\\\{\}]
{\color{incolor}In [{\color{incolor}12}]:} \PY{n}{machado\PYZus{}2gram}\PY{o}{.}\PY{n}{conditions}\PY{p}{(}\PY{p}{)}\PY{p}{[}\PY{p}{:}\PY{l+m+mi}{10}\PY{p}{]}
\end{Verbatim}

            \begin{Verbatim}[commandchars=\\\{\}]
{\color{outcolor}Out[{\color{outcolor}12}]:} ['policia',
          'abat',
          'victa',
          'deveres',
          'dinástica',
          'consistiam',
          'frenéticos',
          'Diga',
          'bolsinha',
          'federação']
\end{Verbatim}
        
    Podemos obter a contagem de palavras que se seguem à uma palavra
específica.

    \begin{Verbatim}[commandchars=\\\{\}]
{\color{incolor}In [{\color{incolor}13}]:} \PY{n}{machado\PYZus{}2gram}\PY{p}{[}\PY{l+s+s1}{\PYZsq{}}\PY{l+s+s1}{exercia}\PY{l+s+s1}{\PYZsq{}}\PY{p}{]}
\end{Verbatim}

            \begin{Verbatim}[commandchars=\\\{\}]
{\color{outcolor}Out[{\color{outcolor}13}]:} FreqDist(\{'-': 1,
                   'a': 3,
                   'aquele': 1,
                   'as': 1,
                   'com': 1,
                   'de': 1,
                   'desde': 2,
                   'em': 2,
                   'então': 1,
                   'esse': 1,
                   'interinamente': 1,
                   'mais': 1,
                   'na': 1,
                   'nela': 3,
                   'nele': 1,
                   'no': 1,
                   'o': 4,
                   'outra': 1,
                   'realmente': 1,
                   'sobre': 4,
                   'um': 3,
                   'uma': 1\})
\end{Verbatim}
        
    Ou podemos obter a Probabilidade por máxima verossimilhança:

    \begin{Verbatim}[commandchars=\\\{\}]
{\color{incolor}In [{\color{incolor}14}]:} \PY{n}{machado\PYZus{}2gram\PYZus{}cp} \PY{o}{=} \PY{n}{nltk}\PY{o}{.}\PY{n}{ConditionalProbDist}\PY{p}{(}\PY{n}{machado\PYZus{}2gram}\PY{p}{,} \PY{n}{nltk}\PY{o}{.}\PY{n}{MLEProbDist}\PY{p}{)}
\end{Verbatim}

    \begin{Verbatim}[commandchars=\\\{\}]
{\color{incolor}In [{\color{incolor}15}]:} \PY{n}{cp} \PY{o}{=} \PY{n}{machado\PYZus{}2gram\PYZus{}cp}\PY{p}{[}\PY{l+s+s1}{\PYZsq{}}\PY{l+s+s1}{exercia}\PY{l+s+s1}{\PYZsq{}}\PY{p}{]}
         \PY{n}{cp}\PY{o}{.}\PY{n}{samples}\PY{p}{(}\PY{p}{)}
\end{Verbatim}

            \begin{Verbatim}[commandchars=\\\{\}]
{\color{outcolor}Out[{\color{outcolor}15}]:} dict\_keys(['realmente', 'um', 'a', 'interinamente', 'em', 'na', 'uma', '-', 'as', 'o', 'esse', 'de', 'desde', 'então', 'nele', 'outra', 'sobre', 'nela', 'com', 'aquele', 'mais', 'no'])
\end{Verbatim}
        
    \begin{Verbatim}[commandchars=\\\{\}]
{\color{incolor}In [{\color{incolor}16}]:} \PY{n}{cp}\PY{o}{.}\PY{n}{prob}\PY{p}{(}\PY{l+s+s1}{\PYZsq{}}\PY{l+s+s1}{sobre}\PY{l+s+s1}{\PYZsq{}}\PY{p}{)}
\end{Verbatim}

            \begin{Verbatim}[commandchars=\\\{\}]
{\color{outcolor}Out[{\color{outcolor}16}]:} 0.1111111111111111
\end{Verbatim}
        
    \begin{Verbatim}[commandchars=\\\{\}]
{\color{incolor}In [{\color{incolor}17}]:} \PY{k}{def} \PY{n+nf}{gera\PYZus{}texto}\PY{p}{(}\PY{n}{palavra\PYZus{}inicial}\PY{p}{,} \PY{n}{cpd}\PY{p}{,}\PY{n}{numero}\PY{o}{=}\PY{l+m+mi}{50}\PY{p}{)}\PY{p}{:}
             
             \PY{n}{w} \PY{o}{=} \PY{n}{palavra\PYZus{}inicial}
             \PY{n}{texto} \PY{o}{=} \PY{n}{w}
             \PY{k}{for} \PY{n}{i} \PY{o+ow}{in} \PY{n+nb}{range}\PY{p}{(}\PY{n}{numero}\PY{p}{)}\PY{p}{:}
                 \PY{n}{w2} \PY{o}{=} \PY{n}{cpd}\PY{p}{[}\PY{n}{w}\PY{p}{]}\PY{o}{.}\PY{n}{generate}\PY{p}{(}\PY{p}{)}
                 \PY{n}{texto} \PY{o}{+}\PY{o}{=} \PY{l+s+s1}{\PYZsq{}}\PY{l+s+s1}{ }\PY{l+s+s1}{\PYZsq{}} \PY{o}{+} \PY{n}{w2}
                 \PY{n}{w} \PY{o}{=} \PY{n}{w2}
             \PY{k}{return} \PY{n}{texto}
\end{Verbatim}

    \begin{Verbatim}[commandchars=\\\{\}]
{\color{incolor}In [{\color{incolor}18}]:} \PY{n}{gera\PYZus{}texto}\PY{p}{(}\PY{l+s+s1}{\PYZsq{}}\PY{l+s+s1}{Cresceu}\PY{l+s+s1}{\PYZsq{}}\PY{p}{,}\PY{n}{machado\PYZus{}2gram\PYZus{}cp}\PY{p}{)}
\end{Verbatim}

            \begin{Verbatim}[commandchars=\\\{\}]
{\color{outcolor}Out[{\color{outcolor}18}]:} 'Cresceu ainda hoje e conclui ? VULCANO É cruel , meias {\ldots} Pedro é ? \textbackslash{}x97 Ora essa esperança . Paulo . Excia . Maria Benedita . Tendo o primeiro ocorre , quando um do século não acreditava firmemente a Com os olhos da filha , e adorados que até a'
\end{Verbatim}
        
    \begin{Verbatim}[commandchars=\\\{\}]
{\color{incolor}In [{\color{incolor} }]:} 
\end{Verbatim}


    % Add a bibliography block to the postdoc
    
    
    
    