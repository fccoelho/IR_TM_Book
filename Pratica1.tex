
% Default to the notebook output style

    


% Inherit from the specified cell style.




    
\documentclass[11pt]{article}

    
    
    \usepackage[T1]{fontenc}
    % Nicer default font than Computer Modern for most use cases
    \usepackage{palatino}

    % Basic figure setup, for now with no caption control since it's done
    % automatically by Pandoc (which extracts ![](path) syntax from Markdown).
    \usepackage{graphicx}
    % We will generate all images so they have a width \maxwidth. This means
    % that they will get their normal width if they fit onto the page, but
    % are scaled down if they would overflow the margins.
    \makeatletter
    \def\maxwidth{\ifdim\Gin@nat@width>\linewidth\linewidth
    \else\Gin@nat@width\fi}
    \makeatother
    \let\Oldincludegraphics\includegraphics
    % Set max figure width to be 80% of text width, for now hardcoded.
    \renewcommand{\includegraphics}[1]{\Oldincludegraphics[width=.8\maxwidth]{#1}}
    % Ensure that by default, figures have no caption (until we provide a
    % proper Figure object with a Caption API and a way to capture that
    % in the conversion process - todo).
    \usepackage{caption}
    \DeclareCaptionLabelFormat{nolabel}{}
    \captionsetup{labelformat=nolabel}

    \usepackage{adjustbox} % Used to constrain images to a maximum size 
    \usepackage{xcolor} % Allow colors to be defined
    \usepackage{enumerate} % Needed for markdown enumerations to work
    \usepackage{geometry} % Used to adjust the document margins
    \usepackage{amsmath} % Equations
    \usepackage{amssymb} % Equations
    \usepackage{textcomp} % defines textquotesingle
    % Hack from http://tex.stackexchange.com/a/47451/13684:
    \AtBeginDocument{%
        \def\PYZsq{\textquotesingle}% Upright quotes in Pygmentized code
    }
    \usepackage{upquote} % Upright quotes for verbatim code
    \usepackage{eurosym} % defines \euro
    \usepackage[mathletters]{ucs} % Extended unicode (utf-8) support
    \usepackage[utf8x]{inputenc} % Allow utf-8 characters in the tex document
    \usepackage{fancyvrb} % verbatim replacement that allows latex
    \usepackage{grffile} % extends the file name processing of package graphics 
                         % to support a larger range 
    % The hyperref package gives us a pdf with properly built
    % internal navigation ('pdf bookmarks' for the table of contents,
    % internal cross-reference links, web links for URLs, etc.)
    \usepackage{hyperref}
    \usepackage{longtable} % longtable support required by pandoc >1.10
    \usepackage{booktabs}  % table support for pandoc > 1.12.2
    \usepackage[normalem]{ulem} % ulem is needed to support strikethroughs (\sout)
                                % normalem makes italics be italics, not underlines
    

    
    
    % Colors for the hyperref package
    \definecolor{urlcolor}{rgb}{0,.145,.698}
    \definecolor{linkcolor}{rgb}{.71,0.21,0.01}
    \definecolor{citecolor}{rgb}{.12,.54,.11}

    % ANSI colors
    \definecolor{ansi-black}{HTML}{3E424D}
    \definecolor{ansi-black-intense}{HTML}{282C36}
    \definecolor{ansi-red}{HTML}{E75C58}
    \definecolor{ansi-red-intense}{HTML}{B22B31}
    \definecolor{ansi-green}{HTML}{00A250}
    \definecolor{ansi-green-intense}{HTML}{007427}
    \definecolor{ansi-yellow}{HTML}{DDB62B}
    \definecolor{ansi-yellow-intense}{HTML}{B27D12}
    \definecolor{ansi-blue}{HTML}{208FFB}
    \definecolor{ansi-blue-intense}{HTML}{0065CA}
    \definecolor{ansi-magenta}{HTML}{D160C4}
    \definecolor{ansi-magenta-intense}{HTML}{A03196}
    \definecolor{ansi-cyan}{HTML}{60C6C8}
    \definecolor{ansi-cyan-intense}{HTML}{258F8F}
    \definecolor{ansi-white}{HTML}{C5C1B4}
    \definecolor{ansi-white-intense}{HTML}{A1A6B2}

    % commands and environments needed by pandoc snippets
    % extracted from the output of `pandoc -s`
    \providecommand{\tightlist}{%
      \setlength{\itemsep}{0pt}\setlength{\parskip}{0pt}}
    \DefineVerbatimEnvironment{Highlighting}{Verbatim}{commandchars=\\\{\}}
    % Add ',fontsize=\small' for more characters per line
    \newenvironment{Shaded}{}{}
    \newcommand{\KeywordTok}[1]{\textcolor[rgb]{0.00,0.44,0.13}{\textbf{{#1}}}}
    \newcommand{\DataTypeTok}[1]{\textcolor[rgb]{0.56,0.13,0.00}{{#1}}}
    \newcommand{\DecValTok}[1]{\textcolor[rgb]{0.25,0.63,0.44}{{#1}}}
    \newcommand{\BaseNTok}[1]{\textcolor[rgb]{0.25,0.63,0.44}{{#1}}}
    \newcommand{\FloatTok}[1]{\textcolor[rgb]{0.25,0.63,0.44}{{#1}}}
    \newcommand{\CharTok}[1]{\textcolor[rgb]{0.25,0.44,0.63}{{#1}}}
    \newcommand{\StringTok}[1]{\textcolor[rgb]{0.25,0.44,0.63}{{#1}}}
    \newcommand{\CommentTok}[1]{\textcolor[rgb]{0.38,0.63,0.69}{\textit{{#1}}}}
    \newcommand{\OtherTok}[1]{\textcolor[rgb]{0.00,0.44,0.13}{{#1}}}
    \newcommand{\AlertTok}[1]{\textcolor[rgb]{1.00,0.00,0.00}{\textbf{{#1}}}}
    \newcommand{\FunctionTok}[1]{\textcolor[rgb]{0.02,0.16,0.49}{{#1}}}
    \newcommand{\RegionMarkerTok}[1]{{#1}}
    \newcommand{\ErrorTok}[1]{\textcolor[rgb]{1.00,0.00,0.00}{\textbf{{#1}}}}
    \newcommand{\NormalTok}[1]{{#1}}
    
    % Additional commands for more recent versions of Pandoc
    \newcommand{\ConstantTok}[1]{\textcolor[rgb]{0.53,0.00,0.00}{{#1}}}
    \newcommand{\SpecialCharTok}[1]{\textcolor[rgb]{0.25,0.44,0.63}{{#1}}}
    \newcommand{\VerbatimStringTok}[1]{\textcolor[rgb]{0.25,0.44,0.63}{{#1}}}
    \newcommand{\SpecialStringTok}[1]{\textcolor[rgb]{0.73,0.40,0.53}{{#1}}}
    \newcommand{\ImportTok}[1]{{#1}}
    \newcommand{\DocumentationTok}[1]{\textcolor[rgb]{0.73,0.13,0.13}{\textit{{#1}}}}
    \newcommand{\AnnotationTok}[1]{\textcolor[rgb]{0.38,0.63,0.69}{\textbf{\textit{{#1}}}}}
    \newcommand{\CommentVarTok}[1]{\textcolor[rgb]{0.38,0.63,0.69}{\textbf{\textit{{#1}}}}}
    \newcommand{\VariableTok}[1]{\textcolor[rgb]{0.10,0.09,0.49}{{#1}}}
    \newcommand{\ControlFlowTok}[1]{\textcolor[rgb]{0.00,0.44,0.13}{\textbf{{#1}}}}
    \newcommand{\OperatorTok}[1]{\textcolor[rgb]{0.40,0.40,0.40}{{#1}}}
    \newcommand{\BuiltInTok}[1]{{#1}}
    \newcommand{\ExtensionTok}[1]{{#1}}
    \newcommand{\PreprocessorTok}[1]{\textcolor[rgb]{0.74,0.48,0.00}{{#1}}}
    \newcommand{\AttributeTok}[1]{\textcolor[rgb]{0.49,0.56,0.16}{{#1}}}
    \newcommand{\InformationTok}[1]{\textcolor[rgb]{0.38,0.63,0.69}{\textbf{\textit{{#1}}}}}
    \newcommand{\WarningTok}[1]{\textcolor[rgb]{0.38,0.63,0.69}{\textbf{\textit{{#1}}}}}
    
    
    % Define a nice break command that doesn't care if a line doesn't already
    % exist.
    \def\br{\hspace*{\fill} \\* }
    % Math Jax compatability definitions
    \def\gt{>}
    \def\lt{<}
    % Document parameters
    \title{Pratica1}
    
    
    

    % Pygments definitions
    
\makeatletter
\def\PY@reset{\let\PY@it=\relax \let\PY@bf=\relax%
    \let\PY@ul=\relax \let\PY@tc=\relax%
    \let\PY@bc=\relax \let\PY@ff=\relax}
\def\PY@tok#1{\csname PY@tok@#1\endcsname}
\def\PY@toks#1+{\ifx\relax#1\empty\else%
    \PY@tok{#1}\expandafter\PY@toks\fi}
\def\PY@do#1{\PY@bc{\PY@tc{\PY@ul{%
    \PY@it{\PY@bf{\PY@ff{#1}}}}}}}
\def\PY#1#2{\PY@reset\PY@toks#1+\relax+\PY@do{#2}}

\expandafter\def\csname PY@tok@w\endcsname{\def\PY@tc##1{\textcolor[rgb]{0.73,0.73,0.73}{##1}}}
\expandafter\def\csname PY@tok@gh\endcsname{\let\PY@bf=\textbf\def\PY@tc##1{\textcolor[rgb]{0.00,0.00,0.50}{##1}}}
\expandafter\def\csname PY@tok@nl\endcsname{\def\PY@tc##1{\textcolor[rgb]{0.63,0.63,0.00}{##1}}}
\expandafter\def\csname PY@tok@sb\endcsname{\def\PY@tc##1{\textcolor[rgb]{0.73,0.13,0.13}{##1}}}
\expandafter\def\csname PY@tok@gp\endcsname{\let\PY@bf=\textbf\def\PY@tc##1{\textcolor[rgb]{0.00,0.00,0.50}{##1}}}
\expandafter\def\csname PY@tok@sx\endcsname{\def\PY@tc##1{\textcolor[rgb]{0.00,0.50,0.00}{##1}}}
\expandafter\def\csname PY@tok@vg\endcsname{\def\PY@tc##1{\textcolor[rgb]{0.10,0.09,0.49}{##1}}}
\expandafter\def\csname PY@tok@ni\endcsname{\let\PY@bf=\textbf\def\PY@tc##1{\textcolor[rgb]{0.60,0.60,0.60}{##1}}}
\expandafter\def\csname PY@tok@mh\endcsname{\def\PY@tc##1{\textcolor[rgb]{0.40,0.40,0.40}{##1}}}
\expandafter\def\csname PY@tok@mb\endcsname{\def\PY@tc##1{\textcolor[rgb]{0.40,0.40,0.40}{##1}}}
\expandafter\def\csname PY@tok@cs\endcsname{\let\PY@it=\textit\def\PY@tc##1{\textcolor[rgb]{0.25,0.50,0.50}{##1}}}
\expandafter\def\csname PY@tok@sh\endcsname{\def\PY@tc##1{\textcolor[rgb]{0.73,0.13,0.13}{##1}}}
\expandafter\def\csname PY@tok@nc\endcsname{\let\PY@bf=\textbf\def\PY@tc##1{\textcolor[rgb]{0.00,0.00,1.00}{##1}}}
\expandafter\def\csname PY@tok@cp\endcsname{\def\PY@tc##1{\textcolor[rgb]{0.74,0.48,0.00}{##1}}}
\expandafter\def\csname PY@tok@gd\endcsname{\def\PY@tc##1{\textcolor[rgb]{0.63,0.00,0.00}{##1}}}
\expandafter\def\csname PY@tok@bp\endcsname{\def\PY@tc##1{\textcolor[rgb]{0.00,0.50,0.00}{##1}}}
\expandafter\def\csname PY@tok@gt\endcsname{\def\PY@tc##1{\textcolor[rgb]{0.00,0.27,0.87}{##1}}}
\expandafter\def\csname PY@tok@ow\endcsname{\let\PY@bf=\textbf\def\PY@tc##1{\textcolor[rgb]{0.67,0.13,1.00}{##1}}}
\expandafter\def\csname PY@tok@cpf\endcsname{\let\PY@it=\textit\def\PY@tc##1{\textcolor[rgb]{0.25,0.50,0.50}{##1}}}
\expandafter\def\csname PY@tok@sd\endcsname{\let\PY@it=\textit\def\PY@tc##1{\textcolor[rgb]{0.73,0.13,0.13}{##1}}}
\expandafter\def\csname PY@tok@kr\endcsname{\let\PY@bf=\textbf\def\PY@tc##1{\textcolor[rgb]{0.00,0.50,0.00}{##1}}}
\expandafter\def\csname PY@tok@o\endcsname{\def\PY@tc##1{\textcolor[rgb]{0.40,0.40,0.40}{##1}}}
\expandafter\def\csname PY@tok@si\endcsname{\let\PY@bf=\textbf\def\PY@tc##1{\textcolor[rgb]{0.73,0.40,0.53}{##1}}}
\expandafter\def\csname PY@tok@nf\endcsname{\def\PY@tc##1{\textcolor[rgb]{0.00,0.00,1.00}{##1}}}
\expandafter\def\csname PY@tok@kc\endcsname{\let\PY@bf=\textbf\def\PY@tc##1{\textcolor[rgb]{0.00,0.50,0.00}{##1}}}
\expandafter\def\csname PY@tok@nn\endcsname{\let\PY@bf=\textbf\def\PY@tc##1{\textcolor[rgb]{0.00,0.00,1.00}{##1}}}
\expandafter\def\csname PY@tok@sr\endcsname{\def\PY@tc##1{\textcolor[rgb]{0.73,0.40,0.53}{##1}}}
\expandafter\def\csname PY@tok@c1\endcsname{\let\PY@it=\textit\def\PY@tc##1{\textcolor[rgb]{0.25,0.50,0.50}{##1}}}
\expandafter\def\csname PY@tok@na\endcsname{\def\PY@tc##1{\textcolor[rgb]{0.49,0.56,0.16}{##1}}}
\expandafter\def\csname PY@tok@no\endcsname{\def\PY@tc##1{\textcolor[rgb]{0.53,0.00,0.00}{##1}}}
\expandafter\def\csname PY@tok@gs\endcsname{\let\PY@bf=\textbf}
\expandafter\def\csname PY@tok@vi\endcsname{\def\PY@tc##1{\textcolor[rgb]{0.10,0.09,0.49}{##1}}}
\expandafter\def\csname PY@tok@nt\endcsname{\let\PY@bf=\textbf\def\PY@tc##1{\textcolor[rgb]{0.00,0.50,0.00}{##1}}}
\expandafter\def\csname PY@tok@m\endcsname{\def\PY@tc##1{\textcolor[rgb]{0.40,0.40,0.40}{##1}}}
\expandafter\def\csname PY@tok@mo\endcsname{\def\PY@tc##1{\textcolor[rgb]{0.40,0.40,0.40}{##1}}}
\expandafter\def\csname PY@tok@mi\endcsname{\def\PY@tc##1{\textcolor[rgb]{0.40,0.40,0.40}{##1}}}
\expandafter\def\csname PY@tok@se\endcsname{\let\PY@bf=\textbf\def\PY@tc##1{\textcolor[rgb]{0.73,0.40,0.13}{##1}}}
\expandafter\def\csname PY@tok@nb\endcsname{\def\PY@tc##1{\textcolor[rgb]{0.00,0.50,0.00}{##1}}}
\expandafter\def\csname PY@tok@err\endcsname{\def\PY@bc##1{\setlength{\fboxsep}{0pt}\fcolorbox[rgb]{1.00,0.00,0.00}{1,1,1}{\strut ##1}}}
\expandafter\def\csname PY@tok@nd\endcsname{\def\PY@tc##1{\textcolor[rgb]{0.67,0.13,1.00}{##1}}}
\expandafter\def\csname PY@tok@kp\endcsname{\def\PY@tc##1{\textcolor[rgb]{0.00,0.50,0.00}{##1}}}
\expandafter\def\csname PY@tok@nv\endcsname{\def\PY@tc##1{\textcolor[rgb]{0.10,0.09,0.49}{##1}}}
\expandafter\def\csname PY@tok@k\endcsname{\let\PY@bf=\textbf\def\PY@tc##1{\textcolor[rgb]{0.00,0.50,0.00}{##1}}}
\expandafter\def\csname PY@tok@cm\endcsname{\let\PY@it=\textit\def\PY@tc##1{\textcolor[rgb]{0.25,0.50,0.50}{##1}}}
\expandafter\def\csname PY@tok@gr\endcsname{\def\PY@tc##1{\textcolor[rgb]{1.00,0.00,0.00}{##1}}}
\expandafter\def\csname PY@tok@kt\endcsname{\def\PY@tc##1{\textcolor[rgb]{0.69,0.00,0.25}{##1}}}
\expandafter\def\csname PY@tok@s2\endcsname{\def\PY@tc##1{\textcolor[rgb]{0.73,0.13,0.13}{##1}}}
\expandafter\def\csname PY@tok@ch\endcsname{\let\PY@it=\textit\def\PY@tc##1{\textcolor[rgb]{0.25,0.50,0.50}{##1}}}
\expandafter\def\csname PY@tok@s\endcsname{\def\PY@tc##1{\textcolor[rgb]{0.73,0.13,0.13}{##1}}}
\expandafter\def\csname PY@tok@gi\endcsname{\def\PY@tc##1{\textcolor[rgb]{0.00,0.63,0.00}{##1}}}
\expandafter\def\csname PY@tok@c\endcsname{\let\PY@it=\textit\def\PY@tc##1{\textcolor[rgb]{0.25,0.50,0.50}{##1}}}
\expandafter\def\csname PY@tok@s1\endcsname{\def\PY@tc##1{\textcolor[rgb]{0.73,0.13,0.13}{##1}}}
\expandafter\def\csname PY@tok@ss\endcsname{\def\PY@tc##1{\textcolor[rgb]{0.10,0.09,0.49}{##1}}}
\expandafter\def\csname PY@tok@gu\endcsname{\let\PY@bf=\textbf\def\PY@tc##1{\textcolor[rgb]{0.50,0.00,0.50}{##1}}}
\expandafter\def\csname PY@tok@ge\endcsname{\let\PY@it=\textit}
\expandafter\def\csname PY@tok@vc\endcsname{\def\PY@tc##1{\textcolor[rgb]{0.10,0.09,0.49}{##1}}}
\expandafter\def\csname PY@tok@ne\endcsname{\let\PY@bf=\textbf\def\PY@tc##1{\textcolor[rgb]{0.82,0.25,0.23}{##1}}}
\expandafter\def\csname PY@tok@kn\endcsname{\let\PY@bf=\textbf\def\PY@tc##1{\textcolor[rgb]{0.00,0.50,0.00}{##1}}}
\expandafter\def\csname PY@tok@kd\endcsname{\let\PY@bf=\textbf\def\PY@tc##1{\textcolor[rgb]{0.00,0.50,0.00}{##1}}}
\expandafter\def\csname PY@tok@mf\endcsname{\def\PY@tc##1{\textcolor[rgb]{0.40,0.40,0.40}{##1}}}
\expandafter\def\csname PY@tok@sc\endcsname{\def\PY@tc##1{\textcolor[rgb]{0.73,0.13,0.13}{##1}}}
\expandafter\def\csname PY@tok@go\endcsname{\def\PY@tc##1{\textcolor[rgb]{0.53,0.53,0.53}{##1}}}
\expandafter\def\csname PY@tok@il\endcsname{\def\PY@tc##1{\textcolor[rgb]{0.40,0.40,0.40}{##1}}}

\def\PYZbs{\char`\\}
\def\PYZus{\char`\_}
\def\PYZob{\char`\{}
\def\PYZcb{\char`\}}
\def\PYZca{\char`\^}
\def\PYZam{\char`\&}
\def\PYZlt{\char`\<}
\def\PYZgt{\char`\>}
\def\PYZsh{\char`\#}
\def\PYZpc{\char`\%}
\def\PYZdl{\char`\$}
\def\PYZhy{\char`\-}
\def\PYZsq{\char`\'}
\def\PYZdq{\char`\"}
\def\PYZti{\char`\~}
% for compatibility with earlier versions
\def\PYZat{@}
\def\PYZlb{[}
\def\PYZrb{]}
\makeatother


    % Exact colors from NB
    \definecolor{incolor}{rgb}{0.0, 0.0, 0.5}
    \definecolor{outcolor}{rgb}{0.545, 0.0, 0.0}



    
    % Prevent overflowing lines due to hard-to-break entities
    \sloppy 
    % Setup hyperref package
    \hypersetup{
      breaklinks=true,  % so long urls are correctly broken across lines
      colorlinks=true,
      urlcolor=urlcolor,
      linkcolor=linkcolor,
      citecolor=citecolor,
      }
    % Slightly bigger margins than the latex defaults
    
    \geometry{verbose,tmargin=1in,bmargin=1in,lmargin=1in,rmargin=1in}
    
    

    \begin{document}
    
    
    \maketitle
    
    

    
    \section{Índices invertidos e busca
booleana}\label{uxedndices-invertidos-e-busca-booleana}

\emph{Flávio Codeço Coelho (Com a contribuição dos alunos do curso de
Sistemas de recuperação de Informações da EMAp)}

Nesta prática, vamos contruir um indice invertido e uma máquina de busca
booleana simples.

Para agilizar nosso trabalho, vamos utilizar a biblioteca
\href{http://nltk.org}{NLTK} para processamento de linguagem natural.

    \begin{Verbatim}[commandchars=\\\{\}]
{\color{incolor}In [{\color{incolor}17}]:} \PY{k+kn}{import} \PY{n+nn}{nltk}
         \PY{k+kn}{import} \PY{n+nn}{os}
\end{Verbatim}

    Em seguida vamos importar mais coisas necessárias para o nosso trabalho.
Note que estamos baixando a obra completa de Machado de Assis, com a
qual iremos alimentar nosso índice.

    \begin{Verbatim}[commandchars=\\\{\}]
{\color{incolor}In [{\color{incolor}2}]:} \PY{k+kn}{from} \PY{n+nn}{nltk.corpus} \PY{k+kn}{import} \PY{n}{machado}\PY{p}{,} \PY{n}{mac\PYZus{}morpho}
        \PY{k+kn}{from} \PY{n+nn}{nltk.tokenize} \PY{k+kn}{import} \PY{n}{WordPunctTokenizer}
        \PY{k+kn}{from} \PY{n+nn}{nltk.corpus} \PY{k+kn}{import} \PY{n}{stopwords}
        \PY{k+kn}{import} \PY{n+nn}{string}
        \PY{k+kn}{from} \PY{n+nn}{collections} \PY{k+kn}{import} \PY{n}{defaultdict}
        \PY{k+kn}{from} \PY{n+nn}{nltk.stem.snowball} \PY{k+kn}{import} \PY{n}{PortugueseStemmer}
\end{Verbatim}

    Vamos também baixar o banco de \emph{stopwords} do NLTK. Stop words são
um conjunto de palavras que normalmente carregam baixo conteúdo
semântico e portanto não são alvo de buscas.

    \begin{Verbatim}[commandchars=\\\{\}]
{\color{incolor}In [{\color{incolor}3}]:} \PY{n}{nltk}\PY{o}{.}\PY{n}{download}\PY{p}{(}\PY{l+s+s1}{\PYZsq{}}\PY{l+s+s1}{stopwords}\PY{l+s+s1}{\PYZsq{}}\PY{p}{)}
        \PY{n}{nltk}\PY{o}{.}\PY{n}{download}\PY{p}{(}\PY{l+s+s1}{\PYZsq{}}\PY{l+s+s1}{machado}\PY{l+s+s1}{\PYZsq{}}\PY{p}{)}
\end{Verbatim}

    \begin{Verbatim}[commandchars=\\\{\}]
[nltk\_data] Downloading package stopwords to
[nltk\_data]     C:\textbackslash{}Users\textbackslash{}fccoelho\textbackslash{}AppData\textbackslash{}Roaming\textbackslash{}nltk\_data{\ldots}
[nltk\_data]   Unzipping corpora\textbackslash{}stopwords.zip.
[nltk\_data] Downloading package machado to
[nltk\_data]     C:\textbackslash{}Users\textbackslash{}fccoelho\textbackslash{}AppData\textbackslash{}Roaming\textbackslash{}nltk\_data{\ldots}

    \end{Verbatim}

            \begin{Verbatim}[commandchars=\\\{\}]
{\color{outcolor}Out[{\color{outcolor}3}]:} True
\end{Verbatim}
        
    \begin{Verbatim}[commandchars=\\\{\}]
{\color{incolor}In [{\color{incolor}21}]:} \PY{n}{textos} \PY{o}{=} \PY{p}{[}\PY{p}{]}
         \PY{k}{for} \PY{n}{p}\PY{p}{,} \PY{n}{d}\PY{p}{,} \PY{n}{f} \PY{o+ow}{in} \PY{n}{os}\PY{o}{.}\PY{n}{walk}\PY{p}{(}\PY{l+s+s1}{r\PYZsq{}}\PY{l+s+s1}{machado}\PY{l+s+s1}{\PYZbs{}}\PY{l+s+s1}{machado}\PY{l+s+s1}{\PYZsq{}}\PY{p}{)}\PY{p}{:}
             \PY{c+c1}{\PYZsh{}print( p,d,f)}
             \PY{k}{if} \PY{n}{f}\PY{p}{:}
                 \PY{k}{for} \PY{n}{fileid}  \PY{o+ow}{in} \PY{n}{f}\PY{p}{:}
                     \PY{k}{if} \PY{o+ow}{not} \PY{n}{fileid}\PY{o}{.}\PY{n}{endswith}\PY{p}{(}\PY{l+s+s1}{\PYZsq{}}\PY{l+s+s1}{.txt}\PY{l+s+s1}{\PYZsq{}}\PY{p}{)}\PY{p}{:}
                         \PY{k}{continue}
                     \PY{k}{with} \PY{n+nb}{open}\PY{p}{(}\PY{n}{os}\PY{o}{.}\PY{n}{path}\PY{o}{.}\PY{n}{join}\PY{p}{(}\PY{n}{p}\PY{p}{,}\PY{n}{fileid}\PY{p}{)}\PY{p}{,}\PY{l+s+s1}{\PYZsq{}}\PY{l+s+s1}{r}\PY{l+s+s1}{\PYZsq{}}\PY{p}{)} \PY{k}{as} \PY{n}{g}\PY{p}{:}
                         \PY{n}{textos}\PY{o}{.}\PY{n}{append}\PY{p}{(}\PY{n}{g}\PY{o}{.}\PY{n}{read}\PY{p}{(}\PY{p}{)}\PY{p}{)}
\end{Verbatim}

    \begin{Verbatim}[commandchars=\\\{\}]
{\color{incolor}In [{\color{incolor}22}]:} \PY{n}{textos}\PY{p}{[}\PY{l+m+mi}{0}\PY{p}{]}
\end{Verbatim}

            \begin{Verbatim}[commandchars=\\\{\}]
{\color{outcolor}Out[{\color{outcolor}22}]:} "Conto, Contos Fluminenses, 1870\textbackslash{}n\textbackslash{}nContos Fluminenses\textbackslash{}n\textbackslash{}nTexto-fonte:\textbackslash{}n\textbackslash{}nObra Completa, Machado de Assis, vol. II,\textbackslash{}n\textbackslash{}nRio de Janeiro: Nova Aguilar, 1994.\textbackslash{}n\textbackslash{}nPublicado originalmente pela\textbackslash{}nEditora Garnier, Rio de Janeiro, em 1870.\textbackslash{}n\textbackslash{}nÍNDICE\textbackslash{}n\textbackslash{}nMISS DOLLAR\textbackslash{}n\textbackslash{}nLUÍS\textbackslash{}nSOARES\textbackslash{}n\textbackslash{}nA MULHER DE\textbackslash{}nPRETO\textbackslash{}n\textbackslash{}nO\textbackslash{}nSEGREDO DE AUGUSTA\textbackslash{}n\textbackslash{}nCONFISSÕES DE UMA VIÚVA MOÇA\textbackslash{}n\textbackslash{}nLINHA\textbackslash{}nRETA E LINHA CURVA\textbackslash{}n\textbackslash{}nFREI\textbackslash{}nSIMÃO\textbackslash{}n\textbackslash{}nMISS\textbackslash{}nDOLLAR\textbackslash{}n\textbackslash{}nÍNDICE\textbackslash{}n\textbackslash{}nCapítulo Primeiro\textbackslash{}n\textbackslash{}nCapítulo II\textbackslash{}n\textbackslash{}nCapítulo iii\textbackslash{}n\textbackslash{}nCapítulo iv\textbackslash{}n\textbackslash{}nCapítulo v\textbackslash{}n\textbackslash{}nCapítulo vI\textbackslash{}n\textbackslash{}nCapítulo vII\textbackslash{}n\textbackslash{}nCAPÍTULO VIII\textbackslash{}n\textbackslash{}nCAPÍTULO PRIMEIRO\textbackslash{}n\textbackslash{}nEra conveniente ao romance que o leitor\textbackslash{}nficasse muito tempo sem saber quem era Miss Dollar. Mas por outro lado,\textbackslash{}nsem a apresentação de Miss Dollar, seria o autor obrigado a longas\textbackslash{}ndigressões, que encheriam o papel sem adiantar a ação. Não há hesitação\textbackslash{}npossível: vou apresentar-lhes Miss Dollar.\textbackslash{}n\textbackslash{}nSe o leitor é rapaz e dado ao gênio\textbackslash{}nmelancólico, imagina que Miss Dollar é uma inglesa pálida e delgada,\textbackslash{}nescassa de carnes e de sangue, abrindo à flor do rosto dois grandes olhos azuis\textbackslash{}ne sacudindo ao vento umas longas tranças loiras. A moça em questão deve ser\textbackslash{}nvaporosa e ideal como uma criação de Shakespeare; deve ser o contraste do roastbeef\textbackslash{}nbritânico, com que se alimenta a liberdade do Reino Unido. Uma tal Miss\textbackslash{}nDollar deve ter o poeta Tennyson de cor e ler Lamartine no original; se\textbackslash{}nsouber o português deve deliciar-se com a leitura dos sonetos de Camões ou os Cantos\textbackslash{}nde Gonçalves Dias. O chá e o leite devem ser a alimentação de semelhante\textbackslash{}ncriatura, adicionando-se-lhe alguns confeitos e biscoitos para acudir às\textbackslash{}nurgências do estômago. A sua fala deve ser um murmúrio de harpa eólia; o seu\textbackslash{}namor um desmaio, a sua vida uma contemplação, a sua morte um suspiro.\textbackslash{}n\textbackslash{}nA figura é poética, mas não é a da\textbackslash{}nheroína do romance.\textbackslash{}n\textbackslash{}nSuponhamos que o leitor não é dado a\textbackslash{}nestes devaneios e melancolias; nesse caso imagina uma Miss Dollar totalmente\textbackslash{}ndiferente da outra. Desta vez será uma robusta americana, vertendo sangue pelas\textbackslash{}nfaces, formas arredondadas, olhos vivos e ardentes, mulher feita, refeita e\textbackslash{}nperfeita. Amiga da boa mesa e do bom copo, esta Miss Dollar preferirá um\textbackslash{}nquarto de carneiro a uma página de Longfellow, coisa naturalíssima quando o\textbackslash{}nestômago reclama, e nunca chegará a compreender a poesia do pôr-do-sol. Será\textbackslash{}numa boa mãe de família segundo a doutrina de alguns padres-mestres da\textbackslash{}ncivilização, isto é, fecunda e ignorante.\textbackslash{}n\textbackslash{}nJá não será do mesmo sentir o leitor que\textbackslash{}ntiver passado a segunda mocidade e vir diante de si uma velhice sem recurso.\textbackslash{}nPara esse, a Miss Dollar verdadeiramente digna de ser contada em algumas\textbackslash{}npáginas, seria uma boa inglesa de cinqüenta anos, dotada com algumas mil libras\textbackslash{}nesterlinas, e que, aportando ao Brasil em procura de assunto para escrever um romance,\textbackslash{}nrealizasse um romance verdadeiro, casando com o leitor aludido. Uma tal Miss\textbackslash{}nDollar seria incompleta se não tivesse óculos verdes e um grande cacho de\textbackslash{}ncabelo grisalho em cada fonte. Luvas de renda branca e chapéu de linho em forma\textbackslash{}nde cuia, seriam a última demão deste magnífico tipo de ultramar.\textbackslash{}n\textbackslash{}nMais esperto que os outros, acode um\textbackslash{}nleitor dizendo que a heroína do romance não é nem foi inglesa, mas brasileira\textbackslash{}ndos quatro costados, e que o nome de Miss Dollar quer dizer simplesmente\textbackslash{}nque a rapariga é rica.\textbackslash{}n\textbackslash{}nA descoberta seria excelente, se fosse\textbackslash{}nexata; infelizmente nem esta nem as outras são exatas. A Miss Dollar do\textbackslash{}nromance não é a menina romântica, nem a mulher robusta, nem a velha literata,\textbackslash{}nnem a brasileira rica. Falha desta vez a proverbial perspicácia dos leitores; Miss\textbackslash{}nDollar é uma cadelinha galga.\textbackslash{}n\textbackslash{}nPara algumas pessoas a qualidade da\textbackslash{}nheroína fará perder o interesse do romance. Erro manifesto. Miss Dollar,\textbackslash{}napesar de não ser mais que uma cadelinha galga, teve as honras de ver o seu\textbackslash{}nnome nos papéis públicos, antes de entrar para este livro. O Jornal do\textbackslash{}nComércio e o Correio Mercantil publicaram nas colunas dos anúncios\textbackslash{}nas seguintes linhas reverberantes de promessa:\textbackslash{}n\textbackslash{}nDesencaminhou-se uma cadelinha galga, na\textbackslash{}nnoite de ontem, 30. Acode ao nome de Miss Dollar. Quem a achou e quiser\textbackslash{}nlevar à Rua de Mata-cavalos no{\ldots}, receberá duzentos mil-réis de\textbackslash{}nrecompensa. Miss Dollar tem uma coleira ao pescoço fechada por um\textbackslash{}ncadeado em que se lêem as seguintes palavras: De tout mon coeur.\textbackslash{}n\textbackslash{}nTodas as pessoas que sentiam necessidade\textbackslash{}nurgente de duzentos mil-réis, e tiveram a felicidade de ler aquele anúncio,\textbackslash{}nandaram nesse dia com extremo cuidado nas ruas do Rio de Janeiro, a ver se\textbackslash{}ndavam com a fugitiva Miss Dollar. Galgo que aparecesse ao longe era\textbackslash{}nperseguido com tenacidade até verificar-se que não era o animal procurado. Mas\textbackslash{}ntoda esta caçada dos duzentos mil-réis era completamente inútil, visto que, no\textbackslash{}ndia em que apareceu o anúncio, já Miss Dollar estava aboletada na casa\textbackslash{}nde um sujeito morador nos Cajueiros que fazia coleção de cães.\textbackslash{}n\textbackslash{}nCAPÍTULO II\textbackslash{}n\textbackslash{}nQuais as razões que induziram o Dr.\textbackslash{}nMendonça a fazer coleção de cães, é coisa que ninguém podia dizer; uns queriam\textbackslash{}nque fosse simplesmente paixão por esse símbolo da fidelidade ou do servilismo;\textbackslash{}noutros pensavam antes que, cheio de profundo desgosto pelos homens, Mendonça\textbackslash{}nachou que era de boa guerra adorar os cães.\textbackslash{}n\textbackslash{}nFossem quais fossem as razões, o certo é\textbackslash{}nque ninguém possuía mais bonita e variada coleção do que ele. Tinha-os de todas\textbackslash{}nas raças, tamanhos e cores. Cuidava deles como se fossem seus filhos; se algum\textbackslash{}nlhe morria ficava melancólico. Quase se pode dizer que, no espírito de\textbackslash{}nMendonça, o cão pesava tanto como o amor, segundo uma expressão célebre: tirai\textbackslash{}ndo mundo o cão, e o mundo será um ermo.\textbackslash{}n\textbackslash{}nO leitor superficial conclui daqui que o\textbackslash{}nnosso Mendonça era um homem excêntrico. Não era. Mendonça era um homem como os\textbackslash{}noutros; gostava de cães como outros gostam de flores. Os cães eram as suas\textbackslash{}nrosas e violetas; cultivava-os com o mesmíssimo esmero. De flores gostava\textbackslash{}ntambém; mas gostava delas nas plantas em que nasciam: cortar um jasmim ou\textbackslash{}nprender um canário parecia-lhe idêntico atentado.\textbackslash{}n\textbackslash{}nEra o Dr. Mendonça homem de seus trinta e\textbackslash{}nquatro anos, bem apessoado, maneiras francas e distintas. Tinha-se formado em\textbackslash{}nmedicina e tratou algum tempo de doentes; a clínica estava já adiantada quando\textbackslash{}nsobreveio uma epidemia na capital; o Dr. Mendonça inventou um elixir contra a\textbackslash{}ndoença; e tão excelente era o elixir, que o autor ganhou um bom par de contos\textbackslash{}nde réis. Agora exercia a medicina como amador. Tinha quanto bastava para si e a\textbackslash{}nfamília. A família compunha-se dos animais citados acima.\textbackslash{}n\textbackslash{}nNa memorável noite em que se\textbackslash{}ndesencaminhou Miss Dollar, voltava Mendonça para casa quando teve a\textbackslash{}nventura de encontrar a fugitiva no Rocio. A cadelinha entrou a acompanhá-lo, e\textbackslash{}nele, notando que era animal sem dono visível, levou-a consigo para os\textbackslash{}nCajueiros.\textbackslash{}n\textbackslash{}nApenas entrou em casa examinou\textbackslash{}ncuidadosamente a cadelinha, Miss Dollar era realmente um mimo; tinha as\textbackslash{}nformas delgadas e graciosas da sua fidalga raça; os olhos castanhos e\textbackslash{}naveludados pareciam exprimir a mais completa felicidade deste mundo, tão\textbackslash{}nalegres e serenos eram. Mendonça contemplou-a e examinou minuciosamente. Leu o\textbackslash{}ndístico do cadeado que fechava a coleira, e convenceu-se finalmente de que a\textbackslash{}ncadelinha era animal de grande estimação da parte de quem quer que fosse dono\textbackslash{}ndela.\textbackslash{}n\textbackslash{}n— Se não aparecer o dono, fica comigo,\textbackslash{}ndisse ele entregando Miss Dollar ao moleque encarregado dos cães.\textbackslash{}n\textbackslash{}nTratou o moleque de dar comida a Miss\textbackslash{}nDollar, enquanto Mendonça planeava um bom futuro à nova hóspede, cuja\textbackslash{}nfamília devia perpetuar-se na casa.\textbackslash{}n\textbackslash{}nO plano de Mendonça durou o que duram os\textbackslash{}nsonhos: o espaço de uma noite. No dia seguinte, lendo os jornais, viu o anúncio\textbackslash{}ntranscrito acima, prometendo duzentos mil-réis a quem entregasse a cadelinha\textbackslash{}nfugitiva. A sua paixão pelos cães deu-lhe a medida da dor que devia sofrer o\textbackslash{}ndono ou dona de Miss Dollar, visto que chegava a oferecer duzentos\textbackslash{}nmil-réis de gratificação a quem apresentasse a galga. Conseqüentemente resolveu\textbackslash{}nrestituí-la, com bastante mágoa do coração. Chegou a hesitar por alguns\textbackslash{}ninstantes; mas afinal venceram os sentimentos de probidade e compaixão, que\textbackslash{}neram o apanágio daquela alma. E, como se lhe custasse despedir-se do animal,\textbackslash{}nainda recente na casa, dispôs-se a levá-lo ele mesmo, e para esse fim\textbackslash{}npreparou-se. Almoçou, e depois de averiguar bem se Miss Dollar havia\textbackslash{}nfeito a mesma operação, saíram ambos de casa com direção a Mata-cavalos.\textbackslash{}n\textbackslash{}nNaquele tempo ainda o Barão do Amazonas\textbackslash{}nnão tinha salvo a independência das repúblicas platinas mediante a vitória de\textbackslash{}nRiachuelo, nome com que depois a Câmara Municipal crismou a Rua de\textbackslash{}nMata-cavalos. Vigorava, portanto, o nome tradicional da rua, que não queria\textbackslash{}ndizer coisa nenhuma de jeito.\textbackslash{}n\textbackslash{}nA casa que tinha o número indicado no\textbackslash{}nanúncio era de bonita aparência e indicava certa abastança nos haveres de quem\textbackslash{}nlá morasse. Antes mesmo que Mendonça batesse palmas no corredor, já Miss\textbackslash{}nDollar, reconhecendo os pátrios lares, começava a pular de contente e a\textbackslash{}nsoltar uns sons alegres e guturais que, se houvesse entre os cães literatura,\textbackslash{}ndeviam ser um hino de ação de graças.\textbackslash{}n\textbackslash{}nVeio um moleque saber quem estava;\textbackslash{}nMendonça disse que vinha restituir a galga fugitiva. Expansão do rosto do\textbackslash{}nmoleque, que correu a anunciar a boa nova. Miss Dollar, aproveitando uma\textbackslash{}nfresta, precipitou-se pelas escadas acima. Dispunha-se Mendonça a descer, pois\textbackslash{}nestava cumprida a sua tarefa, quando o moleque voltou dizendo-lhe que subisse e\textbackslash{}nentrasse para a sala.\textbackslash{}n\textbackslash{}nNa sala não havia ninguém. Algumas\textbackslash{}npessoas, que têm salas elegantemente dispostas, costumam deixar tempo de serem\textbackslash{}nestas admiradas pelas visitas, antes de as virem cumprimentar. É possível que\textbackslash{}nesse fosse o costume dos donos daquela casa, mas desta vez não se cuidou em\textbackslash{}nsemelhante coisa, porque mal o médico entrou pela porta do corredor surgiu de\textbackslash{}noutra interior uma velha com Miss Dollar nos braços e a alegria no\textbackslash{}nrosto.\textbackslash{}n\textbackslash{}n— Queira ter a bondade de sentar-se,\textbackslash{}ndisse ela designando uma cadeira à Mendonça.\textbackslash{}n\textbackslash{}n— A minha demora é pequena, disse o\textbackslash{}nmédico sentando-se. Vim trazer-lhe a cadelinha que está comigo desde ontem{\ldots}\textbackslash{}n\textbackslash{}n— Não imagina que desassossego causou cá\textbackslash{}nem casa a ausência de Miss Dollar{\ldots}\textbackslash{}n\textbackslash{}n— Imagino, minha senhora; eu também sou\textbackslash{}napreciador de cães, e se me faltasse um sentiria profundamente. A sua Miss\textbackslash{}nDollar{\ldots}\textbackslash{}n\textbackslash{}n— Perdão! interrompeu a velha; minha não;\textbackslash{}nMiss Dollar não é minha, é de minha sobrinha.\textbackslash{}n\textbackslash{}n— Ah!{\ldots}\textbackslash{}n\textbackslash{}n— Ela aí vem.\textbackslash{}n\textbackslash{}nMendonça levantou-se justamente quando\textbackslash{}nentrava na sala a sobrinha em questão. Era uma moça que representava vinte e\textbackslash{}noito anos, no pleno desenvolvimento da sua beleza, uma dessas mulheres que\textbackslash{}nanunciam velhice tardia e imponente. O vestido de seda escura dava singular\textbackslash{}nrealce à cor imensamente branca da sua pele. Era roçagante o vestido, o que lhe\textbackslash{}naumentava a majestade do porte e da estatura. O corpinho do vestido cobria-lhe\textbackslash{}ntodo o colo; mas adivinhava-se por baixo da seda um belo tronco de mármore\textbackslash{}nmodelado por escultor divino. Os cabelos castanhos e naturalmente ondeados estavam\textbackslash{}npenteados com essa simplicidade caseira, que é a melhor de todas as modas\textbackslash{}nconhecidas; ornavam-lhe graciosamente a fronte como uma coroa doada pela\textbackslash{}nnatureza. A extrema brancura da pele não tinha o menor tom cor-de-rosa que lhe\textbackslash{}nfizesse harmonia e contraste. A boca era pequena, e tinha uma certa expressão\textbackslash{}nimperiosa. Mas a grande distinção daquele rosto, aquilo que mais prendia os\textbackslash{}nolhos, eram os olhos; imaginem duas esmeraldas nadando em leite.\textbackslash{}n\textbackslash{}nMendonça nunca vira olhos verdes em toda\textbackslash{}na sua vida; disseram-lhe que existiam olhos verdes, ele sabia de cor uns versos\textbackslash{}ncélebres de Gonçalves Dias; mas até então os olhos verdes eram para ele a mesma\textbackslash{}ncoisa que a fênix dos antigos. Um dia, conversando com uns amigos a propósito\textbackslash{}ndisto, afirmava que se alguma vez encontrasse um par de olhos verdes fugiria\textbackslash{}ndeles com terror.\textbackslash{}n\textbackslash{}n— Por quê? perguntou-lhe um dos\textbackslash{}ncircunstantes admirado.\textbackslash{}n\textbackslash{}n— A cor verde é a cor do mar, respondeu\textbackslash{}nMendonça; evito as tempestades de um; evitarei as tempestades dos outros.\textbackslash{}n\textbackslash{}nEu deixo ao critério do leitor esta\textbackslash{}nsingularidade de Mendonça, que de mais a mais é preciosa, no sentido de\textbackslash{}nMolière.\textbackslash{}n\textbackslash{}nCAPÍTULO III\textbackslash{}n\textbackslash{}nMendonça cumprimentou respeitosamente a\textbackslash{}nrecém-chegada, e esta, com um gesto, convidou-o a sentar-se outra vez.\textbackslash{}n\textbackslash{}n— Agradeço-lhe infinitamente o ter-me\textbackslash{}nrestituído este pobre animal, que me merece grande estima, disse Margarida\textbackslash{}nsentando-se.\textbackslash{}n\textbackslash{}n— E eu dou graças a Deus por tê-lo\textbackslash{}nachado; podia ter caído em mãos que o não restituíssem.\textbackslash{}n\textbackslash{}nMargarida fez um gesto a Miss Dollar,\textbackslash{}ne a cadelinha, saltando do regaço da velha, foi ter com Margarida; levantou as\textbackslash{}npatas dianteiras e pôs-lhas sobre os joelhos; Margarida e Miss Dollar\textbackslash{}ntrocaram um longo olhar de afeto. Durante esse tempo uma das mãos da moça\textbackslash{}nbrincava com uma das orelhas da galga, e dava assim lugar a que Mendonça\textbackslash{}nadmirasse os seus belíssimos dedos armados com unhas agudíssimas.\textbackslash{}n\textbackslash{}nMas, conquanto Mendonça tivesse sumo\textbackslash{}nprazer em estar ali, reparou que era esquisita e humilhante a sua demora.\textbackslash{}nPareceria estar esperando a gratificação. Para escapar a essa interpretação\textbackslash{}ndesairosa, sacrificou o prazer da conversa e a contemplação da moça;\textbackslash{}nlevantou-se dizendo:\textbackslash{}n\textbackslash{}n— A minha missão está cumprida{\ldots}\textbackslash{}n\textbackslash{}n— Mas{\ldots} interrompeu a velha.\textbackslash{}n\textbackslash{}nMendonça compreendeu a ameaça da\textbackslash{}ninterrupção da velha.\textbackslash{}n\textbackslash{}n— A alegria, disse ele, que restituí a\textbackslash{}nesta casa é a maior recompensa que eu podia ambicionar. Agora peço-lhes\textbackslash{}nlicença{\ldots}\textbackslash{}n\textbackslash{}nAs duas senhoras compreenderam a intenção\textbackslash{}nde Mendonça; a moça pagou-lhe a cortesia com um sorriso; e a velha, reunindo no\textbackslash{}npulso quantas forças ainda lhe restavam pelo corpo todo, apertou com amizade\textbackslash{}na mão do rapaz.\textbackslash{}n\textbackslash{}nMendonça saiu impressionado pela\textbackslash{}ninteressante Margarida. Notava-lhe principalmente, além da beleza, que era de\textbackslash{}nprimeira água, certa severidade triste no olhar e nos modos. Se aquilo era\textbackslash{}ncaráter da moça, dava-se bem com a índole de médico; se era resultado de algum\textbackslash{}nepisódio da vida, era uma página do romance que devia ser decifrada por olhos\textbackslash{}nhábeis. A falar verdade, o único defeito que Mendonça lhe achou foi a cor dos olhos,\textbackslash{}nnão porque a cor fosse feia, mas porque ele tinha prevenção contra os olhos\textbackslash{}nverdes. A prevenção, cumpre dizê-lo, era mais literária que outra coisa;\textbackslash{}nMendonça apegava-se à frase que uma vez proferira, e foi acima citada, e a\textbackslash{}nfrase é que lhe produziu a prevenção. Não mo acusem de chofre; Mendonça era\textbackslash{}nhomem inteligente, instruído e dotado de bom senso; tinha, além disso, grande\textbackslash{}ntendência para as afeições românticas; mas apesar disso lá tinha calcanhar o\textbackslash{}nnosso Aquiles. Era homem como os outros, outros Aquiles andam por aí que são da\textbackslash{}ncabeça aos pés um imenso calcanhar. O ponto vulnerável de Mendonça era esse; o\textbackslash{}namor de uma frase era capaz de violentar-lhe afetos; sacrificava uma situação a\textbackslash{}num período arredondado.\textbackslash{}n\textbackslash{}nReferindo a um amigo o episódio da galga\textbackslash{}ne a entrevista com Margarida, Mendonça disse que poderia vir a gostar dela se\textbackslash{}nnão tivesse olhos verdes. O amigo riu com certo ar de sarcasmo.\textbackslash{}n\textbackslash{}n— Mas, doutor, disse-lhe ele, não\textbackslash{}ncompreendo essa prevenção; eu ouço até dizer que os olhos verdes são de\textbackslash{}nordinário núncios de boa alma. Além de que, a cor dos olhos não vale nada, a\textbackslash{}nquestão é a expressão deles. Podem ser azuis como o céu e pérfidos como o mar.\textbackslash{}n\textbackslash{}nA observação deste amigo anônimo tinha a\textbackslash{}nvantagem de ser tão poética como a de Mendonça. Por isso abalou profundamente o\textbackslash{}nânimo do médico. Não ficou este como o asno de Buridan entre a selha d’água e a\textbackslash{}nquarta de cevada; o asno hesitaria, Mendonça não hesitou. Acudiu-lhe de pronto\textbackslash{}na lição do casuísta Sánchez, e das duas opiniões tomou a que lhe pareceu provável.\textbackslash{}n\textbackslash{}nAlgum leitor grave achará pueril esta\textbackslash{}ncircunstância dos olhos verdes e esta controvérsia sobre a qualidade provável\textbackslash{}ndeles. Provará com isso que tem pouca prática do mundo. Os almanaques\textbackslash{}npitorescos citam até à saciedade mil excentricidades e senões dos grandes\textbackslash{}nvarões que a humanidade admira, já por instruídos nas letras, já por valentes\textbackslash{}nnas armas; e nem por isso deixamos de admirar esses mesmos varões. Não queira o\textbackslash{}nleitor abrir uma exceção só para encaixar nela o nosso doutor. Aceitemo-lo com\textbackslash{}nos seus ridículos; quem os não tem? O ridículo é uma espécie de lastro da alma\textbackslash{}nquando ela entra no mar da vida; algumas fazem toda a navegação sem outra\textbackslash{}nespécie de carregamento.\textbackslash{}n\textbackslash{}nPara compensar essas fraquezas, já disse\textbackslash{}nque Mendonça tinha qualidades não vulgares. Adotando a opinião que lhe pareceu\textbackslash{}nmais provável, que foi a do amigo, Mendonça disse consigo que nas mãos de\textbackslash{}nMargarida estava talvez a chave do seu futuro. Ideou nesse sentido um plano de\textbackslash{}nfelicidade; uma casa num ermo, olhando para o mar ao lado do ocidente, a fim de\textbackslash{}npoder assistir ao espetáculo do pôr-do-sol. Margarida e ele, unidos pelo amor e\textbackslash{}npela Igreja, beberiam ali, gota a gota, a taça inteira da celeste felicidade. O\textbackslash{}nsonho de Mendonça continha outras particularidades que seria ocioso mencionar\textbackslash{}naqui. Mendonça pensou nisto alguns dias; chegou a passar algumas vezes por\textbackslash{}nMata-cavalos; mas tão infeliz que nunca viu Margarida nem a tia; afinal\textbackslash{}ndesistiu da empresa e voltou aos cães.\textbackslash{}n\textbackslash{}nA coleção de cães era uma verdadeira\textbackslash{}ngaleria de homens ilustres. O mais estimado deles chamava-se Diógenes;\textbackslash{}nhavia um galgo que acudia ao nome de César; um cão d’água que se chamava\textbackslash{}nNelson; Cornélia chamava-se uma cadelinha rateira, e Calígula\textbackslash{}num enorme cão de fila, vera-efígie do grande monstro que a sociedade romana\textbackslash{}nproduziu. Quando se achava entre toda essa gente, ilustre por diferentes\textbackslash{}ntítulos, dizia Mendonça que entrava na história; era assim que se esquecia do\textbackslash{}nresto do mundo.\textbackslash{}n\textbackslash{}nCAPÍTULO IV\textbackslash{}n\textbackslash{}nAchava-se Mendonça uma vez à porta do\textbackslash{}nCarceller, onde acabava de tomar sorvete em companhia de um indivíduo, amigo\textbackslash{}ndele, quando viu passar um carro, e dentro do carro duas senhoras que lhe\textbackslash{}npareceram as senhoras de Mata-cavalos. Mendonça fez um movimento de espanto que\textbackslash{}nnão escapou ao amigo.\textbackslash{}n\textbackslash{}n— Que foi? perguntou-lhe este.\textbackslash{}n\textbackslash{}n— Nada; pareceu-me conhecer aquelas\textbackslash{}nsenhoras. Viste-as, Andrade?\textbackslash{}n\textbackslash{}n— Não.\textbackslash{}n\textbackslash{}nO carro entrara na Rua do Ouvidor; os\textbackslash{}ndois subiram pela mesma rua. Logo acima da Rua da Quitanda, parara o carro à\textbackslash{}nporta de uma loja, e as senhoras apearam-se e entraram. Mendonça não as viu\textbackslash{}nsair; mas viu o carro e suspeitou que fosse o mesmo. Apressou o passo sem dizer\textbackslash{}nnada a Andrade, que fez o mesmo, movido por essa natural curiosidade que sente\textbackslash{}num homem quando percebe algum segredo oculto.\textbackslash{}n\textbackslash{}nPoucos instantes depois estavam à porta\textbackslash{}nda loja; Mendonça verificou que eram as duas senhoras de Mata-cavalos. Entrou\textbackslash{}nafoito, com ar de quem ia comprar alguma coisa, e aproximou-se das senhoras. A\textbackslash{}nprimeira que o conheceu foi a tia. Mendonça cumprimentou-as respeitosamente.\textbackslash{}nElas receberam o cumprimento com afabilidade. Ao pé de Margarida estava Miss\textbackslash{}nDollar, que, por esse admirável faro que a natureza concedeu aos cães e aos\textbackslash{}ncortesãos da fortuna, deu dois saltos de alegria apenas viu Mendonça, chegando\textbackslash{}na tocar-lhe o estômago com as patas dianteiras.\textbackslash{}n\textbackslash{}n— Parece que Miss Dollar ficou com\textbackslash{}nboas recordações suas, disse D. Antônia (assim se chamava a tia de Margarida).\textbackslash{}n\textbackslash{}n— Creio que sim, respondeu Mendonça\textbackslash{}nbrincando com a galga e olhando para Margarida.\textbackslash{}n\textbackslash{}nJustamente nesse momento entrou Andrade.\textbackslash{}n\textbackslash{}n— Só agora as reconheci, disse ele\textbackslash{}ndirigindo-se às senhoras.\textbackslash{}n\textbackslash{}nAndrade apertou a mão das duas senhoras,\textbackslash{}nou antes apertou a mão de Antônia e os dedos de Margarida.\textbackslash{}n\textbackslash{}nMendonça não contava com este incidente,\textbackslash{}ne alegrou-se com ele por ter à mão o meio de tornar íntimas as relações\textbackslash{}nsuperficiais que tinha com a família.\textbackslash{}n\textbackslash{}n— Seria bom, disse ele a Andrade, que me\textbackslash{}napresentasses a estas senhoras.\textbackslash{}n\textbackslash{}n— Pois não as conheces? perguntou Andrade\textbackslash{}nestupefato.\textbackslash{}n\textbackslash{}n— Conhece-nos sem nos conhecer, respondeu\textbackslash{}nsorrindo a velha tia; por ora quem o apresentou foi Miss Dollar.\textbackslash{}n\textbackslash{}nAntônia referiu a Andrade a perda e o\textbackslash{}nachado da cadelinha.\textbackslash{}n\textbackslash{}n— Pois, nesse caso, respondeu Andrade,\textbackslash{}napresento-o já.\textbackslash{}n\textbackslash{}nFeita a apresentação oficial, o caixeiro trouxe\textbackslash{}na Margarida os objetos que ela havia comprado, e as duas senhoras despediram-se\textbackslash{}ndos rapazes pedindo-lhes que as fossem ver.\textbackslash{}n\textbackslash{}nNão citei nenhuma palavra de Margarida no\textbackslash{}ndiálogo acima transcrito, porque, a falar verdade, a moça só proferiu duas\textbackslash{}npalavras a cada um dos rapazes.\textbackslash{}n\textbackslash{}n— Passe bem, disse-lhes ela dando as\textbackslash{}npontas dos dedos e saindo para entrar no carro.\textbackslash{}n\textbackslash{}nFicando sós, saíram também os dois\textbackslash{}nrapazes e seguiram pela Rua do Ouvidor acima, ambos calados. Mendonça pensava\textbackslash{}nem Margarida; Andrade pensava nos meios de entrar na confidência de Mendonça. A\textbackslash{}nvaidade tem mil formas de manifestar-se como o fabuloso Proteu. A vaidade de\textbackslash{}nAndrade era ser confidente dos outros; parecia-lhe assim obter da confiança\textbackslash{}naquilo que só alcançava da indiscrição. Não lhe foi difícil apanhar o segredo\textbackslash{}nde Mendonça; antes de chegar à esquina da Rua dos Ourives já Andrade sabia de\textbackslash{}ntudo.\textbackslash{}n\textbackslash{}n— Compreendes agora, disse Mendonça, que\textbackslash{}neu preciso ir à casa dela; tenho necessidade de vê-la; quero ver se consigo{\ldots}\textbackslash{}n\textbackslash{}nMendonça estacou.\textbackslash{}n\textbackslash{}n— Acaba! disse Andrade; se consegues ser\textbackslash{}namado. Por que não? Mas desde já te digo que não será fácil.\textbackslash{}n\textbackslash{}n— Por quê?\textbackslash{}n\textbackslash{}n— Margarida tem rejeitado cinco\textbackslash{}ncasamentos.\textbackslash{}n\textbackslash{}n— Naturalmente não amava os pretendentes,\textbackslash{}ndisse Mendonça com o ar de um geômetra que acha uma solução.\textbackslash{}n\textbackslash{}n— Amava apaixonadamente o primeiro,\textbackslash{}nrespondeu Andrade, e não era indiferente ao último.\textbackslash{}n\textbackslash{}n— Houve naturalmente intriga.\textbackslash{}n\textbackslash{}n— Também não. Admiras-te? É o que me\textbackslash{}nacontece. É uma rapariga esquisita. Se te achas com força de ser o Colombo\textbackslash{}ndaquele mundo, lança-te ao mar com a armada; mas toma cuidado com a revolta das\textbackslash{}npaixões, que são os ferozes marujos destas navegações de descoberta.\textbackslash{}n\textbackslash{}nEntusiasmado com esta alusão, histórica\textbackslash{}ndebaixo da forma de alegoria, Andrade olhou para Mendonça, que, desta vez\textbackslash{}nentregue ao pensamento da moça, não atendeu à frase do amigo. Andrade\textbackslash{}ncontentou-se com o seu próprio sufrágio, e sorriu com o mesmo ar de satisfação\textbackslash{}nque deve ter um poeta quando escreve o último verso de um poema.\textbackslash{}n\textbackslash{}nCAPÍTULO V\textbackslash{}n\textbackslash{}nDias depois, Andrade e Mendonça foram à\textbackslash{}ncasa de Margarida, e lá passaram meia hora em conversa cerimoniosa. As visitas\textbackslash{}nrepetiram-se; eram porém mais freqüentes da parte de Mendonça que de Andrade.\textbackslash{}nD. Antônia mostrou-se mais familiar que Margarida; só depois de algum tempo\textbackslash{}nMargarida desceu do Olimpo do silêncio em que habitualmente se encerrara.\textbackslash{}n\textbackslash{}nEra difícil deixar de o fazer. Mendonça,\textbackslash{}nconquanto não fosse dado à convivência das salas, era um cavalheiro próprio\textbackslash{}npara entreter duas senhoras que pareciam mortalmente aborrecidas. O médico\textbackslash{}nsabia piano e tocava agradavelmente; a sua conversa era animada; sabia esses\textbackslash{}nmil nadas que entretêm geralmente as senhoras quando elas não gostam ou não\textbackslash{}npodem entrar no terreno elevado da arte, da história e da filosofia. Não foi\textbackslash{}ndifícil ao rapaz estabelecer intimidade com a família.\textbackslash{}n\textbackslash{}nPosteriormente às primeiras visitas,\textbackslash{}nsoube Mendonça, por via de Andrade, que Margarida era viúva. Mendonça não\textbackslash{}nreprimiu o gesto de espanto.\textbackslash{}n\textbackslash{}n— Mas tu falaste de um modo que parecias\textbackslash{}ntratar de uma solteira, disse ele ao amigo.\textbackslash{}n\textbackslash{}n— É verdade que não me expliquei bem; os\textbackslash{}ncasamentos recusados foram todos propostos depois da viuvez.\textbackslash{}n\textbackslash{}n— Há que tempo está viúva?\textbackslash{}n\textbackslash{}n— Há três anos.\textbackslash{}n\textbackslash{}n— Tudo se explica, disse Mendonça depois\textbackslash{}nde algum silêncio; quer ficar fiel à sepultura; é uma Artemisa do século.\textbackslash{}n\textbackslash{}nAndrade era cético a respeito de\textbackslash{}nArtemisas; sorriu à observação do amigo, e, como este insistisse, replicou:\textbackslash{}n\textbackslash{}n— Mas se eu já te disse que ela amava\textbackslash{}napaixonadamente o primeiro pretendente e não era indiferente ao último.\textbackslash{}n\textbackslash{}n— Então, não compreendo.\textbackslash{}n\textbackslash{}n— Nem eu.\textbackslash{}n\textbackslash{}nMendonça desde esse momento tratou de\textbackslash{}ncortejar assiduamente a viúva; Margarida recebeu os primeiros olhares de\textbackslash{}nMendonça com um ar de tão supremo desdém, que o rapaz esteve quase a abandonar\textbackslash{}na empresa; mas, a viúva, ao mesmo tempo que parecia recusar amor, não lhe\textbackslash{}nrecusava estima, e tratava-o com a maior meiguice deste mundo sempre que ele a\textbackslash{}nolhava como toda a gente.\textbackslash{}n\textbackslash{}nAmor repelido é amor multiplicado. Cada\textbackslash{}nrepulsa de Margarida aumentava a paixão de Mendonça. Nem já lhe mereciam\textbackslash{}natenção o feroz Calígula, nem o elegante Júlio César. Os dois\textbackslash{}nescravos de Mendonça começaram a notar a profunda diferença que havia entre os\textbackslash{}nhábitos de hoje e os de outro tempo. Supuseram logo que alguma coisa o\textbackslash{}npreocupava. Convenceram-se disso quando Mendonça, entrando uma vez em casa, deu\textbackslash{}ncom a ponta do botim no focinho de Cornélia, na ocasião em que esta\textbackslash{}ninteressante cadelinha, mãe de dois Gracos rateiros, festejava a chegada\textbackslash{}ndo doutor.\textbackslash{}n\textbackslash{}nAndrade não foi insensível aos\textbackslash{}nsofrimentos do amigo e procurou consolá-lo. Toda a consolação nestes casos é\textbackslash{}ntão desejada quanto inútil; Mendonça ouvia as palavras de Andrade e\textbackslash{}nconfiava-lhe todas as suas penas. Andrade lembrou a Mendonça um excelente meio\textbackslash{}nde fazer cessar a paixão: era ausentar-se da casa. A isto respondeu Mendonça\textbackslash{}ncitando La Rochefoucauld:\textbackslash{}n\textbackslash{}n'A ausência diminui as paixões\textbackslash{}nmedíocres e aumenta as grandes, como o vento apaga as velas e atiça as\textbackslash{}nfogueiras.'\textbackslash{}n\textbackslash{}nA citação teve o mérito de tapar a boca\textbackslash{}nde Andrade, que acreditava tanto na constância como nas Artemisas, mas que não\textbackslash{}nqueria contrariar a autoridade do moralista, nem a resolução de Mendonça.\textbackslash{}n\textbackslash{}nCAPÍTULO VI\textbackslash{}n\textbackslash{}nCorreram assim três meses. A corte de Mendonça\textbackslash{}nnão adiantava um passo; mas a viúva nunca deixou de ser amável com ele. Era\textbackslash{}nisto o que principalmente retinha o médico aos pés da insensível viúva; não o\textbackslash{}nabandonava a esperança de vencê-la.\textbackslash{}n\textbackslash{}nAlgum leitor conspícuo desejaria antes\textbackslash{}nque Mendonça não fosse tão assíduo na casa de uma senhora exposta às calúnias\textbackslash{}ndo mundo. Pensou nisso o médico e consolou a consciência com a presença de um\textbackslash{}nindivíduo, até aqui não nomeado por motivo de sua nulidade, e que era nada\textbackslash{}nmenos que o filho da Sra. D. Antônia e a menina dos seus olhos. Chamava-se\textbackslash{}nJorge esse rapaz, que gastava duzentos mil-réis por mês, sem os ganhar, graças\textbackslash{}nà longanimidade da mãe. Freqüentava as casas dos cabeleireiros, onde gastava\textbackslash{}nmais tempo que uma romana da decadência às mãos das suas servas latinas. Não\textbackslash{}nperdia representação de importância no Alcazar; montava bons cavalos, e\textbackslash{}nenriquecia com despesas extraordinárias as algibeiras de algumas damas célebres\textbackslash{}ne de vários parasitas obscuros. Calçava luvas da letra E e botas nº 36, duas\textbackslash{}nqualidades que lançava à cara de todos os seus amigos que não desciam do nº 40\textbackslash{}ne da letra H. A presença deste gentil pimpolho, achava Mendonça que salvava a\textbackslash{}nsituação. Mendonça queria dar esta satisfação ao mundo, isto é, à opinião dos\textbackslash{}nociosos da cidade. Mas bastaria isso para tapar a boca aos ociosos?\textbackslash{}n\textbackslash{}nMargarida parecia indiferente às\textbackslash{}ninterpretações do mundo como à assiduidade do rapaz. Seria ela tão indiferente\textbackslash{}na tudo mais neste mundo? Não; amava a mãe, tinha um capricho por Miss Dollar,\textbackslash{}ngostava da boa música, e lia romances. Vestia-se bem, sem ser rigorista em\textbackslash{}nmatéria de moda; não valsava; quando muito dançava alguma quadrilha nos saraus\textbackslash{}na que era convidada. Não falava muito, mas exprimia-se bem. Tinha o gesto\textbackslash{}ngracioso e animado, mas sem pretensão nem faceirice.\textbackslash{}n\textbackslash{}nQuando Mendonça aparecia lá, Margarida\textbackslash{}nrecebia-o com visível contentamento. O médico iludia-se sempre, apesar de já\textbackslash{}nacostumado a essas manifestações. Com efeito, Margarida gostava imenso da\textbackslash{}npresença do rapaz, mas não parecia dar-lhe uma importância que lisonjeasse o\textbackslash{}ncoração dele. Gostava de o ver como se gosta de ver um dia bonito, sem morrer\textbackslash{}nde amores pelo sol.\textbackslash{}n\textbackslash{}nNão era possível sofrer por muito tempo a\textbackslash{}nposição em que se achava o médico. Uma noite, por um esforço de que antes disso\textbackslash{}nse não julgaria capaz, Mendonça dirigiu a Margarida esta pergunta indiscreta:\textbackslash{}n\textbackslash{}n— Foi feliz com seu marido?\textbackslash{}n\textbackslash{}nMargarida franziu a testa com espanto e\textbackslash{}ncravou os olhos nos do médico, que pareciam continuar mudamente a pergunta.\textbackslash{}n\textbackslash{}n— Fui, disse ela no fim de alguns\textbackslash{}ninstantes.\textbackslash{}n\textbackslash{}nMendonça não disse palavra; não contava\textbackslash{}ncom aquela resposta. Confiava demais na intimidade que reinava entre ambos; e\textbackslash{}nqueria descobrir por algum modo a causa da insensibilidade da viúva. Falhou o\textbackslash{}ncálculo; Margarida tornou-se séria durante algum tempo; a chegada de D. Antônia\textbackslash{}nsalvou uma situação esquerda para Mendonça. Pouco depois Margarida voltava às\textbackslash{}nboas, e a conversa tornou-se animada e íntima como sempre. A chegada de Jorge\textbackslash{}nlevou a animação da conversa a proporções maiores; D. Antônia, com olhos e ouvidos\textbackslash{}nde mãe, achava que o filho era o rapaz mais engraçado deste mundo; mas a\textbackslash{}nverdade é que não havia em toda a cristandade espírito mais frívolo. A mãe\textbackslash{}nria-se de tudo quanto o filho dizia; o filho enchia, só ele, a conversa,\textbackslash{}nreferindo anedotas e reproduzindo ditos e sestros do Alcazar. Mendonça via\textbackslash{}ntodas essas feições do rapaz, e aturava-o com resignação evangélica.\textbackslash{}n\textbackslash{}nA entrada de Jorge, animando a conversa,\textbackslash{}nacelerou as horas; às dez retirou-se o médico, acompanhado pelo filho de D.\textbackslash{}nAntônia, que ia cear. Mendonça recusou o convite que Jorge lhe fez, e\textbackslash{}ndespediu-se dele na Rua do Conde, esquina da do Lavradio.\textbackslash{}n\textbackslash{}nNessa mesma noite resolveu Mendonça dar\textbackslash{}num golpe decisivo; resolveu escrever uma carta a Margarida. Era temerário para\textbackslash{}nquem conhecesse o caráter da viúva; mas, com os precedentes já mencionados, era\textbackslash{}nloucura. Entretanto, não hesitou o médico em empregar a carta, confiando que no\textbackslash{}npapel diria as coisas de muito melhor maneira que de boca. A carta foi escrita\textbackslash{}ncom febril impaciência; no dia seguinte, logo depois de almoçar, Mendonça meteu\textbackslash{}na carta dentro de um volume de George Sand, mandou-o pelo moleque a Margarida.\textbackslash{}n\textbackslash{}nA viúva rompeu a capa de papel que\textbackslash{}nembrulhava o volume, e pôs o livro sobre a mesa da sala; meia hora depois\textbackslash{}nvoltou e pegou no livro para ler. Apenas o abriu, caiu-lhe a carta aos pés.\textbackslash{}nAbriu-a e leu o seguinte:\textbackslash{}n\textbackslash{}nQualquer que seja a causa da sua\textbackslash{}nesquivança, respeito-a, não me insurjo contra ela. Mas, se não me é dado\textbackslash{}ninsurgir-me, não me será lícito queixar-me? Há de ter compreendido o meu amor,\textbackslash{}ndo mesmo modo que tenho compreendido a sua indiferença; mas, por maior que seja\textbackslash{}nessa indiferença está longe de ombrear com o amor profundo e imperioso que se\textbackslash{}napossou de meu coração quando eu mais longe me cuidava destas paixões dos\textbackslash{}nprimeiros anos. Não lhe contarei as insônias e as lágrimas, as esperanças e os\textbackslash{}ndesencantos, páginas tristes deste livro que o destino põe nas mãos do homem\textbackslash{}npara que duas almas o leiam. É-lhe indiferente isso.\textbackslash{}n\textbackslash{}nNão ouso interrogá-la sobre a esquivança\textbackslash{}nque tem mostrado em relação a mim; mas por que motivo se estende essa\textbackslash{}nesquivança a tantos mais? Na idade das paixões férvidas, ornada pelo céu com\textbackslash{}numa beleza rara, por que motivo quer esconder-se ao mundo e defraudar a\textbackslash{}nnatureza e o coração de seus incontestáveis direitos? Perdoe-me a audácia da\textbackslash{}npergunta; acho-me diante de um enigma que o meu coração desejaria decifrar.\textbackslash{}nPenso às vezes que alguma grande dor a atormenta, e quisera ser o médico do seu\textbackslash{}ncoração; ambicionava, confesso, restaurar-lhe alguma ilusão perdida. Parece que\textbackslash{}nnão há ofensa nesta ambição.\textbackslash{}n\textbackslash{}nSe, porém, essa esquivança denota\textbackslash{}nsimplesmente um sentimento de orgulho legítimo, perdoe-me se ousei escrever-lhe\textbackslash{}nquando seus olhos expressamente mo proibiram. Rasgue a carta que não pode\textbackslash{}nvaler-lhe uma recordação, nem representar uma arma.\textbackslash{}n\textbackslash{}nA carta era toda de reflexão; a frase\textbackslash{}nfria e medida não exprimia o fogo do sentimento. Não terá, porém, escapado ao\textbackslash{}nleitor a sinceridade e a simplicidade com que Mendonça pedia uma explicação que\textbackslash{}nMargarida provavelmente não podia dar.\textbackslash{}n\textbackslash{}nQuando Mendonça disse a Andrade haver\textbackslash{}nescrito a Margarida, o amigo do médico entrou a rir despregadamente.\textbackslash{}n\textbackslash{}n— Fiz mal? perguntou Mendonça.\textbackslash{}n\textbackslash{}n— Estragaste tudo. Os outros pretendentes\textbackslash{}ncomeçaram também por carta; foi justamente a certidão de óbito do amor.\textbackslash{}n\textbackslash{}n— Paciência, se acontecer o mesmo, disse\textbackslash{}nMendonça levantando os ombros com aparente indiferença; mas eu desejava que não\textbackslash{}nestivesses sempre a falar nos pretendentes; eu não sou pretendente no sentido\textbackslash{}ndesses.\textbackslash{}n\textbackslash{}n— Não querias casar com ela?\textbackslash{}n\textbackslash{}n— Sem dúvida, se fosse possível,\textbackslash{}nrespondeu Mendonça.\textbackslash{}n\textbackslash{}n— Pois era justamente o que os outros\textbackslash{}nqueriam; casar-te-ias e entrarias na mansa posse dos bens que lhe couberam em\textbackslash{}npartilha e que sobem a muito mais de cem contos. Meu rico, se falo em\textbackslash{}npretendentes não é por te ofender, porque um dos quatro pretendentes despedidos\textbackslash{}nfui eu.\textbackslash{}n\textbackslash{}n— Tu?\textbackslash{}n\textbackslash{}n— É verdade; mas descansa, não fui o\textbackslash{}nprimeiro, nem ao menos o último.\textbackslash{}n\textbackslash{}n— Escreveste?\textbackslash{}n\textbackslash{}n— Como os outros; como eles, não obtive\textbackslash{}nresposta; isto é, obtive uma: devolveu-me a carta. Portanto, já que lhe\textbackslash{}nescreveste, espera o resto; verás se o que te digo é ou não exato. Estás\textbackslash{}nperdido, Mendonça; fizeste muito mal.\textbackslash{}n\textbackslash{}nAndrade tinha esta feição característica\textbackslash{}nde não omitir nenhuma das cores sombrias de uma situação, com o pretexto de que\textbackslash{}naos amigos se deve a verdade. Desenhado o quadro, despediu-se de Mendonça, e\textbackslash{}nfoi adiante.\textbackslash{}n\textbackslash{}nMendonça foi para casa, onde passou a\textbackslash{}nnoite em claro.\textbackslash{}n\textbackslash{}nCAPÍTULO VII\textbackslash{}n\textbackslash{}nEnganara-se Andrade; a viúva respondeu à carta\textbackslash{}ndo médico. A carta dela limitou-se a isto:\textbackslash{}n\textbackslash{}nPerdôo-lhe tudo; não lhe perdoarei se me\textbackslash{}nescrever outra vez. A minha esquivança não tem nenhuma causa; é questão de\textbackslash{}ntemperamento.\textbackslash{}n\textbackslash{}nO sentido da carta era ainda mais\textbackslash{}nlacônico do que a expressão. Mendonça leu-a muitas vezes, a ver se a\textbackslash{}ncompletava; mas foi trabalho perdido. Uma coisa concluiu ele logo; era que\textbackslash{}nhavia coisa oculta que arredava Margarida do casamento; depois concluiu outra,\textbackslash{}nera que Margarida ainda lhe perdoaria segunda carta se lha escrevesse.\textbackslash{}n\textbackslash{}nA primeira vez que Mendonça foi a\textbackslash{}nMata-cavalos achou-se embaraçado sobre a maneira por que falaria a Margarida; a\textbackslash{}nviúva tirou-o do embaraço, tratando-o como se nada houvesse entre ambos.\textbackslash{}nMendonça não teve ocasião de aludir às cartas por causa da presença de D.\textbackslash{}nAntônia, mas estimou isso mesmo, porque não sabia o que lhe diria caso viessem\textbackslash{}na ficar sós os dois.\textbackslash{}n\textbackslash{}nDias depois, Mendonça escreveu segunda\textbackslash{}ncarta à viúva e mandou-lha pelo mesmo canal da outra. A carta foi-lhe devolvida\textbackslash{}nsem resposta. Mendonça arrependeu-se de ter abusado da ordem da moça, e\textbackslash{}nresolveu, de uma vez por todas, não voltar à casa de Mata-cavalos. Nem tinha\textbackslash{}nânimo de lá aparecer, nem julgava conveniente estar junto de uma pessoa a quem\textbackslash{}namava sem esperança.\textbackslash{}n\textbackslash{}nAo cabo de um mês não tinha perdido uma\textbackslash{}npartícula sequer do sentimento que nutria pela viúva. Amava-a com o mesmíssimo\textbackslash{}nardor. A ausência, como ele pensara, aumentou-lhe o amor, como o vento ateia um\textbackslash{}nincêndio. Debalde lia ou buscava distrair-se na vida agitada do Rio de Janeiro;\textbackslash{}nentrou a escrever um estudo sobre a teoria do ouvido, mas a pena\textbackslash{}nescapava-se-lhe para o coração, e saiu o escrito com uma mistura de nervos e\textbackslash{}nsentimentos. Estava então na sua maior nomeada o romance de Renan sobre\textbackslash{}na vida de Jesus; Mendonça encheu o gabinete com todos os folhetos publicados de\textbackslash{}nparte a parte, e entrou a estudar profundamente o misterioso drama da Judéia.\textbackslash{}nFez quanto pôde para absorver o espírito e esquecer a esquiva Margarida;\textbackslash{}nera-lhe impossível.\textbackslash{}n\textbackslash{}nUm dia de manhã apareceu-lhe em casa o\textbackslash{}nfilho de D. Antônia; traziam-no dois motivos: perguntar-lhe por que não ia a\textbackslash{}nMata-cavalos, e mostrar-lhe umas calças novas. Mendonça aprovou as calças, e\textbackslash{}ndesculpou como pôde a ausência, dizendo que andava atarefado. Jorge não era\textbackslash{}nalma que compreendesse a verdade escondida por baixo de uma palavra\textbackslash{}nindiferente; vendo Mendonça mergulhado no meio de uma chusma de livros e\textbackslash{}nfolhetos, perguntou-lhe se estava estudando para ser deputado. Jorge cuidava\textbackslash{}nque se estudava para ser deputado!\textbackslash{}n\textbackslash{}n— Não, respondeu Mendonça.\textbackslash{}n\textbackslash{}n— É verdade que a prima também lá anda\textbackslash{}ncom livros, e não creio que pretenda ir à câmara.\textbackslash{}n\textbackslash{}n— Ah! sua prima?\textbackslash{}n\textbackslash{}n— Não imagina; não faz outra coisa.\textbackslash{}nFecha-se no quarto, e passa os dias inteiros a ler.\textbackslash{}n\textbackslash{}nInformado por Jorge, Mendonça supôs que\textbackslash{}nMargarida era nada menos que uma mulher de letras, alguma modesta poetisa, que\textbackslash{}nesquecia o amor dos homens nos braços das musas. A suposição era gratuita e\textbackslash{}nfilha mesmo de um espírito cego pelo amor como o de Mendonça. Há várias razões\textbackslash{}npara ler muito sem ter comércio com as musas.\textbackslash{}n\textbackslash{}n— Note que a prima nunca leu tanto; agora\textbackslash{}né que lhe deu para isso, disse Jorge tirando da charuteira um magnífico havana\textbackslash{}ndo valor de três tostões, e oferecendo outro a Mendonça. Fume isto, continuou\textbackslash{}nele, fume e diga-me se há ninguém como o Bernardo para ter charutos bons.\textbackslash{}n\textbackslash{}nGastos os charutos, Jorge despediu-se do\textbackslash{}nmédico, levando a promessa de que este iria à casa de D. Antônia o mais cedo\textbackslash{}nque pudesse.\textbackslash{}n\textbackslash{}nNo fim de quinze dias Mendonça voltou a\textbackslash{}nMata-cavalos.\textbackslash{}n\textbackslash{}nEncontrou na sala Andrade e D. Antônia,\textbackslash{}nque o receberam com aleluias. Mendonça parecia com efeito ressurgir de um\textbackslash{}ntúmulo; tinha emagrecido e empalidecido. A melancolia dava-lhe ao rosto maior\textbackslash{}nexpressão de abatimento. Alegou trabalhos extraordinários, e entrou a conversar\textbackslash{}nalegremente como dantes. Mas essa alegria, como se compreende, era toda\textbackslash{}nforçada. No fim de um quarto de hora a tristeza apossou-se-lhe outra vez do\textbackslash{}nrosto. Durante esse tempo, Margarida não apareceu na sala; Mendonça, que até\textbackslash{}nentão não perguntara por ela, não sei por que razão, vendo que ela não\textbackslash{}naparecia, perguntou se estava doente. D. Antônia respondeu-lhe que Margarida\textbackslash{}nestava um pouco incomodada.\textbackslash{}n\textbackslash{}nO incômodo de Margarida durou uns três\textbackslash{}ndias; era uma simples dor de cabeça, que o primo atribuiu à aturada leitura.\textbackslash{}n\textbackslash{}nNo fim de alguns dias mais, D. Antônia\textbackslash{}nfoi surpreendida com uma lembrança de Margarida; a viúva queria ir viver na\textbackslash{}nroça algum tempo.\textbackslash{}n\textbackslash{}n— Aborrece-te a cidade? perguntou a boa\textbackslash{}nvelha.\textbackslash{}n\textbackslash{}n— Alguma coisa, respondeu Margarida; queria\textbackslash{}nir viver uns dois meses na roça.\textbackslash{}n\textbackslash{}nD. Antônia não podia recusar nada à\textbackslash{}nsobrinha; concordou em ir para a roça; e começaram os preparativos. Mendonça\textbackslash{}nsoube da mudança no Rocio, andando a passear de noite; disse-lho Jorge na\textbackslash{}nocasião de ir para o Alcazar. Para o rapaz era uma fortuna aquela mudança,\textbackslash{}nporque suprimia-lhe a única obrigação que ainda tinha neste mundo, que era a de\textbackslash{}nir jantar com a mãe.\textbackslash{}n\textbackslash{}nNão achou Mendonça nada que admirar na\textbackslash{}nresolução; as resoluções de Margarida começavam a parecer-lhe simplicidades.\textbackslash{}n\textbackslash{}nQuando voltou para casa encontrou um\textbackslash{}nbilhete de D. Antônia concebido nestes termos:\textbackslash{}n\textbackslash{}nTemos de ir para fora alguns meses;\textbackslash{}nespero que não nos deixe sem despedir-se de nós. A partida é sábado; e eu quero\textbackslash{}nincumbi-lo de uma coisa.\textbackslash{}n\textbackslash{}nMendonça tomou chá, e dispôs-se a dormir.\textbackslash{}nNão pôde. Quis ler; estava incapaz disso. Era cedo; saiu. Insensivelmente\textbackslash{}ndirigiu os passos para Mata-cavalos. A casa de D. Antônia estava fechada e\textbackslash{}nsilenciosa; evidentemente estavam já dormindo. Mendonça passou adiante, e parou\textbackslash{}njunto da grade do jardim adjacente à casa. De fora podia ver a janela do quarto\textbackslash{}nde Margarida, pouco elevada, e dando para o jardim. Havia luz dentro;\textbackslash{}nnaturalmente Margarida estava acordada. Mendonça deu mais alguns passos; a\textbackslash{}nporta do jardim estava aberta. Mendonça sentiu pulsar-lhe o coração com força\textbackslash{}ndesconhecida. Surgiu-lhe no espírito uma suspeita. Não há coração confiante que\textbackslash{}nnão tenha desfalecimentos destes; além de que, seria errada a suspeita?\textbackslash{}nMendonça, entretanto, não tinha nenhum direito à viúva; fora repelido\textbackslash{}ncategoricamente. Se havia algum dever da parte dele era a retirada e o\textbackslash{}nsilêncio.\textbackslash{}n\textbackslash{}nMendonça quis conservar-se no limite que\textbackslash{}nlhe estava marcado; a porta aberta do jardim podia ser esquecimento da parte\textbackslash{}ndos fâmulos. O médico refletiu bem que aquilo tudo era fortuito, e fazendo um\textbackslash{}nesforço afastou-se do lugar. Adiante parou e refletiu; havia um demônio que o\textbackslash{}nimpelia por aquela porta dentro. Mendonça voltou, e entrou com precaução.\textbackslash{}n\textbackslash{}nApenas dera alguns passos surgiu-lhe em\textbackslash{}nfrente Miss Dollar latindo; parece que a galga saíra de casa sem ser\textbackslash{}npressentida; Mendonça amimou-a e a cadelinha parece que reconheceu o médico,\textbackslash{}nporque trocou os latidos em festas. Na parede do quarto de Margarida\textbackslash{}ndesenhou-se uma sombra de mulher; era a viúva que chegava à janela para ver a\textbackslash{}ncausa do ruído. Mendonça coseu-se como pôde com uns arbustos que ficavam junto\textbackslash{}nda grade; não vendo ninguém, Margarida voltou para dentro.\textbackslash{}n\textbackslash{}nPassados alguns minutos, Mendonça saiu do\textbackslash{}nlugar em que se achava e dirigiu-se para o lado da janela da viúva.\textbackslash{}nAcompanhava-o Miss Dollar. Do jardim não podia olhar, ainda que fosse\textbackslash{}nmais alto, para o aposento da moça. A cadelinha apenas chegou àquele ponto,\textbackslash{}nsubiu ligeira uma escada de pedra que comunicava o jardim com a casa; a porta\textbackslash{}ndo quarto de Margarida ficava justamente no corredor que se seguia à escada; a\textbackslash{}nporta estava aberta. O rapaz imitou a cadelinha; subiu os seis degraus de pedra\textbackslash{}nvagarosamente; quando pôs o pé no último ouviu Miss Dollar pulando no\textbackslash{}nquarto e vindo latir à porta, como que avisando a Margarida de que se\textbackslash{}naproximava um estranho.\textbackslash{}n\textbackslash{}nMendonça deu mais um passo. Mas nesse\textbackslash{}nmomento atravessou o jardim um escravo que acudia ao latido da cadelinha; o\textbackslash{}nescravo examinou o jardim, e não vendo ninguém retirou-se. Margarida foi à\textbackslash{}njanela e perguntou o que era; o escravo explicou-lho e tranqüilizou-a dizendo\textbackslash{}nque não havia ninguém.\textbackslash{}n\textbackslash{}nJustamente quando ela saía da janela\textbackslash{}naparecia à porta a figura de Mendonça. Margarida estremeceu por um abalo\textbackslash{}nnervoso; ficou mais pálida do que era; depois, concentrando nos olhos toda a\textbackslash{}nsoma de indignação que pode conter um coração, perguntou-lhe com voz trêmula:\textbackslash{}n\textbackslash{}n— Que quer aqui?\textbackslash{}n\textbackslash{}nFoi nesse momento, e só então, que\textbackslash{}nMendonça reconheceu toda a baixeza do seu procedimento, ou para falar mais\textbackslash{}nacertadamente, toda a alucinação do seu espírito. Pareceu-lhe ver em Margarida\textbackslash{}na figura da sua consciência, a exprobrar-lhe tamanha indignidade. O pobre rapaz\textbackslash{}nnão procurou desculpar-se; a sua resposta foi singela e verdadeira.\textbackslash{}n\textbackslash{}n— Sei que cometi um ato infame, disse\textbackslash{}nele; não tinha razão para isso; estava louco; agora conheço a extensão do mal.\textbackslash{}nNão lhe peço que me desculpe, D. Margarida; não mereço perdão; mereço desprezo;\textbackslash{}nadeus!\textbackslash{}n\textbackslash{}n— Compreendo, senhor, disse Margarida;\textbackslash{}nquer obrigar-me pela força do descrédito quando me não pode obrigar pelo\textbackslash{}ncoração. Não é de cavalheiro.\textbackslash{}n\textbackslash{}n— Oh! isso{\ldots} juro-lhe que não foi tal o\textbackslash{}nmeu pensamento{\ldots}\textbackslash{}n\textbackslash{}nMargarida caiu numa cadeira parecendo\textbackslash{}nchorar. Mendonça deu um passo para entrar, visto que até então não saíra da\textbackslash{}nporta; Margarida levantou os olhos cobertos de lágrimas, e com um gesto\textbackslash{}nimperioso mostrou-lhe que saísse.\textbackslash{}n\textbackslash{}nMendonça obedeceu; nem um nem outro\textbackslash{}ndormiram nessa noite. Ambos curvavam-se ao peso da vergonha: mas, por honra de\textbackslash{}nMendonça, a dele era maior que a dela; e a dor de uma não ombreava com o\textbackslash{}nremorso de outro.\textbackslash{}n\textbackslash{}nCAPÍTULO\textbackslash{}nVIII\textbackslash{}n\textbackslash{}nNo dia seguinte estava Mendonça em casa\textbackslash{}nfumando charutos sobre charutos, recurso das grandes ocasiões, quando parou à\textbackslash{}nporta dele um carro, apeando-se pouco depois a mãe de Jorge. A visita pareceu\textbackslash{}nde mau agouro ao médico. Mas apenas a velha entrou, dissipou-lhe o receio.\textbackslash{}n\textbackslash{}n— Creio, disse D. Antônia, que a minha\textbackslash{}nidade permite visitar um homem solteiro.\textbackslash{}n\textbackslash{}nMendonça procurou sorrir ouvindo este\textbackslash{}ngracejo; mas não pôde. Convidou a boa senhora a sentar-se, e sentou-se ele\textbackslash{}ntambém esperando que ela lhe explicasse a causa da visita.\textbackslash{}n\textbackslash{}n— Escrevi-lhe ontem, disse ela, para que\textbackslash{}nfosse ver-me hoje; preferi vir cá, receando que por qualquer motivo não fosse a\textbackslash{}nMata-cavalos.\textbackslash{}n\textbackslash{}n— Queria então incumbir-me?\textbackslash{}n\textbackslash{}n— De coisa nenhuma, respondeu a velha\textbackslash{}nsorrindo; incumbir disse-lhe eu, como diria qualquer outra coisa indiferente;\textbackslash{}nquero informá-lo.\textbackslash{}n\textbackslash{}n— Ah! de quê?\textbackslash{}n\textbackslash{}n— Sabe quem ficou hoje de cama?\textbackslash{}n\textbackslash{}n— D. Margarida?\textbackslash{}n\textbackslash{}n— É verdade; amanheceu um pouco doente;\textbackslash{}ndiz que passou a noite mal. Eu creio que sei a razão, acrescentou D. Antônia\textbackslash{}nrindo maliciosamente para Mendonça.\textbackslash{}n\textbackslash{}n— Qual será então a razão? perguntou o\textbackslash{}nmédico.\textbackslash{}n\textbackslash{}n— Pois não percebe?\textbackslash{}n\textbackslash{}n— Não.\textbackslash{}n\textbackslash{}n— Margarida ama-o.\textbackslash{}n\textbackslash{}nMendonça levantou-se da cadeira como por\textbackslash{}numa mola. A declaração da tia da viúva era tão inesperada que o rapaz cuidou\textbackslash{}nestar sonhando.\textbackslash{}n\textbackslash{}n— Ama-o, repetiu D. Antônia.\textbackslash{}n\textbackslash{}n— Não creio, respondeu Mendonça depois de\textbackslash{}nalgum silêncio; há de ser engano seu.\textbackslash{}n\textbackslash{}n— Engano! disse a velha.\textbackslash{}n\textbackslash{}nD. Antônia contou a Mendonça que, curiosa\textbackslash{}npor saber a causa das vigílias de Margarida, descobrira no quarto dela um diário\textbackslash{}nde impressões, escrito por ela, à imitação de não sei quantas heroínas de\textbackslash{}nromances; aí lera a verdade que lhe acabava de dizer.\textbackslash{}n\textbackslash{}n— Mas se me ama, observou Mendonça\textbackslash{}nsentindo entrar-lhe n’alma um mundo de esperanças, se me ama, por que recusa o\textbackslash{}nmeu coração?\textbackslash{}n\textbackslash{}n— O diário explica isso mesmo; eu\textbackslash{}nlhe digo. Margarida foi infeliz no casamento; o marido teve unicamente em vista\textbackslash{}ngozar da riqueza dela; Margarida adquiriu a certeza de que nunca será amada por\textbackslash{}nsi, mas pelos cabedais que possui; atribui o seu amor à cobiça. Está\textbackslash{}nconvencido?\textbackslash{}n\textbackslash{}nMendonça começou a protestar.\textbackslash{}n\textbackslash{}n— É inútil, disse D. Antônia, eu creio na\textbackslash{}nsinceridade do seu afeto; já de há muito percebi isso mesmo; mas como convencer\textbackslash{}num coração desconfiado?\textbackslash{}n\textbackslash{}n— Não sei.\textbackslash{}n\textbackslash{}n— Nem eu, disse a velha, mas para isso é\textbackslash{}nque eu vim cá; peço-lhe que veja se pode fazer com que a minha Margarida torne\textbackslash{}na ser feliz, se lhe influi a crença no amor que lhe tem.\textbackslash{}n\textbackslash{}n— Acho que é impossível{\ldots}\textbackslash{}n\textbackslash{}nMendonça lembrou-se de contar a D.\textbackslash{}nAntônia a cena da véspera; mas arrependeu-se a tempo.\textbackslash{}n\textbackslash{}nD. Antônia saiu pouco depois.\textbackslash{}n\textbackslash{}nA situação de Mendonça, ao passo que se\textbackslash{}ntornara mais clara, estava mais difícil que dantes. Era possível tentar alguma\textbackslash{}ncoisa antes da cena do quarto; mas depois, achava Mendonça impossível conseguir\textbackslash{}nnada.\textbackslash{}n\textbackslash{}nA doença de Margarida durou dois dias, no\textbackslash{}nfim dos quais levantou-se a viúva um pouco abatida, e a primeira coisa que fez\textbackslash{}nfoi escrever a Mendonça pedindo-lhe que fosse lá à casa.\textbackslash{}n\textbackslash{}nMendonça admirou-se bastante do convite,\textbackslash{}ne obedeceu de pronto.\textbackslash{}n\textbackslash{}n— Depois do que se deu há três dias,\textbackslash{}ndisse-lhe Margarida, compreende o senhor que eu não posso ficar debaixo da ação\textbackslash{}nda maledicência{\ldots} Diz que me ama; pois bem, o nosso casamento é inevitável.\textbackslash{}n\textbackslash{}nInevitável! amargou esta\textbackslash{}npalavra ao médico, que aliás não podia recusar uma reparação. Lembrava-se ao\textbackslash{}nmesmo tempo que era amado; e conquanto a idéia lhe sorrisse ao espírito, outra\textbackslash{}nvinha dissipar esse instantâneo prazer, e era a suspeita que Margarida nutria a\textbackslash{}nseu respeito.\textbackslash{}n\textbackslash{}n— Estou às suas ordens, respondeu ele.\textbackslash{}n\textbackslash{}nAdmirou-se D. Antônia da presteza do\textbackslash{}ncasamento quando Margarida lho anunciou nesse mesmo dia. Supôs que fosse\textbackslash{}nmilagre do rapaz. Pelo tempo adiante reparou que os noivos tinham cara mais de\textbackslash{}nenterro que de casamento. Interrogou a sobrinha a esse respeito; obteve uma\textbackslash{}nresposta evasiva.\textbackslash{}n\textbackslash{}nFoi modesta e reservada a cerimônia do\textbackslash{}ncasamento. Andrade serviu de padrinho, D. Antônia de madrinha; Jorge falou no\textbackslash{}nAlcazar a um padre, seu amigo, para celebrar o ato.\textbackslash{}n\textbackslash{}nD. Antônia quis que os noivos ficassem\textbackslash{}nresidindo em casa com ela. Quando Mendonça se achou a sós com Margarida,\textbackslash{}ndisse-lhe:\textbackslash{}n\textbackslash{}n— Casei-me para salvar-lhe a reputação;\textbackslash{}nnão quero obrigar pela fatalidade das coisas um coração que me não pertence.\textbackslash{}nTer-me-á por seu amigo; até amanhã.\textbackslash{}n\textbackslash{}nSaiu Mendonça depois deste speech,\textbackslash{}ndeixando Margarida suspensa entre o conceito que fazia dele e a impressão das\textbackslash{}nsuas palavras agora.\textbackslash{}n\textbackslash{}nNão havia posição mais singular do que a\textbackslash{}ndestes noivos separados por uma quimera. O mais belo dia da vida tornava-se\textbackslash{}npara eles um dia de desgraça e de solidão; a formalidade do casamento foi\textbackslash{}nsimplesmente o prelúdio do mais completo divórcio. Menos ceticismo da parte de\textbackslash{}nMargarida, mais cavalheirismo da parte do rapaz, teriam poupado o desenlace\textbackslash{}nsombrio da comédia do coração. Vale mais imaginar que descrever as torturas\textbackslash{}ndaquela primeira noite de noivado.\textbackslash{}n\textbackslash{}nMas aquilo que o espírito do homem não\textbackslash{}nvence, há de vencê-lo o tempo, a quem cabe final razão. O tempo convenceu\textbackslash{}nMargarida de que a sua suspeita era gratuita; e, coincidindo com ele o coração,\textbackslash{}nveio a tornar-se efetivo o casamento apenas celebrado.\textbackslash{}n\textbackslash{}nAndrade ignorou estas coisas; cada vez\textbackslash{}nque encontrava Mendonça chamava-lhe Colombo do amor; tinha Andrade a mania de\textbackslash{}ntodo o sujeito a quem as idéias ocorrem trimestralmente; apenas pilhada alguma\textbackslash{}nde jeito repetia-a até a saciedade.\textbackslash{}n\textbackslash{}nOs dois esposos são ainda noivos e\textbackslash{}nprometem sê-lo até a morte. Andrade meteu-se na diplomacia e promete ser um dos\textbackslash{}nluzeiros da nossa representação internacional. Jorge continua a ser um bom\textbackslash{}npândego; D. Antônia prepara-se para despedir-se do mundo.\textbackslash{}n\textbackslash{}nQuanto a Miss Dollar, causa\textbackslash{}nindireta de todos estes acontecimentos, saindo um dia à rua foi pisada por um\textbackslash{}ncarro; faleceu pouco depois. Margarida não pôde reter algumas lágrimas pela\textbackslash{}nnobre cadelinha; foi o corpo enterrado na chácara, à sombra de uma laranjeira;\textbackslash{}ncobre a sepultura uma lápide com esta simples inscrição:\textbackslash{}n\textbackslash{}nA Miss Dollar\textbackslash{}n\textbackslash{}nLuís Soares\textbackslash{}n\textbackslash{}nÍNDICE\textbackslash{}n\textbackslash{}nCapítulo Primeiro\textbackslash{}n\textbackslash{}nCapítulo II\textbackslash{}n\textbackslash{}nCapítulo iii\textbackslash{}n\textbackslash{}nCapítulo iv\textbackslash{}n\textbackslash{}nCapítulo v\textbackslash{}n\textbackslash{}nCapítulo vI\textbackslash{}n\textbackslash{}nCAPÍTULO PRIMEIRO\textbackslash{}n\textbackslash{}nTrocar o dia pela noite, dizia Luís\textbackslash{}nSoares, é restaurar o império da natureza corrigindo a obra da sociedade. O\textbackslash{}ncalor do sol está dizendo aos homens que vão descansar e dormir, ao passo que a\textbackslash{}nfrescura relativa da noite é a verdadeira estação em que se deve viver. Livre\textbackslash{}nem todas as minhas ações, não quero sujeitar-me à lei absurda que a sociedade\textbackslash{}nme impõe: velarei de noite, dormirei de dia.\textbackslash{}n\textbackslash{}nContrariamente a vários ministérios,\textbackslash{}nSoares cumpria este programa com um escrúpulo digno de uma grande consciência. A\textbackslash{}naurora para ele era o crepúsculo, o crepúsculo era a aurora. Dormia doze horas\textbackslash{}nconsecutivas durante o dia, quer dizer das seis da manhã às seis da tarde.\textbackslash{}nAlmoçava às sete e jantava às duas da madrugada. Não ceava. A sua ceia\textbackslash{}nlimitava-se a uma xícara de chocolate que o criado lhe dava às cinco horas da\textbackslash{}nmanhã quando ele entrava para casa. Soares engolia o chocolate, fumava dois\textbackslash{}ncharutos, fazia alguns trocadilhos com o criado, lia uma página de algum\textbackslash{}nromance, e deitava-se.\textbackslash{}n\textbackslash{}nNão lia jornais. Achava que um jornal era\textbackslash{}na coisa mais inútil deste mundo, depois da Câmara dos Deputados, das obras dos\textbackslash{}npoetas e das missas. Não quer isto dizer que Soares fosse ateu em religião,\textbackslash{}npolítica e poesia. Não. Soares era apenas indiferente. Olhava para todas as\textbackslash{}ngrandes coisas com a mesma cara com que via uma mulher feia. Podia vir a ser um\textbackslash{}ngrande perverso; até então era apenas uma grande inutilidade.\textbackslash{}n\textbackslash{}nGraças a uma boa fortuna que lhe deixara\textbackslash{}no pai, Soares podia gozar a vida que levava, esquivando-se a todo o gênero de\textbackslash{}ntrabalho e entregue somente aos instintos da sua natureza e aos caprichos do\textbackslash{}nseu coração. Coração é talvez demais. Era duvidoso que Soares o tivesse. Ele\textbackslash{}nmesmo o dizia. Quando alguma dama lhe pedia que ele a amasse, Soares respondia:\textbackslash{}n\textbackslash{}n— Minha rica pequena, eu nasci com a\textbackslash{}ngrande vantagem de não ter coisa nenhuma dentro do peito nem dentro da cabeça.\textbackslash{}nIsso que chamam juízo e sentimento são para mim verdadeiros mistérios. Não os\textbackslash{}ncompreendo porque os não sinto.\textbackslash{}n\textbackslash{}nSoares acrescentava que a fortuna\textbackslash{}nsuplantara a natureza deitando-lhe no berço em que nasceu uma boa soma de\textbackslash{}ncontos de réis. Mas esquecia que a fortuna, apesar de generosa, é exigente, e\textbackslash{}nquer da parte dos seus afilhados algum esforço próprio. A fortuna não é\textbackslash{}nDanaide. Quando vê que um tonel esgota a água que se lhe põe dentro vai levar\textbackslash{}nos seus cântaros a outra parte. Soares não pensava nisto. Cuidava que os seus\textbackslash{}nbens eram renascentes como as cabeças da hidra antiga. Gastava às mãos largas;\textbackslash{}ne os contos de réis, tão dificilmente acumulados por seu pai, escapavam-se-lhes\textbackslash{}ndas mãos como pássaros sequiosos por gozarem do ar livre.\textbackslash{}n\textbackslash{}nAchou-se, portanto, pobre quando menos o\textbackslash{}nesperava. Um dia de manhã, quer dizer às ave-marias, os olhos de Soares viram\textbackslash{}nescritas as palavras fatídicas do festim babilônico. Era uma carta que o criado\textbackslash{}nlhe entregara dizendo que o banqueiro de Soares a havia deixado à meia-noite. O\textbackslash{}ncriado falava como o amo vivia: ao meio-dia chamava meia-noite.\textbackslash{}n\textbackslash{}n— Já te disse, respondeu Soares, que eu\textbackslash{}nsó recebo cartas dos meus amigos, ou então{\ldots}\textbackslash{}n\textbackslash{}n— De alguma rapariga, bem sei. É por isso\textbackslash{}nque lhe não tenho dado as cartas que o banqueiro tem trazido há um mês. Hoje,\textbackslash{}nporém, o homem disse que era indispensável que lhe eu desse esta.\textbackslash{}n\textbackslash{}nSoares sentou-se na cama, e perguntou ao\textbackslash{}ncriado meio alegre e meio zangado:\textbackslash{}n\textbackslash{}n— Então tu és criado dele ou meu?\textbackslash{}n\textbackslash{}n— Meu amo, o banqueiro disse que se trata\textbackslash{}nde um grande perigo.\textbackslash{}n\textbackslash{}n— Que perigo?\textbackslash{}n\textbackslash{}n— Não sei.\textbackslash{}n\textbackslash{}n— Deixa ver a carta.\textbackslash{}n\textbackslash{}nO criado entregou-lhe a carta.\textbackslash{}n\textbackslash{}nSoares abriu-a e leu-a duas vezes. Dizia\textbackslash{}na carta que o rapaz não possuía mais que seis contos de réis. Para Soares seis\textbackslash{}ncontos de réis eram menos que seis vinténs.\textbackslash{}n\textbackslash{}nPela primeira vez na sua vida Soares\textbackslash{}nsentiu uma grande comoção. A idéia de não ter dinheiro nunca lhe havia acudido\textbackslash{}nao espírito; não imaginava que um dia se achasse na posição de qualquer outro\textbackslash{}nhomem que precisava de trabalhar.\textbackslash{}n\textbackslash{}nAlmoçou sem vontade e saiu. Foi ao\textbackslash{}nAlcazar. Os amigos acharam-no triste; perguntaram-lhe se era alguma mágoa de\textbackslash{}namor. Soares respondeu que estava doente. As Laís da localidade acharam que era\textbackslash{}nde bom gosto ficarem tristes também. A consternação foi geral.\textbackslash{}n\textbackslash{}nUm dos seus amigos, José Pires, propôs um\textbackslash{}npasseio a Botafogo para distrair as melancolias de Soares. O rapaz aceitou. Mas\textbackslash{}no passeio a Botafogo era tão comum que não podia distraí-lo. Lembraram-se de ir\textbackslash{}nao Corcovado, idéia que foi aceita e executada imediatamente.\textbackslash{}n\textbackslash{}nMas que há que possa distrair um rapaz\textbackslash{}nnas condições de Soares? A viagem ao Corcovado apenas lhe produziu uma grande\textbackslash{}nfadiga, aliás útil, porque, na volta, dormiu o rapaz a sono solto.\textbackslash{}n\textbackslash{}nQuando acordou mandou dizer ao Pires que\textbackslash{}nviesse falar-lhe imediatamente. Daí a uma hora parava um carro à porta: era o\textbackslash{}nPires que chegava, mas acompanhado de uma rapariga morena que respondia ao nome\textbackslash{}nde Vitória. Entraram os dois pela sala de Soares com a franqueza e o estrépito\textbackslash{}nnaturais entre pessoas de família.\textbackslash{}n\textbackslash{}n— Não está doente? perguntou Vitória ao\textbackslash{}ndono da casa.\textbackslash{}n\textbackslash{}n— Não, respondeu este; mas por que veio\textbackslash{}nvocê?\textbackslash{}n\textbackslash{}n— É boa! disse José Pires; veio porque é\textbackslash{}na minha xícara inseparável{\ldots} Querias falar-me em particular?\textbackslash{}n\textbackslash{}n— Queria.\textbackslash{}n\textbackslash{}n— Pois falemos aí em qualquer canto;\textbackslash{}nVitória fica na sala vendo os álbuns.\textbackslash{}n\textbackslash{}n— Nada, interrompeu a moça; nesse caso\textbackslash{}nvou-me embora. É melhor; só imponho uma condição: é que ambos hão de ir depois\textbackslash{}nlá para casa; temos ceata.\textbackslash{}n\textbackslash{}n— Valeu! disse Pires.\textbackslash{}n\textbackslash{}nVitória saiu; os dois rapazes ficaram\textbackslash{}nsós.\textbackslash{}n\textbackslash{}nPires era o tipo do bisbilhoteiro e leviano.\textbackslash{}nEm lhe cheirando novidade preparava-se para instruir-se de tudo. Lisonjeava-o a\textbackslash{}nconfiança de Soares, e adivinhava que o rapaz ia comunicar-lhe alguma coisa\textbackslash{}nimportante. Para isso assumiu um ar condigno com a situação. Sentou-se\textbackslash{}ncomodamente em uma cadeira de braços; pôs o castão da bengala na boca e começou\textbackslash{}no ataque com estas palavras:\textbackslash{}n\textbackslash{}n— Estamos sós; que me queres?\textbackslash{}n\textbackslash{}nSoares confiou-lhe tudo; leu-lhe a carta\textbackslash{}ndo banqueiro; mostrou-lhe em toda a nudez a sua miséria. Disse-lhe que naquela\textbackslash{}nsituação não via solução possível, e confessou ingenuamente que a idéia do\textbackslash{}nsuicídio o havia alimentado durante longas horas.\textbackslash{}n\textbackslash{}n— Um suicídio! exclamou Pires; estás\textbackslash{}ndoido.\textbackslash{}n\textbackslash{}n— Doido! respondeu Soares; entretanto não\textbackslash{}nvejo outra saída neste beco. Demais, é apenas meio suicídio, porque a pobreza\textbackslash{}njá é meia morte.\textbackslash{}n\textbackslash{}n— Convenho que a pobreza não é coisa\textbackslash{}nagradável, e até acho{\ldots}\textbackslash{}n\textbackslash{}nPires interrompeu-se; uma idéia súbita\textbackslash{}natravessara-lhe o espírito: a idéia de que Soares acabasse a conferência por\textbackslash{}npedir-lhe dinheiro. Pires tinha um preceito na sua vida: era não emprestar\textbackslash{}ndinheiro aos amigos. Não se empresta sangue, dizia ele.\textbackslash{}n\textbackslash{}nSoares não reparou na frase cortada do\textbackslash{}namigo, e disse:\textbackslash{}n\textbackslash{}n— Viver pobre depois de ter sido rico{\ldots}\textbackslash{}né impossível.\textbackslash{}n\textbackslash{}n— Nesse caso que me queres tu? perguntou\textbackslash{}nPires, a quem pareceu que era bom atacar o touro de frente.\textbackslash{}n\textbackslash{}n— Um conselho.\textbackslash{}n\textbackslash{}n— Inútil conselho, pois que já tens uma\textbackslash{}nidéia fixa.\textbackslash{}n\textbackslash{}n— Talvez. Entretanto confesso que não se\textbackslash{}ndeixa a vida com facilidade, e má ou boa, sempre custa morrer. Por outro lado, ostentar\textbackslash{}na minha miséria diante das pessoas que me viram rico é uma humilhação que eu\textbackslash{}nnão aceito. Que farias tu no meu lugar?\textbackslash{}n\textbackslash{}n— Homem, respondeu Pires, há muitos\textbackslash{}nmeios{\ldots}\textbackslash{}n\textbackslash{}n— Venha um.\textbackslash{}n\textbackslash{}n— Primeiro meio. Vai para Nova Iorque e\textbackslash{}nprocura uma fortuna.\textbackslash{}n\textbackslash{}n— Não me convém; nesse caso fico no Rio\textbackslash{}nde Janeiro.\textbackslash{}n\textbackslash{}n— Segundo meio. Arranja um casamento\textbackslash{}nrico.\textbackslash{}n\textbackslash{}n— É bom de dizer. Onde está esse\textbackslash{}ncasamento?\textbackslash{}n\textbackslash{}n— Procura. Não tens uma prima que gosta\textbackslash{}nde ti?\textbackslash{}n\textbackslash{}n— Creio que já não gosta; e demais não é\textbackslash{}nrica; tem apenas trinta contos; despesa de um ano.\textbackslash{}n\textbackslash{}n— É um bom princípio de vida.\textbackslash{}n\textbackslash{}n— Nada; outro meio.\textbackslash{}n\textbackslash{}n— Terceiro meio, e o melhor. Vai à casa\textbackslash{}nde teu tio, angaria-lhe a estima, dize que estás arrependido da vida passada,\textbackslash{}naceita um emprego, enfim vê se te constituis seu herdeiro universal.\textbackslash{}n\textbackslash{}nSoares não respondeu; a idéia pareceu-lhe\textbackslash{}nboa.\textbackslash{}n\textbackslash{}n— Aposto que te agrada o terceiro meio?\textbackslash{}nperguntou Pires rindo.\textbackslash{}n\textbackslash{}n— Não é mau. Aceito; e bem sei que é\textbackslash{}ndifícil e demorado; mas eu não tenho muitos à escolha.\textbackslash{}n\textbackslash{}n— Ainda bem, disse Pires levantando-se.\textbackslash{}nAgora o que se quer é algum juízo. Há de custar-te o sacrifício, mas lembra-te\textbackslash{}nque é o meio único de teres dentro de pouco tempo uma fortuna. Teu tio é um\textbackslash{}nhomem achacado de moléstias; qualquer dia bate a bota. Aproveita o tempo. E\textbackslash{}nagora vamos à ceia da Vitória.\textbackslash{}n\textbackslash{}n— Não vou, disse Soares; quero\textbackslash{}nacostumar-me desde já a viver vida nova.\textbackslash{}n\textbackslash{}n— Bem; adeus.\textbackslash{}n\textbackslash{}n— Olha; confiei-te isto a ti só;\textbackslash{}nguarda-me segredo.\textbackslash{}n\textbackslash{}n— Sou um túmulo, respondeu Pires descendo\textbackslash{}na escada.\textbackslash{}n\textbackslash{}nMas no dia seguinte já os rapazes e\textbackslash{}nraparigas sabiam que Soares ia fazer-se anacoreta{\ldots} por não ter dinheiro\textbackslash{}nnenhum. O próprio Soares reconheceu isto no rosto dos amigos. Todos pareciam\textbackslash{}ndizer-lhe: É pena! que pândego vamos nós perder!\textbackslash{}n\textbackslash{}nPires nunca mais o visitou.\textbackslash{}n\textbackslash{}nCAPÍTULO II\textbackslash{}n\textbackslash{}nO tio de Soares chamava-se o Major Luís\textbackslash{}nda Cunha Vilela, e era com efeito um homem já velho e adoentado. Contudo não se\textbackslash{}npodia dizer que morreria cedo. O Major Vilela observava um rigoroso regímen que\textbackslash{}nlhe ia entretendo a vida. Tinha uns bons sessenta anos. Era um velho alegre e\textbackslash{}nsevero ao mesmo tempo. Gostava de rir, mas era implacável com os maus costumes.\textbackslash{}nConstitucional por necessidade, era no fundo de sua alma absolutista. Chorava\textbackslash{}npela sociedade antiga; criticava constantemente a nova. Enfim foi o último\textbackslash{}nhomem que abandonou a cabeleira de rabicho.\textbackslash{}n\textbackslash{}nVivia o Major Vilela em Catumbi,\textbackslash{}nacompanhado de sua sobrinha Adelaide, e mais uma velha parenta. A sua vida era\textbackslash{}npatriarcal. Importando-se pouco ou nada com o que ia por fora, o major\textbackslash{}nentregava-se todo ao cuidado de sua casa, aonde poucos amigos e algumas\textbackslash{}nfamílias da vizinhança o iam ver, e passar as noites com ele. O major\textbackslash{}nconservava sempre a mesma alegria, ainda nas ocasiões em que o reumatismo o\textbackslash{}nprostrava. Os reumáticos dificilmente acreditarão nisto; mas eu posso afirmar\textbackslash{}nque era verdade.\textbackslash{}n\textbackslash{}nFoi num dia de manhã, felizmente um dia\textbackslash{}nem que o major não sentia o menor achaque, e ria e brincava com as duas\textbackslash{}nparentas, que Soares apareceu em Catumbi à porta do tio.\textbackslash{}n\textbackslash{}nQuando o major recebeu o cartão com o\textbackslash{}nnome do sobrinho, supôs que era alguma caçoada. Podia contar com todos em casa,\textbackslash{}nmenos o sobrinho. Fazia já dois anos que o não via, e entre a última e a\textbackslash{}npenúltima vez tinha mediado ano e meio. Mas o moleque disse-lhe tão seriamente\textbackslash{}nque o nhonhô Luís estava na sala de espera, que o velho acabou por acreditar.\textbackslash{}n\textbackslash{}n— Que te parece, Adelaide?\textbackslash{}n\textbackslash{}nA moça não respondeu.\textbackslash{}n\textbackslash{}nO velho foi à sala de visitas.\textbackslash{}n\textbackslash{}nSoares tinha pensado no meio de aparecer\textbackslash{}nao tio. Ajoelhar-se era dramático demais; cair-lhe nos braços exigia certo\textbackslash{}nimpulso íntimo que ele não tinha; além de que, Soares vexava-se de ter ou\textbackslash{}nfingir uma comoção. Lembrou-se de começar uma conversação alheia ao fim que o\textbackslash{}nlevava lá, e acabar por confessar-se disposto a arrepiar carreira. Mas este\textbackslash{}nmeio tinha o inconveniente de fazer preceder a reconciliação por um sermão, que\textbackslash{}no rapaz dispensava. Ainda não se resolvera a aceitar um dos muitos meios que\textbackslash{}nlhe vieram à idéia, quando o major apareceu à porta da sala.\textbackslash{}n\textbackslash{}nO major parou à porta sem dizer palavra e\textbackslash{}nlançou sobre o sobrinho um olhar severo e interrogador.\textbackslash{}n\textbackslash{}nSoares hesitou um instante; mas como a\textbackslash{}nsituação podia prolongar-se sem benefício seu, o rapaz seguiu um movimento\textbackslash{}nnatural: foi ao tio e estendeu-lhe a mão.\textbackslash{}n\textbackslash{}n— Meu tio, disse ele, não precisa dizer\textbackslash{}nmais nada; o seu olhar diz-me tudo. Fui pecador e arrependo-me. Aqui estou.\textbackslash{}n\textbackslash{}nO major estendeu-lhe a mão, que o rapaz\textbackslash{}nbeijou com o respeito de que era suscetível.\textbackslash{}n\textbackslash{}nDepois encaminhou-se para uma cadeira e\textbackslash{}nsentou-se; o rapaz ficou de pé.\textbackslash{}n\textbackslash{}n— Se o teu arrependimento é sincero,\textbackslash{}nabro-te a minha porta e o meu coração. Se não é sincero podes ir embora; há\textbackslash{}nmuito tempo que não freqüento a casa da ópera: não gosto de comediantes.\textbackslash{}n\textbackslash{}nSoares protestou que era sincero. Disse\textbackslash{}nque fora dissipado e doido, mas que aos trinta anos era justo ter juízo.\textbackslash{}nReconhecia agora que o tio sempre tivera razão. Supôs ao princípio que eram\textbackslash{}nsimples rabugices de velho, e mais nada; mas não era natural esta leviandade num\textbackslash{}nrapaz educado no vício? Felizmente corrigia-se a tempo. O que ele agora queria\textbackslash{}nera entrar em bom viver, e começava por aceitar um emprego público que o\textbackslash{}nobrigasse a trabalhar e fazer-se sério. Tratava-se de ganhar uma posição.\textbackslash{}n\textbackslash{}nOuvindo o discurso de que fiz o extrato\textbackslash{}nacima, o major procurava adivinhar o fundo do pensamento de Soares. Seria ele\textbackslash{}nsincero? O velho concluiu que o sobrinho falava com a alma nas mãos. A sua\textbackslash{}nilusão chegou ao ponto de ver-lhe uma lágrima nos olhos, lágrima que não\textbackslash{}napareceu, nem mesmo fingida.\textbackslash{}n\textbackslash{}nQuando Soares acabou, o major\textbackslash{}nestendeu-lhe a mão e apertou a que o rapaz lhe estendeu também.\textbackslash{}n\textbackslash{}n— Creio, Luís. Ainda bem que te\textbackslash{}narrependeste a tempo. Isso que vivias não era vida nem morte; a vida é mais\textbackslash{}ndigna e a morte mais tranqüila do que a existência que malbarataste. Entras\textbackslash{}nagora em casa como um filho pródigo. Terás o melhor lugar à mesa. Esta família\textbackslash{}né a mesma família.\textbackslash{}n\textbackslash{}nO major continuou por este tom; Soares\textbackslash{}nouviu a pé quedo o discurso do tio. Dizia consigo que era a amostra da pena que\textbackslash{}nia sofrer, e um grande desconto dos seus pecados.\textbackslash{}n\textbackslash{}nO major acabou levando o rapaz para\textbackslash{}ndentro, onde os esperava o almoço.\textbackslash{}n\textbackslash{}nNa sala de jantar estavam Adelaide e a\textbackslash{}nvelha parenta. A Sra. Antônia de Moura Vilela recebeu Soares com grandes\textbackslash{}nexclamações que envergonharam sinceramente o rapaz. Quanto a Adelaide, apenas o\textbackslash{}ncumprimentou sem olhar para ele; Soares retribuiu o cumprimento.\textbackslash{}n\textbackslash{}nO major reparou na frieza; mas parece que\textbackslash{}nsabia alguma coisa, porque apenas deu uma risadinha amarela, coisa que lhe era\textbackslash{}npeculiar.\textbackslash{}n\textbackslash{}nSentaram-se à mesa, e o almoço correu\textbackslash{}nentre as pilhérias do major, as recriminações da Sra. Antônia, as explicações\textbackslash{}ndo rapaz e o silêncio de Adelaide. Quando o almoço acabou, o major disse ao\textbackslash{}nsobrinho que fumasse, concessão enorme que o rapaz a custo aceitou. As duas\textbackslash{}nsenhoras saíram; ficaram os dois à mesa.\textbackslash{}n\textbackslash{}n— Estás então disposto a trabalhar?\textbackslash{}n\textbackslash{}n— Estou, meu tio.\textbackslash{}n\textbackslash{}n— Bem; vou ver se te arranjo um emprego.\textbackslash{}nQue emprego preferes?\textbackslash{}n\textbackslash{}n— O que quiser, meu tio, contanto que eu\textbackslash{}ntrabalhe.\textbackslash{}n\textbackslash{}n— Bem. Levarás amanhã, uma carta minha a\textbackslash{}num dos ministros. Deus queira que possas obter o emprego sem dificuldade. Quero\textbackslash{}nver-te trabalhador e sério; quero ver-te homem. As dissipações não produzem\textbackslash{}nnada, a não serem dívidas e desgostos{\ldots} Tens dívidas?\textbackslash{}n\textbackslash{}n— Nenhuma, respondeu Soares.\textbackslash{}n\textbackslash{}nSoares mentia. Tinha uma dívida de\textbackslash{}nalfaiate, relativamente pequena; queria pagá-la sem que o tio soubesse.\textbackslash{}n\textbackslash{}nNo dia seguinte o major escreveu a carta\textbackslash{}nprometida, que o sobrinho levou ao ministro; e tão feliz foi, que daí a um mês\textbackslash{}nestava empregado em uma secretaria com um bom ordenado.\textbackslash{}n\textbackslash{}nCumpre fazer justiça ao rapaz. O\textbackslash{}nsacrifício que fez de transformar os seus hábitos da vida foi enorme, e a\textbackslash{}njulgá-lo pelos seus antecedentes, ninguém o julgara capaz de tal. Mas o desejo\textbackslash{}nde perpetuar uma vida de dissipação pode explicar a mudança e o sacrifício.\textbackslash{}nAquilo na existência de Soares não passava de um parêntesis mais ou menos\textbackslash{}nextenso. Almejava por fechá-lo e continuar o período como havia começado, isto\textbackslash{}né, vivendo com Aspásia e pagodeando com Alcibíades.\textbackslash{}n\textbackslash{}nO tio não desconfiava de nada; mas temia\textbackslash{}nque o rapaz fosse novamente tentado à fuga, ou porque o seduzisse a lembrança\textbackslash{}ndas dissipações antigas, ou porque o aborrecesse a monotonia e a fadiga do\textbackslash{}ntrabalho. Com o fim de impedir o desastre, lembrou-se de inspirar-lhe ambição\textbackslash{}npolítica. Pensava o major que a política seria um remédio decisivo para aquele\textbackslash{}ndoente, como se não fosse conhecido que os louros de Lovelace e os de Turgot andam\textbackslash{}nmuita vez na mesma cabeça.\textbackslash{}n\textbackslash{}nSoares não desanimou o major. Disse que\textbackslash{}nera natural acabar a sua existência na política, e chegou a dizer que algumas\textbackslash{}nvezes sonhara com uma cadeira no parlamento.\textbackslash{}n\textbackslash{}n— Pois eu verei se te posso arranjar\textbackslash{}nisto, respondeu o tio. O que é preciso é que estudes a ciência da política, a\textbackslash{}nhistória do nosso parlamento e do nosso governo; e principalmente é preciso que\textbackslash{}ncontinues a ser o que és hoje: um rapaz sério.\textbackslash{}n\textbackslash{}nSe bem o dizia o major, melhor o fazia\textbackslash{}nSoares, que desde então meteu-se com os livros e lia com afinco as discussões\textbackslash{}ndas câmaras.\textbackslash{}n\textbackslash{}nSoares não morava com o tio, mas passava\textbackslash{}nlá todo o tempo que lhe sobrava do trabalho, e voltava para casa depois do chá,\textbackslash{}nque era patriarcal, e bem diferente das ceatas do antigo tempo.\textbackslash{}n\textbackslash{}nNão afirmo que entre as duas fases da\textbackslash{}nexistência de Luís Soares não houvesse algum elo de união, e que o emigrante\textbackslash{}ndas terras de Gnido não fizesse de quando em quando excursões à pátria. Em todo\textbackslash{}no caso essas excursões eram tão secretas que ninguém sabia delas, nem talvez os\textbackslash{}nhabitantes das referidas terras, com exceção dos poucos escolhidos para\textbackslash{}nreceberem o expatriado. O caso era singular, porque naquele país não se\textbackslash{}nreconhece o cidadão naturalizado estrangeiro, ao contrário da Inglaterra, que\textbackslash{}nnão dá aos súditos da rainha o direito de escolherem outra pátria.\textbackslash{}n\textbackslash{}nSoares encontrava-se de quando em quando\textbackslash{}ncom Pires. O confidente do convertido manifestava a sua amizade antiga\textbackslash{}noferecendo-lhe um charuto de Havana e contando-lhe algumas boas fortunas\textbackslash{}nhavidas nas campanhas do amor, em que o alarve supunha ser consumado general.\textbackslash{}n\textbackslash{}nHavia já cinco meses que o sobrinho do\textbackslash{}nMajor Vilela se achava empregado, e ainda os chefes da repartição não tinham\textbackslash{}ntido um só motivo de queixa contra ele. A dedicação era digna de melhor causa.\textbackslash{}nExteriormente via-se em Luís Soares um monge; raspando-se um pouco achava-se o\textbackslash{}ndiabo.\textbackslash{}n\textbackslash{}nOra, o diabo viu de longe uma\textbackslash{}nconquista{\ldots}\textbackslash{}n\textbackslash{}nCAPÍTULO III\textbackslash{}n\textbackslash{}nA prima Adelaide tinha vinte e quatro\textbackslash{}nanos, e a sua beleza, no pleno desenvolvimento da sua mocidade, tinha em si o\textbackslash{}ncondão de fazer morrer de amores. Era alta e bem proporcionada; tinha uma\textbackslash{}ncabeça modelada pelo tipo antigo; a testa era espaçosa e alta, os olhos\textbackslash{}nrasgados e negros, o nariz levemente aquilino. Quem a contemplava durante\textbackslash{}nalguns momentos sentia que ela tinha todas as energias, a das paixões e a da\textbackslash{}nvontade.\textbackslash{}n\textbackslash{}nHá de lembrar-se o leitor do frio\textbackslash{}ncumprimento trocado entre Adelaide e seu primo; também se há de lembrar que\textbackslash{}nSoares disse ao amigo Pires ter sido amado por sua prima. Ligam-se estas duas\textbackslash{}ncoisas. A frieza de Adelaide resultava de uma lembrança que era dolorosa para a\textbackslash{}nmoça; Adelaide amara o primo, não com um simples amor de primos, que em geral\textbackslash{}nresulta da convivência e não de uma súbita atração. Amara-o com todo o vigor e\textbackslash{}ncalor de sua alma; mas já então o rapaz iniciava os seus passos em outras\textbackslash{}nregiões e ficou indiferente aos afetos da moça. Um amigo que sabia do segredo\textbackslash{}nperguntou-lhe um dia por que razão não se casava com Adelaide, ao que o rapaz\textbackslash{}nrespondeu friamente:\textbackslash{}n\textbackslash{}n— Quem tem a minha fortuna não se casa;\textbackslash{}nmas se se casa é sempre com quem tenha mais. Os bens de Adelaide são a quinta\textbackslash{}nparte dos meus; para ela é negócio da China; para mim é um mau negócio.\textbackslash{}n\textbackslash{}nO amigo que ouvira esta resposta não\textbackslash{}ndeixou de dar uma prova da sua afeição ao rapaz indo contar tudo à moça. O\textbackslash{}ngolpe foi tremendo, não tanto pela certeza que lhe dava de não ser amada, como\textbackslash{}npela circunstância de nem ao menos ficar-lhe o direito de estima. A confissão\textbackslash{}nde Soares era um corpo de delito. O confidente oficioso esperava talvez colher\textbackslash{}nos despojos da derrota; mas Adelaide, tão depressa ouviu a delação como\textbackslash{}ndesprezou o delator.\textbackslash{}n\textbackslash{}nO incidente não passou disto.\textbackslash{}n\textbackslash{}nQuando Soares voltou à casa do tio, a\textbackslash{}nmoça achou-se em dolorosa situação; era obrigada a conviver com um homem ao\textbackslash{}nqual nem podia dar apreço. Pela sua parte, o rapaz também se achava acanhado,\textbackslash{}nnão porque lhe doessem as palavras que dissera um dia, mas por causa do tio,\textbackslash{}nque ignorava tudo. Não ignorava; o moço é que o supunha. O major soube da\textbackslash{}npaixão de Adelaide e soube também da repulsa que tivera no coração do rapaz.\textbackslash{}nTalvez não soubesse das palavras textuais repetidas à moça pelo amigo de\textbackslash{}nSoares; mas se não conhecia o texto, conhecia o espírito; sabia que, pelo\textbackslash{}nmotivo de ser amado, o rapaz entrara a aborrecer a prima, e que esta, vendo-se\textbackslash{}nrepelida, entrara a aborrecer o rapaz. O major supôs até durante algum tempo\textbackslash{}nque a ausência de Soares tinha por motivo a presença da moça em casa.\textbackslash{}n\textbackslash{}nAdelaide era filha de um irmão do major,\textbackslash{}nhomem muito rico e igualmente excêntrico, que morrera havia dez anos deixando a\textbackslash{}nmoça entregue aos cuidados do irmão. Como o pai de Adelaide fizera muitas\textbackslash{}nviagens, parece que gastou nelas a maior parte da sua fortuna. Quando morreu\textbackslash{}napenas coube a Adelaide, filha única, cerca de trinta contos, que o tio\textbackslash{}nconservou intactos para serem o dote da pupila.\textbackslash{}n\textbackslash{}nSoares houve-se como pôde na singular\textbackslash{}nsituação em que se achava. Não conversava com a prima; apenas trocava com ela\textbackslash{}nas palavras estritamente necessárias para não chamar a atenção do tio. A moça\textbackslash{}nfazia o mesmo.\textbackslash{}n\textbackslash{}nMas quem pode ter mão ao coração? A prima\textbackslash{}nde Luís Soares sentiu que pouco a pouco lhe ia renascendo o antigo afeto.\textbackslash{}nProcurou combatê-lo sinceramente; mas não se impede o crescimento de uma planta\textbackslash{}nsenão arrancando-lhe as raízes. As raízes existiam ainda. Apesar dos esforços\textbackslash{}nda moça o amor veio pouco a pouco invadindo o lugar do ódio, e se até então o\textbackslash{}nsuplício era grande, agora era enorme. Travara-se uma luta entre o orgulho e o\textbackslash{}namor. A moça sofreu consigo; não articulou uma palavra.\textbackslash{}n\textbackslash{}nLuís Soares reparava que quando os seus\textbackslash{}ndedos tocavam os da prima, esta experimentava uma grande emoção: corava e\textbackslash{}nempalidecia. Era um grande navegador aquele rapaz nos mares do amor:\textbackslash{}nconhecia-lhe a calma e a tempestade. Convenceu-se de que a prima o amava outra\textbackslash{}nvez. A descoberta não o alegrou; pelo contrário, foi-lhe motivo de grande\textbackslash{}nirritação. Receava que o tio, descobrindo o sentimento da sobrinha, propusesse\textbackslash{}no casamento ao rapaz; e recusá-lo não seria comprometer no futuro a esperada\textbackslash{}nherança? A herança sem o casamento era o ideal do moço. 'Dar-me asas,\textbackslash{}npensava ele, atando-me os pés, é o mesmo que condenar-me à prisão. É o destino\textbackslash{}ndo papagaio doméstico; não aspiro a tê-lo.'\textbackslash{}n\textbackslash{}nRealizaram-se as previsões do rapaz. O\textbackslash{}nmajor descobriu a causa da tristeza da moça e resolveu pôr termo àquela\textbackslash{}nsituação propondo ao sobrinho o casamento.\textbackslash{}n\textbackslash{}nSoares não podia recusar abertamente sem\textbackslash{}ncomprometer o edifício da sua fortuna.\textbackslash{}n\textbackslash{}n— Este casamento, disse-lhe o tio, é complemento\textbackslash{}nda minha felicidade. De um só lance reúno duas pessoas que tanto estimo, e\textbackslash{}nmorro tranqüilo sem levar nenhum pesar para o outro mundo. Estou que aceitarás.\textbackslash{}n\textbackslash{}n— Aceito, meu tio; mas observo que o\textbackslash{}ncasamento assenta no amor, e eu não amo minha prima.\textbackslash{}n\textbackslash{}n— Bem; hás de amá-la; casa-te primeiro{\ldots}\textbackslash{}n\textbackslash{}n— Não desejo expô-la a uma desilusão.\textbackslash{}n\textbackslash{}n— Qual desilusão! disse o major sorrindo.\textbackslash{}nGosto de ouvir-te falar essa linguagem poética, mas casamento não é poesia. É\textbackslash{}nverdade que é bom que duas pessoas antes de se casarem se tenham já alguma\textbackslash{}nestima mútua. Isso creio que tens. Lá fogos ardentes, meu rico sobrinho, são\textbackslash{}ncoisas que ficam bem em verso, e mesmo em prosa; mas na vida, que não é prosa\textbackslash{}nnem verso, o casamento apenas exige certa conformidade de gênio, de educação e\textbackslash{}nde estima.\textbackslash{}n\textbackslash{}n— Meu tio sabe que eu não me recuso a uma\textbackslash{}nordem sua.\textbackslash{}n\textbackslash{}n— Ordem, não! Não te ordeno, proponho.\textbackslash{}nDizes que não amas tua prima; pois bem, faze por isso, e daqui a algum tempo\textbackslash{}ncasem-se que me darão gosto. O que eu quero é que seja cedo, porque não estou\textbackslash{}nlonge de dar à casca.\textbackslash{}n\textbackslash{}nO rapaz disse que sim. Adiou a\textbackslash{}ndificuldade não podendo resolvê-la. O major ficou satisfeito com o arranjo e\textbackslash{}nconsolou a sobrinha com a promessa de que podia casar-se um dia com o primo.\textbackslash{}nEra a primeira vez que o velho tocava em semelhante assunto, e Adelaide não\textbackslash{}ndissimulou o seu espanto, espanto que lisonjeou profundamente a perspicácia do\textbackslash{}nmajor.\textbackslash{}n\textbackslash{}n— Ah! tu pensas, disse ele, que eu por\textbackslash{}nser velho já perdi os olhos do coração? Vejo tudo, Adelaide; vejo aquilo mesmo\textbackslash{}nque se quer esconder.\textbackslash{}n\textbackslash{}nA moça não pôde reter algumas lágrimas, e\textbackslash{}ncomo o velho a consolasse dando-lhe esperanças, ela respondeu abanando a\textbackslash{}ncabeça:\textbackslash{}n\textbackslash{}n— Esperanças, nenhuma!\textbackslash{}n\textbackslash{}n— Descansa em mim! disse o major.\textbackslash{}n\textbackslash{}nConquanto a dedicação do tio fosse toda espontânea\textbackslash{}ne filha do amor que votava à sobrinha, esta compreendeu que semelhante\textbackslash{}nintervenção podia fazer supor ao primo que ela esmolava os afetos do seu\textbackslash{}ncoração.\textbackslash{}n\textbackslash{}nAqui falou o orgulho da mulher, que\textbackslash{}npreferia o sofrimento à humilhação. Quando ela expôs estas objeções ao tio, o\textbackslash{}nmajor sorriu-se afavelmente e procurou acalmar a suscetibilidade da moça.\textbackslash{}n\textbackslash{}nPassaram-se alguns dias sem mais\textbackslash{}nincidente; o rapaz estava no gozo da dilação que lhe dera o tio. Adelaide\textbackslash{}nreadquiriu o seu ar frio e indiferente. Soares compreendia o motivo, e àquela\textbackslash{}nmanifestação do orgulho respondia com um sorriso. Duas vezes notou Adelaide\textbackslash{}nessa expressão de desdém da parte do primo. Que mais precisava para reconhecer\textbackslash{}nque o rapaz sentia por ela a mesma indiferença de outro tempo! Acrescia que\textbackslash{}nsempre que os dois se encontravam sós, Soares era o primeiro que se afastava\textbackslash{}ndela. Era o mesmo homem.\textbackslash{}n\textbackslash{}n'Não me ama, não me amará\textbackslash{}nnunca!' dizia a moça consigo.\textbackslash{}n\textbackslash{}nCAPÍTULO IV\textbackslash{}n\textbackslash{}nUm dia de manhã o major Vilela recebeu a\textbackslash{}nseguinte carta:\textbackslash{}n\textbackslash{}nMeu valente major.\textbackslash{}n\textbackslash{}nCheguei da Bahia hoje mesmo, e lá irei de\textbackslash{}ntarde para ver-te e abraçar-te. Prepara um jantar. Creio que me não hás de\textbackslash{}nreceber como qualquer indivíduo. Não esqueças o vatapá.\textbackslash{}n\textbackslash{}nTeu amigo,\textbackslash{}nAnselmo.\textbackslash{}n\textbackslash{}n— Bravo! disse o major. Temos cá o Anselmo;\textbackslash{}nprima Antônia, mande fazer um bom vatapá.\textbackslash{}n\textbackslash{}nO Anselmo que chegara da Bahia chamava-se\textbackslash{}nAnselmo Barroso de Vasconcelos. Era um fazendeiro rico, e veterano da\textbackslash{}nindependência. Com os seus setenta e oito anos ainda se mostrava rijo e capaz\textbackslash{}nde grandes feitos. Tinha sido íntimo amigo do pai de Adelaide, que o apresentou\textbackslash{}nao major, vindo a ficar amigo deste depois que o outro morrera. Anselmo\textbackslash{}nacompanhou o amigo até os seus últimos instantes; e chorou a perda como se fora\textbackslash{}nseu próprio irmão. As lágrimas cimentaram a amizade entre ele e o major.\textbackslash{}n\textbackslash{}nDe tarde apareceu Anselmo galhofeiro e\textbackslash{}nvivo como se começasse para ele uma nova mocidade. Abraçou a todos; deu um\textbackslash{}nbeijo em Adelaide, a quem felicitou pelo desenvolvimento das suas graças.\textbackslash{}n\textbackslash{}n— Não se ria de mim, disse-lhe ele, eu\textbackslash{}nfui o maior amigo de seu pai. Pobre amigo! morreu nos meus braços.\textbackslash{}n\textbackslash{}nSoares, que sofria com a monotonia da\textbackslash{}nvida que levava em casa do tio, alegrou-se com a presença do galhofeiro ancião,\textbackslash{}nque era um verdadeiro fogo de artifício. Anselmo é que pareceu não simpatizar\textbackslash{}ncom o sobrinho do major. Quando o major ouviu isto, disse:\textbackslash{}n\textbackslash{}n— Sinto muito, porque Soares é um rapaz\textbackslash{}nsério.\textbackslash{}n\textbackslash{}n— Creio que é sério demais. Rapaz que não\textbackslash{}nri{\ldots}\textbackslash{}n\textbackslash{}nNão sei que incidente interrompeu a frase\textbackslash{}ndo fazendeiro.\textbackslash{}n\textbackslash{}nDepois do jantar Anselmo disse ao major:\textbackslash{}n\textbackslash{}n— Quantos são amanhã?\textbackslash{}n\textbackslash{}n— Quinze.\textbackslash{}n\textbackslash{}n— De que mês?\textbackslash{}n\textbackslash{}n— É boa! de dezembro.\textbackslash{}n\textbackslash{}n— Bem; amanhã 15 de dezembro preciso ter\textbackslash{}numa conferência contigo e os teus parentes. Se o vapor se demora um dia em\textbackslash{}ncaminho pregava-me uma boa peça.\textbackslash{}n\textbackslash{}nNo dia seguinte verificou-se a\textbackslash{}nconferência pedida por Anselmo. Estavam presentes o major, Soares, Adelaide e\textbackslash{}nD. Antônia, únicos parentes do finado.\textbackslash{}n\textbackslash{}n— Faz hoje dez anos que faleceu o pai\textbackslash{}ndesta menina, disse Anselmo apontando para Adelaide. Como sabem, o Dr. Bento\textbackslash{}nVarela foi o meu melhor amigo, e eu tenho consciência de haver correspondido à\textbackslash{}nsua afeição até aos últimos instantes. Sabem que ele era um gênio excêntrico;\textbackslash{}ntoda a sua vida foi uma grande originalidade. Ideava vinte projetos, qual mais\textbackslash{}ngrandioso, qual mais impossível, sem chegar ao cabo de nenhum, porque o seu\textbackslash{}nespírito criador tão depressa compunha uma coisa como entrava a planear outra.\textbackslash{}n\textbackslash{}n— É verdade, interrompeu o major.\textbackslash{}n\textbackslash{}n— O Bento morreu nos meus braços, e como\textbackslash{}nderradeira prova da sua amizade confiou-me um papel com a declaração de que eu\textbackslash{}nsó o abrisse em presença dos seus parentes dez anos depois de sua morte. No\textbackslash{}ncaso de eu morrer os meus herdeiros assumiriam essa obrigação; em falta deles,\textbackslash{}no major, a Sra. D. Adelaide, enfim qualquer pessoa que por laço de sangue\textbackslash{}nestivesse ligada a ele. Enfim, se ninguém houvesse na classe mencionada, ficava\textbackslash{}nincumbido um tabelião. Tudo isto havia eu declarado em testamento, que vou\textbackslash{}nreformar. O papel a que me refiro, tenho aqui no bolso.\textbackslash{}n\textbackslash{}nHouve um movimento de curiosidade.\textbackslash{}n\textbackslash{}nAnselmo tirou do bolso uma carta fechada\textbackslash{}ncom lacre preto.\textbackslash{}n\textbackslash{}n— É este, disse ele. Está intacto. Não\textbackslash{}nconheço o texto; mas posso mais ou menos saber o que está dentro por\textbackslash{}ncircunstâncias que vou referir.\textbackslash{}n\textbackslash{}nRedobrou a atenção geral.\textbackslash{}n\textbackslash{}n— Antes de morrer, continuou Anselmo, o\textbackslash{}nmeu querido amigo entregou-me uma parte da sua fortuna, quero dizer a maior\textbackslash{}nparte, porque a menina recebeu apenas trinta contos. Eu recebi dele trezentos\textbackslash{}ncontos, que guardei até hoje intactos, e que devo restituir segundo as\textbackslash{}nindicações desta carta.\textbackslash{}n\textbackslash{}nA um movimento de espanto em todos\textbackslash{}nseguiu-se um movimento de ansiedade. Qual seria a vontade misteriosa do pai de\textbackslash{}nAdelaide? D. Antônia lembrou-se que em rapariga fora namorada do defunto, e por\textbackslash{}num momento lisonjeou-se com a idéia de que o velho maníaco se houvesse lembrado\textbackslash{}ndela às portas da morte.\textbackslash{}n\textbackslash{}n— Nisto reconheço eu o mano Bento, disse\textbackslash{}no major tomando uma pitada; era o homem dos mistérios, das surpresas e das\textbackslash{}nidéias extravagantes, seja dito sem agravo aos seus pecados, se é que os\textbackslash{}nteve{\ldots}\textbackslash{}n\textbackslash{}nAnselmo tinha aberto a carta. Todos\textbackslash{}nprestaram ouvidos. O veterano leu o seguinte:\textbackslash{}n\textbackslash{}nMeu bom e estimadíssimo Anselmo.\textbackslash{}n\textbackslash{}nQuero que me prestes o último favor. Tens\textbackslash{}ncontigo a maior parte da minha fortuna, e eu diria a melhor se tivesse de\textbackslash{}naludir à minha querida filha Adelaide. Guarda esses trezentos contos até daqui\textbackslash{}na dez anos, e ao terminar o prazo, lê esta carta diante dos meus parentes.\textbackslash{}n\textbackslash{}nSe nessa época a minha filha Adelaide for\textbackslash{}nviva e casada entrega-lhe a fortuna. Se não estiver casada, entrega-lha também,\textbackslash{}nmas com uma condição: é que se case com o sobrinho Luís Soares, filho de minha\textbackslash{}nirmã Luísa; quero-lhe muito, e apesar de ser rico, desejo que entre na posse da\textbackslash{}nfortuna com minha filha. No caso em que esta se recuse a esta condição, fica tu\textbackslash{}ncom a fortuna toda.\textbackslash{}n\textbackslash{}nQuando Anselmo acabou de ler esta carta\textbackslash{}nseguiu-se um silêncio de surpresa geral, de que partilhava o próprio veterano,\textbackslash{}nalheio até então ao conteúdo da carta.\textbackslash{}n\textbackslash{}nSoares tinha os olhos em Adelaide; esta\textbackslash{}ntinha-os no chão.\textbackslash{}n\textbackslash{}nComo o silêncio se prolongasse, Anselmo\textbackslash{}nresolveu rompê-lo.\textbackslash{}n\textbackslash{}n— Ignorava, como todos, disse ele, o que\textbackslash{}nesta carta contém; felizmente chega ela a tempo de se realizar a última vontade\textbackslash{}ndo meu finado amigo.\textbackslash{}n\textbackslash{}n— Sem dúvida nenhuma, disse o major.\textbackslash{}n\textbackslash{}nOuvindo isto, a moça levantou\textbackslash{}ninsensivelmente os olhos para o primo, e os dela encontraram-se com os dele. Os\textbackslash{}ndele transbordavam de contentamento e ternura; a moça fitou-os durante alguns\textbackslash{}ninstantes. Um sorriso, já não zombeteiro, passou pelos lábios do rapaz. A moça\textbackslash{}nsorriu com tamanho desdém às zumbaias de um cortesão.\textbackslash{}n\textbackslash{}nAnselmo levantou-se.\textbackslash{}n\textbackslash{}n— Agora que estão cientes disto, disse\textbackslash{}nele aos dois primos, espero que resolvam, e como o resultado não pode ser duvidoso,\textbackslash{}ndesde já os felicito. Entretanto, hão de dar-me licença, que tenho de ir a\textbackslash{}noutras partes.\textbackslash{}n\textbackslash{}nCom a saída de Anselmo dispersara-se a\textbackslash{}nreunião. Adelaide foi para o seu quarto com a velha parenta. O tio e o sobrinho\textbackslash{}nficaram na sala.\textbackslash{}n\textbackslash{}n— Luís, disse o primeiro, és o homem mais\textbackslash{}nfeliz do mundo.\textbackslash{}n\textbackslash{}n— Parece-lhe, meu tio? disse o moço\textbackslash{}nprocurando disfarçar a sua alegria.\textbackslash{}n\textbackslash{}n— És. Tens uma moça que te ama\textbackslash{}nloucamente. De repente cai-lhe nas mãos uma fortuna inesperada; e essa fortuna\textbackslash{}nsó pode havê-la com a condição de se casar contigo. Até os mortos trabalham a\textbackslash{}nteu favor.\textbackslash{}n\textbackslash{}n— Afirmo-lhe, meu tio, que a fortuna não\textbackslash{}npesa nada nestes casos, e se eu assentar em casar com a prima será por outro\textbackslash{}nmotivo.\textbackslash{}n\textbackslash{}n— Bem sei que a riqueza não é essencial; não\textbackslash{}né. Mas enfim vale alguma coisa. É melhor ter trezentos contos que trinta;\textbackslash{}nsempre é mais uma cifra. Contudo não te aconselho que te cases com ela se não\textbackslash{}ntiveres alguma afeição. Nota que eu não me refiro a essas paixões de que me\textbackslash{}nfalaste. Casar mal, apesar da riqueza, é sempre casar mal.\textbackslash{}n\textbackslash{}n— Estou convencido disto, meu tio. Por\textbackslash{}nisso ainda não dei a minha resposta, nem dou por ora. Se eu vier a afeiçoar-me\textbackslash{}nà prima estou pronto a entrar na posse dessa inesperada riqueza.\textbackslash{}n\textbackslash{}nComo o leitor terá adivinhado, a resolução\textbackslash{}ndo casamento estava assentada no espírito de Soares. Em vez de esperar a morte\textbackslash{}ndo tio, parecia-lhe melhor entrar desde logo na posse de um excelente pecúlio,\textbackslash{}no que se lhe afigurava tanto mais fácil, quanto que era a voz do túmulo que o\textbackslash{}nimpunha.\textbackslash{}n\textbackslash{}nSoares contava também com a profunda\textbackslash{}nveneração de Adelaide por seu pai. Isto, ligado ao amor que a rapariga sentia\textbackslash{}npor ele, devia produzir o desejado efeito.\textbackslash{}n\textbackslash{}nNessa noite o rapaz dormiu pouco. Sonhou\textbackslash{}ncom o Oriente. Pintou-lhe a imaginação um harém recendente das melhores\textbackslash{}nessências da Arábia, forrado o chão com tapetes da Pérsia; sobre moles divãs\textbackslash{}nostentavam-se as mais perfeitas belezas do mundo. Uma circassiana dançava no\textbackslash{}nmeio do salão ao som de um pandeiro de marfim. Mas um furioso eunuco,\textbackslash{}nprecipitando-se na sala com o iatagã desembainhado, enterrou-o todo no peito de\textbackslash{}nSoares, que acordou com o pesadelo, e não pôde mais conciliar o sono.\textbackslash{}n\textbackslash{}nLevantou-se mais cedo e foi passear até\textbackslash{}nchegar a hora do almoço e da repartição.\textbackslash{}n\textbackslash{}nCAPÍTULO V\textbackslash{}n\textbackslash{}nO plano de Luís Soares estava feito.\textbackslash{}n\textbackslash{}nTratava-se de abater as armas pouco a\textbackslash{}npouco, simulando-se vencido diante da influência de Adelaide. A circunstância\textbackslash{}nda riqueza tornava necessária toda a discrição. A transição devia ser lenta.\textbackslash{}nCumpria ser diplomata.\textbackslash{}n\textbackslash{}nOs leitores terão visto que, apesar de\textbackslash{}ncerta argúcia da parte de Soares, não tinha ele a perfeita compreensão das\textbackslash{}ncoisas, e por outro lado o seu caráter era indeciso e vário.\textbackslash{}n\textbackslash{}nHesitara em casar com Adelaide quando o\textbackslash{}ntio lhe falou nisso, quando era certo que viria a obter mais tarde a fortuna do\textbackslash{}nmajor. Dizia então que não tinha vocação de papagaio. A situação agora era a\textbackslash{}nmesma; aceitava uma fortuna mediante uma prisão. É verdade que se esta\textbackslash{}nresolução era contrária à primeira, podia ter por causa o cansaço que lhe ia\textbackslash{}nproduzindo a vida que levava. Além de que, desta vez, a riqueza não se fazia\textbackslash{}nesperar; era entregue logo depois do consórcio.\textbackslash{}n\textbackslash{}n'Trezentos contos, pensava o rapaz,\textbackslash{}né quanto basta para eu ser mais do que fui. O que não hão de dizer os\textbackslash{}noutros!'\textbackslash{}n\textbackslash{}nAntevendo uma felicidade que era certa\textbackslash{}npara ele, Soares começou o assédio da praça, aliás praça rendida.\textbackslash{}n\textbackslash{}nJá o rapaz procurava os olhos da prima,\textbackslash{}njá os encontrava, já lhes pedia aquilo que recusara até então, o amor da moça.\textbackslash{}nQuando, à mesa, as suas mãos se encontravam, Soares tinha o cuidado de demorar\textbackslash{}no contato, e se a moça retirava a sua mão, o rapaz nem por isso desanimava.\textbackslash{}nQuando se encontrava a sós com ela, não fugia como outrora, antes lhe dirigia\textbackslash{}nalguma palavra, a que Adelaide respondia com fria polidez.\textbackslash{}n\textbackslash{}n'Quer vender o peixe caro',\textbackslash{}npensava Soares.\textbackslash{}n\textbackslash{}nUma vez atreveu-se a mais. Adelaide\textbackslash{}ntocava piano quando ele entrou sem que ela o visse. Quando a moça acabou,\textbackslash{}nSoares estava por trás dela.\textbackslash{}n\textbackslash{}n— Que lindo! disse o rapaz; deixe-me\textbackslash{}nbeijar-lhe essas mãos inspiradas.\textbackslash{}n\textbackslash{}nA moça olhou séria para ele, pegou no\textbackslash{}nlenço que pusera sobre o piano, e saiu sem dizer palavra.\textbackslash{}n\textbackslash{}nEsta cena mostrou a Soares toda a\textbackslash{}ndificuldade da empresa; mas o rapaz confiava em si, não porque se reconhecesse\textbackslash{}ncapaz de grandes energias, mas por espécie de esperança na sua boa estrela.\textbackslash{}n\textbackslash{}n— É difícil subir a corrente, disse ele,\textbackslash{}nmas sobe-se. Não se fazem Alexandres na conquista de praças desarmadas.\textbackslash{}n\textbackslash{}nContudo, as desilusões iam-se sucedendo,\textbackslash{}ne o rapaz, se o não alentasse a idéia da riqueza, teria abatido as armas.\textbackslash{}n\textbackslash{}nUm dia lembrou-se de escrever-lhe uma\textbackslash{}ncarta. Lembrou-se de que era difícil expor-lhe de viva voz tudo quanto sentia;\textbackslash{}nmas que uma carta, por muito ódio que ela lhe tivesse, sempre seria lida.\textbackslash{}n\textbackslash{}nAdelaide devolveu a carta pelo moleque da\textbackslash{}ncasa que lha havia entregue.\textbackslash{}n\textbackslash{}nA segunda carta teve a mesma sorte.\textbackslash{}nQuando mandou a terceira, o moleque não a quis receber.\textbackslash{}n\textbackslash{}nLuís Soares teve um instante de\textbackslash{}ndesengano. Indiferente à moça, já começava a odiá-la; se casasse com ela era\textbackslash{}nprovável que a tratasse como inimigo mortal.\textbackslash{}n\textbackslash{}nA situação tornava-se ridícula para ele;\textbackslash{}nou antes, já o era há muito, mas Soares só então o compreendeu. Para escapar ao\textbackslash{}nridículo, resolveu dar um golpe final, mas grande. Aproveitou a primeira\textbackslash{}nocasião que pôde, e fez uma declaração positiva à moça, cheia de súplicas, de\textbackslash{}nsuspiros, talvez de lágrimas. Confessou os seus erros; reconheceu que não a\textbackslash{}nhavia compreendido; mas arrependera-se e confessava tudo. A influência dela\textbackslash{}nacabara por abatê-lo.\textbackslash{}n\textbackslash{}n— Abatê-lo! disse ela; não compreendo. A\textbackslash{}nque influência alude?\textbackslash{}n\textbackslash{}n— Bem sabe; à influência da sua beleza,\textbackslash{}ndo seu amor{\ldots} Não suponha que lhe estou mentindo. Sinto-me hoje tão apaixonado\textbackslash{}nque era capaz de cometer um crime!\textbackslash{}n\textbackslash{}n— Um crime?\textbackslash{}n\textbackslash{}n— Não é crime o suicídio? De que me\textbackslash{}nserviria a vida sem o seu amor? Vamos, fale!\textbackslash{}n\textbackslash{}nA moça olhou para ele durante alguns\textbackslash{}ninstantes sem dizer palavra.\textbackslash{}n\textbackslash{}nO rapaz ajoelhou-se.\textbackslash{}n\textbackslash{}n— Ou seja a morte, ou seja a felicidade,\textbackslash{}ndisse ele, quero recebê-la de joelhos.\textbackslash{}n\textbackslash{}nAdelaide sorriu e soltou lentamente estas\textbackslash{}npalavras:\textbackslash{}n\textbackslash{}n— Trezentos contos! É muito dinheiro para\textbackslash{}ncomprar um miserável.\textbackslash{}n\textbackslash{}nE deu-lhe as costas.\textbackslash{}n\textbackslash{}nSoares ficou petrificado. Durante alguns\textbackslash{}nminutos conservou-se na mesma posição, com os olhos fitos na moça que se\textbackslash{}nafastava lentamente. O rapaz dobrava-se ao peso da humilhação. Não previra tão\textbackslash{}ncruel desforra da parte de Adelaide. Nem uma palavra de ódio, nem um indício de\textbackslash{}nraiva; apenas um calmo desdém, um desprezo tranqüilo e soberano. Soares sofrera\textbackslash{}nmuito quando perdeu a fortuna; mas agora que o seu orgulho foi humilhado, a sua\textbackslash{}ndor foi infinitamente maior.\textbackslash{}n\textbackslash{}nPobre rapaz!\textbackslash{}n\textbackslash{}nA moça foi para dentro. Parece que\textbackslash{}ncontava com aquela cena; porque entrando em casa, foi logo procurar o tio, e\textbackslash{}ndeclarou-lhe que, apesar de quanto venerava a memória do pai, não podia\textbackslash{}nobedecer-lhe, e desistia do casamento.\textbackslash{}n\textbackslash{}n— Mas não o amas tu? perguntou-lhe o\textbackslash{}nmajor.\textbackslash{}n\textbackslash{}n— Amei-o.\textbackslash{}n\textbackslash{}n— Amas a outro?\textbackslash{}n\textbackslash{}n— Não.\textbackslash{}n\textbackslash{}n— Então explica-te.\textbackslash{}n\textbackslash{}nAdelaide expôs francamente o procedimento\textbackslash{}nde Soares desde que ali entrara, a mudança que fizera, a sua ambição, a cena do\textbackslash{}njardim. O major ouviu atentamente a moça, procurou desculpar o sobrinho, mas no\textbackslash{}nfundo ele acreditava que Soares era um mau caráter.\textbackslash{}n\textbackslash{}nEste, depois que pôde refrear a sua cólera,\textbackslash{}nentrou em casa e foi despedir-se do tio até o dia seguinte.\textbackslash{}n\textbackslash{}nPretextou que tinha um negócio urgente.\textbackslash{}n\textbackslash{}nCAPÍTULO VI\textbackslash{}n\textbackslash{}nAdelaide contou miudamente ao amigo de\textbackslash{}nseu pai os sucessos que a obrigavam a não preencher a condição da carta póstuma\textbackslash{}nconfiada a Anselmo. Em conseqüência desta recusa, a fortuna devia ficar com\textbackslash{}nAnselmo; a moça contentava-se com o que tinha.\textbackslash{}n\textbackslash{}nNão se deu Anselmo por vencido, e antes\textbackslash{}nde aceitar a recusa foi ver se sondava o espírito de Luís Soares.\textbackslash{}n\textbackslash{}nQuando o sobrinho do major viu entrar por\textbackslash{}ncasa o fazendeiro suspeitou que alguma coisa houvesse a respeito do casamento.\textbackslash{}nAnselmo era perspicaz; de modo que, apesar da aparência de vítima com que\textbackslash{}nSoares lhe aparecera, compreendeu ele que Adelaide tinha razão.\textbackslash{}n\textbackslash{}nAssim pois tudo estava acabado. Anselmo\textbackslash{}ndispôs-se a partir para a Bahia, e assim o declarou à família do major.\textbackslash{}n\textbackslash{}nNas vésperas de partir achavam-se todos\textbackslash{}njuntos na sala de visitas, quando Anselmo soltou estas palavras:\textbackslash{}n\textbackslash{}n— Major, está ficando melhor e forte; eu\textbackslash{}ncreio que uma viagem à Europa lhe fará bem. Esta moça também gostará de ver a\textbackslash{}nEuropa, e creio que a Sra. D. Antônia, apesar da idade, lá quererá ir. Pela\textbackslash{}nminha parte sacrifico a Bahia e vou também. Aprovam o conselho?\textbackslash{}n\textbackslash{}n— Homem, disse o major, é preciso\textbackslash{}npensar{\ldots}\textbackslash{}n\textbackslash{}n— Qual pensar! Se pensarem não\textbackslash{}nembarcarão. Que diz a menina?\textbackslash{}n\textbackslash{}n— Eu obedeço ao tio, respondeu Adelaide.\textbackslash{}n\textbackslash{}n— Além de que, disse Anselmo, agora que\textbackslash{}nD. Adelaide está de posse de uma grande fortuna, há de querer apreciar o que há\textbackslash{}nde bonito nos países estrangeiros a fim de poder melhor avaliar o que há no\textbackslash{}nnosso{\ldots}\textbackslash{}n\textbackslash{}n— Sim, disse o major; mas você fala de\textbackslash{}ngrande fortuna{\ldots}\textbackslash{}n\textbackslash{}n— Trezentos contos.\textbackslash{}n\textbackslash{}n— São seus.\textbackslash{}n\textbackslash{}n— Meus! Então sou algum ratoneiro? Que me\textbackslash{}nimporta a mim a fantasia de um generoso amigo? O dinheiro é desta menina, sua\textbackslash{}nlegítima herdeira, e não meu, que aliás tenho bastante.\textbackslash{}n\textbackslash{}n— Isso é bonito, Anselmo!\textbackslash{}n\textbackslash{}n— Mas o que não seria se não fosse isto?\textbackslash{}n\textbackslash{}nA viagem à Europa ficou assentada.\textbackslash{}n\textbackslash{}nLuís Soares ouviu a conversa toda sem\textbackslash{}ndizer palavra; mas a idéia de que talvez pudesse ir com o tio sorriu-lhe ao\textbackslash{}nespírito. No dia seguinte teve um desengano cruel. Disse-lhe o major que, antes\textbackslash{}nde partir, o deixaria recomendado ao ministro.\textbackslash{}n\textbackslash{}nSoares procurou ainda ver se alcançava\textbackslash{}nseguir com a família. Era simples cobiça na fortuna do tio, desejo de ver novas\textbackslash{}nterras, ou impulso de vingança contra a prima? Era tudo isso, talvez.\textbackslash{}n\textbackslash{}nÀ última hora foi-se a derradeira\textbackslash{}nesperança. A família partiu sem ele.\textbackslash{}n\textbackslash{}nAbandonado, pobre, tendo por única\textbackslash{}nperspectiva o trabalho diário, sem esperanças no futuro, e além do mais,\textbackslash{}nhumilhado e ferido em seu amor-próprio, Soares tomou a triste resolução dos\textbackslash{}ncovardes.\textbackslash{}n\textbackslash{}nUm dia de noite o criado ouviu no quarto\textbackslash{}ndele um tiro; correu, achou um cadáver.\textbackslash{}n\textbackslash{}nPires soube na rua da notícia, e correu à\textbackslash{}ncasa de Vitória, que encontrou no toucador.\textbackslash{}n\textbackslash{}n— Sabes de uma coisa? perguntou ele.\textbackslash{}n\textbackslash{}n— Não. Que é?\textbackslash{}n\textbackslash{}n— O Soares matou-se.\textbackslash{}n\textbackslash{}n— Quando?\textbackslash{}n\textbackslash{}n— Neste momento.\textbackslash{}n\textbackslash{}n— Coitado! É sério?\textbackslash{}n\textbackslash{}n— É sério. Vais sair?\textbackslash{}n\textbackslash{}n— Vou ao Alcazar.\textbackslash{}n\textbackslash{}n— Canta-se hoje Barbe-Bleue, não\textbackslash{}né?\textbackslash{}n\textbackslash{}n— É.\textbackslash{}n\textbackslash{}n— Pois eu também vou.\textbackslash{}n\textbackslash{}nE entrou a cantarolar a canção de Barbe-Bleue.\textbackslash{}n\textbackslash{}nLuís Soares não teve outra oração fúnebre\textbackslash{}ndos seus amigos mais íntimos.\textbackslash{}n\textbackslash{}nA MULHER DE PRETO\textbackslash{}n\textbackslash{}nÍNDICE\textbackslash{}n\textbackslash{}nCapítulo Primeiro\textbackslash{}n\textbackslash{}nCapítulo II\textbackslash{}n\textbackslash{}nCapítulo iii\textbackslash{}n\textbackslash{}nCapítulo iv\textbackslash{}n\textbackslash{}nCapítulo v\textbackslash{}n\textbackslash{}nCapítulo vI\textbackslash{}n\textbackslash{}nCapítulo vII\textbackslash{}n\textbackslash{}nCapítulo VIII\textbackslash{}n\textbackslash{}nCapítulo IX\textbackslash{}n\textbackslash{}nCapítulo X\textbackslash{}n\textbackslash{}nCapítulo XI\textbackslash{}n\textbackslash{}nCAPÍTULO PRIMEIRO\textbackslash{}n\textbackslash{}nA primeira vez que o Dr. Estêvão Soares\textbackslash{}nfalou ao deputado Meneses foi no Teatro Lírico no tempo da memorável luta entre\textbackslash{}nlagruístas e chartonistas. Um amigo comum os apresentou ao outro.\textbackslash{}nNo fim da noite separaram-se oferecendo cada um deles os seus serviços e\textbackslash{}ntrocando os respectivos cartões de visita.\textbackslash{}n\textbackslash{}nSó dois meses depois encontraram-se outra\textbackslash{}nvez.\textbackslash{}n\textbackslash{}nEstêvão Soares teve de ir à casa de um\textbackslash{}nministro de Estado para saber de uns papéis relativos a um parente da\textbackslash{}nprovíncia, e aí encontrou o deputado Meneses, que acabava de ter uma\textbackslash{}nconferência política.\textbackslash{}n\textbackslash{}nHouve sincero prazer em ambos\textbackslash{}nencontrando-se pela segunda vez; e Meneses arrancou de Estêvão a promessa de\textbackslash{}nque iria à casa dele daí a poucos dias.\textbackslash{}n\textbackslash{}nO ministro depressa despachou o jovem\textbackslash{}nmédico.\textbackslash{}n\textbackslash{}nChegando ao corredor, Estêvão foi\textbackslash{}nsurpreendido com uma tremenda bátega d'água, que nesse momento caía, e começava\textbackslash{}na alagar a rua.\textbackslash{}n\textbackslash{}nO rapaz olhou a um e outro lado a ver se\textbackslash{}npassava algum veículo vazio, mas procurou inutilmente; todos que passavam iam\textbackslash{}nocupados.\textbackslash{}n\textbackslash{}nApenas à porta estava um coupé\textbackslash{}nvazio à espera de alguém, que o rapaz supôs ser o deputado.\textbackslash{}n\textbackslash{}nDaí a alguns minutos desce com efeito o\textbackslash{}nrepresentante da nação, e admirou-se de ver o médico ainda à porta.\textbackslash{}n\textbackslash{}n— Que quer? disse-lhe Estêvão; a chuva\textbackslash{}nimpediu-me de sair; aqui fiquei a ver se passa um tílburi.\textbackslash{}n\textbackslash{}n— É natural que não passe, e nesse caso\textbackslash{}nofereço-lhe um lugar no meu coupé. Venha.\textbackslash{}n\textbackslash{}n— Perdão; mas é um incômodo{\ldots}\textbackslash{}n\textbackslash{}n— Ora, incômodo! É um prazer. Vou\textbackslash{}ndeixá-lo em casa. Onde mora?\textbackslash{}n\textbackslash{}n— Rua da Misericórdia nº{\ldots}\textbackslash{}n\textbackslash{}n— Bem, suba.\textbackslash{}n\textbackslash{}nEstêvão hesitou um pouco; mas não podia\textbackslash{}ndeixar de subir sem ofender o digno homem que de tão boa vontade lhe fazia um\textbackslash{}nobséquio.\textbackslash{}n\textbackslash{}nSubiram.\textbackslash{}n\textbackslash{}nMas em vez de mandar o cocheiro para a\textbackslash{}nRua da Misericórdia, o deputado gritou:\textbackslash{}n\textbackslash{}n— João, para casa!\textbackslash{}n\textbackslash{}nE entrou.\textbackslash{}n\textbackslash{}nEstêvão olhou para ele admirado.\textbackslash{}n\textbackslash{}n— Já sei, disse-lhe Meneses; admira-se de\textbackslash{}nver que faltei à minha palavra; mas eu desejo apenas que fique conhecendo a\textbackslash{}nminha casa a fim de lá voltar quanto antes.\textbackslash{}n\textbackslash{}nO coupé rolava já pela rua fora\textbackslash{}ndebaixo de uma chuva torrencial. Meneses foi o primeiro que rompeu o silêncio\textbackslash{}nde alguns minutos, dizendo ao jovem amigo:\textbackslash{}n\textbackslash{}n— Espero que o romance da nossa amizade\textbackslash{}nnão termine no primeiro capítulo.\textbackslash{}n\textbackslash{}nEstêvão, que já reparara nas maneiras\textbackslash{}nsolícitas do deputado, ficou inteiramente pasmado quando lhe ouviu falar no\textbackslash{}nromance da amizade. A razão era simples. O amigo que os havia apresentado no\textbackslash{}nTeatro Lírico disse no dia seguinte:\textbackslash{}n\textbackslash{}n— Meneses é um misantropo, e um cético;\textbackslash{}nnão crê em nada, nem estima ninguém. Na política como na sociedade faz um papel\textbackslash{}npuramente negativo.\textbackslash{}n\textbackslash{}nEsta era a impressão com que Estêvão,\textbackslash{}napesar da simpatia que o arrastava, falou a segunda vez a Meneses, e\textbackslash{}nadmirava-se de tudo, das maneiras, das palavras, e do tom de afeto que elas\textbackslash{}npareciam revelar.\textbackslash{}n\textbackslash{}nÀ linguagem do deputado o jovem médico\textbackslash{}nrespondeu com igual franqueza.\textbackslash{}n\textbackslash{}n— Por que acabaremos no primeiro\textbackslash{}ncapítulo? perguntou ele; um amigo não é coisa que se despreze, acolhe-se como\textbackslash{}num presente dos deuses.\textbackslash{}n\textbackslash{}n— Dos deuses! disse Meneses rindo; já\textbackslash{}nvejo que é pagão.\textbackslash{}n\textbackslash{}n— Alguma coisa, é verdade; mas no bom\textbackslash{}nsentido, respondeu Estêvão rindo também. Minha vida assemelha-se um pouco à de\textbackslash{}nUlisses{\ldots}\textbackslash{}n\textbackslash{}n— Tem ao menos uma Ítaca, sua pátria, e\textbackslash{}numa Penélope, sua esposa.\textbackslash{}n\textbackslash{}n— Nem uma nem outra.\textbackslash{}n\textbackslash{}n— Então entender-nos-emos.\textbackslash{}n\textbackslash{}nDizendo isto o deputado voltou a cara\textbackslash{}npara o outro lado, vendo a chuva que caía na vidraça da portinhola.\textbackslash{}n\textbackslash{}nDecorreram dois ou três minutos, durante\textbackslash{}nos quais Estêvão teve tempo de contemplar a seu gosto o companheiro de viagem.\textbackslash{}n\textbackslash{}nMeneses voltou-se e entrou em novo\textbackslash{}nassunto.\textbackslash{}n\textbackslash{}nQuando o coupé entrou na Rua do\textbackslash{}nLavradio, Meneses disse ao médico:\textbackslash{}n\textbackslash{}n— Moro nesta rua; estamos perto de casa.\textbackslash{}nPromete-me que há de vir ver-me algumas vezes?\textbackslash{}n\textbackslash{}n— Amanhã mesmo.\textbackslash{}n\textbackslash{}n— Bem. Como vai a sua clínica?\textbackslash{}n\textbackslash{}n— Apenas começo, disse Estêvão; trabalho\textbackslash{}npouco; mas espero fazer alguma coisa.\textbackslash{}n\textbackslash{}n— O seu companheiro, na noite em que mo\textbackslash{}napresentou, disse-me que o senhor é moço de muito merecimento.\textbackslash{}n\textbackslash{}n— Tenho vontade de fazer alguma coisa.\textbackslash{}n\textbackslash{}nDaí a dez minutos parava o coupé à\textbackslash{}nporta de uma casa da Rua do Lavradio.\textbackslash{}n\textbackslash{}nApearam-se os dois e subiram.\textbackslash{}n\textbackslash{}nMeneses mostrou a Estêvão o seu gabinete\textbackslash{}nde trabalho, onde havia duas longas estantes de livros.\textbackslash{}n\textbackslash{}n— É a minha família, disse o deputado\textbackslash{}nmostrando os livros. História, filosofia, poesia{\ldots} e alguns livros de\textbackslash{}npolítica. Aqui estudo e trabalho. Quando cá vier é aqui que o hei de receber.\textbackslash{}n\textbackslash{}nEstêvão prometeu voltar no dia seguinte,\textbackslash{}ne desceu para entrar no coupé que esperava por ele, e que o levou à Rua\textbackslash{}nda Misericórdia.\textbackslash{}n\textbackslash{}nEntrando em casa Estêvão dizia consigo:\textbackslash{}n\textbackslash{}n'Onde está a misantropia daquele\textbackslash{}nhomem? As maneiras de misantropo são mais rudes do que as dele; salvo se ele,\textbackslash{}nmais feliz do que Diógenes, achou em mim o homem que procurava.'\textbackslash{}n\textbackslash{}nCAPÍTULO II\textbackslash{}n\textbackslash{}nEstêvão era o tipo do rapaz sério. Tinha\textbackslash{}ntalento, ambição e vontade de saber, três armas poderosas nas mãos de um homem\textbackslash{}nque tenha consciência de si. Desde os dezesseis anos a sua vida foi um estudo\textbackslash{}nconstante, aturado e profundo. Destinado ao curso médico, Estêvão entrou na\textbackslash{}nacademia um pouco forçado, não queria desobedecer ao pai. A sua vocação era\textbackslash{}ntoda para as matemáticas. Que importa? disse ele ao saber da resolução paterna;\textbackslash{}nestudarei a medicina e a matemática. Com efeito teve tempo para uma e outra\textbackslash{}ncoisa; teve tempo ainda para estudar a literatura, e as principais obras da\textbackslash{}nantigüidade e contemporâneas eram-lhe tão familiares como os tratados de\textbackslash{}noperações e de higiene.\textbackslash{}n\textbackslash{}nPara estudar tanto, foi-lhe preciso\textbackslash{}nsacrificar uma parte da saúde. Estêvão aos vinte e quatro anos adquirira uma\textbackslash{}nmagreza, que não era a dos dezesseis; tinha a tez pálida e a cabeça pendia-lhe\textbackslash{}num pouco para a frente pelo longo hábito da leitura. Mas esses vestígios de uma\textbackslash{}nlonga aplicação intelectual não lhe alteraram a regularidade e harmonia das\textbackslash{}nfeições, nem os olhos perderam nos livros o brilho e a expressão. Era além\textbackslash{}ndisso naturalmente elegante, não digo enfeitado, que é coisa diferente: era\textbackslash{}nelegante nas maneiras, na atitude, no sorriso, no trajo, tudo mesclado de uma\textbackslash{}ncerta severidade que era o cunho do seu caráter. Podia-se notar-lhe muitas\textbackslash{}ninfrações ao código da moda; ninguém poderia dizer que ele faltasse nunca às\textbackslash{}nboas regras do gentleman.\textbackslash{}n\textbackslash{}nPerdera os pais aos vinte anos, mas\textbackslash{}nficara-lhe bastante juízo para continuar sozinho a viagem do mundo. O estudo\textbackslash{}nserviu-lhe de refúgio e bordão. Não sabia nada do que era o amor. Ocupara-se\textbackslash{}ntanto com a cabeça que esquecera-se de que tinha um coração dentro do peito.\textbackslash{}nNão se infira daqui que Estêvão fosse puramente um positivista. Pelo contrário,\textbackslash{}na alma dele possuía ainda em toda a plenitude da graça e da força as duas asas\textbackslash{}nque a natureza lhe dera. Não raras vezes rompia ela do cárcere da carne para ir\textbackslash{}ncorrer os espaços do céu, em busca de não sei que ideal mal definido, obscuro,\textbackslash{}nincerto. Quando voltava desses êxtases, Estêvão curava-se deles enterrando-se\textbackslash{}nnos volumes à cata de uma verdade científica. Newton era-lhe o antídoto de\textbackslash{}nGoethe.\textbackslash{}n\textbackslash{}nAlém disso, Estêvão tinha idéias\textbackslash{}nsingulares. Havia um padre, amigo dele, rapaz de trinta anos, da escola de\textbackslash{}nFénelon, que entrava com Telêmaco na ilha de Calipso. Ora, o padre dizia muitas\textbackslash{}nvezes a Estêvão que só uma coisa lhe faltava para ser completo: era casar-se.\textbackslash{}n\textbackslash{}n— Quando você tiver, dizia-lhe, uma\textbackslash{}nmulher amada e amante ao pé de si, será um homem feliz e completo. Dividirá\textbackslash{}nentão o tempo entre as duas coisas mais elevadas que a natureza deu ao homem, a\textbackslash{}ninteligência e o coração. Nesse dia quero eu mesmo casá-lo{\ldots}\textbackslash{}n\textbackslash{}n— Padre Luís, respondia Estêvão, faça-me\textbackslash{}nentão o serviço completo: traga-me a mulher e a bênção.\textbackslash{}n\textbackslash{}nO padre sorria-se ao ouvir a resposta do\textbackslash{}nmédico, e como o sorriso parecia a Estêvão uma nova pergunta, o médico\textbackslash{}ncontinuava:\textbackslash{}n\textbackslash{}n— Se encontrar uma mulher tão completa\textbackslash{}ncomo eu exijo, afirmo-lhe que me casarei. Dirá que as obras humanas são\textbackslash{}nimperfeitas, e eu não contestarei, Padre Luís; mas nesse caso deixe-me caminhar\textbackslash{}nsó com as minhas imperfeições.\textbackslash{}n\textbackslash{}nDaqui engendrava-se sempre uma discussão,\textbackslash{}nque se animava e crescia até o ponto em que Estêvão concluía por este modo:\textbackslash{}n\textbackslash{}n— Padre Luís, uma menina que deixa as\textbackslash{}nbonecas para ir decorar mecanicamente alguns livros mal escolhidos; que\textbackslash{}ninterrompe uma lição para ouvir contar uma cena de namoro; que em matéria de\textbackslash{}narte só conhece os figurinos parisienses; que deixa as calças para entrar no\textbackslash{}nbaile, e que antes de suspirar por um homem, examina-lhe a correção da gravata,\textbackslash{}ne o apertado do botim; Padre Luís, esta menina pode vir a ser um esplêndido\textbackslash{}nornamento de salão e até uma fecunda mãe de família, mas nunca será uma mulher.\textbackslash{}n\textbackslash{}nEsta sentença de Estêvão tinha o defeito\textbackslash{}nde certas regras absolutas. Por isso, o padre dizia-lhe sempre:\textbackslash{}n\textbackslash{}n— Tem você razão; mas eu não lhe digo que\textbackslash{}ncase com a regra; procure a exceção que há de encontrar e leve-a ao altar, onde\textbackslash{}neu estarei para os unir.\textbackslash{}n\textbackslash{}nTais eram os sentimentos de Estêvão em relação\textbackslash{}nao amor e à mulher. A natureza dera-lhe em parte esses sentimentos; mas em\textbackslash{}nparte adquiriu-os ele nos livros. Exigia a perfeição intelectual e moral de uma\textbackslash{}nHeloísa; e partia da exceção para estabelecer uma regra. Era intolerante para\textbackslash{}nos erros veniais. Não os reconhecia como tais. Não há erro venial, dizia ele,\textbackslash{}nem matéria de costumes e de amor.\textbackslash{}n\textbackslash{}nContribuíra para esta rigidez de ânimo o\textbackslash{}nespetáculo da própria família de Estêvão. Até aos vinte anos foi ele testemunha\textbackslash{}ndo que era a santidade do amor mantido pela virtude doméstica. Sua mãe, que\textbackslash{}nmorrera com trinta e oito anos, amou o marido até os últimos dias, e poucos\textbackslash{}nmeses lhe sobreviveu. Estêvão soube que fora ardente e entusiástico o amor de\textbackslash{}nseus pais, na estação do noivado, durante a manhã conjugal; conheceu-o assim\textbackslash{}npor tradição; mas na tarde conjugal a que ele assistiu viu o amor calmo,\textbackslash{}nsolícito e confiante, cheio de dedicação e respeito, praticado como um culto;\textbackslash{}nsem recriminações nem pesares, e tão profundo como no primeiro dia. Os pais de\textbackslash{}nEstêvão morreram amados e felizes na tranqüila serenidade do dever.\textbackslash{}n\textbackslash{}nNo ânimo de Estêvão, o amor que funda a\textbackslash{}nfamília devia ser aquilo ou não seria nada. Era justiça; mas a intolerância de\textbackslash{}nEstêvão começava na convicção que ele tinha de que com a dele morrera a última\textbackslash{}nfamília, e fora com ela a derradeira tradição do amor. Que era preciso para\textbackslash{}nderrubar todo este sistema, ainda que momentâneo? Uma coisa pequeníssima: um\textbackslash{}nsorriso e dois olhos.\textbackslash{}n\textbackslash{}nMas como esses dois olhos não apareciam,\textbackslash{}nEstêvão entregava-se na maior parte do tempo aos seus estudos científicos,\textbackslash{}nempregando as horas vagas em algumas distrações que o não prendiam por muito\textbackslash{}ntempo.\textbackslash{}n\textbackslash{}nMorava só; tinha um escravo, da mesma\textbackslash{}nidade que ele, e cria da casa do pai, - mais irmão do que escravo, na dedicação\textbackslash{}ne no afeto. Recebia alguns amigos, a quem visitava de quando em quando de\textbackslash{}nquando, entre os quais incluímos o jovem Padre Luís, a quem Estevão chamava -\textbackslash{}nPlatão de sotaina.\textbackslash{}n\textbackslash{}nNaturalmente bom e afetuoso, generoso e\textbackslash{}ncavalheiresco, sem ódios nem rancores, entusiasta por todas as coisas boas e\textbackslash{}nverdadeiras, tal era o Dr. Estevão Soares, aos vinte e quatro anos de idade.\textbackslash{}n\textbackslash{}nDo seu retrato físico já dissemos alguma\textbackslash{}ncoisa. Bastará acrescentar que tinha uma bela cabeça, coberta de bastos cabelos\textbackslash{}ncastanhos, dois olhos da mesma cor, vivos e observadores; a palidez do rosto\textbackslash{}nfazia realçar o bigode naturalmente encaracolado. Era alto e tinha mãos\textbackslash{}nadmiráveis.\textbackslash{}n\textbackslash{}nCAPÍTULO III\textbackslash{}n\textbackslash{}nEstêvão Soares visitou Meneses no dia\textbackslash{}nseguinte.\textbackslash{}n\textbackslash{}nO deputado esperava-o, e recebeu-o como\textbackslash{}nse fosse um amigo velho. Estêvão marcara a hora da visita, que impossibilitava\textbackslash{}na presença de Meneses na Câmara; mas o deputado importou-se pouco com isso: não\textbackslash{}nfoi à Câmara. Mas teve a delicadeza de o não dizer a Estevão.\textbackslash{}n\textbackslash{}nMeneses estava no gabinete quando o\textbackslash{}ncriado anunciou-lhe a chegada do médico. Foi recebê-lo à porta.\textbackslash{}n\textbackslash{}n— Pontual como um rei, disse-lhe\textbackslash{}nalegremente.\textbackslash{}n\textbackslash{}n— Era dever. Lembro-lhe que não me\textbackslash{}nesqueci.\textbackslash{}n\textbackslash{}n— E agradeço-lho.\textbackslash{}n\textbackslash{}nSentaram-se os dois.\textbackslash{}n\textbackslash{}n— Agradeço-lhe porque eu receava sobretudo\textbackslash{}nque me houvesse compreendido mal; e que os impulsos da minha simpatia não\textbackslash{}nmerecessem da sua parte nenhuma consideração{\ldots}\textbackslash{}n\textbackslash{}nEstêvão ia protestar.\textbackslash{}n\textbackslash{}n— Perdão, continuou Meneses, bem vejo que\textbackslash{}nme enganei, e é por isso que lhe agradeço. Eu não sou rapaz; tenho 47 anos; e\textbackslash{}npara a sua idade as relações de um homem como eu já não têm valor.\textbackslash{}n\textbackslash{}n— A velhice, quando é respeitável, deve\textbackslash{}nser respeitada; e amada quando é amável. Mas V. Excia. não é velho; tem os\textbackslash{}ncabelos apenas grisalhos: pode-se dizer que está na segunda mocidade.\textbackslash{}n\textbackslash{}n— Parece-lhe isso{\ldots}\textbackslash{}n\textbackslash{}n— Parece e é.\textbackslash{}n\textbackslash{}n— Seja como for, disse Meneses, a verdade\textbackslash{}né que podemos ser amigos. Quantos anos tem?\textbackslash{}n\textbackslash{}n— Olhe lá, podia ser meu filho. Tem seus\textbackslash{}npais vivos?\textbackslash{}n\textbackslash{}n— Morreram há quatro anos.\textbackslash{}n\textbackslash{}n— Lembra-me haver dito que era\textbackslash{}nsolteiro{\ldots}\textbackslash{}n\textbackslash{}n— É verdade.\textbackslash{}n\textbackslash{}n— De maneira que os seus cuidados são\textbackslash{}ntodos para a ciência?\textbackslash{}n\textbackslash{}n— É a minha esposa.\textbackslash{}n\textbackslash{}n— Sim, a sua esposa intelectual; mas essa\textbackslash{}nnão basta a um homem como o senhor{\ldots} Enfim, isso é com o tempo; está ainda\textbackslash{}nmoço.\textbackslash{}n\textbackslash{}nDurante este diálogo, Estevão contemplava\textbackslash{}ne observava Meneses, em cujo rosto batia a claridade que entrava por uma das\textbackslash{}njanelas. Era uma cabeça severa, cheia de cabelos já grisalhos, que lhe caíam em\textbackslash{}ngracioso desalinho. Tinha os olhos negros e um pouco amortecidos; adivinhava-se\textbackslash{}nporém que deviam ter sido vivos e ardentes. As suíças também grisalhas eram\textbackslash{}ncomo as de lorde Palmerston, segundo dizem as gravuras. Não tinha rugas\textbackslash{}nde velhice; tinha uma ruga na testa, entre as sobrancelhas, indício de\textbackslash{}nconcentração de espírito, e não vestígio do tempo. A testa era alta, o queixo e\textbackslash{}nas maçãs do rosto um pouco salientes. Adivinhava-se que devia ter sido formoso\textbackslash{}nno tempo da primeira mocidade; e antevia-se já uma velhice imponente e augusta.\textbackslash{}nSorria de quando em quando; e o sorriso, embora aquele rosto não fosse de um\textbackslash{}nancião, produzia uma impressão singular; parecia um raio de lua no meio de uma\textbackslash{}nvelha ruína. É que o sorriso era amável, mas não era alegre.\textbackslash{}n\textbackslash{}nTodo aquele conjunto impressionava e\textbackslash{}natraía; Estêvão sentia-se cada vez mais arrastado para aquele homem, que o\textbackslash{}nprocurava, e lhe estendia a mão.\textbackslash{}n\textbackslash{}nA conversa continuou no tom afetuoso com\textbackslash{}nque começara; a primeira entrevista da amizade é o oposto da primeira\textbackslash{}nentrevista do amor; nesta a mudez é a grande eloqüência; naquela inspira-se e\textbackslash{}nganha-se a confiança, pela exposição franca dos sentimentos e das idéias.\textbackslash{}n\textbackslash{}nNão se falou de política. Estêvão aludiu\textbackslash{}nde passagem às funções de Meneses, mas foi um verdadeiro incidente a que o\textbackslash{}ndeputado não prestou atenção.\textbackslash{}n\textbackslash{}nNo fim de uma hora, Estêvão levantou-se\textbackslash{}npara sair; tinha de ir ver um doente.\textbackslash{}n\textbackslash{}n— O motivo é sagrado; senão retinha-o.\textbackslash{}n\textbackslash{}n— Mas eu voltarei outras vezes.\textbackslash{}n\textbackslash{}n— Sem dúvida alguma, e eu irei vê-lo algumas\textbackslash{}nvezes. Se no fim de quinze dias não se aborrecer{\ldots} Olhe, venha de tarde; janta\textbackslash{}nalgumas vezes comigo; depois da Câmara estou completamente livre.\textbackslash{}n\textbackslash{}nEstêvão saiu prometendo tudo.\textbackslash{}n\textbackslash{}nVoltou lá, com efeito, e jantou duas\textbackslash{}nvezes com o deputado, que também visitou Estêvão em casa; foram ao teatro\textbackslash{}njuntos; relacionaram-se intimamente com as famílias conhecidas. No fim de um\textbackslash{}nmês eram dois amigos velhos. Tinham observado reciprocamente o caráter e os\textbackslash{}nsentimentos. Meneses gostava de ver a seriedade do médico e o seu bom senso,\textbackslash{}nestimava-o com as suas intolerâncias, aplaudindo-lhe a generosa ambição que o\textbackslash{}ndominava. Pela sua parte o médico via em Meneses um homem que sabia ligar a\textbackslash{}nausteridade dos anos à amabilidade de cavalheiro, modesto nas suas maneiras,\textbackslash{}ninstruído, sentimental. Da misantropia anunciada não encontrou vestígios. É\textbackslash{}nverdade que em algumas ocasiões Meneses parecia mais disposto a ouvir do que a\textbackslash{}nfalar; e então o olhar tornava-se-lhe sombrio e parado, como se em vez de ver\textbackslash{}nos objetos exteriores, estivesse contemplando a sua própria consciência. Mas\textbackslash{}neram rápidos esses momentos, e Meneses voltava logo aos seus modos habituais.\textbackslash{}n\textbackslash{}n'Não é um misantropo, pensava então\textbackslash{}nEstêvão; mas este homem tem um drama dentro de si.'\textbackslash{}n\textbackslash{}nA observação de Estêvão adquiriu certo caráter\textbackslash{}nde verossimilhança quando uma noite em que se achavam no Teatro Lírico, Estêvão\textbackslash{}nchamou a atenção de Meneses para uma mulher vestida de preto que se achava em\textbackslash{}num camarote da primeira ordem.\textbackslash{}n\textbackslash{}n— Não conheço aquela mulher, disse\textbackslash{}nEstêvão. Sabe quem é?\textbackslash{}n\textbackslash{}nMeneses olhou para o camarote indicado,\textbackslash{}ncontemplou a mulher por alguns instantes e respondeu:\textbackslash{}n\textbackslash{}n— Não conheço.\textbackslash{}n\textbackslash{}nA conversa ficou aí; mas o médico reparou\textbackslash{}nque a mulher duas vezes olhou para Meneses, e este duas vezes para ela, encontrando-se\textbackslash{}nos olhos de ambos.\textbackslash{}n\textbackslash{}nNo fim do espetáculo, os dois amigos\textbackslash{}ndirigiram-se pelo corredor do lado em que estivera a mulher de preto. Estêvão\textbackslash{}nteve apenas nova curiosidade, a curiosidade de artista: quis vê-la de perto.\textbackslash{}nMas a porta do camarote estava fechada. Teria já saído ou não? Era impossível\textbackslash{}nsabê-lo. Meneses passou sem olhar. Ao chegarem ao patamar da escada que dá para\textbackslash{}no lado da Rua dos Ciganos, pararam os dois porque havia grande afluência de\textbackslash{}ngente. Daí a pouco ouviu-se passo apressado; Meneses voltou o rosto, e dando o\textbackslash{}nbraço a Estêvão desceu imediatamente, apesar da dificuldade.\textbackslash{}n\textbackslash{}nEstêvão compreendeu, mas nada viu.\textbackslash{}n\textbackslash{}nPela sua parte, Meneses não deu sinal\textbackslash{}nalgum.\textbackslash{}n\textbackslash{}nApenas se desembaraçaram da multidão, o\textbackslash{}ndeputado encetou uma alegre conversa com o médico.\textbackslash{}n\textbackslash{}n— Que efeito lhe faz, perguntou ele,\textbackslash{}nquando passa no meio de tantas damas elegantes, aquela confusão de sedas e de\textbackslash{}nperfumes?\textbackslash{}n\textbackslash{}nEstêvão respondeu distraidamente, e\textbackslash{}nMeneses continuou a conversa no mesmo estilo; daí a cinco minutos a aventura do\textbackslash{}nteatro tinha-se-lhe varrido da memória.\textbackslash{}n\textbackslash{}nCAPÍTULO IV\textbackslash{}n\textbackslash{}nUm dia Estêvão Soares foi convidado para\textbackslash{}num baile em casa de um velho amigo de seu pai.\textbackslash{}n\textbackslash{}nA sociedade era luzida e numerosa; Estêvão,\textbackslash{}nembora vivesse muito arredado, achou ali grande número de conhecidas. Não\textbackslash{}ndançou; viu, conversou, riu um pouco e saiu.\textbackslash{}n\textbackslash{}nMas ao entrar levava o coração livre; ao\textbackslash{}nsair trouxe nele uma flecha, para falar a linguagem dos poetas da Arcádia; era\textbackslash{}na flecha do amor.\textbackslash{}n\textbackslash{}nDo amor? A falar a verdade não se pode\textbackslash{}ndar este nome ao sentimento experimentado por Estêvão; não era ainda o amor,\textbackslash{}nmas bem pode ser que viesse a sê-lo. Por enquanto era um sentimento de\textbackslash{}nfascinação doce e branda; uma mulher que lá estava produzira nele a impressão\textbackslash{}nque as fadas produziam nos príncipes errantes ou nas princesas perseguidas,\textbackslash{}nsegundo nos rezam os contos das velhas.\textbackslash{}n\textbackslash{}nA mulher em questão não era uma virgem;\textbackslash{}nera uma viúva de trinta e quatro anos, bela como o dia, graciosa e terna. Estêvão\textbackslash{}nvia-a pela primeira vez; pelo menos não se lembrava daquelas feições. Conversou\textbackslash{}ncom ela durante meia hora, e tão encantado ficou com as maneiras, a voz, a\textbackslash{}nbeleza de Madalena, que ao chegar à casa não pôde dormir.\textbackslash{}n\textbackslash{}nComo verdadeiro médico que era, sentia em\textbackslash{}nsi os sintomas dessa hipertrofia do coração que se chama amor e procurou\textbackslash{}ncombater a enfermidade nascente. Leu algumas páginas de matemáticas, isto é,\textbackslash{}npercorreu-as com os olhos; porque apenas começava a ler o espírito alheava do\textbackslash{}nlivro onde apenas ficavam os olhos: o espírito ia ter com a viúva.\textbackslash{}n\textbackslash{}nO cansaço foi mais feliz que Euclides:\textbackslash{}nsobre a madrugada Estêvão Soares adormeceu.\textbackslash{}n\textbackslash{}nMas sonhou com a viúva.\textbackslash{}n\textbackslash{}nSonhou que a apertava em seus braços, que\textbackslash{}na cobria de beijos, que era seu esposo perante a Igreja e perante a sociedade.\textbackslash{}n\textbackslash{}nQuando acordou e lembrou-se do sonho,\textbackslash{}nEstêvão sorriu.\textbackslash{}n\textbackslash{}n— Casar-me! disse ele. Era o que me\textbackslash{}nfaltava. Como poderia eu ser feliz com o espírito receoso e ambicioso que a\textbackslash{}nnatureza me deu? Acabemos com isto; nunca mais verei aquela mulher{\ldots} e boa\textbackslash{}nnoite.\textbackslash{}n\textbackslash{}nComeçou a vestir-se.\textbackslash{}n\textbackslash{}nTrouxeram-lhe o almoço; Estêvão comeu\textbackslash{}nrapidamente, porque era tarde, e saiu para ir ver alguns doentes.\textbackslash{}n\textbackslash{}nMas ao passar pela Rua do Conde\textbackslash{}nlembrou-se que Madalena lhe dissera morar ali; mas aonde? A viúva disse-lhe o\textbackslash{}nnúmero; o médico porém estava tão embebido em ouvi-la falar que não o decorou.\textbackslash{}n\textbackslash{}nQueria e não queria; protestava\textbackslash{}nesquecê-la, e contudo daria o que se lhe pedisse para saber o número da casa\textbackslash{}nnaquele momento.\textbackslash{}n\textbackslash{}nComo ninguém podia dizer-lhe, o rapaz\textbackslash{}ntomou o partido de ir-se embora.\textbackslash{}n\textbackslash{}nNo dia seguinte, porém, teve o cuidado de\textbackslash{}npassar duas vezes pela Rua do Conde a ver se descobria a encantadora viúva. Não\textbackslash{}ndescobriu nada; mas quando ia tomar um tílburi e voltar para casa encontrou o\textbackslash{}namigo de seu pai em cuja casa encontrara Madalena.\textbackslash{}n\textbackslash{}nEstêvão já tinha pensado nele; mas\textbackslash{}nimediatamente tirou dali o pensamento, porque ir perguntar-lhe onde morava a\textbackslash{}nviúva era uma coisa que podia traí-lo.\textbackslash{}n\textbackslash{}nEstêvão já empregava o verbo trair.\textbackslash{}n\textbackslash{}nO homem em questão, depois de\textbackslash{}ncumprimentar ao médico, e trocar com ele algumas palavras, disse-lhe que ia à\textbackslash{}ncasa de Madalena, e despediu-se.\textbackslash{}n\textbackslash{}nEstêvão estremeceu de satisfação.\textbackslash{}n\textbackslash{}nAcompanhou de longe o amigo e viu-o\textbackslash{}nentrar em uma casa.\textbackslash{}n\textbackslash{}n'É ali', pensou ele.\textbackslash{}n\textbackslash{}nE afastou-se rapidamente.\textbackslash{}n\textbackslash{}nQuando entrou em casa achou uma carta\textbackslash{}npara ele; a letra, que lhe era desconhecida, estava traçada com elegância e\textbackslash{}ncuidado: a carta recendia de sândalo.\textbackslash{}n\textbackslash{}nO médico rompeu o lacre.\textbackslash{}n\textbackslash{}nA carta dizia assim:\textbackslash{}n\textbackslash{}nAmanhã toma-se chá em minha casa. Se\textbackslash{}nquiser vir passar algumas horas conosco dar-nos-á sumo prazer.\textbackslash{}n\textbackslash{}nMadalena C{\ldots}\textbackslash{}n\textbackslash{}nEstêvão leu e releu o bilhete; teve idéia\textbackslash{}nde levá-lo aos lábios, mas envergonhado diante de si próprio por uma idéia que\textbackslash{}nlhe parecia de fraqueza, cheirou simplesmente o bilhete e meteu-o no bolso.\textbackslash{}n\textbackslash{}nEstêvão era um pouco fatalista.\textbackslash{}n\textbackslash{}n'Se eu não fosse àquele baile não\textbackslash{}nconhecia esta mulher, não andava agora com estes cuidados, e tinha conjurado\textbackslash{}numa desgraça ou uma felicidade, porque ambas as coisas podem nascer deste\textbackslash{}nencontro fortuito. Que será? Eis-me na dúvida de Hamleto. Devo ir à casa dela?\textbackslash{}nA cortesia pede que vá. Devo ir; mas irei encouraçado contra tudo. É preciso\textbackslash{}nromper com estas idéias, e continuar a vida tranqüila que tenho tido.'\textbackslash{}n\textbackslash{}nEstava nisto quando Meneses lhe entrou\textbackslash{}npor casa. Vinha buscá-lo para jantar. Estêvão saiu com o deputado. Em caminho\textbackslash{}nfez-lhe perguntas curiosas.\textbackslash{}n\textbackslash{}nPor exemplo:\textbackslash{}n\textbackslash{}n— Acredita no destino, meu amigo? Pensa\textbackslash{}nque há um deus do bem e um deus do mal, em conflito travado sobre a vida do\textbackslash{}nhomem?\textbackslash{}n\textbackslash{}n— O destino é a vontade, respondia\textbackslash{}nMeneses; cada homem faz o seu destino.\textbackslash{}n\textbackslash{}n— Mas enfim nós temos pressentimentos{\ldots}\textbackslash{}nÀs vezes adivinhamos acontecimentos em que não tomamos parte; não lhe parece\textbackslash{}nque é um deus benfazejo que no-los segreda?\textbackslash{}n\textbackslash{}n— Fala como um pagão; eu não creio em\textbackslash{}nnada disso. Creio que tenho o estômago vazio, e o que melhor podemos fazer é\textbackslash{}njantar aqui mesmo no Hotel de Europa em vez de ir à Rua do Lavradio.\textbackslash{}n\textbackslash{}nSubiram ao Hotel de Europa.\textbackslash{}n\textbackslash{}nAli havia vários deputados que\textbackslash{}nconversavam de política, e os quais se reuniram a Meneses. Estêvão ouvia e\textbackslash{}nrespondia, sem esquecer nunca a viúva, a carta e o sândalo.\textbackslash{}n\textbackslash{}nAssim, pois, davam-se contrastes\textbackslash{}nsingulares entre a conversa geral e o pensamento de Estêvão.\textbackslash{}n\textbackslash{}nDizia por exemplo um deputado:\textbackslash{}n\textbackslash{}n— O governo é reator; as províncias não\textbackslash{}npodem mais suportá-lo. Os princípios estão todos preteridos, na minha província\textbackslash{}nforam demitidos alguns subdelegados pela circunstância única de serem meus\textbackslash{}nparentes; meu cunhado, que era diretor das rendas, foi posto fora do lugar, e\textbackslash{}neste deu-se a um peralta contraparente dos Valadares. Eu confesso que vou\textbackslash{}nromper amanhã a oposição.\textbackslash{}n\textbackslash{}nEstêvão olhava para o deputado; mas no\textbackslash{}ninterior estava dizendo isto:\textbackslash{}n\textbackslash{}n'Com efeito, Madalena é bela, é\textbackslash{}nadmiravelmente bela. Tem uns olhos de matar. Os cabelos são lindíssimos: tudo\textbackslash{}nnela é fascinador. Se pudesse ser minha mulher, eu seria feliz; mas quem\textbackslash{}nsabe?{\ldots} Contudo sinto que vou amá-la. Já é irresistível; é preciso amá-la; e\textbackslash{}nela? que quer dizer aquele convite? Amar-me-á?'\textbackslash{}n\textbackslash{}nEstêvão embebera-se tanto nesta\textbackslash{}ncontemplação ideal, que, acontecendo perguntar-lhe um deputado se não achava a\textbackslash{}nsituação negra e carrancuda, Estêvão entregue ao seu pensamento respondeu:\textbackslash{}n\textbackslash{}n— É lindíssima!\textbackslash{}n\textbackslash{}n— Ah! disse o deputado, vejo que o senhor\textbackslash{}né ministerialista.\textbackslash{}n\textbackslash{}nEstêvão sorriu; mas Meneses franziu o\textbackslash{}nsobrolho.\textbackslash{}n\textbackslash{}nCompreendera tudo.\textbackslash{}n\textbackslash{}nCAPÍTULO V\textbackslash{}n\textbackslash{}nQuando saíram, o deputado disse ao\textbackslash{}nmédico:\textbackslash{}n\textbackslash{}n— Meu amigo, você é desleal comigo{\ldots}\textbackslash{}n\textbackslash{}n— Por quê? perguntou Estêvão meio sério e\textbackslash{}nmeio risonho, não compreendendo a observação do deputado.\textbackslash{}n\textbackslash{}n— Sim, continuou Meneses; você esconde-me\textbackslash{}num segredo{\ldots}\textbackslash{}n\textbackslash{}n— Eu?\textbackslash{}n\textbackslash{}n— É verdade: e um segredo de amor.\textbackslash{}n\textbackslash{}n— Ah!{\ldots} disse Estêvão; por que diz isso?\textbackslash{}n\textbackslash{}n— Reparei há pouco que, ao passo que os\textbackslash{}nmais conversavam em política, você pensava em uma mulher, e mulher{\ldots} lindíssima{\ldots}\textbackslash{}n\textbackslash{}nEstêvão compreendeu que estava\textbackslash{}ndescoberto; não negou.\textbackslash{}n\textbackslash{}n— É verdade, pensava em uma mulher.\textbackslash{}n\textbackslash{}n— E eu serei o último a saber?\textbackslash{}n\textbackslash{}n— Mas saber o quê? Não há amor, não há\textbackslash{}nnada. Encontrei uma mulher que me impressionou e ainda agora me preocupa; mas é\textbackslash{}nbem possível que não passe disto. Aí está. É um capítulo interrompido; um\textbackslash{}nromance que fica na primeira página. Eu lhe digo: há de me ser difícil amar.\textbackslash{}n\textbackslash{}n— Por quê?\textbackslash{}n\textbackslash{}n— Eu sei? custa-me a crer no amor.\textbackslash{}n\textbackslash{}nMeneses olhou fixamente para Estêvão,\textbackslash{}nsorriu, abanou a cabeça e disse:\textbackslash{}n\textbackslash{}n— Olhe, deixe a descrença para os que já\textbackslash{}nsofreram as decepções; o senhor está moço, não conhece ainda nada desse sentimento.\textbackslash{}nNa sua idade ninguém é cético{\ldots} Demais, se a mulher é bonita, eu aposto que\textbackslash{}ndaqui a pouco há de dizer-me o contrário.\textbackslash{}n\textbackslash{}n— Pode ser{\ldots} respondeu Estêvão.\textbackslash{}n\textbackslash{}nE ao mesmo tempo entrou a pensar nas\textbackslash{}npalavras de Meneses, palavras que ele comparava ao episódio do Teatro Lírico.\textbackslash{}n\textbackslash{}nEntretanto, Estêvão foi ao convite de\textbackslash{}nMadalena. Preparou-se e perfumou-se como se fosse falar a uma noiva. Que sairia\textbackslash{}ndaquele encontro? Viria de lá livre ou cativo? Já seria amado? Estêvão não\textbackslash{}ndeixou de pensá-lo; aquele convite parecia-lhe uma prova irrecusável. O médico\textbackslash{}nentrando num tílburi começou a formar vários castelos no ar.\textbackslash{}n\textbackslash{}nEnfim chegou à casa.\textbackslash{}n\textbackslash{}nCAPÍTULO VI\textbackslash{}n\textbackslash{}nMadalena estava na sala acompanhada de um\textbackslash{}nfilho.\textbackslash{}n\textbackslash{}nNinguém mais.\textbackslash{}n\textbackslash{}nEram nove horas e meia.\textbackslash{}n\textbackslash{}n— Viria eu cedo demais? perguntou ele à\textbackslash{}ndona da casa.\textbackslash{}n\textbackslash{}n— O senhor nunca vem cedo.\textbackslash{}n\textbackslash{}nEstêvão inclinou-se.\textbackslash{}n\textbackslash{}nMadalena continuou:\textbackslash{}n\textbackslash{}n— Se me acha só, é porque, tendo\textbackslash{}nenfermado um pouco, mandei desavisar as poucas pessoas que eu havia convidado.\textbackslash{}n\textbackslash{}n— Ah! mas eu não recebi{\ldots}\textbackslash{}n\textbackslash{}n— Naturalmente; eu não lhe mandei dizer\textbackslash{}nnada. Era a primeira vez que o convidava; não queria por modo algum arredar de\textbackslash{}ncasa um homem tão distinto.\textbackslash{}n\textbackslash{}nEstas palavras de Madalena não valiam\textbackslash{}ncoisa alguma, nem mesmo como desculpa, porque a desculpa é fraquíssima.\textbackslash{}n\textbackslash{}nEstêvão compreendeu logo que havia algum\textbackslash{}nmotivo oculto.\textbackslash{}n\textbackslash{}nSeria o amor?\textbackslash{}n\textbackslash{}nEstêvão pensou que era, e doeu-se,\textbackslash{}nporque, apesar de tudo, sonhara uma paixão mais reservada e menos precipitada.\textbackslash{}nNão queria, embora lhe agradasse, ser objeto daquela preferência; e mais que\textbackslash{}ntudo achava-se embaraçadíssimo diante de uma mulher a quem começava a amar, e\textbackslash{}nque talvez o amasse. Que lhe diria? Era a primeira vez que o médico achava-se\textbackslash{}nem tais apuros. Há toda a razão para supor que Estêvão naquele momento preferia\textbackslash{}nestar cem léguas distante, e contudo, longe que estivesse pensaria nela.\textbackslash{}n\textbackslash{}nMadalena era excessivamente bela, embora\textbackslash{}nmostrasse no rosto sinais de longo sofrimento. Era alta, cheia, tinha um\textbackslash{}nbelíssimo colo, magníficos braços, olhos castanhos e grandes, boca feita para\textbackslash{}nninho de amores.\textbackslash{}n\textbackslash{}nNaquele momento trajava um vestido preto.\textbackslash{}n\textbackslash{}nA cor preta ia-lhe muito bem.\textbackslash{}n\textbackslash{}nEstêvão contemplava aquela figura com\textbackslash{}namor e adoração; ouvia-a falar e sentia-se encantado e dominado por um sentimento\textbackslash{}nque não podia explicar.\textbackslash{}n\textbackslash{}nEra um misto de amor e de receio.\textbackslash{}n\textbackslash{}nMadalena mostrou-se delicada e solícita.\textbackslash{}nFalou no merecimento do rapaz e na sua nascente reputação, e instou com ele\textbackslash{}npara que fosse algumas vezes visitá-la.\textbackslash{}n\textbackslash{}nÀs 10 horas e meia serviu-se o chá na\textbackslash{}nsala. Estêvão conservou-se lá até às 11 horas.\textbackslash{}n\textbackslash{}nChegando à rua o médico estava\textbackslash{}ncompletamente namorado. Madalena tinha-o atado no seu carro, e o pobre rapaz\textbackslash{}nnem vontade tinha de quebrar o jugo.\textbackslash{}n\textbackslash{}nCaminhando para casa ia ele formando\textbackslash{}nprojetos: via-se casado com ela, amado e amante, causando inveja a todos, e\textbackslash{}nmais que tudo feliz no seu interior.\textbackslash{}n\textbackslash{}nQuando chegou à casa, lembrou-se de\textbackslash{}nescrever uma carta que mandaria no dia seguinte a Meneses. Escreveu cinco e\textbackslash{}nrasgou-as todas.\textbackslash{}n\textbackslash{}nAfinal redigiu um simples bilhete nestes\textbackslash{}ntermos:\textbackslash{}n\textbackslash{}nMeu amigo.\textbackslash{}n\textbackslash{}nVocê tem razão; na minha idade crê-se; eu\textbackslash{}ncreio e amo. Nunca o pensei; mas é verdade. Amo{\ldots} Quer saber a quem? Hei de\textbackslash{}napresentá-lo em casa dela. Há de achá-la bonita{\ldots} Se o é!{\ldots}\textbackslash{}n\textbackslash{}nA carta dizia muitas coisas mais; era\textbackslash{}ntudo, porém, uma glosa do mesmo mote.\textbackslash{}n\textbackslash{}nEstêvão voltou à casa de Madalena e as\textbackslash{}nsuas visitas começaram a ser regulares e assíduas.\textbackslash{}n\textbackslash{}nA viúva usava para com ele de tanta\textbackslash{}nsolicitude que não era possível duvidar do sentimento que a dirigia. Pelo menos\textbackslash{}nEstêvão assim o pensava. Achava-se quase sempre só, e deliciava-se em ouvi-la.\textbackslash{}nA intimidade começou a estabelecer-se.\textbackslash{}n\textbackslash{}nLogo na segunda visita, Estêvão falou-lhe\textbackslash{}nem Meneses pedindo licença para apresentá-lo. A viúva disse que teria muito\textbackslash{}nprazer em receber amigos de Estêvão; mas pedia-lhe que adiasse a apresentação.\textbackslash{}nTodos os pedidos e todas as razões de Madalena eram dignas para o médico; não\textbackslash{}ndisse mais nada.\textbackslash{}n\textbackslash{}nComo era natural, ao passo que as visitas\textbackslash{}nà viúva eram mais assíduas, as visitas ao amigo eram mais raras.\textbackslash{}n\textbackslash{}nMeneses não se queixou; compreendeu, e\textbackslash{}ndisse-o ao rapaz.\textbackslash{}n\textbackslash{}n— Não se desculpe, acrescentou o\textbackslash{}ndeputado; é natural; a amizade deve ceder o passo ao amor. O que eu quero é que\textbackslash{}nseja feliz.\textbackslash{}n\textbackslash{}nUm dia Estêvão pediu ao amigo que lhe\textbackslash{}ncontasse o motivo que o tinha feito descrer do amor, e se algum grande\textbackslash{}ninfortúnio lhe havia acontecido.\textbackslash{}n\textbackslash{}n— Nada me aconteceu, disse Meneses.\textbackslash{}n\textbackslash{}nMas ao mesmo tempo, compreendendo que o\textbackslash{}nmédico merecia-lhe toda a confiança, e podia não acreditá-lo absolutamente,\textbackslash{}ndisse:\textbackslash{}n\textbackslash{}n— Por que negá-lo? Sim, aconteceu-me um\textbackslash{}ngrande infortúnio; amei também, mas não encontrei no amor as doçuras e a\textbackslash{}ndignidade do sentimento; enfim, é um drama íntimo de que não quero falar:\textbackslash{}nlimite-se a pateá-lo.\textbackslash{}n\textbackslash{}nCAPÍTULO VII\textbackslash{}n\textbackslash{}n— Quando quiser que eu lhe apresente o\textbackslash{}nmeu amigo Meneses{\ldots} dizia Estêvão uma noite à viúva Madalena.\textbackslash{}n\textbackslash{}n— Ah! é verdade; um dia destes. Vejo que\textbackslash{}no senhor é amigo dele.\textbackslash{}n\textbackslash{}n— Somos amigos íntimos.\textbackslash{}n\textbackslash{}n— Verdadeiros?\textbackslash{}n\textbackslash{}n— Verdadeiros.\textbackslash{}n\textbackslash{}nMadalena sorriu; e como estava brincando\textbackslash{}ncom os cabelos do filho deu-lhe um beijo na testa.\textbackslash{}n\textbackslash{}nA criança riu alegremente e abraçou a\textbackslash{}nmãe.\textbackslash{}n\textbackslash{}nA idéia de vir a ser pai honorário do\textbackslash{}npequeno apresentou-se ao espírito de Estêvão. Contemplou-o, chamou por ele,\textbackslash{}nacariciou-o e deu-lhe um beijo no mesmo lugar em que pousaram os lábios de\textbackslash{}nMadalena.\textbackslash{}n\textbackslash{}nEstêvão tocava piano, e às vezes\textbackslash{}nexecutava algum pedaço de música a pedido de Madalena.\textbackslash{}n\textbackslash{}nNessas e noutras distrações lá passavam\textbackslash{}nas horas.\textbackslash{}n\textbackslash{}nO amor não adiantava um passo.\textbackslash{}n\textbackslash{}nPodiam ser ambos duas crateras prestes a\textbackslash{}nrebentar a lava; mas até então não davam o menor sinal de si.\textbackslash{}n\textbackslash{}nEsta situação incomodava o rapaz,\textbackslash{}nacanhava-o, e fazia-o sofrer; mas quando ele pensava em dar um ataque decisivo,\textbackslash{}nera exatamente quando se mostrava mais covarde e poltrão.\textbackslash{}n\textbackslash{}nEra o primeiro amor do rapaz: ele nem\textbackslash{}nconhecia as palavras próprias desse sentimento.\textbackslash{}n\textbackslash{}nUm dia resolveu escrever à viúva.\textbackslash{}n\textbackslash{}n'É melhor, pensava ele; uma carta é\textbackslash{}neloqüente e tem a grande vantagem de deixar a gente longe.'\textbackslash{}n\textbackslash{}nEntrou para o gabinete e começou uma\textbackslash{}ncarta.\textbackslash{}n\textbackslash{}nGastou nisso uma hora; cada frase\textbackslash{}nocupava-lhe muito tempo. Estêvão queria fugir à hipótese de ser classificado\textbackslash{}ncomo tolo ou como sensual. Queria que a carta não respirasse sentimentos frívolos\textbackslash{}nnem maus; queria revelar-se puro como era.\textbackslash{}n\textbackslash{}nMas de que não dependem às vezes os\textbackslash{}nacontecimentos? Estêvão estava relendo e emendando a carta quando lhe entrou\textbackslash{}npor casa um rapazola que tinha intimidade com ele. Chamava-se Oliveira e\textbackslash{}npassava por ser o primeiro janota do Rio de Janeiro.\textbackslash{}n\textbackslash{}nEntrou com um rolo de papel na mão.\textbackslash{}n\textbackslash{}nEstêvão escondeu rapidamente a carta.\textbackslash{}n\textbackslash{}n— Adeus, Estêvão! disse o recém-chegado.\textbackslash{}nEstavas escrevendo algum libelo ou carta de namoro?\textbackslash{}n\textbackslash{}n— Nem uma nem outra coisa, respondeu\textbackslash{}nEstêvão secamente.\textbackslash{}n\textbackslash{}n— Dou-te uma notícia.\textbackslash{}n\textbackslash{}n— Que é?\textbackslash{}n\textbackslash{}n— Entrei na literatura.\textbackslash{}n\textbackslash{}n— Ah!\textbackslash{}n\textbackslash{}n— É verdade, e venho ler-te a primeira\textbackslash{}ncomédia.\textbackslash{}n\textbackslash{}n— Deus me livre! disse Estêvão\textbackslash{}nlevantando-se.\textbackslash{}n\textbackslash{}n— Hás de ouvir, meu amigo; ao menos\textbackslash{}nalgumas cenas; dar-se-á caso que não me protejas nas letras? Anda cá; ao menos\textbackslash{}nduas cenas. Sim? É pouca coisa.\textbackslash{}n\textbackslash{}nEstêvão sentou-se.\textbackslash{}n\textbackslash{}nO dramaturgo continuou:\textbackslash{}n\textbackslash{}n— Talvez prefiras ouvir a minha tragédia\textbackslash{}nintitulada — O Punhal de Bruto{\ldots}\textbackslash{}n\textbackslash{}n— Não, não; prefiro a comédia: é menos\textbackslash{}nsanguinária. Vamos lá.\textbackslash{}n\textbackslash{}nO Oliveira abriu o rolo, arranjou as\textbackslash{}nfolhas, tossiu e começou a ler o que se segue, com voz pausada e fanhosa:\textbackslash{}n\textbackslash{}nCENA I\textbackslash{}n\textbackslash{}nCÉSAR (entrando pela direita);\textbackslash{}nJOÃO (pela esquerda)\textbackslash{}n\textbackslash{}nCÉSAR — Fechada! A sinhá já se levantou?\textbackslash{}n\textbackslash{}nJOÃO — Já, sim senhor; mas está\textbackslash{}nincomodada.\textbackslash{}n\textbackslash{}nCÉSAR — O que tem?\textbackslash{}n\textbackslash{}nJOÃO — Tem{\ldots} está incomodada.\textbackslash{}n\textbackslash{}nCÉSAR — Já sei. (Consigo) 'Os\textbackslash{}nincômodos do Costume'. (A João) Qual é então o remédio hoje?\textbackslash{}n\textbackslash{}nJOÃO — O remédio? (Depois de uma pausa)\textbackslash{}nNão sei.\textbackslash{}n\textbackslash{}nCÉSAR — Está bom, vai-te!\textbackslash{}n\textbackslash{}nCENA II\textbackslash{}n\textbackslash{}nCÉSAR, FREITAS (pela direita)\textbackslash{}n\textbackslash{}nCÉSAR — Bom dia. Sr. procurador{\ldots}\textbackslash{}n\textbackslash{}nFREITAS — De causas perdidas. Só me ocupo\textbackslash{}nem procurar as perdidas. Procurar o que se não perdeu é tolice. A minha constituinte?\textbackslash{}n\textbackslash{}nCÉSAR — Disse-me o João que está\textbackslash{}nincomodada.\textbackslash{}n\textbackslash{}nFREITAS — Mesmo para V.Sa.?\textbackslash{}n\textbackslash{}nCÉSAR — (Sentando-se) Mesmo para\textbackslash{}nmim. Por que me olha com esse olhar? Tem inveja?\textbackslash{}n\textbackslash{}nFREITAS — Não é inveja, é admiração! De\textbackslash{}nordinário ninguém corresponde ao nome que recebeu na pia; mas o Sr. César,\textbackslash{}nbenza-o Deus, não desmente que traz um nome significativo, e trata de ser nas\textbackslash{}npáginas amorosas o que foi o outro nas batalhas campais.\textbackslash{}n\textbackslash{}nCÉSAR — Pois também os procuradores dizem\textbackslash{}ncoisas destas?\textbackslash{}n\textbackslash{}nFREITAS — De vez em quando. (Indo\textbackslash{}nsentar-se) V.Sa. admira-se?\textbackslash{}n\textbackslash{}nCÉSAR — (Tirando charutos) Como\textbackslash{}nnão é de costume{\ldots} quer um charuto?\textbackslash{}n\textbackslash{}nFREITAS — Obrigado{\ldots} Eu tomo rapé. (Tira\textbackslash{}na boceta) Quer uma pitada?\textbackslash{}n\textbackslash{}nCÉSAR — Obrigado.\textbackslash{}n\textbackslash{}nFREITAS — (Sentando-se) Pois a\textbackslash{}ncausa da minha constituinte vai às mil maravilhas. A parte contrária requereu\textbackslash{}nassinação de dez dias, mas eu vou{\ldots}\textbackslash{}n\textbackslash{}nCÉSAR — Está bom, Sr. Freitas, eu\textbackslash{}ndispenso o resto; ou então não me fale linguagem do foro. Em resumo, ela vence?\textbackslash{}n\textbackslash{}nFREITAS — Está claro. Tratando provar\textbackslash{}nque{\ldots}\textbackslash{}n\textbackslash{}nCÉSAR — Vence, é quanto basta.\textbackslash{}n\textbackslash{}nFREITAS — Pudera não vencer! Pois se eu\textbackslash{}nando nisto{\ldots}\textbackslash{}n\textbackslash{}nCÉSAR — Tanto melhor!\textbackslash{}n\textbackslash{}nFREITAS — Ainda não me lembro de ter perdido\textbackslash{}numa só causa: isto é, já perdi uma, mas é porque nas vésperas de ganhar\textbackslash{}ndisse-me o constituinte que desejava perdê-la. Dito e feito. Provei o contrário\textbackslash{}ndo que já tinha provado, e perdi{\ldots} ou antes, ganhei, porque perder assim é\textbackslash{}nganhar.\textbackslash{}n\textbackslash{}nCÉSAR — É a fênix dos procuradores.\textbackslash{}n\textbackslash{}nFREITAS — (Modestamente) São os\textbackslash{}nseus bons olhos{\ldots}\textbackslash{}n\textbackslash{}nCÉSAR — Mas a consciência?\textbackslash{}n\textbackslash{}nFREITAS — Quem é a consciência?\textbackslash{}n\textbackslash{}nCÉSAR — A consciência, a sua consciência?\textbackslash{}n\textbackslash{}nFREITAS — A minha consciência? Ah! essa\textbackslash{}ntambém ganha.\textbackslash{}n\textbackslash{}nCÉSAR — (Levantando-se) Ah!\textbackslash{}ntambém?{\ldots}\textbackslash{}n\textbackslash{}nFREITAS — (O mesmo) Tem V.\textbackslash{}nSA.alguma demandazinha?\textbackslash{}n\textbackslash{}nCÉSAR — Não, não, não tenho; mas, quando\textbackslash{}ntiver, fique descansado, vou bater à sua porta{\ldots}\textbackslash{}n\textbackslash{}nFREITAS — Sempre às ordens de V. Sa.\textbackslash{}n\textbackslash{}nCAPÍTULO VIII\textbackslash{}n\textbackslash{}nEstêvão interrompeu violentamente a\textbackslash{}nleitura, o que desgostou bastante ao poeta novel. O pobre candidato às musas\textbackslash{}nmal pôde balbuciar uma súplica; Estêvão mostrou-se surdo, e o mais que lhe\textbackslash{}nconcedeu foi ficar com a comédia para lê-la depois.\textbackslash{}n\textbackslash{}nOliveira contentou-se com isso; mas não\textbackslash{}nse retirou sem recitar-lhe de cor uma fala do protagonista da tragédia, em\textbackslash{}nversos duros e compridos, dando-lhe por quebra uma estrofe de uma poesia\textbackslash{}nlírica, no estilo do Djinns de Vítor Hugo.\textbackslash{}n\textbackslash{}nEnfim saiu.\textbackslash{}n\textbackslash{}nEntretanto havia passado o tempo.\textbackslash{}n\textbackslash{}nEstêvão releu a carta e quis ainda\textbackslash{}nmandá-la; mas a interrupção do poeta fora proveitosa; relendo a carta, Estêvão\textbackslash{}nachou-a fria e nula; a linguagem era ardente, mas não lhe correspondia ao fogo\textbackslash{}ndo coração.\textbackslash{}n\textbackslash{}n— É inútil, disse ele rasgando a carta em\textbackslash{}nmil pedaços, a língua humana há de ser sempre impotente para exprimir certos\textbackslash{}nafetos da alma; tudo aquilo era frio e diferente do que sinto. Estou condenado\textbackslash{}na não dizer nada ou a dizer mal. Ao pé dela não tenho forças, sinto-me fraco{\ldots}\textbackslash{}n\textbackslash{}nEstêvão parou diante da janela que dava\textbackslash{}npara a rua, no momento em que passava um antigo colega dele, com a mulher de\textbackslash{}nbraço, a mulher que era bonita, e com quem se casara um mês antes.\textbackslash{}n\textbackslash{}nOs dois iam alegres e felizes.\textbackslash{}n\textbackslash{}nEstêvão contemplou aquele quadro com\textbackslash{}nadoração e tristeza. O casamento já não era para ele aquele impossível de que\textbackslash{}nfalava quando apenas tinha idéias e não sentimentos. Agora era uma ventura\textbackslash{}nrealizável.\textbackslash{}n\textbackslash{}nO casal que passara dera-lhe nova força.\textbackslash{}n\textbackslash{}n— É preciso acabar com isto, dizia ele;\textbackslash{}neu não posso deixar de ir àquela mulher e dizer-lhe que a amo, que a adoro, que\textbackslash{}ndesejo ser seu marido. Ela amar-me-á, se já me não ama: sim, ama-me{\ldots}\textbackslash{}n\textbackslash{}nE começou a vestir-se.\textbackslash{}n\textbackslash{}nQuando calçava as luvas e lançava um\textbackslash{}nolhar para o relógio, o criado trouxe-lhe uma carta.\textbackslash{}n\textbackslash{}nEra de Madalena.\textbackslash{}n\textbackslash{}nEspero, meu caro doutor, que não deixe de\textbackslash{}nvir hoje; esperei-o ontem em vão. Desejo falar-lhe.\textbackslash{}n\textbackslash{}nEstêvão acabou de ler este bilhete na\textbackslash{}nescada, com tal pressa descia e tal urgência tinha de achar-se em casa da\textbackslash{}nviúva.\textbackslash{}n\textbackslash{}nO que ele não queria era perder aquele\textbackslash{}nassomo de coragem. Partiu.\textbackslash{}n\textbackslash{}nQuando chegou à casa de Madalena\textbackslash{}nachava-se esta à janela. Recebeu-o com a costumada afabilidade. Estêvão\textbackslash{}ndesculpou-se como pôde por não ter podido vir na véspera, acrescentando que só\textbackslash{}ncom desgosto do seu coração havia faltado.\textbackslash{}n\textbackslash{}nQue melhor ocasião do que era essa para\textbackslash{}nlançar a bomba de uma declaração franca e apaixonada? Estêvão hesitou alguns\textbackslash{}nsegundos; mas tomando ânimo, ia continuar o período, quando a viúva lhe disse:\textbackslash{}n\textbackslash{}n— Estava ansiosa por vê-lo para\textbackslash{}ncomunicar-lhe uma coisa de certa importância, e que só a um homem de honra,\textbackslash{}ncomo o senhor, se pode confiar.\textbackslash{}n\textbackslash{}nEstêvão empalideceu.\textbackslash{}n\textbackslash{}n— Sabe onde foi que eu o vi pela primeira\textbackslash{}nvez?\textbackslash{}n\textbackslash{}n— No baile de ***.\textbackslash{}n\textbackslash{}n— Não; foi antes disso; foi no Teatro Lírico.\textbackslash{}n\textbackslash{}n— Ah!\textbackslash{}n\textbackslash{}n— Lá o vi com o seu amigo Meneses.\textbackslash{}n\textbackslash{}n— Fomos algumas vezes lá!\textbackslash{}n\textbackslash{}nMadalena entrou então em uma longa\textbackslash{}nexposição, que o rapaz ouviu sem pestanejar, mas pálido e agitado por comoções\textbackslash{}níntimas. As últimas palavras da viúva foram estas:\textbackslash{}n\textbackslash{}n— Bem vê, senhor; coisas destas só uma\textbackslash{}ngrande alma pode ouvi-las. As pequenas não as compreendem. Se lhe mereço alguma\textbackslash{}ncoisa, e se esta confiança pode ser paga com um benefício, peço-lhe que faça o\textbackslash{}nque lhe pedi.\textbackslash{}n\textbackslash{}nO médico passou a mão pelos olhos, e\textbackslash{}napenas murmurou:\textbackslash{}n\textbackslash{}n— Mas{\ldots}\textbackslash{}n\textbackslash{}nNeste momento entrava na sala o filhinho\textbackslash{}nde Madalena; a viúva levantou-se e trouxe-o pela mão até o lugar onde se achava\textbackslash{}nEstêvão Soares.\textbackslash{}n\textbackslash{}n— Se não por mim, disse ela, ao menos por\textbackslash{}nesta criança inocente!\textbackslash{}n\textbackslash{}nA criança, sem nada compreender,\textbackslash{}natirou-se aos braços de Estêvão.\textbackslash{}n\textbackslash{}nO moço deu-lhe um beijo na testa, e disse\textbackslash{}npara a viúva:\textbackslash{}n\textbackslash{}n— Se hesitei não foi porque duvidasse do\textbackslash{}nque a senhora acaba de contar-me; foi porque a missão é espinhosa; mas prometo\textbackslash{}nque hei de cumpri-la.\textbackslash{}n\textbackslash{}nCAPÍTULO IX\textbackslash{}n\textbackslash{}nEstêvão saiu da casa da viúva agitado por\textbackslash{}ndiversos sentimentos, com passo trêmulo e a vista turva. A conversa com a viúva\textbackslash{}nfora um longo combate; a última promessa foi um golpe decisivo e mortal. Estêvão\textbackslash{}nsaía dali como um homem que acabava de matar as suas esperanças em flor;\textbackslash{}ncaminhava ao acaso, precisava de ar e queria meter-se em um quarto sombrio;\textbackslash{}nquisera ao mesmo tempo estar solitário e no meio de imensa multidão.\textbackslash{}n\textbackslash{}nNo caminho encontrou Oliveira, o poeta\textbackslash{}nnovel.\textbackslash{}n\textbackslash{}nLembrou-se que a leitura da comédia\textbackslash{}nimpedira a remessa da carta, e portanto poupou-lhe um tristíssimo desengano.\textbackslash{}n\textbackslash{}nEstêvão involuntariamente abraçou o poeta\textbackslash{}ncom toda a efusão d'alma.\textbackslash{}n\textbackslash{}nOliveira correspondeu ao abraço, e quando\textbackslash{}npôde desligar-se do médico, disse-lhe:\textbackslash{}n\textbackslash{}n— Obrigado, meu amigo; estas\textbackslash{}nmanifestações são muito honrosas para mim; sempre te conheci como um perfeito\textbackslash{}njuiz literário, e a prova que acabas de dar-me é uma consolação e uma animação;\textbackslash{}nconsola-me do que tenho sofrido, anima-me para novos cometimentos. Se Torquato\textbackslash{}nTasso{\ldots}\textbackslash{}n\textbackslash{}nDiante desta ameaça de discurso, e\textbackslash{}nsobretudo vendo a interpretação do seu abraço, Estêvão resolveu-se a continuar\textbackslash{}ncaminho abandonando o poeta.\textbackslash{}n\textbackslash{}n— Adeus, tenho pressa\textbackslash{}n\textbackslash{}n— Adeus, obrigado! Estêvão chegou à casa\textbackslash{}ne atirou-se à cama. Ninguém o soube nunca, só as paredes do quarto foram\textbackslash{}ntestemunhas; mas a verdade é que Estêvão chorou lágrimas amargas.\textbackslash{}n\textbackslash{}nEnfim que lhe dissera Madalena e que\textbackslash{}nexigira dele?\textbackslash{}n\textbackslash{}nA viúva não era viúva; era mulher de\textbackslash{}nMeneses; viera do Norte meses antes do marido, que só veio como deputado;\textbackslash{}nMeneses, que a amava doidamente, e que era amado com igual delírio, acusava-a\textbackslash{}nde infidelidade; uma carta e um retrato eram os indícios; ela negou, mas\textbackslash{}nexplicou-se mal; o marido separou-se e mandou-a para o Rio de Janeiro.\textbackslash{}n\textbackslash{}nMadalena aceitou a situação com\textbackslash{}nresignação e coragem: não murmurou nem pediu, cumpriu a ordem do marido.\textbackslash{}n\textbackslash{}nTodavia Madalena não era criminosa; o seu\textbackslash{}ncrime era uma aparência; estava condenada por fidelidade de honra. A carta e o\textbackslash{}nretrato não lhe pertenciam; eram apenas um depósito imprudente e fatal.\textbackslash{}nMadalena podia dizer tudo, mas era trair uma promessa; não quis; preferiu que a\textbackslash{}ntempestade doméstica caísse unicamente sobre ela.\textbackslash{}n\textbackslash{}nAgora, porém, a necessidade do segredo\textbackslash{}nexpirara; Madalena recebeu do Norte uma carta em que a amiga, no leito da\textbackslash{}nmorte, pedia que inutilizasse a carta e o retrato, ou os restituísse ao homem\textbackslash{}nque lhos dera. Essa carta era uma justificação.\textbackslash{}n\textbackslash{}nMadalena podia mandar a carta ao marido,\textbackslash{}nou pedir-lhe uma entrevista; mas receava tudo; sabia que seria inútil, porque\textbackslash{}nMeneses era extremamente severo.\textbackslash{}n\textbackslash{}nVira o médico uma noite no teatro em\textbackslash{}ncompanhia de seu marido; indagara e soube que eram amigos; pedia-lhe pois que\textbackslash{}nfosse mediador entre os dois, que a salvasse e que reconstruísse uma família.\textbackslash{}n\textbackslash{}nNão era pois somente o amor de Estêvão\textbackslash{}nque sofria; era também o seu amor-próprio. Estêvão facilmente compreendeu que\textbackslash{}nnão fora atraído àquela casa para outra coisa. É verdade que a carta só chegara\textbackslash{}nna véspera; mas a carta apenas vinha apressar a resolução. Naturalmente\textbackslash{}nMadalena pedir-lhe-ia, sem haver carta, algum serviço análogo àquele.\textbackslash{}n\textbackslash{}nSe se tratasse de qualquer outro homem,\textbackslash{}nEstêvão recusaria o serviço que lhe pedia a viúva, mas tratava-se do seu\textbackslash{}namigo, de um homem a quem ele devia estima e serviços de amizade.\textbackslash{}n\textbackslash{}nAceitou, pois, a cruel missão.\textbackslash{}n\textbackslash{}n— Cumpra-se o destino, disse ele; hei de\textbackslash{}nir lançar a mulher que amo aos braços de outro; e por desgraça maior, em vez de\textbackslash{}ngozar com este restabelecimento de concórdia doméstica, vejo-me na dura\textbackslash{}nsituação de amar a mulher do meu amigo, isto é, de fugir para longe{\ldots}\textbackslash{}n\textbackslash{}nEstêvão não saiu mais de casa nesse dia.\textbackslash{}n\textbackslash{}nQuis escrever ao deputado contando-lhe tudo;\textbackslash{}nmas pensou que o melhor era falar-lhe de viva voz. Embora lhe custasse mais,\textbackslash{}nera de mais efeito para o desempenho da sua promessa.\textbackslash{}n\textbackslash{}nAdiou, porém, para o dia seguinte, ou\textbackslash{}nantes para o mesmo dia, porque a noite não lhe interrompeu o tempo, visto que Estêvão\textbackslash{}nnão dormiu um minuto sequer.\textbackslash{}n\textbackslash{}nCAPÍTULO X\textbackslash{}n\textbackslash{}nLevantou-se da cama o pobre namorado sem\textbackslash{}nter conseguido dormir. Vinha nascendo o sol.\textbackslash{}n\textbackslash{}nQuis ler os jornais e pediu-os.\textbackslash{}n\textbackslash{}nJá os ia pondo de lado, por haver acabado\textbackslash{}nde ler, quando repentinamente viu o seu nome impresso no Jornal do Comércio.\textbackslash{}n\textbackslash{}nEra um artigo a pedido com o\textbackslash{}ntítulo de 'Uma Obra-Prima.'\textbackslash{}n\textbackslash{}nDizia o artigo:\textbackslash{}n\textbackslash{}nTemos o prazer de anunciar ao país o\textbackslash{}npróximo aparecimento de uma excelente comédia, estréia de um jovem literato\textbackslash{}nfluminense, de nome Antônio Carlos de Oliveira.\textbackslash{}n\textbackslash{}nEste robusto talento, por muito tempo\textbackslash{}nincógnito, vai enfim entrar nos mares da publicidade, e para isso procurou logo\textbackslash{}nensaiar-se em uma obra de certo vulto.\textbackslash{}n\textbackslash{}nConsta-nos que o autor, solicitado por\textbackslash{}nseus numerosos amigos, leu há dias a comédia em casa do Sr. Dr. Estêvão Soares,\textbackslash{}ndiante de um luzido auditório, que aplaudiu muito e profetizou no Sr. Oliveira\textbackslash{}num futuro Shakespeare.\textbackslash{}n\textbackslash{}nO Sr. Dr. Estêvão Soares levou a sua\textbackslash{}namabilidade a ponto de pedir a comédia para ler segunda vez, e ontem ao\textbackslash{}nencontrar-se na rua com o Sr. Oliveira, de tal entusiasmo vinha possuído que o\textbackslash{}nabraçou estreitamente, com grande pasmo dos numerosos transeuntes.\textbackslash{}n\textbackslash{}nDa parte de um juiz tão competente em\textbackslash{}nmatérias literárias este ato é honroso para o Sr. Oliveira.\textbackslash{}n\textbackslash{}nEstamos ansiosos por ler a peça do Sr.\textbackslash{}nOliveira, e ficamos certos de que ela fará fortuna de qualquer teatro.\textbackslash{}n\textbackslash{}nO AMIGO DAS LETRAS.\textbackslash{}n\textbackslash{}nEstêvão, apesar dos sentimentos que o agitavam\textbackslash{}nentão, enfureceu-se com o artigo que acabava de ler. Não havia dúvida que o\textbackslash{}nautor dele era o próprio autor da comédia. O abraço da véspera fora mal\textbackslash{}ninterpretado, e o poetastro aproveitava-o em seu favor. Se ao menos não falasse\textbackslash{}nno nome de Estêvão, este poderia desculpar a vaidadezinha do escritor. Mas o\textbackslash{}nnome ali estava como cúmplice da obra.\textbackslash{}n\textbackslash{}nPondo de lado o Jornal do Comércio,\textbackslash{}nEstêvão lembrou-se de protestar, e ia já escrever um artigo quando recebeu uma\textbackslash{}ncartinha de Oliveira.\textbackslash{}n\textbackslash{}nDizia a carta:\textbackslash{}n\textbackslash{}nMeu Estêvão.\textbackslash{}n\textbackslash{}nLembrou-se um amigo meu de escrever\textbackslash{}nalguma coisa a propósito da minha peça. Expliquei-lhe como se dera a leitura em\textbackslash{}ntua casa, e disse-lhe como é que, apesar do vivo desejo que tinhas de ouvir\textbackslash{}nlê-la, interrompeste-me para ir cuidar de um doente. Apesar de tudo isto, o meu\textbackslash{}nreferido amigo contou hoje no Jornal do Comércio a história alterando\textbackslash{}num pouco a verdade. Desculpa-o; é a linguagem da amizade e da benevolência.\textbackslash{}n\textbackslash{}nOntem entrei para casa tão orgulhoso com\textbackslash{}no teu abraço que escrevi uma ode, e assim manifestou-se em mim a veia lírica,\textbackslash{}ndepois da cômica e da trágica. Aí te mando o rascunho; se não prestar, rasga-a.\textbackslash{}n\textbackslash{}nA carta tinha, por engano, a data da\textbackslash{}nvéspera.\textbackslash{}n\textbackslash{}nA ode era muito comprida; Estêvão nem a\textbackslash{}nleu, atirou-a para um canto.\textbackslash{}n\textbackslash{}nA ode começava assim:\textbackslash{}n\textbackslash{}nSai do teu monte, ó musa!\textbackslash{}n\textbackslash{}nVem inspirar a lira do poeta;\textbackslash{}n\textbackslash{}nEnche de luz a minha fronte ousada,\textbackslash{}n\textbackslash{}nE mandemos aos evos,\textbackslash{}n\textbackslash{}nNas asas de uma estrofe igente e\textbackslash{}naltíssona,\textbackslash{}n\textbackslash{}nDo caro amigo o animador abraço!\textbackslash{}n\textbackslash{}nNão canto os altos feitos\textbackslash{}n\textbackslash{}nDe Aquiles, nem traduzo os sons tremendos\textbackslash{}n\textbackslash{}nDos rufos marciais enchendo os campos!\textbackslash{}n\textbackslash{}nOutro assunto me inspira.\textbackslash{}n\textbackslash{}nNão canto a espada que dá morte e campa;\textbackslash{}n\textbackslash{}nCanto o abraço que dá vida e glória!\textbackslash{}n\textbackslash{}nCAPÍTULO XI\textbackslash{}n\textbackslash{}nComo havia prometido, Estêvão foi logo\textbackslash{}nprocurar o deputado Meneses. Em vez de ir direito ao fim, quis antes sondá-lo a\textbackslash{}nrespeito do seu passado. Era a primeira vez que o moço tocava em tal. Meneses\textbackslash{}nnão desconfiou, mas estranhou; mas tal confiança tinha nele que não recusou\textbackslash{}nnada.\textbackslash{}n\textbackslash{}n— Sempre imaginei, dissera-lhe Estêvão,\textbackslash{}nque há na sua vida um drama. E talvez engano meu, mas a verdade é que ainda não\textbackslash{}nperdi a idéia.\textbackslash{}n\textbackslash{}n— Há, com efeito, um drama; mas um drama\textbackslash{}npateado. Não sorria; é assim. Que supõe então?\textbackslash{}n\textbackslash{}n— Não suponho nada. Imagino que{\ldots}\textbackslash{}n\textbackslash{}n— Pede dramas a um homem político?\textbackslash{}n\textbackslash{}n— Por que não?\textbackslash{}n\textbackslash{}n— Eu lhe digo. Sou político e não sou.\textbackslash{}nNão entrei na vida pública por vocação; entrei como se entra em uma sepultura:\textbackslash{}npara dormir melhor. Por que o fiz? A razão é o drama de que me fala.\textbackslash{}n\textbackslash{}n— Uma mulher, talvez{\ldots}\textbackslash{}n\textbackslash{}n— Sim, uma mulher.\textbackslash{}n\textbackslash{}n— Talvez mesmo, disse Estêvão procurando\textbackslash{}nsorrir, talvez uma esposa.\textbackslash{}n\textbackslash{}nMeneses estremeceu e olhou para o amigo,\textbackslash{}nespantado e desconfiado.\textbackslash{}n\textbackslash{}n— Quem lho disse?\textbackslash{}n\textbackslash{}n— Pergunto.\textbackslash{}n\textbackslash{}n— Uma esposa, sim; mas não lhe direi mais\textbackslash{}nnada. É a primeira pessoa que ouve tanta coisa de mim. Deixemos o passado que\textbackslash{}nmorreu: parce sepultis.\textbackslash{}n\textbackslash{}n— Conforme, disse Estêvão; e se eu\textbackslash{}npertencer a uma seita filosófica que pretenda ressuscitar os mortos, mesmo\textbackslash{}nquando é um passado{\ldots}\textbackslash{}n\textbackslash{}n— As suas palavras, ou querem dizer\textbackslash{}nmuito, ou nada. Qual é a sua intenção?\textbackslash{}n\textbackslash{}n— A minha intenção não é ressuscitar o\textbackslash{}npassado unicamente; é repará-lo, é restaurá-lo em todo o seu esplendor, com\textbackslash{}ntoda a legitimidade do seu direito; o meu fim é dizer-lhe, meu caro amigo, que\textbackslash{}na mulher condenada é uma mulher inocente.\textbackslash{}n\textbackslash{}nOuvindo estas palavras Meneses deu um\textbackslash{}npequeno grito.\textbackslash{}n\textbackslash{}nDepois levantando-se com rapidez pediu a\textbackslash{}nEstêvão que lhe dissesse o que sabia e como sabia.\textbackslash{}n\textbackslash{}nEstêvão referiu tudo.\textbackslash{}n\textbackslash{}nQuando concluiu a sua narração, o deputado\textbackslash{}nabanou a cabeça com aquele último sintoma de incredulidade que é ainda um eco\textbackslash{}ndas grandes catástrofes domésticas.\textbackslash{}n\textbackslash{}nMas Estêvão ia armado contra as objeções\textbackslash{}ndo marido. Protestou energicamente pela defesa da mulher; instou pelo\textbackslash{}ncumprimento do dever.\textbackslash{}n\textbackslash{}nA última resposta de Meneses foi esta:\textbackslash{}n\textbackslash{}n— Meu caro Estêvão, a mulher de César nem\textbackslash{}ndeve ser suspeitada. Acredito em tudo; mas o que está feito, está feito.\textbackslash{}n\textbackslash{}n— O princípio é cruel, meu amigo.\textbackslash{}n\textbackslash{}n— É fatal.\textbackslash{}n\textbackslash{}nEstêvão saiu.\textbackslash{}n\textbackslash{}nFicando só, Meneses caiu em profunda\textbackslash{}nmeditação; ele acreditava em tudo, e amava a mulher; mas não acreditava que os\textbackslash{}nbelos dias pudessem voltar.\textbackslash{}n\textbackslash{}nRecusando, pensava ele, era ficar no\textbackslash{}ntúmulo em que tivera tão brando sono.\textbackslash{}n\textbackslash{}nEstêvão, porém, não desanimou.\textbackslash{}n\textbackslash{}nQuando entrou em casa, escreveu uma longa\textbackslash{}ncarta ao deputado exortando-o a que restaurasse a família um momento separada e\textbackslash{}ndesfeita. Estêvão era eloqüente; o coração de Meneses com pouco se contentava.\textbackslash{}n\textbackslash{}nEnfim, nesta missão diplomática, o médico\textbackslash{}nhouve-se com suprema habilidade. No fim de alguns dias dissipara-se a nuvem do\textbackslash{}npassado, e o casal reunira-se.\textbackslash{}n\textbackslash{}nComo?\textbackslash{}n\textbackslash{}nMadalena soube das disposições de Meneses\textbackslash{}ne recebeu o anúncio de uma visita de seu marido.\textbackslash{}n\textbackslash{}nQuando o deputado preparava-se para sair,\textbackslash{}nvieram dizer-lhe que uma senhora o procurava.\textbackslash{}n\textbackslash{}nA senhora era Madalena.\textbackslash{}n\textbackslash{}nMeneses nem quis abraçá-la;\textbackslash{}najoelhou-se-lhe aos pés.\textbackslash{}n\textbackslash{}nTudo estava esquecido.\textbackslash{}n\textbackslash{}nQuiseram celebrar a reconciliação, e\textbackslash{}nEstêvão foi convidado para lá passar o dia em companhia dos seus amigos, que\textbackslash{}nlhe deviam a felicidade.\textbackslash{}n\textbackslash{}nEstêvão não foi.\textbackslash{}n\textbackslash{}nMas no dia seguinte Meneses recebeu este\textbackslash{}nbilhete:\textbackslash{}n\textbackslash{}nDesculpe, meu amigo, se não vou\textbackslash{}ndespedir-me pessoalmente. Sou obrigado a partir repentinamente para Minas.\textbackslash{}nVoltarei daqui a alguns meses.\textbackslash{}n\textbackslash{}nEstimo que sejam felizes, e espero que\textbackslash{}nnão se esqueçam de mim.\textbackslash{}n\textbackslash{}nMeneses foi apressadamente à casa de\textbackslash{}nEstêvão, e ainda o achou preparando as malas. Achou singular a viagem, e mais\textbackslash{}nsingular o bilhete; mas o médico não revelou por modo nenhum o verdadeiro\textbackslash{}nmotivo da sua partida.\textbackslash{}n\textbackslash{}nQuando Meneses voltou, comunicou à mulher\textbackslash{}nas suas impressões; e perguntou se ela compreendia aquilo.\textbackslash{}n\textbackslash{}n— Não, respondeu Madalena.\textbackslash{}n\textbackslash{}nMas tinha compreendido enfim.\textbackslash{}n\textbackslash{}n'Nobre alma!' disse ela\textbackslash{}nconsigo.\textbackslash{}n\textbackslash{}nNada disse ao marido; nisso mostrava-se esposa\textbackslash{}nsolícita pela tranqüilidade conjugal; mas mostrava-se sobretudo mulher.\textbackslash{}n\textbackslash{}nMeneses não foi à Câmara durante muitos\textbackslash{}ndias, e no primeiro paquete seguiu para o Norte.\textbackslash{}n\textbackslash{}nA ausência transtornou algumas votações,\textbackslash{}ne a sua partida logrou muitos cálculos.\textbackslash{}n\textbackslash{}nMas o homem tem o direito de procurar a\textbackslash{}nsua felicidade e a felicidade de Meneses era independente da política.\textbackslash{}n\textbackslash{}nO SEGREDO DE\textbackslash{}nAUGUSTA\textbackslash{}n\textbackslash{}nÍNDICE\textbackslash{}n\textbackslash{}nCapítulo Primeiro\textbackslash{}n\textbackslash{}nCapítulo II\textbackslash{}n\textbackslash{}nCapítulo iii\textbackslash{}n\textbackslash{}nCapítulo iv\textbackslash{}n\textbackslash{}nCapítulo v\textbackslash{}n\textbackslash{}nCapítulo vI\textbackslash{}n\textbackslash{}nCapítulo vII\textbackslash{}n\textbackslash{}nCAPÍTULO PRIMEIRO\textbackslash{}n\textbackslash{}nSão onze horas da manhã.\textbackslash{}n\textbackslash{}nD. Augusta Vasconcelos está reclinada\textbackslash{}nsobre um sofá, com um livro na mão. Adelaide, sua filha, passa os dedos pelo\textbackslash{}nteclado do piano.\textbackslash{}n\textbackslash{}n— Papai já acordou? pergunta Adelaide à\textbackslash{}nsua mãe.\textbackslash{}n\textbackslash{}n— Não, responde esta sem levantar os\textbackslash{}nolhos do livro.\textbackslash{}n\textbackslash{}nAdelaide levantou-se e foi ter com\textbackslash{}nAugusta.\textbackslash{}n\textbackslash{}n— Mas é tão tarde, mamãe, disse ela. São\textbackslash{}nonze horas. Papai dorme muito.\textbackslash{}n\textbackslash{}nAugusta deixou cair o livro no regaço, e\textbackslash{}ndisse olhando para Adelaide:\textbackslash{}n\textbackslash{}n— É que naturalmente recolheu-se tarde.\textbackslash{}n\textbackslash{}n— Reparei já que nunca me despeço de\textbackslash{}npapai quando me vou deitar. Anda sempre fora.\textbackslash{}n\textbackslash{}nAugusta sorriu.\textbackslash{}n\textbackslash{}n— És uma roceira, disse ela; dormes com\textbackslash{}nas galinhas. Aqui o costume é outro. Teu pai tem que fazer de noite.\textbackslash{}n\textbackslash{}n— É política, mamãe? perguntou Adelaide.\textbackslash{}n\textbackslash{}n— Não sei, respondeu Augusta.\textbackslash{}n\textbackslash{}nComecei dizendo que Adelaide era filha de\textbackslash{}nAugusta, e esta informação, necessária no romance, não o era menos na vida real\textbackslash{}nem que se passou o episódio que vou contar, porque à primeira vista ninguém\textbackslash{}ndiria que havia ali mãe e filha; pareciam duas irmãs, tão jovem era a mulher de\textbackslash{}nVasconcelos.\textbackslash{}n\textbackslash{}nTinha Augusta trinta anos e Adelaide\textbackslash{}nquinze; mas comparativamente a mãe parecia mais moça ainda que a filha.\textbackslash{}nConservava a mesma frescura dos quinze anos, e tinha de mais o que faltava a\textbackslash{}nAdelaide, que era a consciência da beleza e da mocidade; consciência que seria\textbackslash{}nlouvável se não tivesse como conseqüência uma imensa e profunda vaidade. A sua\textbackslash{}nestatura era mediana, mas imponente. Era muito alva e muito corada. Tinha os\textbackslash{}ncabelos castanhos, e os olhos garços. As mãos compridas e bem feitas pareciam\textbackslash{}ncriadas para os afagos de amor. Augusta dava melhor emprego às suas mãos;\textbackslash{}ncalçava-as de macia pelica.\textbackslash{}n\textbackslash{}nAs graças de Augusta estavam todas em\textbackslash{}nAdelaide, mas em embrião. Adivinhava-se que aos vinte anos Adelaide devia\textbackslash{}nrivalizar com Augusta; mas por enquanto havia na menina uns restos da infância\textbackslash{}nque não davam realce aos elementos que a natureza pusera nela.\textbackslash{}n\textbackslash{}nTodavia, era bem capaz de apaixonar um\textbackslash{}nhomem, sobretudo se ele fosse poeta, e gostasse das virgens de quinze anos, até\textbackslash{}nporque era um pouco pálida, e os poetas em todos os tempos tiveram sempre queda\textbackslash{}npara as criaturas descoradas.\textbackslash{}n\textbackslash{}nAugusta vestia com suprema elegância;\textbackslash{}ngastava muito, é verdade; mas aproveitava bem as enormes despesas, se acaso é\textbackslash{}nisso aproveitá-las. Deve-se fazer-lhe uma justiça; Augusta não regateava nunca;\textbackslash{}npagava o preço que lhe pediam por qualquer coisa. Punha nisso a sua grandeza, e\textbackslash{}nachava que o procedimento contrário era ridículo e de baixa esfera.\textbackslash{}n\textbackslash{}nNeste ponto Augusta partilhava os\textbackslash{}nsentimentos e servia aos interesses de alguns mercadores, que entendem ser uma\textbackslash{}ndesonra abater alguma coisa no preço das suas mercadorias.\textbackslash{}n\textbackslash{}nO fornecedor de fazendas de Augusta,\textbackslash{}nquando falava a este respeito, costumava dizer-lhe:\textbackslash{}n\textbackslash{}n— Pedir um preço e dar a fazenda por\textbackslash{}noutro preço menor, é confessar que havia intenção de esbulhar o freguês.\textbackslash{}n\textbackslash{}nO fornecedor preferia fazer a coisa sem a\textbackslash{}nconfissão.\textbackslash{}n\textbackslash{}nOutra justiça que devemos reconhecer era\textbackslash{}nque Augusta não poupava esforços para que Adelaide fosse tão elegante como ela.\textbackslash{}n\textbackslash{}nNão era pequeno o trabalho.\textbackslash{}n\textbackslash{}nAdelaide desde a idade de cinco anos fora\textbackslash{}neducada na roça em casa de uns parentes de Augusta, mais dados ao cultivo do café\textbackslash{}nque às despesas do vestuário. Adelaide foi educada nesses hábitos e nessas\textbackslash{}nidéias. Por isso quando chegou à corte, onde se reuniu à família, houve para\textbackslash{}nela uma verdadeira transformação. Passava de uma civilização para outra; viveu\textbackslash{}nnuma longa série de anos. O que lhe valeu é que tinha em sua mãe uma excelente\textbackslash{}nmestra. Adelaide reformou-se, e no dia em que começa esta narração já era\textbackslash{}noutra; todavia estava ainda muito longe de Augusta.\textbackslash{}n\textbackslash{}nNo momento em que Augusta respondia à\textbackslash{}ncuriosa pergunta de sua filha acerca das ocupações de Vasconcelos, parou um\textbackslash{}ncarro à porta.\textbackslash{}n\textbackslash{}nAdelaide correu à janela.\textbackslash{}n\textbackslash{}n— É D. Carlota, mamãe, disse a menina\textbackslash{}nvoltando-se para dentro.\textbackslash{}n\textbackslash{}nDaí a alguns minutos entrava na sala a D.\textbackslash{}nCarlota em questão. Os leitores ficarão conhecendo esta nova personagem com a\textbackslash{}nsimples indicação de que era um segundo volume de Augusta; bela, como ela;\textbackslash{}nelegante, como ela; vaidosa, como ela.\textbackslash{}n\textbackslash{}nTudo isto quer dizer que eram ambas as\textbackslash{}nmais afáveis inimigas que pode haver neste mundo.\textbackslash{}n\textbackslash{}nCarlota vinha pedir a Augusta para ir\textbackslash{}ncantar num concerto que ia dar em casa, imaginado por ela para o fim de\textbackslash{}ninaugurar um magnífico vestido novo.\textbackslash{}n\textbackslash{}nAugusta de boa vontade acedeu ao pedido.\textbackslash{}n\textbackslash{}n— Como está seu marido? perguntou ela a\textbackslash{}nCarlota.\textbackslash{}n\textbackslash{}n— Foi para a praça; e o seu?\textbackslash{}n\textbackslash{}n— O meu dorme.\textbackslash{}n\textbackslash{}n— Como um justo? perguntou Carlota\textbackslash{}nsorrindo maliciosamente.\textbackslash{}n\textbackslash{}n— Parece, respondeu Augusta.\textbackslash{}n\textbackslash{}nNeste momento, Adelaide, que por pedido\textbackslash{}nde Carlota tinha ido tocar um noturno ao piano, voltou para o grupo.\textbackslash{}n\textbackslash{}nA amiga de Augusta perguntou-lhe:\textbackslash{}n\textbackslash{}n— Aposto que já tem algum noivo em vista?\textbackslash{}n\textbackslash{}nA menina corou muito, e balbuciou:\textbackslash{}n\textbackslash{}n— Não fale nisso.\textbackslash{}n\textbackslash{}n— Ora, há de ter! Ou então aproxima-se da\textbackslash{}népoca em que há de ter um noivo, e eu já lhe profetizo que há de ser bonito{\ldots}\textbackslash{}n\textbackslash{}n— É muito cedo, disse Augusta.\textbackslash{}n\textbackslash{}n— Cedo!\textbackslash{}n\textbackslash{}n— Sim, está muito criança; casar-se-á\textbackslash{}nquando for tempo, e o tempo está longe{\ldots}\textbackslash{}n\textbackslash{}n— Já sei, disse Carlota rindo, quer\textbackslash{}nprepará-la bem{\ldots} Aprovo-lhe a intenção. Mas nesse caso não lhe tire as\textbackslash{}nbonecas.\textbackslash{}n\textbackslash{}n— Já não as tem.\textbackslash{}n\textbackslash{}n— Então é difícil impedir os namorados.\textbackslash{}nUma coisa substitui a outra.\textbackslash{}n\textbackslash{}nAugusta sorriu, e Carlota levantou-se\textbackslash{}npara sair.\textbackslash{}n\textbackslash{}n— Já? disse Augusta.\textbackslash{}n\textbackslash{}n— É preciso; adeus!\textbackslash{}n\textbackslash{}n— Adeus!\textbackslash{}n\textbackslash{}nTrocaram-se alguns beijos e Carlota saiu\textbackslash{}nlogo.\textbackslash{}n\textbackslash{}nLogo depois chegaram dois caixeiros: um com\textbackslash{}nalguns vestidos e outro com um romance; eram encomendas feitas na véspera. Os\textbackslash{}nvestidos eram caríssimos, e o romance tinha este título: Fanny, por\textbackslash{}nErnesto Feydeau.\textbackslash{}n\textbackslash{}nCAPÍTULO II\textbackslash{}n\textbackslash{}nPela uma hora da tarde do mesmo dia\textbackslash{}nlevantou-se Vasconcelos da cama.\textbackslash{}n\textbackslash{}nVasconcelos era um homem de quarenta\textbackslash{}nanos, bem apessoado, dotado de um maravilhoso par de suíças grisalhas, que lhe\textbackslash{}ndavam um ar de diplomata, coisa de que estava afastado umas boas cem léguas.\textbackslash{}nTinha a cara risonha e expansiva; todo ele respirava uma robusta saúde.\textbackslash{}n\textbackslash{}nPossuía uma boa fortuna e não trabalhava,\textbackslash{}nisto é, trabalhava muito na destruição da referida fortuna, obra em que sua\textbackslash{}nmulher colaborava conscienciosamente.\textbackslash{}n\textbackslash{}nA observação de Adelaide era verídica;\textbackslash{}nVasconcelos recolhia-se tarde; acordava sempre depois do meio-dia; e saía às\textbackslash{}nave-marias para voltar na madrugada seguinte. Quer dizer que fazia com\textbackslash{}nregularidade algumas pequenas excursões à casa da família.\textbackslash{}n\textbackslash{}nSó uma pessoa tinha o direito de exigir\textbackslash{}nde Vasconcelos mais alguma assiduidade em casa: era Augusta; mas ela nada lhe\textbackslash{}ndizia. Nem por isso se davam mal, porque o marido em compensação da tolerância\textbackslash{}nde sua esposa não lhe negava nada, e todos os caprichos dela eram de pronto\textbackslash{}nsatisfeitos.\textbackslash{}n\textbackslash{}nSe acontecia que Vasconcelos não pudesse\textbackslash{}nacompanhá-la a todos os passeios e bailes, incumbia-se disso um irmão dele,\textbackslash{}ncomendador de duas ordens, político de oposição, excelente jogador de\textbackslash{}nvoltarete, e homem amável nas horas vagas, que eram bem poucas. O irmão\textbackslash{}nLourenço era o que se pode chamar um irmão terrível. Obedecia a todos os\textbackslash{}ndesejos da cunhada, mas não poupava de quando em quando um sermão ao irmão. Boa\textbackslash{}nsemente que não pegava.\textbackslash{}n\textbackslash{}nAcordou, pois, Vasconcelos, e acordou de\textbackslash{}nbom humor. A filha alegrou-se muito ao vê-lo, e ele mostrou-se de uma grande\textbackslash{}nafabilidade com a mulher, que lhe retribuiu do mesmo modo.\textbackslash{}n\textbackslash{}n— Por que acorda tão tarde? perguntou\textbackslash{}nAdelaide acariciando as suíças de Vasconcelos.\textbackslash{}n\textbackslash{}n— Porque me deito tarde.\textbackslash{}n\textbackslash{}n— Mas por que se deita tarde?\textbackslash{}n\textbackslash{}n— Isso agora é muito perguntar! disse\textbackslash{}nVasconcelos sorrindo.\textbackslash{}n\textbackslash{}nE continuou:\textbackslash{}n\textbackslash{}n— Deito-me tarde porque assim o pedem as\textbackslash{}nnecessidades políticas. Tu não sabes o que é política; é uma coisa muito feia,\textbackslash{}nmas muito necessária.\textbackslash{}n\textbackslash{}n— Sei o que é política, sim! disse\textbackslash{}nAdelaide.\textbackslash{}n\textbackslash{}n— Ah! explica-me lá então o que é.\textbackslash{}n\textbackslash{}n— Lá na roça, quando quebraram a cabeça\textbackslash{}nao juiz de paz, disseram que era por política; o que eu achei esquisito, porque\textbackslash{}na política seria não quebrar a cabeça{\ldots}\textbackslash{}n\textbackslash{}nVasconcelos riu muito com a observação da\textbackslash{}nfilha, e foi almoçar, exatamente quando entrava o irmão, que não pôde deixar de\textbackslash{}nexclamar:\textbackslash{}n\textbackslash{}n— A boa hora almoças tu!\textbackslash{}n\textbackslash{}n— Aí vens tu com as tuas reprimendas. Eu\textbackslash{}nalmoço quando tenho fome{\ldots} Vê se me queres agora escravizar às horas e às\textbackslash{}ndenominações. Chama-lhe almoço ou lunch, a verdade é que estou comendo.\textbackslash{}n\textbackslash{}nLourenço respondeu com uma careta.\textbackslash{}n\textbackslash{}nTerminado o almoço, anunciou-se a chegada\textbackslash{}ndo Sr. Batista. Vasconcelos foi recebê-lo no gabinete particular.\textbackslash{}n\textbackslash{}nBatista era um rapaz de vinte e cinco\textbackslash{}nanos; era o tipo acabado do pândego; excelente companheiro numa ceia de\textbackslash{}nsociedade equívoca, nulo conviva numa sociedade honesta. Tinha chiste e certa\textbackslash{}ninteligência, mas era preciso que estivesse em clima próprio para que se lhe\textbackslash{}ndesenvolvessem essas qualidades. No mais era bonito; tinha um lindo bigode;\textbackslash{}ncalçava botins do Campas, e vestia no mais apurado gosto; fumava tanto como um\textbackslash{}nsoldado e tão bem como um lord.\textbackslash{}n\textbackslash{}n— Aposto que acordaste agora? disse\textbackslash{}nBatista entrando no gabinete do Vasconcelos.\textbackslash{}n\textbackslash{}n— Há três quartos de hora; almocei neste\textbackslash{}ninstante. Toma um charuto.\textbackslash{}n\textbackslash{}nBatista aceitou o charuto, e estirou-se\textbackslash{}nnuma cadeira americana, enquanto Vasconcelos acendia um fósforo.\textbackslash{}n\textbackslash{}n— Viste o Gomes? perguntou Vasconcelos.\textbackslash{}n\textbackslash{}n— Vi-o ontem. Grande notícia; rompeu com\textbackslash{}na sociedade.\textbackslash{}n\textbackslash{}n— Deveras?\textbackslash{}n\textbackslash{}n— Quando lhe perguntei por que motivo\textbackslash{}nninguém o via há um mês, respondeu-me que estava passando por uma\textbackslash{}ntransformação, e que do Gomes que foi só ficará lembrança. Parece incrível, mas\textbackslash{}no rapaz fala com convicção.\textbackslash{}n\textbackslash{}n— Não creio; aquilo é alguma caçoada que\textbackslash{}nnos quer fazer. Que novidades há?\textbackslash{}n\textbackslash{}n— Nada; isto é, tu é que deves saber\textbackslash{}nalguma coisa.\textbackslash{}n\textbackslash{}n— Eu, nada{\ldots}\textbackslash{}n\textbackslash{}n— Ora essa! não foste ontem ao Jardim?\textbackslash{}n\textbackslash{}n— Fui, sim; houve uma ceia{\ldots}\textbackslash{}n\textbackslash{}n— De família, sim. Eu fui ao Alcazar. A\textbackslash{}nque horas acabou a reunião?\textbackslash{}n\textbackslash{}n— Às quatro da manhã{\ldots}\textbackslash{}n\textbackslash{}nVasconcelos estendeu-se numa rede, e a\textbackslash{}nconversa continuou por esse tom, até que um moleque veio dizer a Vasconcelos\textbackslash{}nque estava na sala o Sr. Gomes.\textbackslash{}n\textbackslash{}n— Eis o homem! disse Batista.\textbackslash{}n\textbackslash{}n— Manda subir, ordenou Vasconcelos.\textbackslash{}n\textbackslash{}nO moleque desceu para dar o recado; mas\textbackslash{}nsó um quarto de hora depois é que Gomes apareceu, por demorar-se algum tempo em\textbackslash{}nbaixo conversando com Augusta e Adelaide.\textbackslash{}n\textbackslash{}n— Quem é vivo sempre aparece, disse\textbackslash{}nVasconcelos ao avistar o rapaz.\textbackslash{}n\textbackslash{}n— Não me procuram{\ldots}, disse ele.\textbackslash{}n\textbackslash{}n— Perdão; eu já lá fui duas vezes, e\textbackslash{}ndisseram-me que havias saído.\textbackslash{}n\textbackslash{}n— Só por grande fatalidade, porque eu\textbackslash{}nquase nunca saio.\textbackslash{}n\textbackslash{}n— Mas então estás completamente ermitão?\textbackslash{}n\textbackslash{}n— Estou crisálida; vou reaparecer\textbackslash{}nborboleta, disse Gomes sentando-se.\textbackslash{}n\textbackslash{}n— Temos poesia{\ldots} Guarda debaixo,\textbackslash{}nVasconcelos{\ldots}\textbackslash{}n\textbackslash{}nO novo personagem, o Gomes tão desejado e\textbackslash{}ntão escondido, representava ter cerca de trinta anos. Ele, Vasconcelos e\textbackslash{}nBatista eram a trindade do prazer e da dissipação, ligada por uma indissolúvel\textbackslash{}namizade. Quando Gomes, cerca de um mês antes, deixou de aparecer nos círculos\textbackslash{}ndo costume, todos repararam nisso, mas só Vasconcelos e Batista sentiram\textbackslash{}ndeveras. Todavia, não insistiram muito em arrancá-lo à solidão, somente pela\textbackslash{}nconsideração de que talvez houvesse nisso algum interesse do rapaz.\textbackslash{}n\textbackslash{}nGomes foi portanto recebido como um filho\textbackslash{}npródigo.\textbackslash{}n\textbackslash{}n— Mas onde te meteste? que é isso de\textbackslash{}ncrisálida e de borboleta? Cuidas que eu sou do mangue?\textbackslash{}n\textbackslash{}n— É o que lhes digo, meus amigos. Estou\textbackslash{}ncriando asas.\textbackslash{}n\textbackslash{}n— Asas! disse Batista sufocando uma\textbackslash{}nrisada.\textbackslash{}n\textbackslash{}n— Só se são asas de gavião para cair{\ldots}\textbackslash{}n\textbackslash{}n— Não, estou falando sério.\textbackslash{}n\textbackslash{}nE com efeito Gomes apresentava um ar\textbackslash{}nsério e convencido.\textbackslash{}n\textbackslash{}nVasconcelos e Batista olharam um para o\textbackslash{}noutro.\textbackslash{}n\textbackslash{}n— Pois se é verdade isso que dizes,\textbackslash{}nexplica-nos lá que asas são essas, e sobretudo para onde é que queres voar.\textbackslash{}n\textbackslash{}nA estas palavras de Vasconcelos,\textbackslash{}nacrescentou Batista:\textbackslash{}n\textbackslash{}n— Sim, deves dar-nos uma explicação, e se\textbackslash{}nnós que somos o teu conselho de família, acharmos que a explicação é boa,\textbackslash{}naprovamo-la; senão, ficas sem asas, e ficas sendo o que sempre foste{\ldots}\textbackslash{}n\textbackslash{}n— Apoiado, disse Vasconcelos.\textbackslash{}n\textbackslash{}n— Pois é simples; estou criando asas de\textbackslash{}nanjo, e quero voar para o céu do amor.\textbackslash{}n\textbackslash{}n— Do amor! disseram os dois amigos de\textbackslash{}nGomes.\textbackslash{}n\textbackslash{}n— É verdade, continuou Gomes. Que fui eu até\textbackslash{}nhoje? Um verdadeiro estróina, um perfeito pândego, gastando às mãos largas a\textbackslash{}nminha fortuna e o meu coração. Mas isto é bastante para encher a vida? Parece\textbackslash{}nque não{\ldots}\textbackslash{}n\textbackslash{}n— Até aí concordo{\ldots} isso não basta; é preciso\textbackslash{}nque haja outra coisa; a diferença está na maneira de{\ldots}\textbackslash{}n\textbackslash{}n— É exato, disse Vasconcelos; é exato; é\textbackslash{}nnatural que vocês pensem de modo diverso, mas eu acho que tenho razão em dizer\textbackslash{}nque sem o amor casto e puro a vida é um puro deserto.\textbackslash{}n\textbackslash{}nBatista deu um pulo{\ldots}\textbackslash{}n\textbackslash{}nVasconcelos fitou os olhos em Gomes:\textbackslash{}n\textbackslash{}n— Aposto que vais casar? disse-lhe.\textbackslash{}n\textbackslash{}n— Não sei se vou casar; sei que amo, e\textbackslash{}nespero acabar por casar-me com a mulher a quem amo.\textbackslash{}n\textbackslash{}n— Casar! exclamou Batista.\textbackslash{}n\textbackslash{}nE soltou uma estridente gargalhada.\textbackslash{}n\textbackslash{}nMas Gomes falava tão seriamente, insistia\textbackslash{}ncom tanta gravidade naqueles projetos de regeneração, que os dois amigos\textbackslash{}nacabaram por ouvi-lo com igual seriedade.\textbackslash{}n\textbackslash{}nGomes falava uma linguagem estranha, e\textbackslash{}ninteiramente nova na boca de um rapaz que era o mais doido e ruidoso nos\textbackslash{}nfestins de Baco e de Citera.\textbackslash{}n\textbackslash{}n— Assim, pois, deixas-nos? perguntou\textbackslash{}nVasconcelos.\textbackslash{}n\textbackslash{}n— Eu? Sim e não; encontrar-me-ão nas\textbackslash{}nsalas; nos hotéis e nas casas equívocas, nunca mais.\textbackslash{}n\textbackslash{}n— De profundis{\ldots} cantarolou\textbackslash{}nBatista.\textbackslash{}n\textbackslash{}n— Mas, afinal de contas, disse Vasconcelos,\textbackslash{}nonde está a tua Marion? Pode-se saber quem ela é?\textbackslash{}n\textbackslash{}n— Não é Marion, é Virgínia{\ldots} Pura\textbackslash{}nsimpatia ao princípio, depois afeição pronunciada, hoje paixão verdadeira.\textbackslash{}nLutei enquanto pude; mas abati as armas diante de uma força maior. O meu grande\textbackslash{}nmedo era não ter uma alma capaz de oferecer a essa gentil criatura. Pois\textbackslash{}ntenho-a, e tão fogosa, e tão virgem como no tempo dos meus dezoito anos. Só o\textbackslash{}ncasto olhar de uma virgem poderia descobrir no meu lodo essa pérola divina.\textbackslash{}nRenasço melhor do que era{\ldots}\textbackslash{}n\textbackslash{}n— Está claro, Vasconcelos, o rapaz está\textbackslash{}ndoido; mandemo-lo para a Praia Vermelha; e como pode ter algum acesso, eu\textbackslash{}nvou-me embora{\ldots}\textbackslash{}n\textbackslash{}nBatista pegou no chapéu.\textbackslash{}n\textbackslash{}n— Onde vais? disse-lhe Gomes.\textbackslash{}n\textbackslash{}n— Tenho que fazer; mas logo aparecerei em\textbackslash{}ntua casa; quero ver se ainda é tempo de arrancar-te a esse abismo.\textbackslash{}n\textbackslash{}nE saiu.\textbackslash{}n\textbackslash{}nCAPÍTULO III\textbackslash{}n\textbackslash{}nOs dois ficaram sós.\textbackslash{}n\textbackslash{}n— Então é certo que estás apaixonado?\textbackslash{}n\textbackslash{}n— Estou. Eu bem sabia que vocês\textbackslash{}ndificilmente acreditariam nisto; eu próprio não creio ainda, e contudo é\textbackslash{}nverdade. Acabo por onde tu começaste. Será melhor ou pior? Eu creio que é\textbackslash{}nmelhor.\textbackslash{}n\textbackslash{}n— Tens interesse em ocultar o nome da\textbackslash{}npessoa?\textbackslash{}n\textbackslash{}n— Oculto-o por ora a todos, menos a ti.\textbackslash{}n\textbackslash{}n— É uma prova de confiança{\ldots}\textbackslash{}n\textbackslash{}nGomes sorriu.\textbackslash{}n\textbackslash{}n— Não, disse ele, é uma condição sine\textbackslash{}nqua non; antes de todos tu deves saber quem é a escolhida do meu coração;\textbackslash{}ntrata-se de tua filha.\textbackslash{}n\textbackslash{}n— Adelaide? perguntou Vasconcelos\textbackslash{}nespantado.\textbackslash{}n\textbackslash{}n— Sim, tua filha.\textbackslash{}n\textbackslash{}nA revelação de Gomes caiu como uma bomba.\textbackslash{}nVasconcelos nem por sombras suspeitava semelhante coisa.\textbackslash{}n\textbackslash{}n— Este amor é da tua aprovação?\textbackslash{}nperguntou-lhe Gomes.\textbackslash{}n\textbackslash{}nVasconcelos refletia, e depois de alguns\textbackslash{}nminutos de silêncio, disse:\textbackslash{}n\textbackslash{}n— O meu coração aprova a tua escolha; és meu\textbackslash{}namigo, estás apaixonado, e uma vez que ela te ame{\ldots}\textbackslash{}n\textbackslash{}nGomes ia falar, mas Vasconcelos continuou\textbackslash{}nsorrindo:\textbackslash{}n\textbackslash{}n— Mas a sociedade?\textbackslash{}n\textbackslash{}n— Que sociedade?\textbackslash{}n\textbackslash{}n— A sociedade que nos tem em conta de\textbackslash{}nlibertinos, a ti e a mim, é natural que não aprove o meu ato.\textbackslash{}n\textbackslash{}n— Já vejo que é uma recusa, disse Gomes\textbackslash{}nentristecendo.\textbackslash{}n\textbackslash{}n— Qual recusa, pateta! É uma objeção, que\textbackslash{}ntu poderás destruir dizendo: a sociedade é uma grande caluniadora e uma famosa\textbackslash{}nindiscreta. Minha filha é tua, com uma condição.\textbackslash{}n\textbackslash{}n— Qual?\textbackslash{}n\textbackslash{}n— A condição da reciprocidade. Ama-te\textbackslash{}nela?\textbackslash{}n\textbackslash{}n— Não sei, respondeu Gomes.\textbackslash{}n\textbackslash{}n— Mas desconfias{\ldots}\textbackslash{}n\textbackslash{}n— Não sei; sei que a amo e que daria a\textbackslash{}nminha vida por ela, mas ignoro se sou correspondido.\textbackslash{}n\textbackslash{}n— Hás de ser{\ldots} Eu me incumbirei de\textbackslash{}napalpar o terreno. Daqui a dois dias dou-te a minha resposta. Ah! se ainda\textbackslash{}ntenho de ver-te meu genro!\textbackslash{}n\textbackslash{}nA resposta de Gomes foi cair-lhe nos\textbackslash{}nbraços. A cena já roçava pela comédia quando deram três horas. Gomes lembrou-se\textbackslash{}nque tinha rendez-vous com um amigo; Vasconcelos lembrou-se que tinha de\textbackslash{}nescrever algumas cartas.\textbackslash{}n\textbackslash{}nGomes saiu sem falar às senhoras.\textbackslash{}n\textbackslash{}nPelas quatro horas Vasconcelos\textbackslash{}ndispunha-se a sair, quando vieram anunciar-lhe a visita do Sr. José Brito.\textbackslash{}n\textbackslash{}nAo ouvir este nome o alegre Vasconcelos\textbackslash{}nfranziu o sobrolho.\textbackslash{}n\textbackslash{}nPouco depois entrava no gabinete o Sr.\textbackslash{}nJosé Brito.\textbackslash{}n\textbackslash{}nO Sr. José Brito era para Vasconcelos um\textbackslash{}nverdadeiro fantasma, um eco do abismo, uma voz da realidade; era um credor.\textbackslash{}n\textbackslash{}n— Não contava hoje com a sua visita,\textbackslash{}ndisse Vasconcelos.\textbackslash{}n\textbackslash{}n— Admira, respondeu o Sr. José Brito com\textbackslash{}numa placidez de apunhalar, porque hoje são 21.\textbackslash{}n\textbackslash{}n— Cuidei que eram 19, balbuciou\textbackslash{}nVasconcelos.\textbackslash{}n\textbackslash{}n— Anteontem, sim; mas hoje são 21. Olhe,\textbackslash{}ncontinuou o credor pegando no Jornal do Comércio que se achava numa\textbackslash{}ncadeira: quinta-feira, 21.\textbackslash{}n\textbackslash{}n— Vem buscar o dinheiro?\textbackslash{}n\textbackslash{}n— Aqui está a letra, disse o Sr. José\textbackslash{}nBrito tirando a carteira do bolso e um papel da carteira.\textbackslash{}n\textbackslash{}n— Por que não veio mais cedo? perguntou\textbackslash{}nVasconcelos, procurando assim espaçar a questão principal.\textbackslash{}n\textbackslash{}n— Vim às oito horas da manhã, respondeu o\textbackslash{}ncredor, estava dormindo; vim às nove, idem; vim às dez, idem; vim às onze,\textbackslash{}nidem; vim ao meio-dia, idem. Quis vir à uma hora, mas tinha de mandar um homem\textbackslash{}npara a cadeia, e não me foi possível acabar cedo. Às três jantei, e às quatro\textbackslash{}naqui estou.\textbackslash{}n\textbackslash{}nVasconcelos puxava o charuto a ver se lhe\textbackslash{}nocorria alguma idéia boa de escapar ao pagamento com que ele não contava.\textbackslash{}n\textbackslash{}nNão achava nada; mas o próprio credor\textbackslash{}nforneceu-lhe ensejo.\textbackslash{}n\textbackslash{}n— Além de que, disse ele, a hora não\textbackslash{}nimporta nada, porque eu estava certo de que o senhor me vai pagar.\textbackslash{}n\textbackslash{}n— Ah! disse Vasconcelos, é talvez um\textbackslash{}nengano; eu não contava com o senhor hoje, e não arranjei o dinheiro{\ldots}\textbackslash{}n\textbackslash{}n— Então, como há de ser? perguntou o\textbackslash{}ncredor com ingenuidade.\textbackslash{}n\textbackslash{}nVasconcelos sentiu entrar-lhe n’alma a\textbackslash{}nesperança.\textbackslash{}n\textbackslash{}n— Nada mais simples, disse; o senhor\textbackslash{}nespera até amanhã{\ldots}\textbackslash{}n\textbackslash{}n— Amanhã quero assistir à penhora de um\textbackslash{}nindivíduo que mandei processar por uma larga dívida; não posso{\ldots}\textbackslash{}n\textbackslash{}n— Perdão, eu levo-lhe o dinheiro à sua\textbackslash{}ncasa{\ldots}\textbackslash{}n\textbackslash{}n— Isso seria bom se os negócios\textbackslash{}ncomerciais se arranjassem assim. Se fôssemos dois amigos é natural que eu me\textbackslash{}ncontentasse com a sua promessa, e tudo acabaria amanhã; mas eu sou seu credor,\textbackslash{}ne só tenho em vista salvar o meu interesse{\ldots} Portanto, acho melhor pagar hoje{\ldots}\textbackslash{}n\textbackslash{}nVasconcelos passou a mão pelos cabelos.\textbackslash{}n\textbackslash{}n— Mas se eu não tenho! disse ele.\textbackslash{}n\textbackslash{}n— É uma coisa que o deve incomodar muito,\textbackslash{}nmas que a mim não me causa a menor impressão{\ldots} isto é, deve causar-me alguma,\textbackslash{}nporque o senhor está hoje em situação precária.\textbackslash{}n\textbackslash{}n— Eu?\textbackslash{}n\textbackslash{}n— É verdade; as suas casas da Rua da\textbackslash{}nImperatriz estão hipotecadas; a da Rua de S. Pedro foi vendida, e a importância\textbackslash{}njá vai longe; os seus escravos têm ido a um e um, sem que o senhor o perceba, e\textbackslash{}nas despesas que o senhor há pouco fez para montar uma casa a certa dama da\textbackslash{}nsociedade equívoca são imensas. Eu sei tudo; sei mais do que o senhor{\ldots}\textbackslash{}n\textbackslash{}nVasconcelos estava visivelmente aterrado.\textbackslash{}n\textbackslash{}nO credor dizia a verdade.\textbackslash{}n\textbackslash{}n— Mas enfim, disse Vasconcelos, o que\textbackslash{}nhavemos de fazer?\textbackslash{}n\textbackslash{}n— Uma coisa simples; duplicamos a dívida,\textbackslash{}ne o senhor passa-me agora mesmo um depósito.\textbackslash{}n\textbackslash{}n— Duplicar a dívida! Mas isto é um{\ldots}\textbackslash{}n\textbackslash{}n— Isto é uma tábua de salvação; sou\textbackslash{}nmoderado. Vamos lá, aceite. Escreva-me aí o depósito, e rasga-se a letra.\textbackslash{}n\textbackslash{}nVasconcelos ainda quis fazer objeção; mas\textbackslash{}nera impossível convencer o Sr. José Brito.\textbackslash{}n\textbackslash{}nAssinou o depósito de dezoito contos.\textbackslash{}n\textbackslash{}nQuando o credor saiu, Vasconcelos entrou\textbackslash{}na meditar seriamente na sua vida.\textbackslash{}n\textbackslash{}nAté então gastara tanto e tão cegamente\textbackslash{}nque não reparara no abismo que ele próprio cavara a seus pés.\textbackslash{}n\textbackslash{}nVeio porém adverti-lo a voz de um dos\textbackslash{}nseus algozes.\textbackslash{}n\textbackslash{}nVasconcelos refletiu, calculou,\textbackslash{}nrecapitulou as suas despesas e as suas obrigações, e viu que da fortuna que\textbackslash{}npossuía tinha na realidade menos da quarta parte.\textbackslash{}n\textbackslash{}nPara viver como até ali vivera, aquilo\textbackslash{}nera nada menos que a miséria.\textbackslash{}n\textbackslash{}nQue fazer em tal situação?\textbackslash{}n\textbackslash{}nVasconcelos pegou no chapéu e saiu.\textbackslash{}n\textbackslash{}nVinha caindo a noite.\textbackslash{}n\textbackslash{}nDepois de andar algum tempo pelas ruas\textbackslash{}nentregue às suas meditações, Vasconcelos entrou no Alcazar.\textbackslash{}n\textbackslash{}nEra um meio de distrair-se.\textbackslash{}n\textbackslash{}nAli encontraria a sociedade do costume.\textbackslash{}n\textbackslash{}nBatista veio ao encontro do amigo.\textbackslash{}n\textbackslash{}n— Que cara é essa? disse-lhe.\textbackslash{}n\textbackslash{}n— Não é nada, pisaram-me um calo,\textbackslash{}nrespondeu Vasconcelos, que não encontrava melhor resposta.\textbackslash{}n\textbackslash{}nMas um pedicuro que se achava perto de\textbackslash{}nambos ouviu o dito, e nunca mais perdeu de vista o infeliz Vasconcelos, a quem\textbackslash{}na coisa mais indiferente incomodava. O olhar persistente do pedicuro\textbackslash{}naborreceu-o tanto, que Vasconcelos saiu.\textbackslash{}n\textbackslash{}nEntrou no Hotel de Milão, para jantar. Por\textbackslash{}nmais preocupado que ele estivesse, a exigência do estômago não se demorou.\textbackslash{}n\textbackslash{}nOra, no meio do jantar lembrou-lhe aquilo\textbackslash{}nque não devia ter-lhe saído da cabeça: o pedido de casamento feito nessa tarde\textbackslash{}npor Gomes.\textbackslash{}n\textbackslash{}nFoi um raio de luz.\textbackslash{}n\textbackslash{}n'Gomes é rico, pensou Vasconcelos; o\textbackslash{}nmeio de escapar a maiores desgostos é este; Gomes casa-se com Adelaide, e como\textbackslash{}né meu amigo não me negará o que eu precisar. Pela minha parte procurarei ganhar\textbackslash{}no perdido{\ldots} Que boa fortuna foi aquela lembrança do casamento!”\textbackslash{}n\textbackslash{}nVasconcelos comeu alegremente; voltou\textbackslash{}ndepois ao Alcazar, onde alguns rapazes e outras pessoas fizeram esquecer\textbackslash{}ncompletamente os seus infortúnios.\textbackslash{}n\textbackslash{}nÀs três horas da noite Vasconcelos\textbackslash{}nentrava para casa com a tranqüilidade e regularidade do costume.\textbackslash{}n\textbackslash{}nCAPÍTULO IV\textbackslash{}n\textbackslash{}nNo dia seguinte o primeiro cuidado de\textbackslash{}nVasconcelos foi consultar o coração de Adelaide. Queria porém fazê-lo na\textbackslash{}nausência de Augusta. Felizmente esta precisava de ir ver à Rua da Quitanda umas\textbackslash{}nfazendas novas, e saiu com o cunhado, deixando a Vasconcelos toda a liberdade.\textbackslash{}n\textbackslash{}nComo os leitores já sabem, Adelaide\textbackslash{}nqueria muito ao pai, e era capaz de fazer por ele tudo. Era, além disso, um\textbackslash{}nexcelente coração. Vasconcelos contava com essas duas forças.\textbackslash{}n\textbackslash{}n— Vem cá, Adelaide, disse ele entrando na\textbackslash{}nsala; sabes quantos anos tens?\textbackslash{}n\textbackslash{}n— Tenho quinze.\textbackslash{}n\textbackslash{}n— Sabes quantos anos tem tua mãe?\textbackslash{}n\textbackslash{}n— Vinte e sete, não é?\textbackslash{}n\textbackslash{}n— Tem trinta; quer dizer que tua mãe\textbackslash{}ncasou-se com quinze anos.\textbackslash{}n\textbackslash{}nVasconcelos parou, a fim de ver o efeito que\textbackslash{}nproduziam estas palavras; mas foi inútil a expectativa; Adelaide não\textbackslash{}ncompreendeu nada.\textbackslash{}n\textbackslash{}nO pai continuou:\textbackslash{}n\textbackslash{}n— Não pensaste no casamento?\textbackslash{}n\textbackslash{}nA menina corou muito, hesitou em falar,\textbackslash{}nmas como o pai instasse, respondeu:\textbackslash{}n\textbackslash{}n— Qual, papai! Eu não quero casar{\ldots}\textbackslash{}n\textbackslash{}n— Não queres casar? É boa! por quê?\textbackslash{}n\textbackslash{}n— Porque não tenho vontade, e vivo bem\textbackslash{}naqui.\textbackslash{}n\textbackslash{}n— Mas tu podes casar e continuar a viver\textbackslash{}naqui{\ldots}\textbackslash{}n\textbackslash{}n— Bem; mas não tenho vontade.\textbackslash{}n\textbackslash{}n— Anda lá{\ldots} Amas alguém, confessa.\textbackslash{}n\textbackslash{}n— Não me pergunte isso, papai{\ldots} eu não\textbackslash{}namo ninguém.\textbackslash{}n\textbackslash{}nA linguagem de Adelaide era tão sincera\textbackslash{}nque Vasconcelos não podia duvidar.\textbackslash{}n\textbackslash{}n— Ela fala a verdade, pensou ele; é\textbackslash{}ninútil tentar por esse lado{\ldots}\textbackslash{}n\textbackslash{}nAdelaide sentou-se ao pé dele, e disse:\textbackslash{}n\textbackslash{}n— Portanto, meu paizinho, não falemos\textbackslash{}nmais nisso{\ldots}\textbackslash{}n\textbackslash{}n— Falemos, minha filha; tu és criança,\textbackslash{}nnão sabes calcular. Imagina que eu e a tua mãe morremos amanhã. Quem te há de\textbackslash{}namparar? Só um marido.\textbackslash{}n\textbackslash{}n— Mas se eu não gosto de ninguém{\ldots}\textbackslash{}n\textbackslash{}n— Por ora; mas hás de vir a gostar se o\textbackslash{}nnoivo for um bonito rapaz, de bom coração{\ldots} Eu já escolhi um que te ama muito,\textbackslash{}ne a quem tu hás de amar.\textbackslash{}n\textbackslash{}nAdelaide estremeceu.\textbackslash{}n\textbackslash{}n— Eu? disse ela, Mas{\ldots} quem é?\textbackslash{}n\textbackslash{}n— É o Gomes.\textbackslash{}n\textbackslash{}n— Não o amo, meu pai{\ldots}\textbackslash{}n\textbackslash{}n— Agora, creio; mas não negas que ele é\textbackslash{}ndigno de ser amado. Dentro de dois meses está apaixonada por ele.\textbackslash{}n\textbackslash{}nAdelaide não disse palavra. Curvou a\textbackslash{}ncabeça e começou a torcer nos dedos uma das tranças bastas e negras. O seio\textbackslash{}narfava-lhe com força; a menina tinha os olhos cravados no tapete.\textbackslash{}n\textbackslash{}n— Vamos, está decidido, não? perguntou\textbackslash{}nVasconcelos.\textbackslash{}n\textbackslash{}n— Mas, papai, e se eu for infeliz?{\ldots}\textbackslash{}n\textbackslash{}n— Isso é impossível, minha filha; hás de\textbackslash{}nser muito feliz; e hás de amar muito a teu marido.\textbackslash{}n\textbackslash{}n— Oh! papai, disse-lhe Adelaide com os\textbackslash{}nolhos rasos de água, peço-lhe que não me case ainda{\ldots}\textbackslash{}n\textbackslash{}n— Adelaide, o primeiro dever de uma filha\textbackslash{}né obedecer a seu pai, e eu sou teu pai. Quero que te cases com o Gomes; hás de\textbackslash{}ncasar.\textbackslash{}n\textbackslash{}nEstas palavras, para terem todo o efeito,\textbackslash{}ndeviam ser seguidas de uma retirada rápida. Vasconcelos compreendeu isso, e\textbackslash{}nsaiu da sala deixando Adelaide na maior desolação.\textbackslash{}n\textbackslash{}nAdelaide não amava ninguém. A sua recusa\textbackslash{}nnão tinha por ponto de partida nenhum outro amor; também não era resultado de\textbackslash{}naversão que tivesse pelo seu pretendente.\textbackslash{}n\textbackslash{}nA menina sentia simplesmente uma total\textbackslash{}nindiferença pelo rapaz.\textbackslash{}n\textbackslash{}nNestas condições o casamento não deixava\textbackslash{}nde ser uma odiosa imposição.\textbackslash{}n\textbackslash{}nMas que faria Adelaide? a quem\textbackslash{}nrecorreria?\textbackslash{}n\textbackslash{}nRecorreu às lágrimas.\textbackslash{}n\textbackslash{}nQuanto a Vasconcelos, subiu ao gabinete e\textbackslash{}nescreveu as seguintes linhas ao futuro genro:\textbackslash{}n\textbackslash{}nTudo caminha bem; autorizo-te a vires\textbackslash{}nfazer a corte à pequena, e espero que dentro de dois meses o casamento esteja\textbackslash{}nconcluído.\textbackslash{}n\textbackslash{}nFechou a carta e mandou-a.\textbackslash{}n\textbackslash{}nPouco depois voltaram de fora Augusta e\textbackslash{}nLourenço.\textbackslash{}n\textbackslash{}nEnquanto Augusta subiu para o quarto da toilette\textbackslash{}npara mudar de roupa, Lourenço foi ter com Adelaide, que estava no jardim.\textbackslash{}n\textbackslash{}nReparou que ela tinha os olhos vermelhos,\textbackslash{}ne inquiriu a causa; mas a moça negou que fosse de chorar.\textbackslash{}n\textbackslash{}nLourenço não acreditou nas palavras da sobrinha,\textbackslash{}ne instou com ela para que lhe contasse o que havia.\textbackslash{}n\textbackslash{}nAdelaide tinha grande confiança no tio,\textbackslash{}naté por causa da sua rudeza de maneiras. No fim de alguns minutos de\textbackslash{}ninstâncias, Adelaide contou a Lourenço a cena com o pai.\textbackslash{}n\textbackslash{}n— Então, é por isso que estás chorando,\textbackslash{}npequena?\textbackslash{}n\textbackslash{}n— Pois então? Como fugir ao casamento?\textbackslash{}n\textbackslash{}n— Descansa, não te casarás; eu te prometo\textbackslash{}nque não te hás de casar{\ldots}\textbackslash{}n\textbackslash{}nA moça sentiu um estremecimento de\textbackslash{}nalegria.\textbackslash{}n\textbackslash{}n— Promete, meu tio, que há de convencer a\textbackslash{}npapai?\textbackslash{}n\textbackslash{}n— Hei de vencê-lo ou convencê-lo, não\textbackslash{}nimporta; tu não te hás de casar. Teu pai é um tolo.\textbackslash{}n\textbackslash{}nLourenço subiu ao gabinete de\textbackslash{}nVasconcelos, exatamente no momento em que este se dispunha a sair.\textbackslash{}n\textbackslash{}n— Vais sair? perguntou-lhe Lourenço.\textbackslash{}n\textbackslash{}n— Vou.\textbackslash{}n\textbackslash{}n— Preciso falar-te.\textbackslash{}n\textbackslash{}nLourenço sentou-se, e Vasconcelos, que já\textbackslash{}ntinha o chapéu na cabeça, esperou de pé que ele falasse.\textbackslash{}n\textbackslash{}n— Senta-te, disse Lourenço.\textbackslash{}n\textbackslash{}nVasconcelos sentou-se.\textbackslash{}n\textbackslash{}n— Há dezesseis anos{\ldots}\textbackslash{}n\textbackslash{}n— Começas de muito longe; vê se abrevias\textbackslash{}numa meia dúzia de anos, sem o que não prometo ouvir o que me vais dizer.\textbackslash{}n\textbackslash{}n— Há dezesseis anos, continuou Lourenço,\textbackslash{}nque és casado; mas a diferença entre o primeiro dia e o dia de hoje é grande.\textbackslash{}n\textbackslash{}n— Naturalmente, disse Vasconcelos. Tempora\textbackslash{}nmutantur et{\ldots}\textbackslash{}n\textbackslash{}n— Naquele tempo, continuou Lourenço,\textbackslash{}ndizias que encontraras o paraíso, o verdadeiro paraíso, e foste durante dois ou\textbackslash{}ntrês anos o modelo dos maridos. Depois mudaste completamente; e o paraíso\textbackslash{}ntornar-se-ia verdadeiro inferno se tua mulher não fosse tão indiferente e fria\textbackslash{}ncomo é, evitando assim as mais terríveis cenas domésticas.\textbackslash{}n\textbackslash{}n— Mas, Lourenço, que tens com isso?\textbackslash{}n\textbackslash{}n— Nada; nem é disso que vou falar-te. O\textbackslash{}nque me interessa é que não sacrifiques tua filha por um capricho, entregando-a\textbackslash{}na um dos teus companheiros de vida solta{\ldots}\textbackslash{}n\textbackslash{}nVasconcelos levantou-se:\textbackslash{}n\textbackslash{}n— Estás doido! disse ele.\textbackslash{}n\textbackslash{}n— Estou calmo, e dou-te o prudente\textbackslash{}nconselho de não sacrificares tua filha a um libertino.\textbackslash{}n\textbackslash{}n— Gomes não é libertino; teve uma vida de\textbackslash{}nrapaz, é verdade, mas gosta de Adelaide, e reformou-se completamente. É um bom\textbackslash{}ncasamento, e por isso acho que todos devemos aceitá-lo. É a minha vontade, e\textbackslash{}nnesta casa quem manda sou eu.\textbackslash{}n\textbackslash{}nLourenço procurou falar ainda, mas\textbackslash{}nVasconcelos já ia longe.\textbackslash{}n\textbackslash{}n'Que fazer?' pensou Lourenço.\textbackslash{}n\textbackslash{}nCAPÍTULO V\textbackslash{}n\textbackslash{}nA oposição de Lourenço não causava grande\textbackslash{}nimpressão a Vasconcelos. Ele podia, é verdade, sugerir à sobrinha idéias de\textbackslash{}nresistência; mas Adelaide, que era um espírito fraco, cederia ao último que lhe\textbackslash{}nfalasse, e os conselhos de um dia seriam vencidos pela imposição do dia\textbackslash{}nseguinte.\textbackslash{}n\textbackslash{}nTodavia era conveniente obter o apoio de\textbackslash{}nAugusta. Vasconcelos pensou em tratar disso o mais cedo que lhe fosse possível.\textbackslash{}n\textbackslash{}nEntretanto, urgia organizar os seus\textbackslash{}nnegócios, e Vasconcelos procurou um advogado a quem entregou todos os papéis e\textbackslash{}ninformações, encarregando-o de orientá-lo em todas as necessidades da situação,\textbackslash{}nquais os meios que poderia opor em qualquer caso de reclamação por dívida ou\textbackslash{}nhipoteca.\textbackslash{}n\textbackslash{}nNada disto fazia supor da parte de\textbackslash{}nVasconcelos uma reforma de costumes. Preparava-se apenas para continuar a vida\textbackslash{}nanterior.\textbackslash{}n\textbackslash{}nDois dias depois da conversa com o irmão,\textbackslash{}nVasconcelos procurou Augusta, para tratar francamente do casamento de Adelaide.\textbackslash{}n\textbackslash{}nJá nesse intervalo o futuro noivo,\textbackslash{}nobedecendo ao conselho de Vasconcelos, fazia corte prévia à filha. Era possível\textbackslash{}nque, se o casamento não lhe fosse imposto, Adelaide acabasse por gostar do\textbackslash{}nrapaz. Gomes era um homem belo e elegante; e, além disso, conhecia todos os\textbackslash{}nrecursos de que se deve usar para impressionar uma mulher.\textbackslash{}n\textbackslash{}nTeria Augusta notado a presença assídua\textbackslash{}ndo moço? Vasconcelos fazia essa pergunta ao seu espírito no momento em que\textbackslash{}nentrava na toilette da mulher.\textbackslash{}n\textbackslash{}n— Vais sair? perguntou ele.\textbackslash{}n\textbackslash{}n— Não; tenho visitas.\textbackslash{}n\textbackslash{}n— Ah! quem?\textbackslash{}n\textbackslash{}n— A mulher do Seabra, disse ela.\textbackslash{}n\textbackslash{}nVasconcelos sentou-se, e procurou um meio\textbackslash{}nde encabeçar a conversa especial que ali o levava.\textbackslash{}n\textbackslash{}n— Estás muito bonita hoje!\textbackslash{}n\textbackslash{}n— Deveras? disse ela sorrindo. Pois estou\textbackslash{}nhoje como sempre, e é singular que o digas hoje{\ldots}\textbackslash{}n\textbackslash{}n— Não; realmente hoje estás mais bonita\textbackslash{}ndo que costumas, a ponto que sou capaz de ter ciúmes{\ldots}\textbackslash{}n\textbackslash{}n— Qual! disse Augusta com um sorriso\textbackslash{}nirônico.\textbackslash{}n\textbackslash{}nVasconcelos coçou a cabeça, tirou o\textbackslash{}nrelógio, deu-lhe corda; depois entrou a puxar as barbas, pegou numa folha, leu\textbackslash{}ndois ou três anúncios, atirou a folha ao chão, e afinal, depois de um silêncio\textbackslash{}njá prolongado, Vasconcelos achou melhor atacar a praça de frente.\textbackslash{}n\textbackslash{}n— Tenho pensado ultimamente em Adelaide,\textbackslash{}ndisse ele.\textbackslash{}n\textbackslash{}n— Ah! por quê?\textbackslash{}n\textbackslash{}n— Está moça{\ldots}\textbackslash{}n\textbackslash{}n— Moça! exclamou Augusta, é uma\textbackslash{}ncriança{\ldots}\textbackslash{}n\textbackslash{}n— Está mais velha do que tu quando te\textbackslash{}ncasaste{\ldots}\textbackslash{}n\textbackslash{}nAugusta franziu ligeiramente a testa.\textbackslash{}n\textbackslash{}n— Mas então{\ldots} disse ela.\textbackslash{}n\textbackslash{}n— Então é que desejo fazê-la feliz e\textbackslash{}nfeliz pelo casamento. Um rapaz, digno dela a todos os respeitos, pediu-ma há\textbackslash{}ndias, e eu disse-lhe que sim. Em sabendo quem é, aprovarás a escolha; é o\textbackslash{}nGomes. Casamo-la, não?\textbackslash{}n\textbackslash{}n— Não! respondeu Augusta.\textbackslash{}n\textbackslash{}n— Como, não?\textbackslash{}n\textbackslash{}n— Adelaide é uma criança; não tem juízo\textbackslash{}nnem idade própria{\ldots} Casar-se-á quando for tempo.\textbackslash{}n\textbackslash{}n— Quando for tempo? Estás certa se o\textbackslash{}nnoivo esperará até que seja tempo?\textbackslash{}n\textbackslash{}n— Paciência, disse Augusta.\textbackslash{}n\textbackslash{}n— Tens alguma coisa que notar no Gomes?\textbackslash{}n\textbackslash{}n— Nada. É um moço distinto; mas não\textbackslash{}nconvém a Adelaide.\textbackslash{}n\textbackslash{}nVasconcelos hesitava em continuar;\textbackslash{}nparecia-lhe que nada se podia arranjar; mas a idéia da fortuna deu-lhe forças,\textbackslash{}ne ele perguntou:\textbackslash{}n\textbackslash{}n— Por quê?\textbackslash{}n\textbackslash{}n— Estás certo de que ele convenha a\textbackslash{}nAdelaide? perguntou Augusta, eludindo a pergunta do marido.\textbackslash{}n\textbackslash{}n— Afirmo que convém.\textbackslash{}n\textbackslash{}n— Convenha ou não, a pequena não deve\textbackslash{}ncasar já.\textbackslash{}n\textbackslash{}n— E se ela amasse?{\ldots}\textbackslash{}n\textbackslash{}n— Que importa isso? esperaria!\textbackslash{}n\textbackslash{}n— Entretanto, Augusta, não podemos\textbackslash{}nprescindir deste casamento{\ldots} É uma necessidade fatal.\textbackslash{}n\textbackslash{}n— Fatal? não compreendo.\textbackslash{}n\textbackslash{}n— Vou explicar-me. O Gomes tem uma boa\textbackslash{}nfortuna.\textbackslash{}n\textbackslash{}n— Também nós temos uma{\ldots}\textbackslash{}n\textbackslash{}n— É o teu engano, interrompeu\textbackslash{}nVasconcelos.\textbackslash{}n\textbackslash{}n— Como assim?\textbackslash{}n\textbackslash{}nVasconcelos continuou:\textbackslash{}n\textbackslash{}n— Mais tarde ou mais cedo havias de\textbackslash{}nsabê-lo, e eu estimo ter esta ocasião de dizer-te toda a verdade. A verdade é\textbackslash{}nque, se não estamos pobres, estamos arruinados.\textbackslash{}n\textbackslash{}nAugusta ouviu estas palavras com os olhos\textbackslash{}nespantados. Quando ele acabou, disse:\textbackslash{}n\textbackslash{}n— Não é possível!\textbackslash{}n\textbackslash{}n— Infelizmente é verdade!\textbackslash{}n\textbackslash{}nSeguiu-se algum tempo de silêncio.\textbackslash{}n\textbackslash{}n“Tudo está arranjado”, pensou\textbackslash{}nVasconcelos.\textbackslash{}n\textbackslash{}nAugusta rompeu o silêncio.\textbackslash{}n\textbackslash{}n— Mas, disse ela, se a nossa fortuna está\textbackslash{}nabalada, creio que o senhor tem coisa melhor para fazer do que estar\textbackslash{}nconversando; é reconstruí-la.\textbackslash{}n\textbackslash{}nVasconcelos fez com a cabeça um movimento\textbackslash{}nde espanto, e como se fosse aquilo uma pergunta, Augusta apressou-se a\textbackslash{}nresponder:\textbackslash{}n\textbackslash{}n— Não se admire disto; creio que o seu\textbackslash{}ndever é reconstruir a fortuna.\textbackslash{}n\textbackslash{}n— Não me admira esse dever; admira-me que\textbackslash{}nmo lembres por esse modo. Dir-se-ia que a culpa é minha{\ldots}\textbackslash{}n\textbackslash{}n— Bom! disse Augusta, vais dizer que fui\textbackslash{}neu{\ldots}\textbackslash{}n\textbackslash{}n— A culpa, se culpa há, é de nós ambos.\textbackslash{}n\textbackslash{}n— Por quê? é também minha?\textbackslash{}n\textbackslash{}n— Também. As tuas despesas loucas\textbackslash{}ncontribuíram em grande parte para este resultado; eu nada te recusei nem\textbackslash{}nrecuso, e é nisso que sou culpado. Se é isso que me lanças em rosto, aceito.\textbackslash{}n\textbackslash{}nAugusta levantou os ombros com um gesto de\textbackslash{}ndespeito; e deitou a Vasconcelos um olhar de tamanho desdém que bastaria para\textbackslash{}nintentar uma ação de divórcio.\textbackslash{}n\textbackslash{}nVasconcelos viu o movimento e o olhar.\textbackslash{}n\textbackslash{}n— O amor do luxo e do supérfluo, disse ele,\textbackslash{}nhá de sempre produzir estas conseqüências. São terríveis, mas explicáveis. Para\textbackslash{}nconjurá-las era preciso viver com moderação. Nunca pensaste nisso. No fim de\textbackslash{}nseis meses de casada entraste a viver no turbilhão da moda, e o pequeno regato\textbackslash{}ndas despesas tornou-se um rio imenso de desperdícios. Sabes o que me disse uma\textbackslash{}nvez meu irmão? Disse-me que a idéia de mandar Adelaide para a roça foi-te\textbackslash{}nsugerida pela necessidade de viver sem cuidados de natureza alguma.\textbackslash{}n\textbackslash{}nAugusta tinha-se levantado, e deu alguns\textbackslash{}npassos; estava trêmula e pálida.\textbackslash{}n\textbackslash{}nVasconcelos ia por diante nas suas\textbackslash{}nrecriminações, quando a mulher o interrompeu, dizendo:\textbackslash{}n\textbackslash{}n— Mas por que motivo não impediu o senhor\textbackslash{}nessas despesas que eu fazia?\textbackslash{}n\textbackslash{}n— Queria a paz doméstica.\textbackslash{}n\textbackslash{}n— Não! clamou ela; o senhor queria ter\textbackslash{}npor sua parte uma vida livre e independente; vendo que eu me entregava a essas\textbackslash{}ndespesas imaginou comprar a minha tolerância com a sua tolerância. Eis o único\textbackslash{}nmotivo; a sua vida não será igual à minha; mas é pior{\ldots} Se eu fazia despesas\textbackslash{}nem casa o senhor as fazia na rua{\ldots} É inútil negar, porque eu sei tudo;\textbackslash{}nconheço, de nome, as rivais que sucessivamente o senhor me deu, e nunca lhe\textbackslash{}ndisse uma única palavra, nem agora lho censuro, porque seria inútil e tarde.\textbackslash{}n\textbackslash{}nA situação tinha mudado. Vasconcelos\textbackslash{}ncomeçara constituindo-se juiz, e passara a ser co-réu. Negar era impossível;\textbackslash{}ndiscutir era arriscado e inútil. Preferiu sofismar.\textbackslash{}n\textbackslash{}n— Dado que fosse assim (e eu não discuto\textbackslash{}nesse ponto), em todo caso a culpa será de nós ambos, e não vejo razão para que\textbackslash{}nma lances em rosto. Devo reparar a fortuna, concordo; há um meio, e é este: o\textbackslash{}ncasamento de Adelaide com o Gomes.\textbackslash{}n\textbackslash{}n— Não! disse Augusta.\textbackslash{}n\textbackslash{}n— Bem; seremos pobres, ficaremos piores\textbackslash{}ndo que estamos agora; venderemos tudo{\ldots}\textbackslash{}n\textbackslash{}n— Perdão, disse Augusta, eu não sei por\textbackslash{}nque razão não há de o senhor, que é forte, e tem a maior parte no desastre,\textbackslash{}nempregar esforços para a reconstrução da fortuna destruída.\textbackslash{}n\textbackslash{}n— É trabalho longo; e daqui até lá a vida\textbackslash{}ncontinua e gasta-se. O meio, já lho disse, é este: casar Adelaide com o Gomes.\textbackslash{}n\textbackslash{}n— Não quero! disse Augusta, não consinto\textbackslash{}nem semelhante casamento.\textbackslash{}n\textbackslash{}nVasconcelos ia responder, mas Augusta,\textbackslash{}nlogo depois de proferir estas palavras, tinha saído precipitadamente do\textbackslash{}ngabinete.\textbackslash{}n\textbackslash{}nVasconcelos saiu alguns minutos depois.\textbackslash{}n\textbackslash{}nCAPÍTULO VI\textbackslash{}n\textbackslash{}nLourenço não teve conhecimento da cena\textbackslash{}nentre o irmão e a cunhada, e depois da teima de Vasconcelos resolveu nada mais\textbackslash{}ndizer; entretanto, como queria muito à sobrinha, e não queria vê-la entregue a\textbackslash{}num homem de costumes que ele reprovava, Lourenço esperou que a situação tomasse\textbackslash{}ncaráter mais decisivo para assumir mais ativo papel.\textbackslash{}n\textbackslash{}nMas, a fim de não perder tempo, e poder\textbackslash{}nusar alguma arma poderosa, Lourenço tratou de instaurar uma pesquisa mediante a\textbackslash{}nqual pudesse colher informações minuciosas acerca de Gomes.\textbackslash{}n\textbackslash{}nEste cuidava que o casamento era coisa\textbackslash{}ndecidida, e não perdia um só dia na conquista de Adelaide.\textbackslash{}n\textbackslash{}nNotou, porém, que Augusta tornava-se mais\textbackslash{}nfria e indiferente, sem causa que ele conhecesse, e entrou-lhe no espírito a\textbackslash{}nsuspeita de que viesse dali alguma oposição.\textbackslash{}n\textbackslash{}nQuanto a Vasconcelos, desanimado pela\textbackslash{}ncena da toilette, esperou melhores dias, e contou sobretudo com o\textbackslash{}nimpério da necessidade.\textbackslash{}n\textbackslash{}nUm dia, porém, exatamente quarenta e oito\textbackslash{}nhoras depois da grande discussão com Augusta, Vasconcelos fez dentro de si esta\textbackslash{}npergunta:\textbackslash{}n\textbackslash{}n'Augusta recusa a mão de Adelaide\textbackslash{}npara o Gomes; por quê?'\textbackslash{}n\textbackslash{}nDe pergunta em pergunta, de dedução em\textbackslash{}ndedução, abriu-se no espírito de Vasconcelos campo para uma suspeita dolorosa.\textbackslash{}n\textbackslash{}n'Amá-lo-á ela?' perguntou ele a\textbackslash{}nsi próprio.\textbackslash{}n\textbackslash{}nDepois, como se o abismo atraísse o\textbackslash{}nabismo, e uma suspeita reclamasse outra, Vasconcelos perguntou:\textbackslash{}n\textbackslash{}n— Ter-se-iam eles amado algum tempo?\textbackslash{}n\textbackslash{}nPela primeira vez, Vasconcelos sentiu\textbackslash{}nmorder-lhe no coração a serpe do ciúme.\textbackslash{}n\textbackslash{}nDo ciúme digo eu, por eufemismo; não sei\textbackslash{}nse aquilo era ciúme; era amor-próprio ofendido.\textbackslash{}n\textbackslash{}nAs suspeitas de Vasconcelos teriam razão?\textbackslash{}n\textbackslash{}nDevo dizer a verdade: não tinham. Augusta\textbackslash{}nera vaidosa, mas era fiel ao infiel marido; e isso por dois motivos: um de\textbackslash{}nconsciência, outro de temperamento. Ainda que ela não estivesse convencida do\textbackslash{}nseu dever de esposa, é certo que nunca trairia o juramento conjugal. Não era\textbackslash{}nfeita para as paixões, a não ser as paixões ridículas que a vaidade impõe. Ela\textbackslash{}namava antes de tudo a sua própria beleza; o seu melhor amigo era o que dissesse\textbackslash{}nque ela era mais bela entre as mulheres; mas se lhe dava a sua amizade, não lhe\textbackslash{}ndaria nunca o coração; isso a salvava.\textbackslash{}n\textbackslash{}nA verdade é esta; mas quem o diria a\textbackslash{}nVasconcelos? Uma vez suspeitoso de que a sua honra estava afetada, Vasconcelos começou\textbackslash{}na recapitular toda a sua vida. Gomes freqüentava a sua casa há seis anos, e\textbackslash{}ntinha nela plena liberdade. A traição era fácil. Vasconcelos entrou a recordar\textbackslash{}nas palavras, os gestos, os olhares, tudo que antes lhe foi indiferente, e que\textbackslash{}nnaquele momento tomava um caráter suspeitoso.\textbackslash{}n\textbackslash{}nDois dias andou Vasconcelos cheio deste\textbackslash{}npensamento. Não saía de casa. Quando Gomes chegava, Vasconcelos observava a\textbackslash{}nmulher com desusada persistência; a própria frieza com que ela recebia o rapaz\textbackslash{}nera aos olhos do marido uma prova do delito.\textbackslash{}n\textbackslash{}nEstava nisto, quando na manhã do terceiro\textbackslash{}ndia (Vasconcelos já se levantava cedo) entrou-lhe no gabinete o irmão, sempre\textbackslash{}ncom ar selvagem do costume.\textbackslash{}n\textbackslash{}nA presença de Lourenço inspirou a\textbackslash{}nVasconcelos a idéia de contar-lhe tudo.\textbackslash{}n\textbackslash{}nLourenço era um homem de bom senso, e em\textbackslash{}ncaso de necessidade era um apoio.\textbackslash{}n\textbackslash{}nO irmão ouviu tudo quanto Vasconcelos\textbackslash{}ncontou, e concluindo este, rompeu o seu silêncio com estas palavras:\textbackslash{}n\textbackslash{}n— Tudo isso é uma tolice; se tua mulher recusa\textbackslash{}no casamento, será por qualquer outro motivo que não esse.\textbackslash{}n\textbackslash{}n— Mas é o casamento com o Gomes que ela\textbackslash{}nrecusa.\textbackslash{}n\textbackslash{}n— Sim, porque lhe falaste no Gomes;\textbackslash{}nfala-lhe em outro, talvez recuse do mesmo modo. Há de haver outro motivo;\textbackslash{}ntalvez Adelaide lhe contasse, talvez lhe pedisse para opor-se, porque tua filha\textbackslash{}nnão ama o rapaz, e não pode casar com ele.\textbackslash{}n\textbackslash{}n— Não casará.\textbackslash{}n\textbackslash{}n— Não só por isso, mas até porque{\ldots}\textbackslash{}n\textbackslash{}n— Acaba.\textbackslash{}n\textbackslash{}n— Até porque este casamento é uma\textbackslash{}nespeculação do Gomes.\textbackslash{}n\textbackslash{}n— Uma especulação? perguntou Vasconcelos.\textbackslash{}n\textbackslash{}n— Igual à tua, disse Lourenço. Tu dás-lhe\textbackslash{}na filha com os olhos na fortuna dele; ele aceita-a com os olhos na tua\textbackslash{}nfortuna{\ldots}\textbackslash{}n\textbackslash{}n— Mas ele possui{\ldots}\textbackslash{}n\textbackslash{}n— Não possui nada; está arruinado como\textbackslash{}ntu. Indaguei e soube da verdade. Quer naturalmente continuar a mesma vida\textbackslash{}ndissipada que teve até hoje, e a tua fortuna é um meio{\ldots}\textbackslash{}n\textbackslash{}n— Estás certo disso?\textbackslash{}n\textbackslash{}n— Certíssimo!{\ldots}\textbackslash{}n\textbackslash{}nVasconcelos ficou aterrado. No meio de\textbackslash{}ntodas as suspeitas, ainda lhe restava a esperança de ver a sua honra salva, e realizado\textbackslash{}naquele negócio que lhe daria uma excelente situação.\textbackslash{}n\textbackslash{}nMas a revelação de Lourenço matou-o.\textbackslash{}n\textbackslash{}n— Se queres uma prova, manda chamá-lo, e\textbackslash{}ndize-lhe que estás pobre, e por isso lhe recusas a filha; observa-o bem, e\textbackslash{}nverás o efeito que as tuas palavras lhe hão de produzir.\textbackslash{}n\textbackslash{}nNão foi preciso mandar chamar o\textbackslash{}npretendente. Daí a uma hora apresentou-se ele em casa de Vasconcelos.\textbackslash{}n\textbackslash{}nVasconcelos mandou-o subir ao gabinete.\textbackslash{}n\textbackslash{}nCAPÍTULO VII\textbackslash{}n\textbackslash{}nLogo depois dos primeiros cumprimentos\textbackslash{}nVasconcelos disse:\textbackslash{}n\textbackslash{}n— Ia mandar chamar-te.\textbackslash{}n\textbackslash{}n— Ah! para quê? perguntou Gomes.\textbackslash{}n\textbackslash{}n— Para conversarmos acerca do{\ldots}\textbackslash{}ncasamento.\textbackslash{}n\textbackslash{}n— Ah! há algum obstáculo?\textbackslash{}n\textbackslash{}n— Conversemos.\textbackslash{}n\textbackslash{}nGomes tornou-se mais sério; entrevia\textbackslash{}nalguma dificuldade grande.\textbackslash{}n\textbackslash{}nVasconcelos tomou a palavra.\textbackslash{}n\textbackslash{}n— Há circunstâncias, disse ele, que devem\textbackslash{}nser bem definidas, para que se possa compreender bem{\ldots}\textbackslash{}n\textbackslash{}n— É a minha opinião.\textbackslash{}n\textbackslash{}n— Amas minha filha?\textbackslash{}n\textbackslash{}n— Quantas vezes queres que to diga?\textbackslash{}n\textbackslash{}n— O teu amor está acima de todas as\textbackslash{}ncircunstâncias?{\ldots}\textbackslash{}n\textbackslash{}n— De todas, salvo aquelas que entenderem\textbackslash{}ncom a felicidade dela.\textbackslash{}n\textbackslash{}n— Devemos ser francos; além de amigo que\textbackslash{}nsempre foste, és agora quase meu filho{\ldots} A discrição entre nós seria\textbackslash{}nindiscreta{\ldots}\textbackslash{}n\textbackslash{}n— Sem dúvida! respondeu Gomes.\textbackslash{}n\textbackslash{}n— Vim a saber que os meus negócios param\textbackslash{}nmal; as despesas que fiz alteraram profundamente a economia da minha vida, de\textbackslash{}nmodo que eu não te minto dizendo que estou pobre.\textbackslash{}n\textbackslash{}nGomes reprimiu uma careta.\textbackslash{}n\textbackslash{}n— Adelaide, continuou Vasconcelos, não\textbackslash{}ntem fortuna, não terá mesmo dote; é apenas uma mulher que eu te dou. O que te\textbackslash{}nafianço é que é um anjo, e que há de ser excelente esposa.\textbackslash{}n\textbackslash{}nVasconcelos calou-se, e o seu olhar\textbackslash{}ncravado no rapaz parecia querer arrancar-lhe das feições as impressões da alma.\textbackslash{}n\textbackslash{}nGomes devia responder; mas durante alguns\textbackslash{}nminutos houve entre ambos um profundo silêncio.\textbackslash{}n\textbackslash{}nEnfim o pretendente tomou a palavra.\textbackslash{}n\textbackslash{}n— Aprecio, disse ele, a tua franqueza, e\textbackslash{}nusarei de franqueza igual.\textbackslash{}n\textbackslash{}n— Não peço outra coisa{\ldots}\textbackslash{}n\textbackslash{}n— Não foi por certo o dinheiro que me\textbackslash{}ninspirou este amor; creio que me farás a justiça de crer que eu estou acima\textbackslash{}ndessas considerações. Além de que, no dia em que eu te pedi a querida do meu\textbackslash{}ncoração, acreditava estar rico.\textbackslash{}n\textbackslash{}n— Acreditavas?\textbackslash{}n\textbackslash{}n— Escuta. Só ontem é que o meu procurador\textbackslash{}nme comunicou o estado dos meus negócios.\textbackslash{}n\textbackslash{}n— Mau?\textbackslash{}n\textbackslash{}n— Se fosse isso apenas! Mas imagina que\textbackslash{}nhá seis meses estou vivendo pelos esforços inauditos que o meu procurador fez\textbackslash{}npara apurar algum dinheiro, pois que ele não tinha ânimo de dizer-me a verdade.\textbackslash{}nOntem soube tudo!\textbackslash{}n\textbackslash{}n— Ah!\textbackslash{}n\textbackslash{}n— Calcula qual é o desespero de um homem\textbackslash{}nque acredita estar bem, e reconhece um dia que não tem nada!\textbackslash{}n\textbackslash{}n— Imagino por mim!\textbackslash{}n\textbackslash{}n— Entrei alegre aqui, porque a alegria\textbackslash{}nque eu ainda tenho reside nesta casa; mas a verdade é que estou à beira de um abismo.\textbackslash{}nA sorte castigou-nos a um tempo{\ldots}\textbackslash{}n\textbackslash{}nDepois desta narração, que Vasconcelos\textbackslash{}nouviu sem pestanejar, Gomes entrou no ponto mais difícil da questão.\textbackslash{}n\textbackslash{}n— Aprecio a tua franqueza, e aceito a tua\textbackslash{}nfilha sem fortuna; também eu não tenho, mas ainda me restam forças para\textbackslash{}ntrabalhar.\textbackslash{}n\textbackslash{}n— Aceitas?\textbackslash{}n\textbackslash{}n— Escuta. Aceito D. Adelaide, mediante\textbackslash{}numa condição; é que ela queira esperar algum tempo, a fim de que eu comece a\textbackslash{}nminha vida. Pretendo ir ao governo e pedir um lugar qualquer, se é que ainda me\textbackslash{}nlembro do que aprendi na escola{\ldots} Apenas tenha começado a vida, cá virei\textbackslash{}nbuscá-la. Queres?\textbackslash{}n\textbackslash{}n— Se ela consentir, disse Vasconcelos\textbackslash{}nabraçando esta tábua de salvação, é coisa decidida.\textbackslash{}n\textbackslash{}nGomes continuou:\textbackslash{}n\textbackslash{}n— Bem, falarás nisso amanhã, e\textbackslash{}nmandar-me-ás resposta. Ah! se eu tivesse ainda a minha fortuna! Era agora que\textbackslash{}neu queria provar-te a minha estima!\textbackslash{}n\textbackslash{}n— Bem, ficamos nisto.\textbackslash{}n\textbackslash{}n— Espero a tua resposta.\textbackslash{}n\textbackslash{}nE despediram-se.\textbackslash{}n\textbackslash{}nVasconcelos ficou fazendo esta reflexão:\textbackslash{}n\textbackslash{}n'De tudo quanto ele disse só\textbackslash{}nacredito que já não tem nada. Mas é inútil esperar: duro com duro não faz bom\textbackslash{}nmuro.'\textbackslash{}n\textbackslash{}nPela sua parte Gomes desceu a escada\textbackslash{}ndizendo consigo:\textbackslash{}n\textbackslash{}n'O que acho singular é que estando\textbackslash{}npobre viesse dizer-mo assim tão antecipadamente quando eu estava caído. Mas esperarás\textbackslash{}ndebalde: duas metades de cavalo não fazem um cavalo.'\textbackslash{}n\textbackslash{}nVasconcelos desceu.\textbackslash{}n\textbackslash{}nA sua intenção era comunicar a Augusta o\textbackslash{}nresultado da conversa com o pretendente. Uma coisa, porém, o embaraçava: era a\textbackslash{}ninsistência de Augusta em não consentir no casamento de Adelaide, sem dar\textbackslash{}nnenhuma razão da recusa.\textbackslash{}n\textbackslash{}nIa pensando nisto, quando, ao atravessar\textbackslash{}na sala de espera, ouviu vozes na sala de visitas.\textbackslash{}n\textbackslash{}nEra Augusta que conversava com Carlota.\textbackslash{}n\textbackslash{}nIa entrar quando estas palavras lhe\textbackslash{}nchegaram ao ouvido:\textbackslash{}n\textbackslash{}n— Mas Adelaide é muito criança.\textbackslash{}n\textbackslash{}nEra a voz de Augusta.\textbackslash{}n\textbackslash{}n— Criança! disse Carlota.\textbackslash{}n\textbackslash{}n— Sim; não está em idade de casar.\textbackslash{}n\textbackslash{}n— Mas eu no teu caso não punha embargos\textbackslash{}nao casamento, ainda que fosse daqui a alguns meses, porque o Gomes não me\textbackslash{}nparece mau rapaz{\ldots}\textbackslash{}n\textbackslash{}n— Não é; mas enfim eu não quero que\textbackslash{}nAdelaide se case.\textbackslash{}n\textbackslash{}nVasconcelos colou o ouvido à fechadura, e\textbackslash{}ntemia perder uma só palavra do diálogo.\textbackslash{}n\textbackslash{}n— O que eu não compreendo, disse Carlota,\textbackslash{}né a tua insistência. Mais tarde ou mais cedo Adelaide há de vir a casar-se.\textbackslash{}n\textbackslash{}n— Oh! o mais tarde possível, disse\textbackslash{}nAugusta.\textbackslash{}n\textbackslash{}nHouve um silêncio.\textbackslash{}n\textbackslash{}nVasconcelos estava impaciente.\textbackslash{}n\textbackslash{}n— Ah! continuou Augusta, se soubesses o\textbackslash{}nterror que me dá a idéia do casamento de Adelaide{\ldots}\textbackslash{}n\textbackslash{}n— Por que, meu Deus?\textbackslash{}n\textbackslash{}n— Por que, Carlota? Tu pensas em tudo,\textbackslash{}nmenos numa coisa. Eu tenho medo por causa dos filhos dela que serão meus netos!\textbackslash{}nA idéia de ser avó é horrível, Carlota.\textbackslash{}n\textbackslash{}nVasconcelos respirou, e abriu a porta.\textbackslash{}n\textbackslash{}n— Ah! disse Augusta.\textbackslash{}n\textbackslash{}nVasconcelos cumprimentou Carlota, e\textbackslash{}napenas esta saiu, voltou-se para a mulher, e disse:\textbackslash{}n\textbackslash{}n— Ouvi a tua conversa com aquela\textbackslash{}nmulher{\ldots}\textbackslash{}n\textbackslash{}n— Não era segredo; mas{\ldots} que ouviste?\textbackslash{}n\textbackslash{}nVasconcelos respondeu sorrindo:\textbackslash{}n\textbackslash{}n— Ouvi a causa dos teus terrores. Não\textbackslash{}ncuidei nunca que o amor da própria beleza pudesse levar a tamanho egoísmo. O\textbackslash{}ncasamento com o Gomes não se realiza; mas se Adelaide amar alguém, não sei como\textbackslash{}nlhe recusaremos o nosso consentimento{\ldots}\textbackslash{}n\textbackslash{}n— Até lá{\ldots} esperemos, respondeu Augusta.\textbackslash{}n\textbackslash{}nA conversa parou nisto; porque aqueles\textbackslash{}ndois consortes distanciavam-se muito; um tinha a cabeça nos prazeres ruidosos\textbackslash{}nda mocidade, ao passo que a outra meditava exclusivamente em si.\textbackslash{}n\textbackslash{}nNo dia seguinte Gomes recebeu uma carta\textbackslash{}nde Vasconcelos concebida nestes termos:\textbackslash{}n\textbackslash{}nMeu Gomes.\textbackslash{}n\textbackslash{}nOcorre uma circunstância inesperada; é\textbackslash{}nque Adelaide não quer casar. Gastei a minha lógica, mas não alcancei\textbackslash{}nconvencê-la.\textbackslash{}n\textbackslash{}nTeu Vasconcelos.\textbackslash{}n\textbackslash{}nGomes dobrou a carta e acendeu com ela um\textbackslash{}ncharuto, e começou a fumar fazendo esta reflexão profunda:\textbackslash{}n\textbackslash{}n'Onde acharei eu uma herdeira que me\textbackslash{}nqueira por marido?'\textbackslash{}n\textbackslash{}nSe alguém souber avise-o em tempo.\textbackslash{}n\textbackslash{}nDepois do que acabamos de contar,\textbackslash{}nVasconcelos e Gomes encontram-se às vezes na rua ou no Alcazar; conversam,\textbackslash{}nfumam, dão o braço um ao outro, exatamente como dois amigos, que nunca foram,\textbackslash{}nou como dois velhacos que são.\textbackslash{}n\textbackslash{}nConfissões de UMA VIÚVA MOÇA\textbackslash{}n\textbackslash{}nÍNDICE\textbackslash{}n\textbackslash{}nCapítulo Primeiro\textbackslash{}n\textbackslash{}nCapítulo II\textbackslash{}n\textbackslash{}nCapítulo iii\textbackslash{}n\textbackslash{}nCapítulo iv\textbackslash{}n\textbackslash{}nCapítulo v\textbackslash{}n\textbackslash{}nCapítulo vI\textbackslash{}n\textbackslash{}nCapítulo vII\textbackslash{}n\textbackslash{}nCAPÍTULO PRIMEIRO\textbackslash{}n\textbackslash{}nHá dois anos tomei uma resolução\textbackslash{}nsingular: fui residir em Petrópolis em pleno mês de junho. Esta resolução abriu\textbackslash{}nlargo campo às conjeturas. Tu mesma nas cartas que me escreveste para aqui,\textbackslash{}ndeitaste o espírito a adivinhar e figuraste mil razões, cada qual mais absurda.\textbackslash{}n\textbackslash{}nA estas cartas, em que a tua solicitude\textbackslash{}ntraía a um tempo dois sentimentos, a afeição da amiga e a curiosidade de\textbackslash{}nmulher, a essas cartas não respondi e nem podia responder. Não era oportuno\textbackslash{}nabrir-te o meu coração nem desfiar-te a série de motivos que me arredou da\textbackslash{}ncorte, onde as óperas do Teatro Lírico, as tuas partidas e os serões familiares\textbackslash{}ndo primo Barros deviam distrair-me da recente viuvez.\textbackslash{}n\textbackslash{}nEsta circunstância de viuvez recente\textbackslash{}nacreditavam muitos que fosse o único motivo da minha fuga. Era a versão menos\textbackslash{}nequívoca. Deixei-a passar como todas as outras e conservei-me em Petrópolis.\textbackslash{}n\textbackslash{}nLogo no verão\textbackslash{}nseguinte vieste com teu marido para cá, disposta a não voltar para a corte sem\textbackslash{}nlevar o segredo que eu teimava em não revelar. A palavra não fez mais do que a\textbackslash{}ncarta. Fui discreta como um túmulo, indecifrável como a Esfinge. Depuseste as\textbackslash{}narmas e partiste.\textbackslash{}n\textbackslash{}nDesde então não\textbackslash{}nme trataste senão por tua Esfinge.\textbackslash{}n\textbackslash{}nEra Esfinge, era. E se, como Édipo,\textbackslash{}ntivesses respondido ao meu enigma a palavra 'homem', descobririas o\textbackslash{}nmeu segredo, e desfarias o meu encanto.\textbackslash{}n\textbackslash{}nMas não antecipemos os acontecimentos,\textbackslash{}ncomo se diz nos romances.\textbackslash{}n\textbackslash{}nÉ tempo de contar-te este episódio da\textbackslash{}nminha vida.\textbackslash{}n\textbackslash{}nQuero fazê-lo por cartas e não por boca.\textbackslash{}nTalvez corasse de ti. Deste modo o coração abre-se melhor e a vergonha não vem\textbackslash{}ntolher a palavra nos lábios. Repara que eu não falo em lágrimas, o que é um\textbackslash{}nsintoma de que a paz voltou ao meu espírito.\textbackslash{}n\textbackslash{}nAs minhas cartas irão de oito em oito\textbackslash{}ndias, de maneira que a narrativa pode fazer-te o efeito de um folhetim de\textbackslash{}nperiódico semanal.\textbackslash{}n\textbackslash{}nDou-te a minha palavra de que hás de\textbackslash{}ngostar e aprender.\textbackslash{}n\textbackslash{}nE oito dias depois da minha última carta\textbackslash{}nirei abraçar-te, beijar-te, agradecer-te. Tenho necessidade de viver. Estes\textbackslash{}ndois anos são nulos na conta de minha vida: foram dois anos de tédio, de\textbackslash{}ndesespero íntimo, de orgulho abatido, de amor abafado.\textbackslash{}n\textbackslash{}nLia, é verdade. Mas só o tempo, a\textbackslash{}nausência, a idéia do meu coração enganado, da minha dignidade ofendida, puderam\textbackslash{}ntrazer-me a calma necessária, a calma de hoje.\textbackslash{}n\textbackslash{}nE sabe que não ganhei só isto. Ganhei\textbackslash{}nconhecer um homem cujo retrato trago no espírito e que me parece singularmente\textbackslash{}nparecido com outros muitos. Já não é pouco; e a lição há de servir-me, como a\textbackslash{}nti, como às nossas amigas inexperientes. Mostra-lhes estas cartas; são folhas\textbackslash{}nde um roteiro que se eu tivera antes, talvez não houvesse perdido uma ilusão e\textbackslash{}ndois anos de vida.\textbackslash{}n\textbackslash{}nDevo terminar esta. É o prefácio do meu\textbackslash{}nromance, estudo, conto, o que quiseres. Não questiono sobre a designação, nem\textbackslash{}nconsulto para isso os mestres d'arte.\textbackslash{}n\textbackslash{}nEstudo ou romance, isto é simplesmente um\textbackslash{}nlivro de verdades, um episódio singelamente contado, na confabulação íntima dos\textbackslash{}nespíritos, na plena confiança de dois corações que se estimam e se merecem.\textbackslash{}n\textbackslash{}nAdeus.\textbackslash{}n\textbackslash{}nCAPÍTULO II\textbackslash{}n\textbackslash{}nEra no tempo de meu marido.\textbackslash{}n\textbackslash{}nA Corte estava então animada e não tinha\textbackslash{}nesta cruel monotonia que eu sinto aqui através das tuas cartas e dos jornais de\textbackslash{}nque sou assinante.\textbackslash{}n\textbackslash{}nMinha casa era um ponto de reunião de\textbackslash{}nalguns rapazes conversados e algumas moças elegantes. Eu, rainha eleita pelo\textbackslash{}nvoto universal{\ldots} de minha casa, presidia aos serões familiares. Fora de casa,\textbackslash{}ntínhamos os teatros animados, as partidas das amigas, mil outras distrações que\textbackslash{}ndavam à minha vida certas alegrias exteriores em falta das íntimas, que são as\textbackslash{}núnicas verdadeiras e fecundas.\textbackslash{}n\textbackslash{}nSe eu não era feliz, vivia alegre.\textbackslash{}n\textbackslash{}nE aqui vai o começo do meu romance.\textbackslash{}n\textbackslash{}nUm dia meu marido pediu-me como obséquio\textbackslash{}nespecial que eu não fosse à noite ao Teatro Lírico. Dizia ele que não podia acompanhar-me\textbackslash{}npor ser véspera de saída de paquete.\textbackslash{}n\textbackslash{}nEra razoável o pedido.\textbackslash{}n\textbackslash{}nNão sei, porém, que espírito mau\textbackslash{}nsussurrou-me ao ouvido e eu respondi peremptoriamente que havia de ir ao\textbackslash{}nteatro, e com ele. Insistiu no pedido, insisti na recusa. Pouco bastou para que\textbackslash{}neu julgasse a minha honra empenhada naquilo. Hoje vejo que era a minha vaidade\textbackslash{}nou o meu destino.\textbackslash{}n\textbackslash{}nEu tinha certa superioridade sobre o\textbackslash{}nespírito de meu marido. O meu tom imperioso não admitia recusa; meu marido\textbackslash{}ncedeu a despeito de tudo, e à noite fomos ao Teatro Lírico.\textbackslash{}n\textbackslash{}nHavia pouca gente e os cantores estavam\textbackslash{}nendefluxados. No fim do primeiro ato meu marido, com um sorriso vingativo,\textbackslash{}ndisse-me estas palavras rindo-se:\textbackslash{}n\textbackslash{}n— Estimei isto.\textbackslash{}n\textbackslash{}n— Isto? perguntei eu franzindo a testa.\textbackslash{}n\textbackslash{}n— Este espetáculo deplorável. Fizeste da\textbackslash{}nvinda hoje ao teatro um capítulo de honra; estimo ver que o espetáculo não\textbackslash{}ncorrespondeu à tua expectativa.\textbackslash{}n\textbackslash{}n— Pelo contrário, acho magnífico.\textbackslash{}n\textbackslash{}n— Está bom.\textbackslash{}n\textbackslash{}nDeves compreender que eu tinha interesse\textbackslash{}nem me não dar por vencida; mas acreditas facilmente que no fundo eu estava\textbackslash{}nperfeitamente aborrecida do espetáculo e da noite.\textbackslash{}n\textbackslash{}nMeu marido, que não ousava retorquir,\textbackslash{}ncalou-se com ar de vencido, e adiantando-se um pouco à frente do camarote\textbackslash{}npercorreu com o binóculo as linhas dos poucos camarotes fronteiros em que havia\textbackslash{}ngente.\textbackslash{}n\textbackslash{}nEu recuei a minha cadeira, e, encostada à\textbackslash{}ndivisão do camarote, olhava para o corredor vendo a gente que passava.\textbackslash{}n\textbackslash{}nNo corredor, exatamente em frente à porta\textbackslash{}ndo nosso camarote, estava um sujeito encostado, fumando e com os olhos fitos em\textbackslash{}nmim. Não reparei ao princípio, mas a insistência obrigou-me a isso. Olhei para\textbackslash{}nele a ver se era algum conhecido nosso que esperava ser descoberto a fim de vir\textbackslash{}nentão cumprimentar-nos. A intimidade podia explicar este brinco. Mas não\textbackslash{}nconheci.\textbackslash{}n\textbackslash{}nDepois de alguns segundos, vendo que ele\textbackslash{}nnão tirava os olhos de mim, desviei os meus e cravei-os no pano da boca e na\textbackslash{}nplatéia.\textbackslash{}n\textbackslash{}nMeu marido, tendo acabado o exame dos\textbackslash{}ncamarotes, deu-me o binóculo e sentou-se ao fundo diante de mim.\textbackslash{}n\textbackslash{}nTrocamos algumas palavras.\textbackslash{}n\textbackslash{}nNo fim de um quarto de hora a orquestra\textbackslash{}ncomeçou os prelúdios para o segundo ato. Levantei-me, meu marido aproximou a\textbackslash{}ncadeira para a frente, e nesse ínterim lancei um olhar furtivo para o corredor.\textbackslash{}n\textbackslash{}nO homem estava lá.\textbackslash{}n\textbackslash{}nDisse a meu marido que fechasse a porta.\textbackslash{}n\textbackslash{}nComeçou o segundo ato.\textbackslash{}n\textbackslash{}nEntão, por um espírito de curiosidade,\textbackslash{}nprocurei ver se o meu observador entrava para as cadeiras. Queria conhecê-lo\textbackslash{}nmelhor no meio da multidão.\textbackslash{}n\textbackslash{}nMas, ou porque não entrasse, ou porque eu\textbackslash{}nnão tivesse reparado bem, o que é certo é que o não vi.\textbackslash{}n\textbackslash{}nCorreu o segundo ato mais aborrecido do\textbackslash{}nque o primeiro.\textbackslash{}n\textbackslash{}nNo intervalo recuei de novo a cadeira, e\textbackslash{}nmeu marido, a pretexto de que fazia calor, abriu a porta do camarote.\textbackslash{}n\textbackslash{}nLancei um olhar para o corredor.\textbackslash{}n\textbackslash{}nNão vi ninguém; mas daí a poucos minutos\textbackslash{}nchegou o mesmo indivíduo, colocando-se no mesmo lugar, e fitou em mim os mesmos\textbackslash{}nolhos impertinentes.\textbackslash{}n\textbackslash{}nSomos todas vaidosas da nossa beleza e\textbackslash{}ndesejamos que o mundo inteiro nos admire. É por isso que muitas vezes temos a\textbackslash{}nindiscrição de admirar a corte mais ou menos arriscada de um homem. Há, porém,\textbackslash{}numa maneira de fazê-la que nos irrita e nos assusta; irrita-nos por\textbackslash{}nimpertinente, assusta-nos por perigosa. É o que se dava naquele caso.\textbackslash{}n\textbackslash{}nO meu admirador insistia de modo tal que\textbackslash{}nme levava a um dilema: ou ele era vítima de uma paixão louca, ou possuía a\textbackslash{}naudácia mais desfaçada. Em qualquer dos casos não era conveniente que eu\textbackslash{}nanimasse as suas adorações.\textbackslash{}n\textbackslash{}nFiz estas reflexões enquanto decorria o\textbackslash{}ntempo do intervalo. Ia começar o terceiro ato. Esperei que o mudo perseguidor\textbackslash{}nse retirasse e disse a meu marido:\textbackslash{}n\textbackslash{}n— Vamos?\textbackslash{}n\textbackslash{}n— Ah!\textbackslash{}n\textbackslash{}n— Tenho sono simplesmente; mas o\textbackslash{}nespetáculo está magnífico.\textbackslash{}n\textbackslash{}nMeu marido ousou exprimir um sofisma.\textbackslash{}n\textbackslash{}n— Se está magnífico como te faz sono?\textbackslash{}n\textbackslash{}nNão lhe dei resposta.\textbackslash{}n\textbackslash{}nSaímos.\textbackslash{}n\textbackslash{}nNo corredor encontramos a família do\textbackslash{}nAzevedo que voltava de uma visita a um camarote conhecido. Demorei-me um pouco\textbackslash{}npara abraçar as senhoras. Disse-lhes que tinha uma dor de cabeça e que me\textbackslash{}nretirava por isso.\textbackslash{}n\textbackslash{}nChegamos à porta da Rua dos Ciganos.\textbackslash{}n\textbackslash{}nAí esperei o carro por alguns minutos.\textbackslash{}n\textbackslash{}nQuem me havia de aparecer ali, encostado\textbackslash{}nao portal fronteiro?\textbackslash{}n\textbackslash{}nO misterioso.\textbackslash{}n\textbackslash{}nEnraiveci.\textbackslash{}n\textbackslash{}nCobri o rosto o mais que pude com o meu\textbackslash{}ncapuz e esperei o carro, que chegou logo.\textbackslash{}n\textbackslash{}nO misterioso lá ficou tão insensível e\textbackslash{}ntão mudo como o portal a que estava encostado.\textbackslash{}n\textbackslash{}nDurante a viagem a idéia daquele\textbackslash{}nincidente não me saiu da cabeça. Fui despertada na minha distração quando o\textbackslash{}ncarro parou à porta da casa, em Mata-cavalos.\textbackslash{}n\textbackslash{}nFiquei envergonhada de mim mesma e decidi\textbackslash{}nnão pensar mais no que se havia passado.\textbackslash{}n\textbackslash{}nMas acreditarás tu, Carlota? Dormi meia\textbackslash{}nhora mais tarde do que supunha, tanto a minha imaginação teimava em reproduzir\textbackslash{}no corredor, o portal, e o meu admirador platônico.\textbackslash{}n\textbackslash{}nNo dia seguinte pensei menos. No fim de\textbackslash{}noito dias tinha-me varrido do espírito aquela cena, e eu dava graças a Deus por\textbackslash{}nhaver-me salvo de uma preocupação que podia ser-me fatal.\textbackslash{}n\textbackslash{}nQuis acompanhar o auxílio divino,\textbackslash{}nresolvendo não ir ao teatro durante algum tempo.\textbackslash{}n\textbackslash{}nSujeitei-me à vida íntima e limitei-me à\textbackslash{}ndistração das reuniões à noite.\textbackslash{}n\textbackslash{}nEntretanto estava próximo o dia dos anos\textbackslash{}nda tua filhinha. Lembrei-me que para tomar parte na tua festa de família, tinha\textbackslash{}ncomeçado um mês antes um trabalhozinho. Cumpria rematá-lo.\textbackslash{}n\textbackslash{}nUma quinta-feira de manhã mandei vir os\textbackslash{}npreparos da obra e ia continuá-la, quando descobri dentre uma meada de lã um\textbackslash{}ninvólucro azul fechando uma carta.\textbackslash{}n\textbackslash{}nEstranhei aquilo. A carta não tinha\textbackslash{}nindicação. Estava colada e parecia esperar que a abrisse a pessoa a quem era\textbackslash{}nendereçada. Quem seria? Seria meu marido? Acostumada a abrir todas as cartas\textbackslash{}nque lhe eram dirigidas, não hesitei. Rompi o invólucro e descobri o papel\textbackslash{}ncor-de-rosa que vinha dentro.\textbackslash{}n\textbackslash{}nDizia a carta:\textbackslash{}n\textbackslash{}nNão se surpreenda, Eugênia; este meio é o\textbackslash{}ndo desespero, este desespero é o do amor. Amo-a e muito. Até certo tempo\textbackslash{}nprocurei fugir-lhe e abafar este sentimento; não posso mais. Não me viu no\textbackslash{}nTeatro Lírico? Era uma força oculta e interior que me levava ali. Desde então\textbackslash{}nnão a vi mais. Quando a verei? Não a veja embora, paciência; mas que o seu\textbackslash{}ncoração palpite por mim um minuto em cada dia, é quanto basta a um amor que não\textbackslash{}nbusca nem as venturas do gozo, nem as galas da publicidade. Se a ofendo, perdoe\textbackslash{}num pecador; se pode amar-me, faça-me um deus.\textbackslash{}n\textbackslash{}nLi esta carta com a mão trêmula e os\textbackslash{}nolhos anuviados; e ainda durante alguns minutos depois não sabia o que era de\textbackslash{}nmim.\textbackslash{}n\textbackslash{}nCruzavam-se e confundiam-se mil idéias na\textbackslash{}nminha cabeça, como estes pássaros negros que perpassam em bandos no céu nas\textbackslash{}nhoras próximas da tempestade.\textbackslash{}n\textbackslash{}nSeria o amor que movera a mão daquele\textbackslash{}nincógnito? Seria simplesmente aquilo um meio do sedutor calculado? Eu lançava\textbackslash{}num olhar vago em derredor e temia ver entrar meu marido.\textbackslash{}n\textbackslash{}nTinha o papel diante de mim e aquelas letras\textbackslash{}nmisteriosas pareciam-me outros tantos olhos de uma serpente infernal. Com um\textbackslash{}nmovimento nervoso e involuntário amarrotei a carta nas mãos.\textbackslash{}n\textbackslash{}nSe Eva tivesse feito outro tanto à cabeça\textbackslash{}nda serpente que a tentava não houvera pecado. Eu não podia estar certa do mesmo\textbackslash{}nresultado, porque esta que me aparecia ali e cuja cabeça eu esmagava, podia,\textbackslash{}ncomo a hidra de Lerna, brotar muitas outras cabeças.\textbackslash{}n\textbackslash{}nNão cuides que eu fazia então esta dupla\textbackslash{}nevocação bíblica e pagã. Naquele momento, não refletia, desvairava; só muito\textbackslash{}ntempo depois pude ligar duas idéias.\textbackslash{}n\textbackslash{}nDois sentimentos atuavam em mim:\textbackslash{}nprimeiramente, uma espécie de terror que infundia o abismo, abismo profundo que\textbackslash{}neu pressentia atrás daquela carta; depois uma vergonha amarga de ver que eu não\textbackslash{}nestava tão alta na consideração daquele desconhecido, que pudesse demovê-lo do\textbackslash{}nmeio que empregou.\textbackslash{}n\textbackslash{}nQuando o meu espírito se acalmou é que eu\textbackslash{}npude fazer a reflexão que devia acudir-me desde o princípio. Quem poria ali\textbackslash{}naquela carta? Meu primeiro movimento foi para chamar todos os meus fâmulos. Mas\textbackslash{}ndeteve-me logo a idéia de que por uma simples interrogação nada poderia colher\textbackslash{}ne ficava divulgado o achado da carta. De que valia isto?\textbackslash{}n\textbackslash{}nNão chamei ninguém.\textbackslash{}n\textbackslash{}nEntretanto, dizia eu comigo, a empresa\textbackslash{}nfoi audaz; podia falhar a cada trâmite; que móvel impeliu àquele homem a dar\textbackslash{}neste passo? Seria amor ou sedução?\textbackslash{}n\textbackslash{}nVoltando a este dilema, meu espírito,\textbackslash{}napesar dos perigos, comprazia-se em aceitar a primeira hipótese: era a que\textbackslash{}nrespeitava a minha consideração de mulher casada e a minha vaidade de mulher\textbackslash{}nformosa.\textbackslash{}n\textbackslash{}nQuis adivinhar lendo a carta de novo:\textbackslash{}nli-a, não uma, mas duas, três, cinco vezes.\textbackslash{}n\textbackslash{}nUma curiosidade indiscreta prendia-me\textbackslash{}nàquele papel. Fiz um esforço e resolvi aniquilá-lo, protestando que ao segundo\textbackslash{}ncaso nenhum escravo ou criado me ficaria em casa.\textbackslash{}n\textbackslash{}nAtravessei a sala com o papel na mão,\textbackslash{}ndirigi-me para o meu gabinete, onde acendi uma vela e queimei aquela carta que\textbackslash{}nme queimava as mãos e a cabeça.\textbackslash{}n\textbackslash{}nQuando a última faísca do papel enegreceu\textbackslash{}ne voou, senti passos atrás de mim. Era meu marido.\textbackslash{}n\textbackslash{}nTive um movimento espontâneo: atirei-me\textbackslash{}nem seus braços.\textbackslash{}n\textbackslash{}nEle abraçou-me com certo espanto.\textbackslash{}n\textbackslash{}nE quando o meu abraço se prolongava senti\textbackslash{}nque ele me repelia com brandura dizendo-me:\textbackslash{}n\textbackslash{}n— Está bom, olha que me afogas!\textbackslash{}n\textbackslash{}nRecuei.\textbackslash{}n\textbackslash{}nEstristeceu-me ver aquele homem, que\textbackslash{}npodia e devia salvar-me, não compreender, por instinto ao menos, que se eu o\textbackslash{}nabraçava tão estreitamente era como se me agarrasse à idéia do dever.\textbackslash{}n\textbackslash{}nMas este sentimento que me apertava o\textbackslash{}ncoração passou um momento para dar lugar a um sentimento de medo. As cinzas da\textbackslash{}ncarta ainda estavam no chão, a vela conservava-se acesa em pleno dia; era\textbackslash{}nbastante para que ele me interrogasse.\textbackslash{}n\textbackslash{}nNem por curiosidade o fez!\textbackslash{}n\textbackslash{}nDeu dois passos no gabinete e saiu.\textbackslash{}n\textbackslash{}nSenti uma lágrima rolar-me pela face. Não\textbackslash{}nera a primeira lágrima de amargura. Seria a primeira advertência do pecado?\textbackslash{}n\textbackslash{}nCAPÍTULO III\textbackslash{}n\textbackslash{}nDecorreu um mês.\textbackslash{}n\textbackslash{}nNão houve durante esse tempo mudança alguma\textbackslash{}nem casa. Nenhuma carta apareceu mais, e a minha vigilância, que era extrema,\textbackslash{}ntornou-se de todo inútil.\textbackslash{}n\textbackslash{}nNão me podia esquecer o incidente da\textbackslash{}ncarta. Se fosse só isto! As primeiras palavras voltavam-me incessantemente à\textbackslash{}nmemória; depois, as outras, as outras, todas. Eu tinha a carta de cor!\textbackslash{}n\textbackslash{}nLembras-te? Uma das minhas vaidades era\textbackslash{}nter a memória feliz. Até neste dote era castigada. Aquelas palavras\textbackslash{}natordoavam-me, faziam-me arder a cabeça. Por quê? Ah! Carlota! é que eu achava\textbackslash{}nnelas um encanto indefinível, encanto doloroso, porque era acompanhado de um\textbackslash{}nremorso, mas encanto de que eu me não podia libertar.\textbackslash{}n\textbackslash{}nNão era o coração que se empenhava, era a\textbackslash{}nimaginação. A imaginação perdia-me; a luta do dever e da imaginação é cruel e\textbackslash{}nperigosa para os espíritos fracos. Eu era fraca. O mistério fascinava a minha\textbackslash{}nfantasia.\textbackslash{}n\textbackslash{}nEnfim os dias e as diversões puderam\textbackslash{}ndesviar o meu espírito daquele pensamento único. No fim de um mês, se eu não\textbackslash{}ntinha esquecido inteiramente o misterioso e a carta dele, estava, todavia,\textbackslash{}nbastante calma para rir de mim e dos meus temores.\textbackslash{}n\textbackslash{}nNa noite de uma quinta-feira, achavam-se\textbackslash{}nalgumas pessoas em minha casa, e muitas das minhas amigas, menos tu. Meu marido\textbackslash{}nnão tinha voltado, e a ausência dele não era notada nem sentida, visto que,\textbackslash{}napesar de franco cavalheiro como era, não tinha o dom particular de um conviva\textbackslash{}npara tais reuniões.\textbackslash{}n\textbackslash{}nTinha-se cantado, tocado, conversado;\textbackslash{}nreinava em todos a mais franca e expansiva alegria; o tio da Amélia Azevedo\textbackslash{}nfazia rir a todos com as suas excentricidades; a Amélia arrebatava bravos a\textbackslash{}ntodos com as notas da sua garganta celeste; estávamos em um intervalo,\textbackslash{}nesperando a hora do chá.\textbackslash{}n\textbackslash{}nAnunciou-se meu marido.\textbackslash{}n\textbackslash{}nNão vinha só. Vinha ao lado dele um homem\textbackslash{}nalto, magro, elegante. Não pude conhecê-lo. Meu marido adiantou-se, e no meio\textbackslash{}ndo silêncio geral veio apresentar-mo.\textbackslash{}n\textbackslash{}nOuvi de meu marido que o nosso conviva\textbackslash{}nchamava-se Emílio.***\textbackslash{}n\textbackslash{}nFixei nele um olhar e retive um grito.\textbackslash{}n\textbackslash{}nEra ele!\textbackslash{}n\textbackslash{}nO meu grito foi substituído por um gesto\textbackslash{}nde surpresa. Ninguém percebeu. Ele pareceu perceber menos que ninguém. Tinha os\textbackslash{}nolhos fixos em mim, e com um gesto gracioso dirigiu-me algumas palavras de\textbackslash{}nlisonjeira cortesia.\textbackslash{}n\textbackslash{}nRespondi como pude.\textbackslash{}n\textbackslash{}nSeguiram-se as apresentações, e durante\textbackslash{}ndez minutos houve um silêncio de acanhamento em todos.\textbackslash{}n\textbackslash{}nOs olhos voltavam-se todos para o\textbackslash{}nrecém-chegado. Eu também voltei os meus e pude reparar naquela figura em que\textbackslash{}ntudo estava disposto para atrair as atenções: cabeça formosa e altiva, olhar\textbackslash{}nprofundo e magnético, maneiras elegantes e delicadas, certo ar distinto e\textbackslash{}npróprio que fazia contraste com o ar afetado e prosaicamente medido dos outros\textbackslash{}nrapazes.\textbackslash{}n\textbackslash{}nEste exame de minha parte foi rápido. Eu\textbackslash{}nnão podia, nem me convinha encontrar o olhar de Emílio. Tornei a abaixar os\textbackslash{}nolhos e esperei ansiosa que a conversação voltasse de novo ao seu curso.\textbackslash{}n\textbackslash{}nMeu marido encarregou-se de dar o tom.\textbackslash{}nInfelizmente era ainda o novo conviva o motivo da conversa geral.\textbackslash{}n\textbackslash{}nSoubemos então que Emílio era um\textbackslash{}nprovinciano filho de pais opulentos, que recebera uma esmerada educação na\textbackslash{}nEuropa, onde não houve um só recanto que não visitasse.\textbackslash{}n\textbackslash{}nVoltara há pouco tempo ao Brasil, e antes\textbackslash{}nde ir para a província tinha determinado passar algum tempo no Rio de Janeiro.\textbackslash{}n\textbackslash{}nFoi tudo quanto soubemos. Vieram as mil\textbackslash{}nperguntas sobre as viagens de Emílio, e este com a mais amável solicitude,\textbackslash{}nsatisfazia a curiosidade geral.\textbackslash{}n\textbackslash{}nSó eu não era curiosa. É que não podia\textbackslash{}narticular palavra. Pedia interiormente a explicação deste romance misterioso,\textbackslash{}ncomeçado em um corredor do teatro, continuado em uma carta anônima e na\textbackslash{}napresentação em minha casa por intermédio de meu próprio marido.\textbackslash{}n\textbackslash{}nDe quando em quando levantava os olhos\textbackslash{}npara Emílio e achava-o calmo e frio, respondendo polidamente às interrogações\textbackslash{}ndos outros e narrando ele próprio, com uma graça modesta e natural, alguma das\textbackslash{}nsuas aventuras de viagem.\textbackslash{}n\textbackslash{}nOcorreu-me uma idéia. Seria realmente ele\textbackslash{}no misterioso do teatro e da carta? Pareceu-me ao princípio que sim, mas eu\textbackslash{}npodia ter-me enganado; eu não tinha as feições do outro bem presentes à\textbackslash{}nmemória; parecia-me que as duas criaturas eram uma e a mesma; mas não podia\textbackslash{}nexplicar-se o engano por uma semelhança miraculosa?\textbackslash{}n\textbackslash{}nDe reflexão em reflexão, foi-me correndo\textbackslash{}no tempo, e eu assistia à conversa de todos como se não estivesse presente. Veio\textbackslash{}na hora do chá. Depois cantou-se e tocou-se ainda. Emílio ouvia tudo com atenção\textbackslash{}nreligiosa e mostrava-se tão apreciador do gosto como era conversador discreto e\textbackslash{}npertinente.\textbackslash{}n\textbackslash{}nNo fim da noite tinha cativado a todos.\textbackslash{}nMeu marido, sobretudo, estava radiante. Via-se que ele se considerava feliz por\textbackslash{}nter feito a descoberta de mais um amigo para si e um companheiro para as nossas\textbackslash{}nreuniões de família.\textbackslash{}n\textbackslash{}nEmílio saiu prometendo voltar algumas\textbackslash{}nvezes.\textbackslash{}n\textbackslash{}nQuando eu me achei a sós com meu marido,\textbackslash{}nperguntei-lhe:\textbackslash{}n\textbackslash{}n— Donde conheces este homem?\textbackslash{}n\textbackslash{}n— É uma pérola, não é? Foi-me apresentado\textbackslash{}nno escritório há dias; simpatizei logo; parece ser dotado de boa alma, é vivo\textbackslash{}nde espírito e discreto como o bom senso. Não há ninguém que não goste dele{\ldots}\textbackslash{}n\textbackslash{}nE como eu o ouvisse séria e calada, meu\textbackslash{}nmarido interrompeu-se e perguntou-me:\textbackslash{}n\textbackslash{}n— Fiz mal em trazê-lo aqui?\textbackslash{}n\textbackslash{}n— Mal, por quê? perguntei eu.\textbackslash{}n\textbackslash{}n— Por coisa nenhuma. Que mal havia de\textbackslash{}nser? É um homem distinto{\ldots}\textbackslash{}n\textbackslash{}nPus termo ao novo louvor do rapaz,\textbackslash{}nchamando um escravo para dar algumas ordens.\textbackslash{}n\textbackslash{}nE retirei-me ao meu quarto.\textbackslash{}n\textbackslash{}nO sono dessa noite não foi o sono dos\textbackslash{}njustos, podes crer. O que me irritava era a preocupação constante em que eu\textbackslash{}nandava depois destes acontecimentos. Já eu não podia fugir inteiramente a essa\textbackslash{}npreocupação: era involuntária, subjugava-me, arrastava-me. Era a curiosidade do\textbackslash{}ncoração, esse primeiro sinal das tempestades em que sucumbe a nossa vida e o\textbackslash{}nnosso futuro.\textbackslash{}n\textbackslash{}nParece que aquele homem lia na minha alma\textbackslash{}ne sabia apresentar-se no momento mais próprio a ocupar-me a imaginação como uma\textbackslash{}nfigura poética e imponente. Tu, que o conheceste depois, dize-me se, dadas as\textbackslash{}ncircunstâncias anteriores, não era para produzir esta impressão no espírito de\textbackslash{}numa mulher como eu!\textbackslash{}n\textbackslash{}nComo eu, repito. Minhas circunstâncias\textbackslash{}neram especiais; se não o soubeste nunca, suspeitaste-o ao menos.\textbackslash{}n\textbackslash{}nSe meu marido tivesse em mim uma mulher,\textbackslash{}ne se eu tivesse nele um marido, minha salvação era certa. Mas não era assim.\textbackslash{}nEntramos no nosso lar nupcial como dois viajantes estranhos em uma hospedaria,\textbackslash{}ne aos quais a calamidade do tempo e a hora avançada da noite obrigam a aceitar\textbackslash{}npousada sob o teto do mesmo aposento.\textbackslash{}n\textbackslash{}nMeu casamento foi resultado de um cálculo\textbackslash{}ne de uma conveniência. Não inculpo meus pais. Eles cuidavam fazer-me feliz e\textbackslash{}nmorreram na convicção de que o era.\textbackslash{}n\textbackslash{}nEu podia, apesar de tudo, encontrar no\textbackslash{}nmarido que me davam um objeto de felicidade para todos os meus dias. Bastava\textbackslash{}npara isso que meu marido visse em mim uma alma companheira da sua alma, um\textbackslash{}ncoração sócio do seu coração. Não se dava isto; meu marido entendia o casamento\textbackslash{}nao modo da maior parte da gente; via nele a obediência às palavras do Senhor no\textbackslash{}nGênesis.\textbackslash{}n\textbackslash{}nFora disso, fazia-me cercar de certa\textbackslash{}nconsideração e dormia tranqüilo na convicção de que havia cumprido o dever.\textbackslash{}n\textbackslash{}nO dever! esta era a minha tábua de\textbackslash{}nsalvação. Eu sabia que as paixões não eram soberanas e que a nossa vontade pode\textbackslash{}ntriunfar delas. A este respeito eu tinha em mim forças bastantes para repelir\textbackslash{}nidéias más. Mas não era o presente que me abafava e atemorizava; era o futuro.\textbackslash{}nAté então aquele romance influía no meu espírito pela circunstância do mistério\textbackslash{}nem que vinha envolto; a realidade havia de abrir-me os olhos; consolava-me a\textbackslash{}nesperança de que eu triunfaria de um amor culpado. Mas, poderia nesse futuro,\textbackslash{}ncuja proximidade eu não calculava, resistir convenientemente à paixão e salvar\textbackslash{}nintactas a minha consideração e a minha consciência? Esta era a questão.\textbackslash{}n\textbackslash{}nOra, no meio destas oscilações, eu não\textbackslash{}nvia a mão do meu marido estender-se para salvar-me. Pelo contrário, quando na\textbackslash{}nocasião de queimar a carta, atirava-me a ele, lembras-te que ele me repeliu com\textbackslash{}numa palavra de enfado.\textbackslash{}n\textbackslash{}nIsto pensei, isto senti, na longa noite\textbackslash{}nque se seguiu à apresentação de Emílio.\textbackslash{}n\textbackslash{}nNo dia seguinte estava fatigada de\textbackslash{}nespírito; mas, ou fosse calma ou fosse prostração, senti que os pensamentos\textbackslash{}ndolorosos que me haviam torturado durante a noite esvaeceram-se à luz da manhã,\textbackslash{}ncomo verdadeiras aves da noite e da solidão.\textbackslash{}n\textbackslash{}nEntão abriu-se ao meu espírito um raio de\textbackslash{}nluz. Era a repetição do mesmo pensamento que me voltava no meio das\textbackslash{}npreocupações daqueles últimos dias.\textbackslash{}n\textbackslash{}nPor que temer? dizia eu comigo. Sou uma\textbackslash{}ntriste medrosa; e fatigo-me em criar montanhas para cair extenuada no meio da\textbackslash{}nplanície. Eia! nenhum obstáculo se opõe ao meu caminho de mulher virtuosa e\textbackslash{}nconsiderada. Este homem, se é o mesmo, não passa de um mau leitor de romances\textbackslash{}nrealistas. O mistério é que lhe dá algum valor; visto de mais perto há de ser\textbackslash{}nvulgar ou hediondo.\textbackslash{}n\textbackslash{}nCAPÍTULO IV\textbackslash{}n\textbackslash{}nNão te quero fatigar com a narração\textbackslash{}nminuciosa e diária de todos os acontecimentos.\textbackslash{}n\textbackslash{}nEmílio continuou a freqüentar a nossa\textbackslash{}ncasa, mostrando sempre a mesma delicadeza e gravidade, e encantando a todos por\textbackslash{}nsuas maneiras distintas sem afetação, amáveis sem fingimento.\textbackslash{}n\textbackslash{}nNão sei por que meu marido revelava-se\textbackslash{}ncada vez mais amigo de Emílio. Este conseguira despertar nele um entusiasmo\textbackslash{}nnovo para mim e para todos. Que capricho era esse da natureza?\textbackslash{}n\textbackslash{}nMuitas vezes interroguei meu marido\textbackslash{}nacerca desta amizade tão súbita e tão estrepitosa; quis até inventar suspeitas\textbackslash{}nno espírito dele; meu marido era inabalável.\textbackslash{}n\textbackslash{}n— Que queres? respondia-me ele. Não sei\textbackslash{}npor que simpatizo extraordinariamente com este rapaz. Sinto que é uma bela\textbackslash{}npessoa, e eu não posso dissimular o entusiasmo de que me possuo quando estou\textbackslash{}nperto dele.\textbackslash{}n\textbackslash{}n— Mas sem conhecê-lo{\ldots} objetava eu.\textbackslash{}n\textbackslash{}n— Ora essa! Tenho as melhores\textbackslash{}ninformações; e demais, vê-se logo que é uma pessoa distinta{\ldots}\textbackslash{}n\textbackslash{}n— As maneiras enganam muitas vezes.\textbackslash{}n\textbackslash{}n— Conhece-se{\ldots}\textbackslash{}n\textbackslash{}nConfesso, minha amiga, que eu podia impor\textbackslash{}na meu marido o afastamento de Emílio; mas quando esta idéia me vinha à cabeça,\textbackslash{}nnão sei por que ria-me dos meus temores e declarava-me com forças de resistir a\textbackslash{}ntudo o que pudesse sobrevir.\textbackslash{}n\textbackslash{}nDemais, o procedimento de Emílio\textbackslash{}nautorizava-me a desarmar. Ele era para mim de um respeito inalterável,\textbackslash{}ntratava-me como a todas as outras, sem deixar entrever a menor intenção oculta,\textbackslash{}no menor pensamento reservado.\textbackslash{}n\textbackslash{}nSucedeu o que era natural. Diante de tal\textbackslash{}nprocedimento não me ficava bem proceder com rigor e responder com a indiferença\textbackslash{}nà amabilidade.\textbackslash{}n\textbackslash{}nAs coisas marchavam de tal modo que eu\textbackslash{}ncheguei a persuadir-me de que tudo o que sucedera antes não tinha relação\textbackslash{}nalguma com aquele rapaz, e que não havia entre ambos mais do que um fenômeno da\textbackslash{}nsemelhança, o que aliás eu não podia afirmar, porque, como te disse já, não\textbackslash{}npudera reparar bem no homem do teatro.\textbackslash{}n\textbackslash{}nAconteceu que dentro de pouco tempo\textbackslash{}nestávamos na maior intimidade, e eu era para ele o mesmo que todas as outras:\textbackslash{}nadmiradora e admirada.\textbackslash{}n\textbackslash{}nDas reuniões passou Emílio às simples\textbackslash{}nvisitas de dia, nas horas em que meu marido estava presente, e mais tarde,\textbackslash{}nmesmo quando ele se achava ausente.\textbackslash{}n\textbackslash{}nMeu marido de ordinário era quem o\textbackslash{}ntrazia. Emílio vinha então no seu carrinho que ele próprio dirigia, com a maior\textbackslash{}ngraça e elegância. Demorava-se horas e horas em nossa casa, tocando piano ou\textbackslash{}nconversando.\textbackslash{}n\textbackslash{}nA primeira vez que o recebi só, confesso que\textbackslash{}nestremeci; mas foi um susto pueril; Emílio procedeu sempre do modo mais\textbackslash{}nindiferente em relação às minhas suspeitas. Nesse dia, se algumas me ficaram,\textbackslash{}ndesvaneceram-se todas.\textbackslash{}n\textbackslash{}nNisto passaram-se dois meses.\textbackslash{}n\textbackslash{}nUm dia, era de tarde, eu estava só;\textbackslash{}nesperava-te para irmos visitar teu pai enfermo. Parou um carro à porta. Mandei\textbackslash{}nver. Era Emílio.\textbackslash{}n\textbackslash{}nRecebi-o como de costume.\textbackslash{}n\textbackslash{}nDisse-lhe que íamos visitar um doente, e\textbackslash{}nele quis logo sair. Disse-lhe que ficasse até à tua chegada. Ficou como se\textbackslash{}noutro motivo o detivesse além de um dever de cortesia.\textbackslash{}n\textbackslash{}nPassou-se meia hora.\textbackslash{}n\textbackslash{}nNossa conversa foi sobre assuntos\textbackslash{}nindiferentes.\textbackslash{}n\textbackslash{}nEm um dos intervalos da conversa Emílio\textbackslash{}nlevantou-se e foi à janela. Eu levantei-me igualmente para ir ao piano buscar\textbackslash{}num leque. Voltando para o sofá reparei pelo espelho que Emílio me olhava com um\textbackslash{}nolhar estranho. Era uma transfiguração. Parecia que naquele olhar estava\textbackslash{}nconcentrada toda a alma dele.\textbackslash{}n\textbackslash{}nEstremeci.\textbackslash{}n\textbackslash{}nTodavia fiz um esforço sobre mim e fui\textbackslash{}nsentar-me, então mais séria que nunca.\textbackslash{}n\textbackslash{}nEmílio encaminhou-se para mim.\textbackslash{}n\textbackslash{}nOlhei para ele.\textbackslash{}n\textbackslash{}nEra o mesmo olhar.\textbackslash{}n\textbackslash{}nBaixei os meus olhos.\textbackslash{}n\textbackslash{}n— Assustou-se? perguntou-me ele.\textbackslash{}n\textbackslash{}nNão respondi nada. Mas comecei a tremer\textbackslash{}nde novo e parecia-me que o coração me queria pular fora do peito.\textbackslash{}n\textbackslash{}nÉ que naquelas palavras havia a mesma\textbackslash{}nexpressão do olhar; as palavras faziam-me o efeito das palavras da carta.\textbackslash{}n\textbackslash{}n— Assustou-se? repetiu ele.\textbackslash{}n\textbackslash{}n— De quê? perguntei eu procurando rir\textbackslash{}npara não dar maior gravidade à situação.\textbackslash{}n\textbackslash{}n— Pareceu-me.\textbackslash{}n\textbackslash{}nHouve um silêncio.\textbackslash{}n\textbackslash{}n— D. Eugênia, disse ele sentando-se; não\textbackslash{}nquero por mais tempo ocultar o segredo que faz o tormento da minha vida. Fora\textbackslash{}num sacrifício inútil. Feliz ou infeliz, prefiro a certeza da minha situação. D.\textbackslash{}nEugênia, eu amo-a.\textbackslash{}n\textbackslash{}nNão te posso descrever como fiquei,\textbackslash{}nouvindo estas palavras. Senti que empalidecia; minhas mãos estavam geladas.\textbackslash{}nQuis falar: não pude.\textbackslash{}n\textbackslash{}nEmílio continuou:\textbackslash{}n\textbackslash{}n— Oh! eu bem sei a que me exponho. Vejo\textbackslash{}ncomo este amor é culpado. Mas que quer? É fatalidade. Andei tantas léguas,\textbackslash{}npassei à ilharga de tantas belezas, sem que o meu coração pulsasse. Estava-me\textbackslash{}nreservada a ventura rara ou o tremendo infortúnio de ser amado ou desprezado\textbackslash{}npela senhora. Curvo-me ao destino. Qualquer que seja a resposta que eu possa\textbackslash{}nobter, não recuso, aceito. Que me responde?\textbackslash{}n\textbackslash{}nEnquanto ele falava, eu podia,\textbackslash{}nouvindo-lhe as palavras, reunir algumas idéias. Quando ele acabou levantei os\textbackslash{}nolhos e disse:\textbackslash{}n\textbackslash{}n— Que resposta espera de mim?\textbackslash{}n\textbackslash{}n— Qualquer.\textbackslash{}n\textbackslash{}n— Só pode esperar uma{\ldots}\textbackslash{}n\textbackslash{}n— Não me ama?\textbackslash{}n\textbackslash{}n— Não! Nem posso e nem amo, nem amaria se\textbackslash{}npudesse ou quisesse{\ldots} Peço que se retire.\textbackslash{}n\textbackslash{}nE levantei-me.\textbackslash{}n\textbackslash{}nEmílio levantou-se.\textbackslash{}n\textbackslash{}n— Retiro-me, disse ele; e parto com o\textbackslash{}ninferno no coração.\textbackslash{}n\textbackslash{}nLevantei os ombros em sinal de\textbackslash{}nindiferença.\textbackslash{}n\textbackslash{}n— Oh! eu bem sei que isso lhe é\textbackslash{}nindiferente. É isso o que eu mais sinto. Eu preferia o ódio; o ódio, sim; mas a\textbackslash{}nindiferença, acredite, é o pior castigo. Mas eu o recebo resignado. Tamanho\textbackslash{}ncrime deve ter tamanha pena.\textbackslash{}n\textbackslash{}nE tomando o chapéu chegou-se a mim de\textbackslash{}nnovo.\textbackslash{}n\textbackslash{}nEu recuei dois passos.\textbackslash{}n\textbackslash{}n— Oh! não tenha medo. Causo-lhe medo?\textbackslash{}n\textbackslash{}n— Medo? retorqui eu com altivez.\textbackslash{}n\textbackslash{}n— Asco? perguntou ele.\textbackslash{}n\textbackslash{}n— Talvez{\ldots} murmurei.\textbackslash{}n\textbackslash{}n— Uma única resposta, tornou Emílio;\textbackslash{}nconserva aquela carta?\textbackslash{}n\textbackslash{}n— Ah! disse eu. Era o autor da carta?\textbackslash{}n\textbackslash{}n— Era. E aquele misterioso do corredor do\textbackslash{}nTeatro Lírico. Era eu. A carta?\textbackslash{}n\textbackslash{}n— Queimei-a.\textbackslash{}n\textbackslash{}n— Preveniu o meu pensamento.\textbackslash{}n\textbackslash{}nE cumprimentando-me friamente dirigiu-se\textbackslash{}npara a porta. Quase a chegar à porta senti que ele vacilava e levava a mão ao\textbackslash{}npeito.\textbackslash{}n\textbackslash{}nTive um momento de piedade. Mas era\textbackslash{}nnecessário que ele se fosse, quer sofresse quer não. Todavia, dei um passo para\textbackslash{}nele e perguntei-lhe de longe:\textbackslash{}n\textbackslash{}n— Quer dar-me uma resposta?\textbackslash{}n\textbackslash{}nEle parou e voltou-se.\textbackslash{}n\textbackslash{}n— Pois não!\textbackslash{}n\textbackslash{}n— Como é que para praticar o que praticou\textbackslash{}nfingiu-se amigo de meu marido?\textbackslash{}n\textbackslash{}n— Foi um ato indigno, eu sei; mas o meu\textbackslash{}namor é daqueles que não recuam ante a indignidade. É o único que eu compreendo.\textbackslash{}nMas, perdão; não quero enfadá-la mais. Adeus! Para sempre!\textbackslash{}n\textbackslash{}nE saiu.\textbackslash{}n\textbackslash{}nPareceu-me ouvir um soluço.\textbackslash{}n\textbackslash{}nFui sentar-me ao sofá. Daí a pouco ouvi o\textbackslash{}nrodar do carro.\textbackslash{}n\textbackslash{}nO tempo que mediou entre a partida dele e\textbackslash{}na tua chegada não sei como se passou. No lugar em que fiquei aí me achaste.\textbackslash{}n\textbackslash{}nAté então eu não tinha visto o amor senão\textbackslash{}nnos livros. Aquele homem parecia-me realizar o amor que eu sonhara e vira\textbackslash{}ndescrito. A idéia de que o coração de Emílio sangrava naquele momento,\textbackslash{}ndespertou em mim um sentimento vivo de piedade. A piedade foi um primeiro\textbackslash{}npasso.\textbackslash{}n\textbackslash{}n'Quem sabe, dizia eu comigo mesma, o\textbackslash{}nque ele está agora sofrendo? E que culpa é a dele, afinal de contas? Ama-me,\textbackslash{}ndisse-mo; o amor foi mais forte do que a razão; não viu que eu era sagrada para\textbackslash{}nele; revelou-se. Ama, é a sua desculpa.'\textbackslash{}n\textbackslash{}nDepois repassava na memória todas as\textbackslash{}npalavras dele e procurava recordar-me do tom em que ele as proferira.\textbackslash{}nLembrava-me também do que eu dissera e o tom com que respondera às suas\textbackslash{}nconfissões.\textbackslash{}n\textbackslash{}nFui talvez severa demais. Podia manter a\textbackslash{}nminha dignidade sem abrir-lhe uma chaga no coração. Se eu falasse com mais\textbackslash{}nbrandura podia adquirir dele o respeito e a veneração. Agora há de amar-me\textbackslash{}nainda, mas não se recordará do que se passou sem um sentimento de amargura.\textbackslash{}n\textbackslash{}nEstava nestas reflexões quando entraste.\textbackslash{}n\textbackslash{}nLembras-te que me achaste triste e\textbackslash{}nperguntaste a causa disso. Nada te respondi. Fomos à casa da tua tia, sem que\textbackslash{}neu nada mudasse do ar que tinha antes.\textbackslash{}n\textbackslash{}nÀ noite quando meu marido me perguntou\textbackslash{}npor Emílio, respondi sem saber o que respondia:\textbackslash{}n\textbackslash{}n— Não veio cá hoje.\textbackslash{}n\textbackslash{}n— Deveras? disse ele. Então está doente.\textbackslash{}n\textbackslash{}n— Não sei.\textbackslash{}n\textbackslash{}n— Lá vou amanhã.\textbackslash{}n\textbackslash{}n— Lá onde?\textbackslash{}n\textbackslash{}n— À casa dele.\textbackslash{}n\textbackslash{}n— Para quê?\textbackslash{}n\textbackslash{}n— Talvez esteja doente.\textbackslash{}n\textbackslash{}n— Não creio; esperemos até ver{\ldots}\textbackslash{}n\textbackslash{}nPassei uma noite angustiosa. A idéia de\textbackslash{}nEmílio perturbava-me o sono. Afigurava-se-me que ele estaria àquela hora\textbackslash{}nchorando lágrimas de sangue no desespero do amor não aceito.\textbackslash{}n\textbackslash{}nEra piedade? Era amor?\textbackslash{}n\textbackslash{}nCarlota, era uma e outra coisa. Que podia\textbackslash{}nser mais? Eu tinha posto o pé em uma senda fatal; uma força me atraía. Eu\textbackslash{}nfraca, podendo ser forte. Não me inculpo senão a mim.\textbackslash{}n\textbackslash{}nAté domingo.\textbackslash{}n\textbackslash{}nCAPÍTULO V\textbackslash{}n\textbackslash{}nNa tarde seguinte, quando meu marido\textbackslash{}nvoltou perguntei por Emílio.\textbackslash{}n\textbackslash{}n— Não o procurei, respondeu-me ele; tomei\textbackslash{}no conselho; se não vier hoje, sim.\textbackslash{}n\textbackslash{}nPassou-se, pois, um dia sem ter notícias\textbackslash{}ndele.\textbackslash{}n\textbackslash{}nNo dia seguinte, não tendo aparecido, meu\textbackslash{}nmarido foi lá.\textbackslash{}n\textbackslash{}nSerei franca contigo, eu mesma lembrei\textbackslash{}nisso a meu marido.\textbackslash{}n\textbackslash{}nEsperei ansiosa a resposta.\textbackslash{}n\textbackslash{}nMeu marido voltou pela tarde. Tinha um\textbackslash{}ncerto ar triste. Perguntei o que havia.\textbackslash{}n\textbackslash{}n— Não sei. Fui encontrar o rapaz de cama.\textbackslash{}nDisse-me que era uma ligeira constipação; mas eu creio que não é isso só{\ldots}\textbackslash{}n\textbackslash{}n— Que será então? perguntei eu, fitando\textbackslash{}num olhar em meu marido.\textbackslash{}n\textbackslash{}n— Alguma coisa mais. O rapaz falou-me em\textbackslash{}nembarcar para o Norte. Está triste, distraído, preocupado. Ao mesmo tempo que\textbackslash{}nmanifesta a esperança de ver os pais, revela receios de não tornar a vê-los.\textbackslash{}nTem idéias de morrer na viagem. Não sei que lhe aconteceu, mas foi alguma\textbackslash{}ncoisa. Talvez{\ldots}\textbackslash{}n\textbackslash{}n— Talvez?\textbackslash{}n\textbackslash{}n— Talvez alguma perda de dinheiro.\textbackslash{}n\textbackslash{}nEsta resposta transtornou o meu espírito.\textbackslash{}nPosso afirmar-te que esta resposta entrou por muito nos acontecimentos\textbackslash{}nposteriores.\textbackslash{}n\textbackslash{}nDepois de algum silêncio perguntei:\textbackslash{}n\textbackslash{}n— Mas que pretendes fazer?\textbackslash{}n\textbackslash{}n— Abrir-me com ele. Perguntar o que é, e\textbackslash{}nacudir-lhe se for possível. Em qualquer caso não o deixarei partir. Que achas?\textbackslash{}n\textbackslash{}n— Acho que sim.\textbackslash{}n\textbackslash{}nTudo o que ia acontecendo contribuía\textbackslash{}npoderosamente para tornar a idéia de Emílio cada vez mais presente à minha\textbackslash{}nmemória, e, é com dor que o confesso, não pensava já nele sem pulsações do\textbackslash{}ncoração.\textbackslash{}n\textbackslash{}nNa noite do dia seguinte estávamos\textbackslash{}nreunidas algumas pessoas. Eu não dava grande vida à reunião. Estava triste e\textbackslash{}ndesconsolada. Estava com raiva de mim própria. Fazia-me algoz de Emílio e\textbackslash{}ndoía-me a idéia de que ele padecesse ainda mais por mim.\textbackslash{}n\textbackslash{}nMas, seriam nove horas, quando meu marido\textbackslash{}napareceu trazendo Emílio pelo braço.\textbackslash{}n\textbackslash{}nHouve um movimento geral de surpresa.\textbackslash{}n\textbackslash{}nRealmente porque Emílio não aparecia\textbackslash{}nalguns dias já todos começavam a perguntar por ele; depois, porque o pobre moço\textbackslash{}nvinha pálido de cera.\textbackslash{}n\textbackslash{}nNão te direi o que se passou nessa noite.\textbackslash{}nEmílio parecia sofrer, não estava alegre como dantes; ao contrário, era naquela\textbackslash{}nnoite de uma taciturnidade, de uma tristeza que incomodava a todos, mas que me\textbackslash{}nmortificava atrozmente, a mim que me fazia causa das suas dores.\textbackslash{}n\textbackslash{}nPude falar-lhe em uma ocasião, a alguma\textbackslash{}ndistância das outras pessoas.\textbackslash{}n\textbackslash{}n— Desculpe-me, disse-lhe eu, se alguma\textbackslash{}npalavra dura lhe disse. Compreende a minha posição. Ouvindo bruscamente o que\textbackslash{}nme disse não pude pensar no que dizia. Sei que sofreu; peço-lhe que não sofra\textbackslash{}nmais, que esqueça{\ldots}\textbackslash{}n\textbackslash{}n— Obrigado, murmurou ele.\textbackslash{}n\textbackslash{}n— Meu marido falou-me de projetos seus{\ldots}\textbackslash{}n\textbackslash{}n— De voltar à minha província, é verdade.\textbackslash{}n\textbackslash{}n— Mas doente{\ldots}\textbackslash{}n\textbackslash{}n— Esta doença há de passar.\textbackslash{}n\textbackslash{}nE dizendo isto lançou-me um olhar tão sinistro\textbackslash{}nque eu tive medo.\textbackslash{}n\textbackslash{}n— Passar? passar como?\textbackslash{}n\textbackslash{}n— De algum modo.\textbackslash{}n\textbackslash{}n— Não diga isso{\ldots}\textbackslash{}n\textbackslash{}n— Que me resta mais na terra?\textbackslash{}n\textbackslash{}nE voltou os olhos para enxugar uma\textbackslash{}nlágrima.\textbackslash{}n\textbackslash{}n— Que é isso? disse eu. Está chorando?\textbackslash{}n\textbackslash{}n— As últimas lágrimas.\textbackslash{}n\textbackslash{}n— Oh! se soubesse como me faz sofrer! Não\textbackslash{}nchore; eu lho peço. Peço-lhe mais. Peço-lhe que viva.\textbackslash{}n\textbackslash{}n— Oh!\textbackslash{}n\textbackslash{}n— Ordeno-lhe.\textbackslash{}n\textbackslash{}n— Ordena-me? E se eu não obedecer? Se eu\textbackslash{}nnão puder?{\ldots} Acredita que se possa viver com um espinho no coração?\textbackslash{}n\textbackslash{}nIsto que te escrevo é feio. A maneira por\textbackslash{}nque ele falava é que era apaixonada, dolorosa, comovente. Eu ouvia sem saber de\textbackslash{}nmim. Aproximavam-se algumas pessoas. Quis pôr termo à conversa e disse-lhe:\textbackslash{}n\textbackslash{}n— Ama-me? disse eu. Só o amor pode\textbackslash{}nordenar? Pois é o amor que lhe ordena que viva!\textbackslash{}n\textbackslash{}nEmílio fez um gesto de alegria.\textbackslash{}nLevantei-me para ir falar às pessoas que se aproximavam.\textbackslash{}n\textbackslash{}n— Obrigado, murmurou-me ele aos ouvidos.\textbackslash{}n\textbackslash{}nQuando, no fim do serão, Emílio se despediu\textbackslash{}nde mim, dizendo-me, com um olhar em que a gratidão e o amor irradiavam juntos:\textbackslash{}n— Até amanhã! - não sei que sentimento de confusão e de amor, de remorso e de\textbackslash{}nternura se apoderou de mim.\textbackslash{}n\textbackslash{}n— Bem; Emílio está mais alegre, dizia-me\textbackslash{}nmeu marido.\textbackslash{}n\textbackslash{}nEu olhei para ele sem saber o que\textbackslash{}nresponder.\textbackslash{}n\textbackslash{}nDepois retirei-me precipitadamente.\textbackslash{}nParecia-me que via nele a imagem da minha consciência.\textbackslash{}n\textbackslash{}nNo dia seguinte recebi de Emílio esta\textbackslash{}ncarta:\textbackslash{}n\textbackslash{}nEugênia. Obrigado. Torno-me à vida, e à\textbackslash{}nsenhora o devo. Obrigado! fez de um cadáver um homem, faça agora de um homem um\textbackslash{}ndeus. Ânimo! ânimo!\textbackslash{}n\textbackslash{}nLi esta carta, reli, e{\ldots} dir-to-ei,\textbackslash{}nCarlota? beijei-a. Beijei-a repetidas vezes com alma, com paixão, com delírio.\textbackslash{}nEu amava! eu amava!\textbackslash{}n\textbackslash{}nEntão houve em mim a mesma luta, mas estava\textbackslash{}nmudada a situação dos meus sentimentos. Antes era o coração que fugia à razão,\textbackslash{}nagora a razão fugia ao coração.\textbackslash{}n\textbackslash{}nEra um crime, eu bem o via, bem o sentia;\textbackslash{}nmas não sei qual era a minha fatalidade, qual era a minha natureza; eu achava\textbackslash{}nnas delícias do crime desculpa ao meu erro, e procurava com isso legitimar a\textbackslash{}nminha paixão.\textbackslash{}n\textbackslash{}nQuando meu marido se achava perto de mim\textbackslash{}neu me sentia melhor e mais corajosa{\ldots}\textbackslash{}n\textbackslash{}nParo aqui desta vez. Sinto uma opressão\textbackslash{}nno peito. É a recordação de todos estes acontecimentos.\textbackslash{}n\textbackslash{}nAté domingo.\textbackslash{}n\textbackslash{}nCAPÍTULO VI\textbackslash{}n\textbackslash{}nSeguiram-se alguns dias às cenas que eu\textbackslash{}nte contei na minha carta passada.\textbackslash{}n\textbackslash{}nAtivou-se entre mim e Emílio uma\textbackslash{}ncorrespondência. No fim de quinze dias eu só vivia do pensamento dele.\textbackslash{}n\textbackslash{}nNinguém dos que freqüentavam a nossa casa,\textbackslash{}nnem mesmo tu, pôde descobrir este amor. Éramos dois namorados discretos ao\textbackslash{}núltimo ponto.\textbackslash{}n\textbackslash{}nÉ certo que muitas vezes me perguntavam\textbackslash{}npor que é que eu me distraía tanto e andava tão melancólica; isto chamava-me à\textbackslash{}nvida real e eu mudava logo de parecer.\textbackslash{}n\textbackslash{}nMeu marido sobretudo parecia sofrer com\textbackslash{}nas minhas tristezas.\textbackslash{}n\textbackslash{}nA sua solicitude, confesso,\textbackslash{}nincomodava-me. Muitas vezes lhe respondia mal, não já porque eu o odiasse, mas\textbackslash{}nporque de todos era ele o único a quem eu não quisera ouvir destas\textbackslash{}ninterrogações.\textbackslash{}n\textbackslash{}nUm dia voltando para casa à tarde\textbackslash{}nchegou-se ele a mim e disse:\textbackslash{}n\textbackslash{}n— Eugênia, tenho uma notícia a dar-te.\textbackslash{}n\textbackslash{}n— Qual?\textbackslash{}n\textbackslash{}n— E que te há de agradar muito.\textbackslash{}n\textbackslash{}n— Vejamos qual é.\textbackslash{}n\textbackslash{}n— É um passeio.\textbackslash{}n\textbackslash{}n— Aonde?\textbackslash{}n\textbackslash{}n— A idéia foi minha. Já fui ao Emílio e\textbackslash{}nele aplaudiu muito. O passeio deve ser domingo à Gávea; iremos daqui muito\textbackslash{}ncedinho. Tudo isto, é preciso notar, não está decidido. Depende de ti. O que\textbackslash{}ndizes?\textbackslash{}n\textbackslash{}n— Aprovo a idéia.\textbackslash{}n\textbackslash{}n— Muito bem. A Carlota pode ir.\textbackslash{}n\textbackslash{}n— E deve ir, acrescentei eu; e algumas\textbackslash{}noutras amigas.\textbackslash{}n\textbackslash{}nPouco depois recebias tu e outras um\textbackslash{}nbilhete de convite para o passeio.\textbackslash{}n\textbackslash{}nLembras-te que lá fomos. O que não sabes\textbackslash{}né que nesse passeio, a favor da confusão e a distração geral, houve entre mim e\textbackslash{}nEmílio um diálogo que foi para mim a primeira amargura de amor.\textbackslash{}n\textbackslash{}n— Eugênia, dizia ele dando-me o braço,\textbackslash{}nestás certa de que me amas?\textbackslash{}n\textbackslash{}n— Estou.\textbackslash{}n\textbackslash{}n— Pois bem. O que te peço, nem sou eu que\textbackslash{}nte peço, é o meu coração, o teu coração que te pedem, um movimento nobre e\textbackslash{}ncapaz de nos engrandecer aos nossos próprios olhos. Não haverá um recanto no\textbackslash{}nmundo em que possamos viver, longe de todos e perto do céu?\textbackslash{}n\textbackslash{}n— Fugir?\textbackslash{}n\textbackslash{}n— Sim!\textbackslash{}n\textbackslash{}n— Oh! isso nunca!\textbackslash{}n\textbackslash{}n— Não me amas.\textbackslash{}n\textbackslash{}n— Amo, sim; é já um crime, não quero ir\textbackslash{}nalém.\textbackslash{}n\textbackslash{}n— Recusas a felicidade?\textbackslash{}n\textbackslash{}n— Recuso a desonra.\textbackslash{}n\textbackslash{}n— Não me amas.\textbackslash{}n\textbackslash{}n— Oh! meu Deus, como respondê-lo? Amo,\textbackslash{}nsim; mas desejo ficar a seus olhos a mesma mulher, amorosa é verdade, mas até\textbackslash{}ncerto ponto{\ldots} pura.\textbackslash{}n\textbackslash{}n— O amor que calcula, não é amor.\textbackslash{}n\textbackslash{}nNão respondi. Emílio disse estas palavras\textbackslash{}ncom uma expressão tal de desdém e com uma intenção de ferir-me que eu senti o\textbackslash{}ncoração bater-me apressado, e subir-me o sangue ao rosto.\textbackslash{}n\textbackslash{}nO passeio acabou mal.\textbackslash{}n\textbackslash{}nEsta cena tornou Emílio frio para mim; eu\textbackslash{}nsofria com isso; procurei torná-lo ao estado anterior; mas não consegui.\textbackslash{}n\textbackslash{}nUm dia em que nos achávamos a sós,\textbackslash{}ndisse-lhe:\textbackslash{}n\textbackslash{}n— Emílio, se eu amanhã te acompanhasse, o\textbackslash{}nque farias?\textbackslash{}n\textbackslash{}n— Cumpria essa ordem divina.\textbackslash{}n\textbackslash{}n— Mas depois?\textbackslash{}n\textbackslash{}n— Depois? perguntou Emílio com ar de quem\textbackslash{}nestranhava a pergunta.\textbackslash{}n\textbackslash{}n— Sim, depois? continuei eu. Depois\textbackslash{}nquando o tempo volvesse não me havias de olhar com desprezo?\textbackslash{}n\textbackslash{}n— Desprezo? Não vejo{\ldots}\textbackslash{}n\textbackslash{}n— Como não? Que te mereceria eu depois?\textbackslash{}n\textbackslash{}n— Oh! esse sacrifício seria feito por\textbackslash{}nminha causa, eu fora covarde se te lançasse isso em rosto.\textbackslash{}n\textbackslash{}n— Di-lo-ias no teu íntimo.\textbackslash{}n\textbackslash{}n— Juro que não.\textbackslash{}n\textbackslash{}n— Pois a meus olhos é assim; eu nunca me\textbackslash{}nperdoaria esse erro.\textbackslash{}n\textbackslash{}nEmílio pôs o rosto nas mãos e pareceu\textbackslash{}nchorar. Eu que até ali falava com esforço, fui a ele e tirei-lhe o rosto das\textbackslash{}nmãos.\textbackslash{}n\textbackslash{}n— Que é isto? disse eu. Não vês que me\textbackslash{}nfazes chorar também?\textbackslash{}n\textbackslash{}nEle olhou para mim com os olhos rasos de\textbackslash{}nlágrimas. Eu tinha os meus úmidos.\textbackslash{}n\textbackslash{}n— Adeus, disse ele repentinamente. Vou\textbackslash{}npartir.\textbackslash{}n\textbackslash{}nE deu um passo para a porta.\textbackslash{}n\textbackslash{}n— Se me prometes viver, disse-lhe, parte;\textbackslash{}nse tens alguma idéia sinistra, fica.\textbackslash{}n\textbackslash{}nNão sei o que viu ele no meu olhar, mas\textbackslash{}ntomando a mão que eu lhe estendia beijou-a repetidas vezes (eram os primeiros\textbackslash{}nbeijos) e disse-me com fogo:\textbackslash{}n\textbackslash{}n— Fico, Eugênia!\textbackslash{}n\textbackslash{}nOuvimos um ruído fora. Mandei ver. Era\textbackslash{}nmeu marido que chegava enfermo. Tinha tido um ataque no escritório. Tornara a\textbackslash{}nsi, mas achava-se mal. Alguns amigos o trouxeram dentro de um carro.\textbackslash{}n\textbackslash{}nCorri para a porta. Meu marido vinha\textbackslash{}npálido e desfeito. Mal podia andar ajudado pelos amigos.\textbackslash{}n\textbackslash{}nFiquei desesperada, não cuidei de mais\textbackslash{}ncoisa alguma. O médico que acompanhara meu marido mandou logo fazer algumas\textbackslash{}naplicações de remédios. Eu estava impaciente; perguntava a todos se meu marido\textbackslash{}nestava salvo.\textbackslash{}n\textbackslash{}nTodos me tranqüilizavam.\textbackslash{}n\textbackslash{}nEmílio mostrou-se pesaroso com o\textbackslash{}nacontecimento. Foi a meu marido e apertou-lhe a mão.\textbackslash{}n\textbackslash{}nQuando Emílio quis sair, meu marido\textbackslash{}ndisse-lhe:\textbackslash{}n\textbackslash{}n— Olhe, sei que não pode estar aqui\textbackslash{}nsempre; peço-lhe, porém, que venha, se puder, todos os dias.\textbackslash{}n\textbackslash{}n— Pois não, disse Emílio.\textbackslash{}n\textbackslash{}nE saiu.\textbackslash{}n\textbackslash{}nMeu marido passou mal o resto daquele dia\textbackslash{}ne a noite. Eu não dormi. Passei a noite no quarto.\textbackslash{}n\textbackslash{}nNo dia seguinte estava exausta. Tantas\textbackslash{}ncomoções diversas e uma vigília tão longa deixaram-me prostrada: cedia à força\textbackslash{}nmaior. Mandei chamar a prima Elvira e fui deitar-me.\textbackslash{}n\textbackslash{}nFecho esta carta neste ponto. Pouco falta\textbackslash{}npara chegar ao termo da minha triste narração.\textbackslash{}n\textbackslash{}nAté domingo.\textbackslash{}n\textbackslash{}nCAPÍTULO VII\textbackslash{}n\textbackslash{}nA moléstia de meu marido durou poucos\textbackslash{}ndias. De dia para dia agravava-se. No fim de oito dias os médicos desenganaram\textbackslash{}no doente.\textbackslash{}n\textbackslash{}nQuando recebi esta fatal nova fiquei como\textbackslash{}nlouca. Era meu marido, Carlota, e apesar de tudo eu não podia esquecer que ele\textbackslash{}ntinha sido o companheiro da minha vida e a idéia salvadora nos desvios do meu\textbackslash{}nespírito.\textbackslash{}n\textbackslash{}nEmílio achou-me num estado de desespero.\textbackslash{}nProcurou consolar-me. Eu não lhe ocultei que esta morte era um golpe profundo\textbackslash{}npara mim.\textbackslash{}n\textbackslash{}nUma noite estávamos juntos todos, eu, a\textbackslash{}nprima Elvira, uma parenta de meu marido e Emílio. Fazíamos companhia ao doente.\textbackslash{}nEste, depois de um longo silêncio, voltou-se para mim e disse-me:\textbackslash{}n\textbackslash{}n— A tua mão.\textbackslash{}n\textbackslash{}nE apertando-me a mão com uma energia\textbackslash{}nsuprema, voltou-se para a parede.\textbackslash{}n\textbackslash{}nExpirou.\textbackslash{}n\textbackslash{}n{\ldots}\textbackslash{}n\textbackslash{}nPassaram-se quatro meses depois dos fatos\textbackslash{}nque te contei. Emílio acompanhou-me na dor e foi dos mais assíduos em todas as\textbackslash{}ncerimônias fúnebres que se fizeram ao meu finado marido.\textbackslash{}n\textbackslash{}nTodavia, as visitas começaram a\textbackslash{}nescassear. Era, parecia-me, por motivo de uma delicadeza natural.\textbackslash{}n\textbackslash{}nNo fim do prazo de que te falei, soube,\textbackslash{}npor boca de um dos amigos de meu marido, que Emílio ia partir. Não pude crer.\textbackslash{}nEscrevi-lhe uma carta.\textbackslash{}n\textbackslash{}nEu amava-o então, como dantes, mais\textbackslash{}nainda, agora que estava livre.\textbackslash{}n\textbackslash{}nDizia a carta:\textbackslash{}n\textbackslash{}nEmílio.\textbackslash{}n\textbackslash{}nConstou-me que ias partir. Será possível?\textbackslash{}nEu mesma não posso acreditar nos meus ouvidos! Bem sabes se eu te amo. Não é\textbackslash{}ntempo de coroar os nossos votos; mas não faltará muito para que o mundo nos\textbackslash{}nrevele uma união que o amor nos impõe. Vem tu mesmo responder-me por boca.\textbackslash{}n\textbackslash{}nTua Eugênia.\textbackslash{}n\textbackslash{}nEmílio veio em pessoa. Asseverou-me que,\textbackslash{}nse ia partir, era por negócio de pouco tempo, mas que voltaria logo. A viagem\textbackslash{}ndevia ter lugar daí a oito dias.\textbackslash{}n\textbackslash{}nPedi-lhe que jurasse o que dizia, e ele\textbackslash{}njurou.\textbackslash{}n\textbackslash{}nDeixei-o partir.\textbackslash{}n\textbackslash{}nDaí a quatro dias recebia eu a seguinte\textbackslash{}ncarta dele:\textbackslash{}n\textbackslash{}nMenti, Eugênia; vou partir já. Menti\textbackslash{}nainda, eu não volto. Não volto porque não posso. Uma união contigo seria para\textbackslash{}nmim o ideal da felicidade se eu não fosse homem de hábitos opostos ao\textbackslash{}ncasamento. Adeus. Desculpa-me, e reza para que eu faça uma boa viagem. Adeus.\textbackslash{}n\textbackslash{}nEmílio.\textbackslash{}n\textbackslash{}nAvalias facilmente como fiquei depois de\textbackslash{}nler esta carta. Era um castelo que se desmoronava. Em troca do meu amor, do meu\textbackslash{}nprimeiro amor, recebia deste modo a ingratidão e o desprezo. Era justo: aquele\textbackslash{}namor culpado não podia ter bom fim; eu fui castigada pelas conseqüências mesmo\textbackslash{}ndo meu crime.\textbackslash{}n\textbackslash{}nMas, perguntava eu, como é que este\textbackslash{}nhomem, que parecia amar-me tanto, recusou aquela de cuja honestidade podia estar\textbackslash{}ncerto, visto que pôde opor uma resistência aos desejos de seu coração? Isto me\textbackslash{}npareceu um mistério. Hoje vejo que não era; Emílio era um sedutor vulgar e só\textbackslash{}nse diferençava dos outros em ter um pouco mais de habilidade que eles.\textbackslash{}n\textbackslash{}nTal é a minha história. Imagina o que\textbackslash{}nsofri nestes dois anos. Mas o tempo é um grande médico: estou curada.\textbackslash{}n\textbackslash{}nO amor ofendido e o remorso de haver de\textbackslash{}nalgum modo traído a confiança de meu esposo fizeram-me doer muito. Mas eu creio\textbackslash{}nque caro paguei o meu crime e acho-me reabilitada perante a minha consciência.\textbackslash{}n\textbackslash{}nAchar-me-ei perante Deus?\textbackslash{}n\textbackslash{}nE tu? É o que me hás de explicar amanhã;\textbackslash{}nvinte e quatro horas depois de partir esta carta eu serei contigo.\textbackslash{}n\textbackslash{}nAdeus!\textbackslash{}n\textbackslash{}nLINHA RETA E LINHA CURVA\textbackslash{}n\textbackslash{}nÍNDICE\textbackslash{}n\textbackslash{}nCAPÍTULO\textbackslash{}nPRIMEIRO\textbackslash{}n\textbackslash{}nCAPÍTULO II\textbackslash{}n\textbackslash{}nCAPÍTULO III\textbackslash{}n\textbackslash{}nCAPÍTULO IV\textbackslash{}n\textbackslash{}nCAPÍTULO PRIMEIRO\textbackslash{}n\textbackslash{}nEra em Petrópolis, no ano de 186{\ldots} Já se\textbackslash{}nvê que a minha história não data de longe. É tomada dos anais contemporâneos e\textbackslash{}ndos costumes atuais. Talvez algum dos leitores conheça até as personagens que\textbackslash{}nvão figurar neste pequeno quadro. Não será raro que, encontrando uma delas\textbackslash{}namanhã, Azevedo, por exemplo, um dos meus leitores exclame:\textbackslash{}n\textbackslash{}n— Ah! cá vi uma história em que se falou\textbackslash{}nde ti. Não te tratou mal o autor. Mas a semelhança era tamanha, houve tão pouco\textbackslash{}ncuidado em disfarçar a fisionomia, que eu, à proporção que voltava a página,\textbackslash{}ndizia comigo: É o Azevedo, não há dúvida.\textbackslash{}n\textbackslash{}nFeliz Azevedo! A hora em que começa essa\textbackslash{}nnarrativa é ele um marido feliz, inteiramente feliz. Casado de fresco,\textbackslash{}npossuindo por mulher a mais formosa dama da sociedade, e a melhor alma que ainda\textbackslash{}nse encarnou ao sol da América, dono de algumas propriedades bem situadas e\textbackslash{}nperfeitamente rendosas, acatado, querido, descansado, tal é o nosso Azevedo, a\textbackslash{}nquem por cúmulo de ventura coroam os mais belos vinte e seis anos.\textbackslash{}n\textbackslash{}nDeu-lhe a fortuna um emprego suave: não\textbackslash{}nfazer nada. Possui um diploma de bacharel em direito; mas esse diploma nunca\textbackslash{}nlhe serviu; existe guardado no fundo da lata clássica em que o trouxe da\textbackslash{}nFaculdade de São Paulo. De quando em quando Azevedo faz uma visita ao diploma,\textbackslash{}naliás ganho legitimamente, mas é para não o ver mais senão daí a longo tempo.\textbackslash{}nNão é um diploma, é uma relíquia.\textbackslash{}n\textbackslash{}nQuando Azevedo saiu da faculdade de São\textbackslash{}nPaulo e voltou para a fazenda da província de Minas Gerais, tinha um projeto:\textbackslash{}nir à Europa. No fim de alguns meses o pai consentiu na viagem, e Azevedo\textbackslash{}npreparou-se para realizá-la. Chegou à corte no propósito firme de tomar lugar\textbackslash{}nno primeiro paquete que saísse; mas nem tudo depende da vontade do homem.\textbackslash{}nAzevedo foi a um baile antes de partir; aí estava armada uma rede em que ele\textbackslash{}ndevia ser colhido. Que rede! Vinte anos, uma figura delicada, esbelta,\textbackslash{}nfranzina, uma dessas figuras vaporosas que parecem desfazer-se ao primeiro raio\textbackslash{}ndo sol. Azevedo não foi senhor de si: apaixonou-se; daí a um mês casou-se, e\textbackslash{}ndaí a oito dias partiu para Petrópolis.\textbackslash{}n\textbackslash{}nQue casa encerraria aquele casal tão\textbackslash{}nbelo, tão amante e tão feliz? Não podia ser mais própria a casa escolhida; era\textbackslash{}num edifício leve, delgado, elegante, mais de recreio que de morada; um\textbackslash{}nverdadeiro ninho para aquelas duas pombas fugitivas.\textbackslash{}n\textbackslash{}nA nossa história começa exatamente três\textbackslash{}nmeses depois da ida para Petrópolis. Azevedo e a mulher amavam-se ainda como no\textbackslash{}nprimeiro dia. O amor tomava então uma força maior e nova; é que{\ldots} devo\textbackslash{}ndizê-lo, ó casais de três meses? é que apontava no horizonte o primeiro filho.\textbackslash{}nTambém a terra e o céu se alegram quando aponta no horizonte o primeiro raio do\textbackslash{}nsol. A figura não vem aqui por simples ornato de estilo; é uma dedução lógica:\textbackslash{}na mulher de Azevedo chamava-se Adelaide.\textbackslash{}n\textbackslash{}nEra, pois, em Petrópolis, numa tarde de\textbackslash{}ndezembro de 186{\ldots} Azevedo e Adelaide estavam no jardim que ficava em frente da\textbackslash{}ncasa onde ocultavam a sua felicidade. Azevedo lia alto; Adelaide ouvia-o ler,\textbackslash{}nmas como se ouve um eco do coração, tanto a voz do marido e as palavras da obra\textbackslash{}ncorrespondiam ao sentimento interior da moça.\textbackslash{}n\textbackslash{}nNo fim de algum tempo Azevedo deteve-se e\textbackslash{}nperguntou:\textbackslash{}n\textbackslash{}n— Queres que paremos aqui?\textbackslash{}n\textbackslash{}n— Como quiseres, disse Adelaide.\textbackslash{}n\textbackslash{}n— É melhor, disse Azevedo fechando o\textbackslash{}nlivro. As coisas boas não se gozam de uma assentada. Guardemos um pouco para a\textbackslash{}nnoite. Demais, era já tempo que eu passasse do idílio escrito para o idílio\textbackslash{}nvivo. Deixa-me olhar para ti.\textbackslash{}n\textbackslash{}nAdelaide olhou para ele e disse:\textbackslash{}n\textbackslash{}n— Parece que começamos a lua-de-mel.\textbackslash{}n\textbackslash{}n— Parece e é, acrescentou Azevedo; e se o\textbackslash{}ncasamento não fosse eternamente isto, o que poderia ser? A ligação de duas\textbackslash{}nexistências para meditar discretamente na melhor maneira de comer o maxixe e o\textbackslash{}nrepolho? Ora, pelo amor de Deus! Eu penso que o casamento deve ser um namoro\textbackslash{}neterno. Não pensas como eu?\textbackslash{}n\textbackslash{}n— Sinto, disse Adelaide.\textbackslash{}n\textbackslash{}n— Sentes, é quanto basta.\textbackslash{}n\textbackslash{}n— Mas que as mulheres sintam é natural;\textbackslash{}nos homens{\ldots}\textbackslash{}n\textbackslash{}n— Os homens, são homens.\textbackslash{}n\textbackslash{}n— O que nas mulheres é sentimento, nos\textbackslash{}nhomens é pieguice; desde pequena me dizem isto.\textbackslash{}n\textbackslash{}n— Enganam-te desde pequena, disse Azevedo\textbackslash{}nrindo.\textbackslash{}n\textbackslash{}n— Antes isso!\textbackslash{}n\textbackslash{}n— É a verdade. E desconfia sempre dos que\textbackslash{}nmais falam, sejam homens ou mulheres. Tens perto um exemplo. A Emília fala\textbackslash{}nmuito da sua isenção. Quantas vezes se casou? Até aqui duas, e está nos vinte e\textbackslash{}ncinco anos. Era melhor calar-se mais e casar-se menos.\textbackslash{}n\textbackslash{}n— Mas nela é brincadeira, disse Adelaide.\textbackslash{}n\textbackslash{}n— Pois não. O que não é brincadeira é que\textbackslash{}nos três meses do nosso casamento parecem-me três minutos{\ldots}\textbackslash{}n\textbackslash{}n— Três meses! exclamou Adelaide.\textbackslash{}n\textbackslash{}n— Como foge o tempo! disse Azevedo.\textbackslash{}n\textbackslash{}n— Dirás sempre o mesmo? perguntou\textbackslash{}nAdelaide com um gesto de incredulidade.\textbackslash{}n\textbackslash{}nAzevedo abraçou-a e perguntou:\textbackslash{}n\textbackslash{}n— Duvidas?\textbackslash{}n\textbackslash{}n— Receio. É tão bom ser feliz!\textbackslash{}n\textbackslash{}n— Sê-lo-ás sempre e do mesmo modo. De\textbackslash{}noutro não entendo eu.\textbackslash{}n\textbackslash{}nNeste momento ouviram os dois uma voz que\textbackslash{}npartia da porta do jardim.\textbackslash{}n\textbackslash{}n— O que é que não entendes? dizia essa\textbackslash{}nvoz.\textbackslash{}n\textbackslash{}nOlharam.\textbackslash{}n\textbackslash{}nÀ porta do jardim estava um homem alto,\textbackslash{}nbem parecido, trajando com elegância, luvas cor de palha, chicotinho na mão.\textbackslash{}n\textbackslash{}nAzevedo pareceu ao princípio não\textbackslash{}nconhecê-lo. Adelaide olhava para um e para outro sem compreender nada. Tudo\textbackslash{}nisto, porém, não passou de um minuto; no fim dele Azevedo exclamou:\textbackslash{}n\textbackslash{}n— É o Tito! Entra, Tito!\textbackslash{}n\textbackslash{}nTito entrou galhardamente no jardim;\textbackslash{}nabraçou Azevedo e fez um cumprimento gracioso a Adelaide.\textbackslash{}n\textbackslash{}n— É minha mulher, disse Azevedo\textbackslash{}napresentando Adelaide ao recém-chegado.\textbackslash{}n\textbackslash{}n— Já o suspeitava, respondeu Tito; e\textbackslash{}naproveito a ocasião para dar-te os meus parabéns.\textbackslash{}n\textbackslash{}n— Recebeste a nossa carta de\textbackslash{}nparticipação?\textbackslash{}n\textbackslash{}n— Em Valparaíso.\textbackslash{}n\textbackslash{}n— Anda sentar-te e conta-me a tua viagem.\textbackslash{}n\textbackslash{}n— Isso é longo, disse Tito sentando-se. O\textbackslash{}nque te posso contar é que desembarquei ontem no Rio. Tratei de indagar a tua\textbackslash{}nmorada. Disseram-me que estavas temporariamente em Petrópolis. Descansei, mas logo\textbackslash{}nhoje tomei a barca da Prainha e aqui estou. Eu já suspeitava que com o teu\textbackslash{}nespírito de poeta irias esconder tua felicidade em algum recanto do mundo. Com\textbackslash{}nefeito, isto é verdadeiramente uma nesga do paraíso. Jardim, caramanchões, uma\textbackslash{}ncasa leve e elegante, um livro. Bravo! Marília de Dirceu{\ldots} É completo! Tityre,\textbackslash{}ntu patulae. Caio no meio de um idílio. Pastorinha, onde está o cajado?\textbackslash{}n\textbackslash{}nAdelaide ri às gargalhadas.\textbackslash{}n\textbackslash{}nTito continua:\textbackslash{}n\textbackslash{}n— Ri mesmo como uma pastorinha alegre. E tu,\textbackslash{}nTeócrito, que fazes? Deixas correr os dias como as águas do Paraíba? Feliz\textbackslash{}ncriatura!\textbackslash{}n\textbackslash{}n— Sempre o mesmo! disse Azevedo.\textbackslash{}n\textbackslash{}n— O mesmo doido? Acha que ele tem razão,\textbackslash{}nminha senhora?\textbackslash{}n\textbackslash{}n— Acho, se o não ofendo{\ldots}\textbackslash{}n\textbackslash{}n— Qual ofender! Se eu até me honro com\textbackslash{}nisso; sou um doido inofensivo, isso é verdade. Mas é que realmente são felizes\textbackslash{}ncomo poucos. Há quantos meses se casaram?\textbackslash{}n\textbackslash{}n— Três meses faz domingo, respondeu\textbackslash{}nAdelaide.\textbackslash{}n\textbackslash{}n— Disse há pouco que me pareciam três\textbackslash{}nminutos, acrescentou Azevedo.\textbackslash{}n\textbackslash{}nTito olhou para ambos e disse sorrindo:\textbackslash{}n\textbackslash{}n— Três meses, três minutos! Eis toda a\textbackslash{}nverdade da vida. Se os pusessem sobre uma grelha, como São Lourenço, cinco\textbackslash{}nminutos eram cinco meses. E ainda se fala em tempo! Há lá tempo! O tempo está\textbackslash{}nnas nossas impressões. Há meses para os infelizes e minutos para os venturosos!\textbackslash{}n\textbackslash{}n— Mas que ventura! exclama Azevedo.\textbackslash{}n\textbackslash{}n— Completa, não? Imagino! Marido de um\textbackslash{}nserafim, nas graças e no coração, não reparei que estava aqui{\ldots} mas não\textbackslash{}nprecisa corar!{\ldots} Disto me há de ouvir vinte vezes por dia; o que penso, digo.\textbackslash{}nComo não te hão de invejar os nossos amigos!\textbackslash{}n\textbackslash{}n— Isso não sei.\textbackslash{}n\textbackslash{}n— Pudera! Encafuado neste desvão do\textbackslash{}nmundo, de nada podes saber. E fazes bem. Isto de ser feliz à vista de todos é\textbackslash{}nrepartir a felicidade. Ora, para respeitar o princípio devo ir-me já embora{\ldots}\textbackslash{}n\textbackslash{}nDizendo isto, Tito levantou-se.\textbackslash{}n\textbackslash{}n— Deixa-te disso: fica conosco.\textbackslash{}n\textbackslash{}n— Os verdadeiros amigos também são a\textbackslash{}nfelicidade, disse Adelaide.\textbackslash{}n\textbackslash{}n— Ah!\textbackslash{}n\textbackslash{}n— É até bom que aprendas em nossa escola\textbackslash{}na ciência do casamento, acrescentou Azevedo.\textbackslash{}n\textbackslash{}n— Para quê? perguntou Tito meneando o\textbackslash{}nchicotinho.\textbackslash{}n\textbackslash{}n— Para te casares.\textbackslash{}n\textbackslash{}n— Hum!{\ldots} fez Tito.\textbackslash{}n\textbackslash{}n— Não pretende? perguntou Adelaide.\textbackslash{}n\textbackslash{}n— Estás ainda o mesmo que em outro tempo?\textbackslash{}n\textbackslash{}n— O mesmíssimo, respondeu Tito.\textbackslash{}n\textbackslash{}nAdelaide fez um gesto de curiosidade e\textbackslash{}nperguntou:\textbackslash{}n\textbackslash{}n— Tem horror ao casamento?\textbackslash{}n\textbackslash{}n— Não tenho vocação, respondeu Tito. É\textbackslash{}npuramente um caso de vocação. Quem a não tiver não se meta nisso, que é perder\textbackslash{}no tempo e o sossego. Desde muito tempo estou convencido disto.\textbackslash{}n\textbackslash{}n— Ainda te não bateu a hora.\textbackslash{}n\textbackslash{}n— Nem bate, disse Tito.\textbackslash{}n\textbackslash{}n— Mas, se bem me lembro, disse Azevedo\textbackslash{}noferecendo-lhe um charuto, houve um dia em que fugiste às teorias do costume:\textbackslash{}nandavas então apaixonado{\ldots}\textbackslash{}n\textbackslash{}n— Apaixonado, é engano. Houve um dia em\textbackslash{}nque a Providência trouxe uma confirmação aos meus instintos solitários. Meti-me\textbackslash{}na pretender uma senhora{\ldots}\textbackslash{}n\textbackslash{}n— É verdade: foi um caso engraçado.\textbackslash{}n\textbackslash{}n— Como foi o caso? perguntou Adelaide.\textbackslash{}n\textbackslash{}n— O Tito viu em um baile uma rapariga. No\textbackslash{}ndia seguinte apresenta-se em casa dela, e, sem mais nem menos, pede-lhe a mão.\textbackslash{}nEla responde{\ldots} que te respondeu?\textbackslash{}n\textbackslash{}n— Respondeu por escrito que eu era um\textbackslash{}ntolo e me deixasse daquilo. Não disse positivamente tolo, mas vinha a dar na\textbackslash{}nmesma. É preciso confessar que semelhante resposta não era própria. Voltei\textbackslash{}natrás e nunca mais amei.\textbackslash{}n\textbackslash{}n— Mas amou naquela ocasião? perguntou\textbackslash{}nAdelaide.\textbackslash{}n\textbackslash{}n— Não sei se era amor, respondeu Tito,\textbackslash{}nera uma coisa{\ldots} Mas note, isto foi há uns bons cinco anos. Daí para cá ninguém\textbackslash{}nmais me fez bater o coração.\textbackslash{}n\textbackslash{}n— Pior para ti.\textbackslash{}n\textbackslash{}n— Eu sei! disse Tito levantando os\textbackslash{}nombros. Se não tenho os gozos íntimos do amor, não tenho nem os dissabores, nem\textbackslash{}nos desenganos. É já uma grande fortuna!\textbackslash{}n\textbackslash{}n— No verdadeiro amor não há nada disso, disse\textbackslash{}nsentenciosamente a mulher de Azevedo.\textbackslash{}n\textbackslash{}n— Não há? Deixemos o assunto; eu podia\textbackslash{}nfazer um discurso a propósito, mas prefiro{\ldots}\textbackslash{}n\textbackslash{}n— Ficar conosco, Azevedo atalhou-o. Está\textbackslash{}nsabido.\textbackslash{}n\textbackslash{}n— Não tenho essa intenção.\textbackslash{}n\textbackslash{}n— Mas tenho eu. Hás de ficar.\textbackslash{}n\textbackslash{}n— Mas se eu já mandei o criado tomar\textbackslash{}nalojamento no Hotel de Bragança{\ldots}\textbackslash{}n\textbackslash{}n— Pois manda contra-ordem. Fica comigo.\textbackslash{}n\textbackslash{}n— Insisto em não perturbar a tua paz.\textbackslash{}n\textbackslash{}n— Deixa-te disso.\textbackslash{}n\textbackslash{}n— Fique! disse Adelaide.\textbackslash{}n\textbackslash{}n— Ficarei.\textbackslash{}n\textbackslash{}n— E amanhã, continuou Adelaide, depois de\textbackslash{}nter descansado, há de nos dizer qual é o segredo da isenção de que tanto se\textbackslash{}nufana.\textbackslash{}n\textbackslash{}n— Não há segredo, disse Tito. O que há é\textbackslash{}nisto. Entre um amor que se oferece e{\ldots} uma partida de voltarete, não hesito,\textbackslash{}natiro-me ao voltarete. A propósito, Ernesto, sabes que encontrei no Chile um\textbackslash{}nfamoso parceiro de voltarete? Fez a casca mais temerária que tenho visto{\ldots}\textbackslash{}nsabe o que é uma casca, minha senhora?\textbackslash{}n\textbackslash{}n— Não, respondeu Adelaide.\textbackslash{}n\textbackslash{}n— Pois eu lhe explico.\textbackslash{}n\textbackslash{}nAzevedo olhou para fora e disse:\textbackslash{}n\textbackslash{}n— Aí chega a D. Emília.\textbackslash{}n\textbackslash{}nCom efeito à porta do jardim parava uma\textbackslash{}nsenhora dando o braço a um velho de cinqüenta anos.\textbackslash{}n\textbackslash{}nD. Emília era uma moça a que se pode\textbackslash{}nchamar uma bela mulher; era alta na estatura e altiva de caráter. O amor que\textbackslash{}npudesse infundir seria por imposição. De suas maneiras e das suas graças\textbackslash{}ninspirava um não sei que de rainha que dava vontade de levá-la a um trono.\textbackslash{}n\textbackslash{}nTrajava com elegância e simplicidade. Ela\textbackslash{}ntinha essa elegância natural que é outra elegância diversa da elegância dos\textbackslash{}nenfeites, a propósito da qual já tive ocasião de escrever esta máxima:\textbackslash{}n'Que há pessoas elegantes, e pessoas enfeitadas.'\textbackslash{}n\textbackslash{}nOlhos negros e rasgados, cheios de luz e\textbackslash{}nde grandeza, cabelos castanhos e abundantes, nariz reto como o de Safo, boca\textbackslash{}nvermelha e breve, faces de cetim, colo e braços como os das estátuas, tais eram\textbackslash{}nos traços da beleza de Emília.\textbackslash{}n\textbackslash{}nQuanto ao velho que lhe dava o braço,\textbackslash{}nera, como disse, um homem de cinqüenta anos. Era o que se chama em português\textbackslash{}nchão e rude, - um velho gaiteiro. Pintado, espartilhado, via-se nele uma como que\textbackslash{}nruína do passado reconstruída por mãos modernas, de modo a ter esse aspecto\textbackslash{}nbastardo que não é nem a austeridade da velhice, nem a frescura da mocidade.\textbackslash{}nNão havia dúvida de que o velho devia ter sido um belo rapaz em seus tempos;\textbackslash{}nmas presentemente, se algumas conquistas tivesse feito, só podia contentar-se\textbackslash{}ncom a lembrança delas.\textbackslash{}n\textbackslash{}nQuando Emília entrou no jardim todos se\textbackslash{}nachavam de pé. A recém-chegada apertou a mão a Azevedo e foi beijar Adelaide.\textbackslash{}nIa sentar-se na cadeira que Azevedo lhe oferecera quando reparou em Tito que se\textbackslash{}nachava a um lado.\textbackslash{}n\textbackslash{}nOs dois cumprimentaram-se, mas com ar\textbackslash{}ndiferente. Tito parecia tranqüilo e friamente polido; mas Emília, depois de\textbackslash{}ncumprimentá-lo, conservou os olhos fitos nele, como que avocando uma memória do\textbackslash{}npassado.\textbackslash{}n\textbackslash{}nFeitas as apresentações necessárias, e a\textbackslash{}nDiogo Franco (é o nome do velho braceiro), todos tomaram assentos.\textbackslash{}n\textbackslash{}nA primeira que falou foi Emília:\textbackslash{}n\textbackslash{}n— Ainda hoje não vinha se não fosse a\textbackslash{}nobsequiosidade do Sr. Diogo.\textbackslash{}n\textbackslash{}nAdelaide olhou para o velho e disse:\textbackslash{}n\textbackslash{}n— O Sr. Diogo é uma maravilha.\textbackslash{}n\textbackslash{}nDiogo empertigou-se e murmurou com certo\textbackslash{}ntom de modéstia:\textbackslash{}n\textbackslash{}n— Nem tanto, nem tanto.\textbackslash{}n\textbackslash{}n— É, é, disse Emília. Não é talvez uma,\textbackslash{}nporém duas maravilhas. Ah! sabes que me vai fazer um presente?\textbackslash{}n\textbackslash{}n— Um presente! exclamou Azevedo.\textbackslash{}n\textbackslash{}n— É verdade, continuou Emília, um\textbackslash{}npresente que mandou vir da Europa e lá dos confins; recordações das suas\textbackslash{}nviagens de adolescente{\ldots}\textbackslash{}n\textbackslash{}nDiogo estava radiante.\textbackslash{}n\textbackslash{}n— É uma insignificância, disse ele\textbackslash{}nolhando ternamente para Emília.\textbackslash{}n\textbackslash{}n— Mas o que é? perguntou Adelaide.\textbackslash{}n\textbackslash{}n— É{\ldots} adivinhem? É um urso branco!\textbackslash{}n\textbackslash{}n— Um urso branco!\textbackslash{}n\textbackslash{}n— Deveras?\textbackslash{}n\textbackslash{}n— Está para chegar, mas só ontem é que me\textbackslash{}ndeu notícia dele. Que amável lembrança!\textbackslash{}n\textbackslash{}n— Um urso! exclamou ainda Azevedo.\textbackslash{}n\textbackslash{}nTito inclinou-se ao ouvido do amigo, e disse\textbackslash{}nem voz baixa:\textbackslash{}n\textbackslash{}n— Com ele fazem dois.\textbackslash{}n\textbackslash{}nDiogo jubiloso pelo efeito que causava a\textbackslash{}nnotícia do presente, mas iludido no caráter desse efeito disse:\textbackslash{}n\textbackslash{}n— Não vale a pena. É um urso que eu\textbackslash{}nmandei vir; é verdade que eu pedi dos mais belos. Não sabem o que é um urso\textbackslash{}nbranco. Imaginem que é todo branco.\textbackslash{}n\textbackslash{}n— Ah! disse Tito.\textbackslash{}n\textbackslash{}n— É um animal admirável! tornou Diogo.\textbackslash{}n\textbackslash{}n— Acho que sim, disse Tito. Ora imagina\textbackslash{}ntu o que não será um urso branco que é todo branco. Que faz este sujeito?\textbackslash{}nperguntou ele em seguida a Azevedo.\textbackslash{}n\textbackslash{}n— Namora a Emília; tem cinqüenta contos.\textbackslash{}n\textbackslash{}n— E ela?\textbackslash{}n\textbackslash{}n— Não faz caso dele.\textbackslash{}n\textbackslash{}n— Diz ela?\textbackslash{}n\textbackslash{}n— E é verdade.\textbackslash{}n\textbackslash{}nEnquanto os dois trocavam estas palavras,\textbackslash{}nDiogo brincava com os sinetes do relógio e as duas senhoras conversavam. Depois\textbackslash{}ndas últimas palavras entre Azevedo e Tito, Emília voltou-se para o marido de\textbackslash{}nAdelaide e perguntou:\textbackslash{}n\textbackslash{}n— Dá-se isto, Sr. Azevedo? Então faz-se\textbackslash{}nanos nesta casa e não me convidam?\textbackslash{}n\textbackslash{}n— Mas a chuva? disse Adelaide.\textbackslash{}n\textbackslash{}n— Ingrata! Bem sabes que não há chuva em casos\textbackslash{}ntais.\textbackslash{}n\textbackslash{}n— Demais, acrescentou Azevedo, fez-se a\textbackslash{}nfesta tão à capucha.\textbackslash{}n\textbackslash{}n— Fosse como fosse, eu sou de casa.\textbackslash{}n\textbackslash{}n— É que a lua-de-mel continua apesar de\textbackslash{}ncinco meses, disse Tito.\textbackslash{}n\textbackslash{}n— Aí vens tu com os teus epigramas, disse\textbackslash{}nAzevedo.\textbackslash{}n\textbackslash{}n— Ah! isso é mau, Sr. Tito!\textbackslash{}n\textbackslash{}n— Tito? perguntou Emília a Adelaide em\textbackslash{}nvoz baixa.\textbackslash{}n\textbackslash{}n— Sim.\textbackslash{}n\textbackslash{}n— D. Emília não sabe ainda quem é o nosso\textbackslash{}namigo Tito, disse Azevedo. Eu até tenho medo de dizê-lo.\textbackslash{}n\textbackslash{}n— Então é muito feio o que tem para\textbackslash{}ndizer?\textbackslash{}n\textbackslash{}n— Talvez, disse Tito com indiferença.\textbackslash{}n\textbackslash{}n— Muito feio! exclamou Adelaide.\textbackslash{}n\textbackslash{}n— O que é então? perguntou Emília.\textbackslash{}n\textbackslash{}n— É um homem incapaz de amar, continuou\textbackslash{}nAdelaide. Não pode haver maior indiferença para o amor{\ldots} Em resumo, prefere a\textbackslash{}num amor{\ldots} o quê? um voltarete.\textbackslash{}n\textbackslash{}n— Disse-te isso? perguntou Emília.\textbackslash{}n\textbackslash{}n— E repito, disse Tito. Mas note bem, não\textbackslash{}npor elas, é por mim. Acredito que todas as mulheres sejam credoras da minha\textbackslash{}nadoração; mas eu é que sou feito de modo que nada mais lhes posso conceder do\textbackslash{}nque uma estima desinteressada.\textbackslash{}n\textbackslash{}nEmília olhou para o moço e disse:\textbackslash{}n\textbackslash{}n— Se não é vaidade, é doença.\textbackslash{}n\textbackslash{}n— Há de me perdoar, mas eu creio que não\textbackslash{}né doença, nem vaidade. É natureza: uns aborrecem as laranjas, outros aborrecem\textbackslash{}nos amores: agora se o aborrecimento vem por causa das cascas, não sei; o que é\textbackslash{}ncerto é que é assim.\textbackslash{}n\textbackslash{}n— É ferino! disse Emília olhando para\textbackslash{}nAdelaide.\textbackslash{}n\textbackslash{}n— Ferino, eu? disse Tito levantando-se.\textbackslash{}nSou uma seda, uma dama, um milagre de brandura{\ldots} Dói-me, deveras, que eu não\textbackslash{}npossa estar na linha dos outros homens, e não seja, como todos, propenso a receber\textbackslash{}nas impressões amorosas, mas que quer? a culpa não é minha.\textbackslash{}n\textbackslash{}n— Anda lá, disse Azevedo, o tempo te há\textbackslash{}nde mudar.\textbackslash{}n\textbackslash{}n— Mas quando? Tenho vinte e nove anos\textbackslash{}nfeitos.\textbackslash{}n\textbackslash{}n— Já vinte e nove? perguntou Emília.\textbackslash{}n\textbackslash{}n— Completei-os pela Páscoa.\textbackslash{}n\textbackslash{}n— Não parece.\textbackslash{}n\textbackslash{}n— São os seus bons olhos.\textbackslash{}n\textbackslash{}nA conversa continuou por este modo, até\textbackslash{}nque se anunciou o jantar. Emília e Diogo tinham jantado, ficaram apenas para\textbackslash{}nfazer companhia ao casal Azevedo e a Tito, que declarou desde o princípio estar\textbackslash{}ncaindo de fome.\textbackslash{}n\textbackslash{}nA conversa durante o jantar versou sobre\textbackslash{}ncoisas indiferentes.\textbackslash{}n\textbackslash{}nQuando se servia o café apareceu à porta\textbackslash{}num criado do hotel em que morava Diogo; trazia uma carta para este, com\textbackslash{}nindicação no sobrescrito de que era urgente. Diogo recebeu a carta, leu-a e\textbackslash{}npareceu mudar de cor. Todavia continuou a tomar parte na conversa geral. Aquela\textbackslash{}ncircunstância, porém, deu lugar a que Adelaide perguntasse a Emília:\textbackslash{}n\textbackslash{}n— Quando te deixará este eterno namorado?\textbackslash{}n\textbackslash{}n— Eu sei cá! respondeu Emília. Mas afinal\textbackslash{}nde contas, não é mau homem. Tem aquela mania de me dizer no fim de todas as\textbackslash{}nsemanas que nutre por mim uma ardente paixão.\textbackslash{}n\textbackslash{}n— Enfim, se não passa de declaração\textbackslash{}nsemanal{\ldots}\textbackslash{}n\textbackslash{}n— Não passa. Tem a vantagem de ser um\textbackslash{}nbraceiro infalível para a rua e um realejo menos mau dentro de casa. Já me\textbackslash{}ncontou umas cinqüenta vezes as batalhas amorosas em que entrou. Todo o seu\textbackslash{}ndesejo é acompanhar-me a uma viagem à roda do globo. Quando me fala nisto, se é\textbackslash{}nà noite, e é quase sempre à noite, mando vir o chá, excelente meio de\textbackslash{}naplacar-lhe os ardores amorosos. Gosta do chá que se pela. Gosta tanto como de\textbackslash{}nmim! Mas aquela do urso branco? E se realmente mandou vir um urso?\textbackslash{}n\textbackslash{}n— Aceita.\textbackslash{}n\textbackslash{}n— Pois eu hei de sustentar um urso? Não\textbackslash{}nme faltava mais nada!\textbackslash{}n\textbackslash{}nAdelaide sorriu-se e disse:\textbackslash{}n\textbackslash{}n— Quer me parecer que acabas por te apaixonar{\ldots}\textbackslash{}n\textbackslash{}n— Por quem? Pelo urso?\textbackslash{}n\textbackslash{}n— Não, pelo Diogo.\textbackslash{}n\textbackslash{}nNeste momento achavam-se as duas perto de\textbackslash{}numa janela. Tito conversava no sofá com Azevedo. Diogo refletia profundamente,\textbackslash{}nestendido numa poltrona.\textbackslash{}n\textbackslash{}nEmília tinha os olhos em Tito. Depois de\textbackslash{}num silêncio, disse ela para Adelaide:\textbackslash{}n\textbackslash{}n— Que achas ao tal amigo do teu marido?\textbackslash{}nParece um presumido. Nunca se apaixonou! É crível?\textbackslash{}n\textbackslash{}n— Talvez seja verdade.\textbackslash{}n\textbackslash{}n— Não acredito. Pareces criança! Diz\textbackslash{}naquilo dos dentes para fora{\ldots}\textbackslash{}n\textbackslash{}n— É verdade que não tenho maior conhecimento\textbackslash{}ndele{\ldots}\textbackslash{}n\textbackslash{}n— Quanto a mim, pareceu-me não ser\textbackslash{}nestranha aquela cara{\ldots} mas não me lembro!\textbackslash{}n\textbackslash{}n— Parece ser sincero{\ldots} mas dizer aquilo\textbackslash{}né já atrevimento.\textbackslash{}n\textbackslash{}n— Está claro{\ldots}\textbackslash{}n\textbackslash{}n— De que te ris?\textbackslash{}n\textbackslash{}n— Lembra-me um do mesmo gênero que este,\textbackslash{}ndisse Emília. Foi já há tempos. Andava sempre a gabar-se da sua isenção. Dizia\textbackslash{}nque todas as mulheres eram para ele vasos da China: admirava-as e nada mais.\textbackslash{}nCoitado! Caiu em menos de um mês. Adelaide, vi-o beijar-me a ponta dos\textbackslash{}nsapatos{\ldots} depois do que desprezei-o.\textbackslash{}n\textbackslash{}n— Que fizeste?\textbackslash{}n\textbackslash{}n— Ah! não sei o que fiz. Santa Astúcia\textbackslash{}nfoi quem operou o milagre. Vinguei o sexo e abati um orgulhoso.\textbackslash{}n\textbackslash{}n— Bem feito!\textbackslash{}n\textbackslash{}n— Não era menos do que este. Mas falemos\textbackslash{}nde coisas sérias{\ldots} Recebi as folhas francesas de modas{\ldots}\textbackslash{}n\textbackslash{}n— Que há de novo?\textbackslash{}n\textbackslash{}n— Muita coisa. Amanhã tas mandarei.\textbackslash{}nRepara em um novo corte de mangas. É lindíssimo. Já mandei encomendas para a\textbackslash{}ncorte. Em artigos de passeios há fartura e do melhor.\textbackslash{}n\textbackslash{}n— Para mim quase que é inútil mandar.\textbackslash{}n\textbackslash{}n— Por quê?\textbackslash{}n\textbackslash{}n— Quase nunca saio de casa.\textbackslash{}n\textbackslash{}n— Nem ao menos irás jantar comigo no dia\textbackslash{}nde ano-bom!\textbackslash{}n\textbackslash{}n— Oh! com toda a certeza!\textbackslash{}n\textbackslash{}n— Pois vai{\ldots} Ah! irá o homem? O Sr.\textbackslash{}nTito?\textbackslash{}n\textbackslash{}n— Se estiver cá{\ldots} e quiseres{\ldots}\textbackslash{}n\textbackslash{}n— Pois que vá, não faz mal{\ldots} saberei\textbackslash{}ncontê-lo{\ldots} Creio que não será sempre tão{\ldots} incivil. Nem sei como podes ficar\textbackslash{}ncom esse sangue-frio! A mim faz-me mal aos nervos!\textbackslash{}n\textbackslash{}n— É-me indiferente.\textbackslash{}n\textbackslash{}n— Mas a injúria ao sexo{\ldots} não te\textbackslash{}nindigna?\textbackslash{}n\textbackslash{}n— Pouco.\textbackslash{}n\textbackslash{}n— És feliz.\textbackslash{}n\textbackslash{}n— Que queres que eu faça a um homem que diz\textbackslash{}naquilo? Se não fosse casada era possível que me indignasse mais. Se fosse livre\textbackslash{}nera provável que lhe fizesse o que fizeste ao outro. Mas eu não posso cuidar\textbackslash{}ndessas coisas{\ldots}\textbackslash{}n\textbackslash{}n— Nem ouvindo a preferência do voltarete?\textbackslash{}nPôr-nos abaixo da dama de copas! E o ar com que ele diz aquilo! Que calma, que\textbackslash{}nindiferença!\textbackslash{}n\textbackslash{}n— É mau! é mau!\textbackslash{}n\textbackslash{}n— Merecia castigo{\ldots}\textbackslash{}n\textbackslash{}n— Merecia. Queres tu castigá-lo?\textbackslash{}n\textbackslash{}nEmília fez um gesto de desdém e disse:\textbackslash{}n\textbackslash{}n— Não vale a pena.\textbackslash{}n\textbackslash{}n— Mas tu castigaste o outro.\textbackslash{}n\textbackslash{}n— Sim{\ldots} mas não vale a pena.\textbackslash{}n\textbackslash{}n— Dissimulada!\textbackslash{}n\textbackslash{}n— Por que dizes isso?\textbackslash{}n\textbackslash{}n— Porque já te vejo meio tentada a uma\textbackslash{}nnova vingança{\ldots}\textbackslash{}n\textbackslash{}n— Eu? Ora qual!\textbackslash{}n\textbackslash{}n— Que tem? Não é crime{\ldots}\textbackslash{}n\textbackslash{}n— Não é, decerto; mas{\ldots} veremos.\textbackslash{}n\textbackslash{}n— Ah! serás capaz?\textbackslash{}n\textbackslash{}n— Capaz? disse Emília com um gesto de\textbackslash{}norgulho ofendido.\textbackslash{}n\textbackslash{}n— Beijar-te-á ele a ponta do sapato?\textbackslash{}n\textbackslash{}nEmília ficou silenciosa por alguns\textbackslash{}nmomentos; depois apontando com o leque para a botina que lhe calçava o pé,\textbackslash{}ndisse:\textbackslash{}n\textbackslash{}n— E hão de ser estes.\textbackslash{}n\textbackslash{}nEmília e Adelaide se dirigiram para o lado\textbackslash{}nem que se achavam os homens. Tito, que parecia conversar intimamente com\textbackslash{}nAzevedo, interrompeu a conversa para dar atenção às senhoras. Diogo continuava\textbackslash{}nmergulhado na sua meditação.\textbackslash{}n\textbackslash{}n— Então o que é isso, Sr. Diogo?\textbackslash{}nperguntou Tito. Está meditando?\textbackslash{}n\textbackslash{}n— Ah! perdão, estava distraído!\textbackslash{}n\textbackslash{}n— Coitado! disse Tito baixo a Azevedo.\textbackslash{}n\textbackslash{}nDepois, voltando-se para as senhoras:\textbackslash{}n\textbackslash{}n— Não as incomoda o charuto?\textbackslash{}n\textbackslash{}n— Não senhor, disse Emília.\textbackslash{}n\textbackslash{}n— Então, posso continuar a fumar?\textbackslash{}n\textbackslash{}n— Pode, disse Adelaide.\textbackslash{}n\textbackslash{}n— É um mau vício, mas é o meu único\textbackslash{}nvício. Quando fumo parece que aspiro a eternidade. Enlevo-me todo e mudo de\textbackslash{}nser. Divina invenção!\textbackslash{}n\textbackslash{}n— Dizem que é excelente para os desgostos\textbackslash{}namorosos, disse Emília com intenção.\textbackslash{}n\textbackslash{}n— Isso não sei. Mas não é só isto. Depois\textbackslash{}nda invenção do fumo não há solidão possível. É a melhor companhia deste mundo.\textbackslash{}nDemais, o charuto é um verdadeiro Memento homo: convertendo-se pouco a\textbackslash{}npouco em cinzas, vai lembrando ao homem o fim real e infalível de todas as\textbackslash{}ncoisas: é o aviso filosófico, é a sentença fúnebre que nos acompanha em toda a\textbackslash{}nparte. Já é um grande progresso{\ldots} Mas estou eu a aborrecer com uma dissertação\textbackslash{}ntão pesada. Hão de desculpar{\ldots} que foi descuido. Ora, a falar a verdade, eu já\textbackslash{}nvou desconfiando; Vossa Excelência olha com olhos tão singulares{\ldots}\textbackslash{}n\textbackslash{}nEmília, a quem era dirigida a palavra,\textbackslash{}nrespondeu:\textbackslash{}n\textbackslash{}n— Não sei se são singulares, mas são os\textbackslash{}nmeus.\textbackslash{}n\textbackslash{}n— Penso que não são os do costume. Está\textbackslash{}ntalvez Vossa Excelência a dizer consigo que eu sou um esquisito, um singular,\textbackslash{}num{\ldots}\textbackslash{}n\textbackslash{}n— Um vaidoso, é verdade.\textbackslash{}n\textbackslash{}n— Sétimo mandamento: não levantar falsos\textbackslash{}ntestemunhos.\textbackslash{}n\textbackslash{}n— Falsos, diz o mandamento.\textbackslash{}n\textbackslash{}n— Não me dirá em que sou eu vaidoso?\textbackslash{}n\textbackslash{}n— Ah! a isso não respondo eu.\textbackslash{}n\textbackslash{}n— Por que não quer?\textbackslash{}n\textbackslash{}n— Porque{\ldots} não sei. É uma coisa que se\textbackslash{}nsente, mas que se não pode descobrir. Respira-lhe a vaidade em tudo: no olhar,\textbackslash{}nna palavra, no gesto{\ldots} mas não se atina com a verdadeira origem de tal doença.\textbackslash{}n\textbackslash{}n— É pena. Eu tinha grande prazer em ouvir\textbackslash{}nda sua boca o diagnóstico da minha doença. Em compensação pode ouvir da minha o\textbackslash{}ndiagnóstico da sua{\ldots} A sua doença é{\ldots} Digo?\textbackslash{}n\textbackslash{}n— Pode dizer.\textbackslash{}n\textbackslash{}n— É um despeitozinho.\textbackslash{}n\textbackslash{}n— Deveras?\textbackslash{}n\textbackslash{}n— Vamos ver isso, disse Azevedo rindo-se.\textbackslash{}n\textbackslash{}nTito continuou:\textbackslash{}n\textbackslash{}n— Despeito pelo que eu disse há pouco.\textbackslash{}n\textbackslash{}n— Puro engano! disse Emília rindo-se.\textbackslash{}n\textbackslash{}n— É com toda a certeza. Mas é tudo\textbackslash{}ngratuito. Eu não tenho culpa de coisa alguma. A natureza é que me fez assim.\textbackslash{}n\textbackslash{}n— Só a natureza?\textbackslash{}n\textbackslash{}n— E um tanto de estudo. Ora vou expor-lhe\textbackslash{}nas minhas razões. Veja se posso amar ou pretender: primeiro, não sou bonito{\ldots}\textbackslash{}n\textbackslash{}n— Oh!{\ldots} disse Emília.\textbackslash{}n\textbackslash{}n— Agradeço o protesto, mas continuo na\textbackslash{}nmesma opinião: não sou bonito, não sou{\ldots}\textbackslash{}n\textbackslash{}n— Oh!{\ldots} disse Adelaide.\textbackslash{}n\textbackslash{}n— Segundo: não sou curioso, e o amor, se\textbackslash{}no reduzirmos às suas verdadeiras proporções, não passa de uma curiosidade;\textbackslash{}nterceiro: não sou paciente, e nas conquistas amorosas a paciência é a principal\textbackslash{}nvirtude; quarto, finalmente: não sou idiota, porque, se com todos estes\textbackslash{}ndefeitos pretendesse amar, mostraria a maior falta de razão. Aqui está o que eu\textbackslash{}nsou por natural e por indústria.\textbackslash{}n\textbackslash{}n— Emília, parece que é sincero.\textbackslash{}n\textbackslash{}n— Acreditas?\textbackslash{}n\textbackslash{}n— Sincero como a verdade, disse Tito.\textbackslash{}n\textbackslash{}n— Em último caso, seja ou não seja\textbackslash{}nsincero, que tenho eu com isso?\textbackslash{}n\textbackslash{}n— Eu creio que nada, disse Tito.\textbackslash{}n\textbackslash{}nCAPÍTULO II\textbackslash{}n\textbackslash{}nNo dia seguinte àquele em que se passaram\textbackslash{}nas cenas descritas no capítulo anterior, entendeu o céu que devia regar com as\textbackslash{}nsuas lágrimas o solo da formosa Petrópolis.\textbackslash{}n\textbackslash{}nTito, que destinava esse dia a ver toda a\textbackslash{}ncidade, foi obrigado a conservar-se em casa. Era um amigo que não incomodava,\textbackslash{}nporque quando era de mais sabia escapar-se discretamente, e quando o não era,\textbackslash{}ntornava-se o mais delicioso dos companheiros.\textbackslash{}n\textbackslash{}nTito sabia juntar muita jovialidade a\textbackslash{}nmuita delicadeza; sabia fazer rir sem saltar fora das conveniências. Acrescia\textbackslash{}nque, voltando de uma longa e pitoresca viagem, trazia as algibeiras da memória\textbackslash{}n(deixem passar a frase) cheias de vivas reminiscências. Tinha feito uma viagem\textbackslash{}nde poeta e não de peralvilho. Soube ver e sabia contar. Estas duas qualidades,\textbackslash{}nindispensáveis ao viajante, por desgraça são as mais raras. A maioria das\textbackslash{}npessoas que viajam nem sabem ver, nem sabem contar.\textbackslash{}n\textbackslash{}nTito tinha andado por todas as repúblicas\textbackslash{}ndo mar Pacífico, tinha vivido no México e em alguns Estados americanos. Tinha depois\textbackslash{}nido à Europa no paquete da linha de Nova Iorque. Viu Londres e Paris. Foi à\textbackslash{}nEspanha, onde viveu a vida de Almaviva, dando serenatas às janelas das Rosinas\textbackslash{}nde hoje. Trouxe de lá alguns leques e mantilhas. Passou à Itália e levantou o\textbackslash{}nespírito à altura das recordações da arte clássica. Viu a sombra de Dante nas\textbackslash{}nruas de Florença; viu as almas dos doges pairando saudosas sobre as águas\textbackslash{}nviúvas do mar Adriático; a terra de Rafael, de Virgílio e Miguel Ângelo foi\textbackslash{}npara ele uma fonte viva de recordações do passado e de impressões para o\textbackslash{}nfuturo. Foi à Grécia, onde soube evocar o espírito das gerações extintas que\textbackslash{}nderam ao gênio da arte e da poesia um fulgor que atravessou as sombras dos\textbackslash{}nséculos.\textbackslash{}n\textbackslash{}nViajou ainda mais o nosso herói, e tudo\textbackslash{}nviu com olhos de quem sabe ver e tudo contava com alma de quem sabe contar.\textbackslash{}nAzevedo e Adelaide passavam horas esquecidas.\textbackslash{}n\textbackslash{}n— Do amor, dizia ele, eu só sei que é uma\textbackslash{}npalavra de quatro letras, um tanto eufônica, é verdade, mas núncia de lutas e\textbackslash{}ndesgraças. Os bons amores são cheios de felicidade, porque têm a virtude de não\textbackslash{}nalçarem olhos para as estrelas do céu; contentam-se com ceias à meia-noite e\textbackslash{}nalguns passeios a cavalo ou por mar.\textbackslash{}n\textbackslash{}nEsta era a linguagem constante de Tito.\textbackslash{}nExprimia ela a verdade, ou era uma linguagem de convenção? Todos acreditavam\textbackslash{}nque a verdade estava na primeira hipótese, até porque essa era de acordo com o\textbackslash{}nespírito jovial e folgazão de Tito.\textbackslash{}n\textbackslash{}nNo primeiro dia da residência de Tito em\textbackslash{}nPetrópolis, a chuva, como disse acima, impediu que os diversos personagens desta\textbackslash{}nhistória se encontrassem. Cada qual ficou na sua casa. Mas o dia imediato foi\textbackslash{}nmais benigno; Tito aproveitou o bom tempo para ir ver a risonha cidade da\textbackslash{}nserra. Azevedo e Adelaide quiseram acompanhá-lo; mandaram aparelhar três\textbackslash{}nginetes próprios para o ligeiro passeio.\textbackslash{}n\textbackslash{}nNa volta foram visitar Emília. Durou\textbackslash{}npoucos minutos a visita. A bela viúva recebeu-os com graça e cortesia de\textbackslash{}nprincesa. Era a primeira vez que Tito lá ia; e fosse por isso, ou por outra\textbackslash{}ncircunstância, foi ele quem mereceu as principais atenções da dona da casa.\textbackslash{}n\textbackslash{}nDiogo, que então fazia a sua centésima\textbackslash{}ndeclaração de amor a Emília, e a quem Emília acabava de oferecer uma chávena de\textbackslash{}nchá, não viu com bons olhos a demasiada atenção que o viajante merecia da dama dos\textbackslash{}nseus pensamentos. Essa, e talvez outras circunstâncias, faziam com que o velho\textbackslash{}nAdônis assistisse à conversação com a cara fechada.\textbackslash{}n\textbackslash{}nÀ despedida Emília ofereceu a casa a\textbackslash{}nTito, com a declaração de que teria a mesma satisfação em recebê-lo muitas\textbackslash{}nvezes. Tito aceitou cavalheiramente o oferecimento; feito o que, saíram todos.\textbackslash{}n\textbackslash{}nCinco dias depois desta visita Emília foi\textbackslash{}nà casa de Adelaide. Tito não estava presente; andava a passeio. Azevedo tinha\textbackslash{}nsaído para um negócio, mas voltou daí a alguns minutos. Quando, depois de uma\textbackslash{}nhora de conversa, Emília já de pé preparava-se para voltar à casa, entrou Tito.\textbackslash{}n\textbackslash{}n— Ia sair quando entrou, disse Emília.\textbackslash{}nParece que nos contrariamos em tudo.\textbackslash{}n\textbackslash{}n— Não é por minha vontade, respondeu\textbackslash{}nTito; pelo contrário, meu desejo é não contrariar pessoa alguma, e portanto não\textbackslash{}ncontrariar Vossa Excelência.\textbackslash{}n\textbackslash{}n— Não parece.\textbackslash{}n\textbackslash{}n— Por quê?\textbackslash{}n\textbackslash{}nEmília sorriu e disse com uma inflexão de\textbackslash{}ncensura:\textbackslash{}n\textbackslash{}n— Sabe que me daria prazer se utilizasse\textbackslash{}ndo oferecimento de minha casa; ainda se não utilizou. Foi esquecimento?\textbackslash{}n\textbackslash{}n— Foi.\textbackslash{}n\textbackslash{}n— É muito amável{\ldots}\textbackslash{}n\textbackslash{}n— Sou muito franco. Eu sei que Vossa\textbackslash{}nExcelência preferia uma delicada mentira; mas eu não conheço nada mais delicado\textbackslash{}nque a verdade.\textbackslash{}n\textbackslash{}nEmília sorriu.\textbackslash{}n\textbackslash{}nNesse momento entrou Diogo.\textbackslash{}n\textbackslash{}n— Ia sair, D. Emília? perguntou ele.\textbackslash{}n\textbackslash{}n— Esperava o seu braço.\textbackslash{}n\textbackslash{}n— Aqui o tem.\textbackslash{}n\textbackslash{}nEmília despediu-se de Azevedo e de\textbackslash{}nAdelaide. Quanto a Tito, no momento em que ele curvava-se respeitosamente,\textbackslash{}nEmília disse-lhe com a maior placidez da alma:\textbackslash{}n\textbackslash{}n— Há alguém tão delicado como a verdade:\textbackslash{}né o Sr. Diogo. Espero dizer o mesmo{\ldots}\textbackslash{}n\textbackslash{}n— De mim? interrompeu Tito. Amanhã mesmo.\textbackslash{}n\textbackslash{}nEmília saiu pelo braço de Diogo.\textbackslash{}n\textbackslash{}nNo dia seguinte, com efeito, Tito foi à\textbackslash{}ncasa de Emília. Ela o esperava com certa impaciência. Como não soubesse a hora\textbackslash{}nem que ele devia apresentar-se lá, a bela viúva esperou-o a todos os momentos,\textbackslash{}ndesde manhã. Só ao cair da tarde é que Tito dignou-se aparecer.\textbackslash{}n\textbackslash{}nEmília morava com uma tia velha. Era uma\textbackslash{}nboa senhora, amiga da sobrinha, e inteiramente escrava da sua vontade. Isto\textbackslash{}nquer dizer que não havia em Emília o menor receio que a boa tia não assinasse\textbackslash{}nde antemão.\textbackslash{}n\textbackslash{}nNa sala em que Tito foi recebido não\textbackslash{}nestava ninguém. Ele teve portanto tempo de sobra para examiná-la à vontade. Era\textbackslash{}numa sala pequena, mas mobiliada e adornada com gosto. Móveis leves, elegantes e\textbackslash{}nricos; quatro finíssimas estatuetas, copiadas de Pradier, um piano de Erard,\textbackslash{}ntudo disposto e arranjado com vida.\textbackslash{}n\textbackslash{}nTito gastou o primeiro quarto de hora no\textbackslash{}nexame da sala e dos objetos que a enchiam. Esse exame devia influir muito no\textbackslash{}nestudo que ele quisesse fazer do espírito da moça. Dize-me como moras,\textbackslash{}ndir-te-ei quem és.\textbackslash{}n\textbackslash{}nMas o primeiro quarto de hora correu sem\textbackslash{}nque aparecesse viva alma, nem que se ouvisse rumor de natureza alguma. Tito\textbackslash{}ncomeçou a impacientar-se. Já sabemos que espírito brusco era ele, apesar da\textbackslash{}nsuprema delicadeza que todos lhe reconheciam. Parece, porém, que a sua rudeza,\textbackslash{}nquase sempre exercida contra Emília, era antes estudada que natural. O que é\textbackslash{}ncerto é que no fim de meia hora, aborrecido pela demora, Tito murmurou consigo:\textbackslash{}n\textbackslash{}n— Quer tomar desforra!\textbackslash{}n\textbackslash{}nE tomando o chapéu que havia posto numa\textbackslash{}ncadeira ia dirigindo-se para a porta quando ouviu um farfalhar de sedas. Voltou\textbackslash{}na cabeça; Emília entrava.\textbackslash{}n\textbackslash{}n— Fugia?\textbackslash{}n\textbackslash{}n— É verdade.\textbackslash{}n\textbackslash{}n— Perdoe a demora.\textbackslash{}n\textbackslash{}n— Não há que perdoar; não podia vir, era\textbackslash{}nnatural que fosse por algum motivo sério. Quanto a mim não tenho igualmente de\textbackslash{}nque pedir perdão. Esperei, estava cansado, voltaria em outra ocasião. Tudo isto\textbackslash{}né natural.\textbackslash{}n\textbackslash{}nEmília ofereceu uma cadeira a Tito e\textbackslash{}nsentou-se num sofá.\textbackslash{}n\textbackslash{}n— Realmente, disse ela acomodando o\textbackslash{}nbalão, o Sr. Tito é um homem original.\textbackslash{}n\textbackslash{}n— É a minha glória. Não imagina como eu\textbackslash{}naborreço as cópias. Fazer o que muita gente faz, que mérito há nisso? Não nasci\textbackslash{}npara esses trabalhos de imitação.\textbackslash{}n\textbackslash{}n— Já uma coisa fez como muita gente.\textbackslash{}n\textbackslash{}n— Qual foi?\textbackslash{}n\textbackslash{}n— Prometeu-me ontem esta visita e veio\textbackslash{}ncumprir a promessa.\textbackslash{}n\textbackslash{}n— Ah! minha senhora, não lance isto à\textbackslash{}nconta das minhas virtudes. Podia não vir; vim; não foi vontade, foi{\ldots} acaso.\textbackslash{}n\textbackslash{}n— Em todo caso, agradeço-lhe.\textbackslash{}n\textbackslash{}n— É o meio de me fechar a sua porta.\textbackslash{}n\textbackslash{}n— Por quê?\textbackslash{}n\textbackslash{}n— Porque eu não me dou com esses\textbackslash{}nagradecimentos; nem creio mesmo que eles possam acrescentar nada à minha\textbackslash{}nadmiração pela pessoa de Vossa Excelência. Fui visitar muitas vezes as estátuas\textbackslash{}ndos museus da Europa, mas se elas se lembrassem de me agradecer um dia, dou-lhe\textbackslash{}na minha palavra que não voltava lá.\textbackslash{}n\textbackslash{}nA estas palavras seguiu-se um silêncio de\textbackslash{}nalguns segundos.\textbackslash{}n\textbackslash{}nEmília foi quem falou primeiro.\textbackslash{}n\textbackslash{}n— Há muito tempo que se dá com o marido\textbackslash{}nde Adelaide?\textbackslash{}n\textbackslash{}n— Desde criança, respondeu Tito.\textbackslash{}n\textbackslash{}n— Ah! foi criança?\textbackslash{}n\textbackslash{}n— Ainda hoje sou.\textbackslash{}n\textbackslash{}n— É exatamente o tempo das minhas\textbackslash{}nrelações com Adelaide. Nunca me arrependi.\textbackslash{}n\textbackslash{}n— Nem eu.\textbackslash{}n\textbackslash{}n— Houve um tempo, prosseguiu Emília, em\textbackslash{}nque estivemos separadas; mas isso não trouxe mudança alguma às nossas relações.\textbackslash{}nFoi no tempo do meu primeiro casamento.\textbackslash{}n\textbackslash{}n— Ah! foi casada duas vezes?\textbackslash{}n\textbackslash{}n— Em dois anos.\textbackslash{}n\textbackslash{}n— E por que enviuvou da primeira?\textbackslash{}n\textbackslash{}n— Porque meu marido morreu, disse Emília\textbackslash{}nrindo-se.\textbackslash{}n\textbackslash{}n— Mas eu pergunto outra coisa. Por que se\textbackslash{}nfez viúva, mesmo depois da morte de seu primeiro marido? Creio que poderia\textbackslash{}ncontinuar casada.\textbackslash{}n\textbackslash{}n— De que modo? perguntou Emília com\textbackslash{}nespanto.\textbackslash{}n\textbackslash{}n— Ficando mulher do finado. Se o amor\textbackslash{}nacaba na sepultura acho que não vale a pena de procurá-lo neste mundo.\textbackslash{}n\textbackslash{}n— Realmente o Sr. Tito é um espírito fora\textbackslash{}ndo comum.\textbackslash{}n\textbackslash{}n— Um tanto.\textbackslash{}n\textbackslash{}n— É preciso que o seja para desconhecer\textbackslash{}nque a nossa vida não importa essas exigências da eterna fidelidade. E demais,\textbackslash{}npode-se conservar a lembrança dos que morrem sem renunciar às condições da\textbackslash{}nnossa existência. Agora é que eu lhe pergunto por que me olha com olhos tão\textbackslash{}nsingulares?{\ldots}\textbackslash{}n\textbackslash{}n— Não sei se são singulares, mas são os\textbackslash{}nmeus.\textbackslash{}n\textbackslash{}n— Então, acha que eu cometi uma bigamia?\textbackslash{}n\textbackslash{}n— Eu não acho nada. Ora, deixe-me\textbackslash{}ndizer-lhe a última razão da minha incapacidade para os amores.\textbackslash{}n\textbackslash{}n— Sou toda ouvidos.\textbackslash{}n\textbackslash{}n— Eu não creio na fidelidade.\textbackslash{}n\textbackslash{}n— Em absoluto?\textbackslash{}n\textbackslash{}n— Em absoluto.\textbackslash{}n\textbackslash{}n— Muito obrigada.\textbackslash{}n\textbackslash{}n— Ah! eu sei que isto não é delicado; mas\textbackslash{}nem primeiro lugar, eu tenho a coragem das minhas opiniões, e em segundo foi\textbackslash{}nVossa Excelência quem me provocou. É infelizmente verdade, eu não creio nos\textbackslash{}namores leais e eternos. Quero fazê-la minha confidente. Houve um dia em que eu\textbackslash{}ntentei amar; concentrei todas as forças vivas do meu coração; dispus-me a\textbackslash{}nreunir o meu orgulho e a minha ilusão na cabeça do objeto amado. Que lição\textbackslash{}nmestra! O objeto amado, depois de me alimentar as esperanças, casou-se com\textbackslash{}noutro que não era nem mais bonito, nem mais amante.\textbackslash{}n\textbackslash{}n— Que prova isso? perguntou a viúva.\textbackslash{}n\textbackslash{}n— Prova que me aconteceu o que pode\textbackslash{}nacontecer e acontece diariamente aos outros.\textbackslash{}n\textbackslash{}n— Ora{\ldots}\textbackslash{}n\textbackslash{}n— Há de me perdoar, mas eu creio que é\textbackslash{}numa coisa já metida na massa do sangue{\ldots}\textbackslash{}n\textbackslash{}n— Não diga isso. É certo que podem\textbackslash{}nacontecer casos desses; mas serão todos assim? Não admite uma exceção?\textbackslash{}nAprofunde mais os corações alheios se quiser encontrar a verdade{\ldots} e há de\textbackslash{}nencontrar.\textbackslash{}n\textbackslash{}n— Qual! disse Tito abaixando a cabeça e\textbackslash{}nbatendo com a bengala na ponta do pé.\textbackslash{}n\textbackslash{}n— Posso afirmá-lo, disse Emília.\textbackslash{}n\textbackslash{}n— Duvido.\textbackslash{}n\textbackslash{}n— Tenho pena de uma criatura assim,\textbackslash{}ncontinuou a viúva. Não conhecer o amor é não conhecer a vida! Há nada igual à\textbackslash{}nunião de duas almas que se adoram? Desde que o amor entra no coração, tudo se\textbackslash{}ntransforma, tudo muda, a noite parece dia, a dor assemelha-se ao prazer{\ldots} Se\textbackslash{}nnão conhece nada disto, pode morrer, porque é o mais infeliz dos homens.\textbackslash{}n\textbackslash{}n— Tenho lido isso nos livros, mas ainda\textbackslash{}nnão me convenci{\ldots}\textbackslash{}n\textbackslash{}n— Já reparou na minha sala?\textbackslash{}n\textbackslash{}n— Já vi alguma coisa.\textbackslash{}n\textbackslash{}n— Reparou naquela gravura?\textbackslash{}n\textbackslash{}nTito olhou para a gravura que a viúva lhe\textbackslash{}nindicava.\textbackslash{}n\textbackslash{}n— Se me não engano, disse ele, aquilo é o\textbackslash{}nAmor domando as feras.\textbackslash{}n\textbackslash{}n— Veja e convença-se.\textbackslash{}n\textbackslash{}n— Com a opinião do desenhista? perguntou\textbackslash{}nTito. Não é possível. Tenho visto gravuras vivas. Tenho servido de alvo a\textbackslash{}nmuitas setas; crivam-me todo, mas eu tenho a fortaleza de S. Sebastião;\textbackslash{}nafronto, não me curvo.\textbackslash{}n\textbackslash{}n— Que orgulho!\textbackslash{}n\textbackslash{}n— O que pode fazer dobrar uma altivez\textbackslash{}ndestas? A beleza? Nem Cleópatra. A castidade? Nem Susana. Resuma, se quiser,\textbackslash{}ntodas as qualidades em uma só criatura, e eu não mudarei{\ldots} É isto e nada mais.\textbackslash{}n\textbackslash{}nEmília levantou-se e dirigiu-se para o\textbackslash{}npiano.\textbackslash{}n\textbackslash{}n— Não aborrece a música? perguntou ela\textbackslash{}nabrindo o piano.\textbackslash{}n\textbackslash{}n— Adoro-a, respondeu o moço sem se mover;\textbackslash{}nagora quanto aos executantes só gosto dos bons. Os maus dá-me ímpetos de\textbackslash{}nenforcá-los.\textbackslash{}n\textbackslash{}nEmília executou ao piano os prelúdios de\textbackslash{}numa sinfonia. Tito ouvia-a com a mais profunda atenção. Realmente a bela viúva\textbackslash{}ntocava divinamente.\textbackslash{}n\textbackslash{}n— Então, disse ela levantando-se, devo\textbackslash{}nser enforcada?\textbackslash{}n\textbackslash{}n— Deve ser coroada. Toca perfeitamente.\textbackslash{}n\textbackslash{}n— Outro ponto em que não é original. Toda\textbackslash{}na gente me diz isso.\textbackslash{}n\textbackslash{}n— Ah! eu também não nego a luz do sol.\textbackslash{}n\textbackslash{}nNeste momento entrou na sala a tia de\textbackslash{}nEmília. Esta apresentou-lhe Tito. A conversa tomou então um tom pessoal e\textbackslash{}nreservado; durou pouco, aliás, porque Tito, travando repentinamente do chapéu,\textbackslash{}ndeclarou que tinha que fazer.\textbackslash{}n\textbackslash{}n— Até quando?\textbackslash{}n\textbackslash{}n— Até sempre.\textbackslash{}n\textbackslash{}nDespediu-se e saiu.\textbackslash{}n\textbackslash{}nEmília ainda o acompanhou com os olhos\textbackslash{}npor algum tempo, da janela da casa. Mas Tito, como se o caso não fosse com ele,\textbackslash{}nseguiu sem olhar para trás.\textbackslash{}n\textbackslash{}nMas, exatamente no momento em que Emília\textbackslash{}nvoltava para dentro, Tito encontrava o velho Diogo.\textbackslash{}n\textbackslash{}nDiogo ia na direção da casa da viúva.\textbackslash{}nTinha um ar pensativo. Tão distraído ia que chegou quase a esbarrar com Tito.\textbackslash{}n\textbackslash{}n— Onde vai tão distraído? perguntou Tito.\textbackslash{}n\textbackslash{}n— Ah! é o senhor? Vem da casa de D.\textbackslash{}nEmília?\textbackslash{}n\textbackslash{}n— Venho.\textbackslash{}n\textbackslash{}n— Eu para lá vou. Coitada! há de estar\textbackslash{}nmuito impaciente com a minha demora.\textbackslash{}n\textbackslash{}n— Não está, não senhor, respondeu Tito\textbackslash{}ncom o maior sangue-frio.\textbackslash{}n\textbackslash{}nDiogo lançou-lhe um olhar de despeito.\textbackslash{}n\textbackslash{}nA isso seguiu-se um silêncio de alguns\textbackslash{}nminutos, durante o qual Diogo brincava com a corrente do relógio, e Tito\textbackslash{}nlançava ao ar novelos de fumaça de um primoroso havana. Um desses novelos foi\textbackslash{}ndesenrolar-se na cara de Diogo. O velho tossiu e disse a Tito:\textbackslash{}n\textbackslash{}n— Apre lá, Sr. Tito! É demais!\textbackslash{}n\textbackslash{}n— O quê, meu caro senhor? perguntou o rapaz.\textbackslash{}n\textbackslash{}n— Até a fumaça!\textbackslash{}n\textbackslash{}n— Foi sem reparar. Mas eu não compreendo\textbackslash{}nas suas palavras{\ldots}\textbackslash{}n\textbackslash{}n— Eu me faço explicar, disse o velho\textbackslash{}ntomando um ar risonho. Dê-me o seu braço{\ldots}\textbackslash{}n\textbackslash{}n— Pois não!\textbackslash{}n\textbackslash{}nE os dois seguiram conversando como dois\textbackslash{}namigos velhos.\textbackslash{}n\textbackslash{}n— Estou pronto a ouvir a sua explicação.\textbackslash{}n\textbackslash{}n— Lá vai. Sabe o que eu quero? É que seja\textbackslash{}nfranco. Não ignora que eu suspiro aos pés da viúva. Peço-lhe que não discuta o\textbackslash{}nfato, admita-o simplesmente. Até aqui tudo ia caminhando bem, quando o senhor\textbackslash{}nchegou a Petrópolis.\textbackslash{}n\textbackslash{}n— Mas{\ldots}\textbackslash{}n\textbackslash{}n— Ouça-me silenciosamente. Chegou o\textbackslash{}nsenhor a Petrópolis, e sem que eu lhe tivesse feito mal algum, entendeu de si\textbackslash{}npara si que me havia de tirar do lance. Desde então começou a corte{\ldots}\textbackslash{}n\textbackslash{}n— Meu caro Sr. Diogo, tudo isso é uma\textbackslash{}nfantasia. Eu não faço a corte a D. Emília, nem pretendo fazer-lha. Vê-me acaso\textbackslash{}nfreqüentar a casa dela?\textbackslash{}n\textbackslash{}n— Acaba de sair de lá.\textbackslash{}n\textbackslash{}n— É a primeira vez que a visito.\textbackslash{}n\textbackslash{}n— Quem sabe?\textbackslash{}n\textbackslash{}n— Demais, ainda ontem não ouviu em casa\textbackslash{}nde Azevedo as expressões com que ela se despediu de mim? Não são de mulher\textbackslash{}nque{\ldots}\textbackslash{}n\textbackslash{}n— Ah! isso não prova nada. As mulheres, e\textbackslash{}nsobretudo aquela, nem sempre dizem o que sentem{\ldots}\textbackslash{}n\textbackslash{}n— Então acha que aquela sente alguma\textbackslash{}ncoisa por mim?{\ldots}\textbackslash{}n\textbackslash{}n— Se não fosse isso, não lhe falaria.\textbackslash{}n\textbackslash{}n— Ah! ora eis aí uma novidade.\textbackslash{}n\textbackslash{}n— Suspeito apenas. Ela só me fala do\textbackslash{}nsenhor; indaga-me vinte vezes por dia de sua pessoa, dos seus hábitos, do seu\textbackslash{}npassado e das suas opiniões{\ldots} Eu, como há de acreditar, respondo a tudo que\textbackslash{}nnão sei, mas vou criando um ódio ao senhor, do qual não me poderá jamais\textbackslash{}ncriminar.\textbackslash{}n\textbackslash{}n— É culpa minha se ela gosta de mim? Ora,\textbackslash{}nvá descansado, Sr. Diogo. Nem ela gosta de mim, nem eu gosto dela. Trabalhe\textbackslash{}ndesassombradamente e seja feliz.\textbackslash{}n\textbackslash{}n— Feliz! se eu pudesse ser! Mas não{\ldots}\textbackslash{}nnão creio; a felicidade não se fez para mim. Olhe, Sr. Tito, amo aquela mulher\textbackslash{}ncomo se pode amar a vida. Um olhar dela vale mais para mim que um ano de\textbackslash{}nglórias e de felicidade. É por ela que eu tenho deixado os meus negócios à toa.\textbackslash{}nNão viu outro dia que uma carta me chegou às mãos, cuja leitura me fez\textbackslash{}nentristecer? Perdi uma causa. Tudo por quê? por ela!\textbackslash{}n\textbackslash{}n— Mas, ela não lhe dá esperanças?\textbackslash{}n\textbackslash{}n— Eu sei o que é aquela moça! Ora\textbackslash{}ntrata-me de modo que eu vou ao sétimo céu; ora é tal a sua indiferença que me\textbackslash{}natira ao inferno. Hoje um sorriso, amanhã um gesto de desdém. Ralha-me de não\textbackslash{}nvisitá-la; vou visitá-la, ocupa-se tanto de mim como de Ganimedes; Ganimedes\textbackslash{}né o nome de um cãozinho felpudo que eu lhe dei. Importa-se tanto comigo como\textbackslash{}ncom o cachorro{\ldots} É de propósito. É um enigma aquela moça.\textbackslash{}n\textbackslash{}n— Pois não serei eu quem o decifre, Sr.\textbackslash{}nDiogo. Desejo-lhe muita felicidade. Adeus.\textbackslash{}n\textbackslash{}nE os dois separaram-se. Diogo seguiu para\textbackslash{}na casa de Emília, Tito para a casa de Azevedo.\textbackslash{}n\textbackslash{}nTito acabava de saber que a viúva pensava\textbackslash{}nnele; todavia, isso não lhe dera o menor abalo. Por quê? É o que saberemos mais\textbackslash{}nadiante. O que é preciso dizer desde já, é que as mesmas suspeitas despertadas\textbackslash{}nno espírito de Diogo, tivera a mulher de Azevedo. A intimidade de Emília dava lugar\textbackslash{}na uma franca interrogação e a uma confissão franca. Adelaide, no dia seguinte\textbackslash{}nàquele em que se passou a cena que referi acima, disse a Emília o que pensava.\textbackslash{}n\textbackslash{}nA resposta da viúva foi uma risada.\textbackslash{}n\textbackslash{}n— Não te compreendo, disse a mulher de\textbackslash{}nAzevedo.\textbackslash{}n\textbackslash{}n— É simples, disse a viúva. Julgas-me\textbackslash{}ncapaz de apaixonar-me pelo amigo de teu marido? Enganas-te. Não, eu não o amo.\textbackslash{}nSomente, como te disse no dia em que o vi aqui pela primeira vez, empenho-me em\textbackslash{}ntê-lo a meus pés. Se bem me recordo foste tu mesma quem me deu conselho.\textbackslash{}nAceitei-o. Hei de vingar o nosso sexo. É um pouco de vaidade minha, embora; mas\textbackslash{}neu creio que aquilo que nenhuma fez, fá-lo-ei eu.\textbackslash{}n\textbackslash{}n— Ah! cruelzinha! É isso?\textbackslash{}n\textbackslash{}n— Nem mais, nem menos.\textbackslash{}n\textbackslash{}n— Achas possível?\textbackslash{}n\textbackslash{}n— Por que não?\textbackslash{}n\textbackslash{}n— Reflete que a derrota será dupla{\ldots}\textbackslash{}n\textbackslash{}n— Será, mas não há de haver.\textbackslash{}n\textbackslash{}nEsta conversa foi interrompida por\textbackslash{}nAzevedo. Um sinal de Emília fez calar Adelaide. Ficou convencionado que nem\textbackslash{}nmesmo Azevedo saberia de coisa alguma. E, com efeito, Adelaide nada comunicou a\textbackslash{}nseu marido.\textbackslash{}n\textbackslash{}nCAPÍTULO III\textbackslash{}n\textbackslash{}nTinham-se passado oito dias depois do que\textbackslash{}nacabo de narrar.\textbackslash{}n\textbackslash{}nTito, como o temos visto até aqui, estava\textbackslash{}nno terreno do primeiro dia. Passeava, lia, conversava e parecia inteiramente\textbackslash{}nalheio aos planos que se tramavam em roda dele. Durante esse tempo foi apenas\textbackslash{}nduas vezes à casa de Emília, uma com a família de Azevedo, outra com Diogo.\textbackslash{}nNestas visitas era sempre o mesmo, frio, indiferente, impassível. Não havia\textbackslash{}nolhar, por mais sedutor e significativo, que o abalasse; nem a idéia de que\textbackslash{}nandava no pensamento da viúva era capaz de animá-lo.\textbackslash{}n\textbackslash{}n— Por que, ao menos, se não é capaz de\textbackslash{}namar, não procura entreter um desses namoros de sala, que tanto lisonjeiam a\textbackslash{}nvaidade dos homens?\textbackslash{}n\textbackslash{}nEsta pergunta era feita por Emília a si\textbackslash{}nmesma, sob a impressão da estranheza que lhe causava a indiferença do rapaz.\textbackslash{}nEla não compreendia que Tito pudesse conservar-se de gelo diante dos seus\textbackslash{}nencantos. Mas infelizmente era assim.\textbackslash{}n\textbackslash{}nCansada de trabalhar em vão, a viúva\textbackslash{}ndeterminou dar um golpe mais decisivo. Encaminhou a conversa para as doçuras do\textbackslash{}ncasamento e lamentou o estado de sua viuvez. O casal Azevedo era para ela o\textbackslash{}ntipo da perfeita felicidade conjugal. Apresentava-o aos olhos de Tito como um\textbackslash{}nincentivo para quem queria ser venturoso na terra. Nada, nem a tese, nem a\textbackslash{}nhipótese, nada moveu a frieza de Tito.\textbackslash{}n\textbackslash{}nEmília jogava um jogo perigoso. Era\textbackslash{}npreciso decidir entre os seus desejos de vingar o sexo e as conveniências da\textbackslash{}nsua posição; mas ela era de um caráter imperioso; respeitava muito os\textbackslash{}nprincípios de sua moral severa, mas não acatava do mesmo modo as conveniências\textbackslash{}nde que a sociedade cercava essa moral. A vaidade impunha no espírito dela, com\textbackslash{}nforça prodigiosa. Assim que a bela viúva foi usando todos os meios que era\textbackslash{}nlícito empregar para fazer apaixonar Tito.\textbackslash{}n\textbackslash{}nMas, apaixonado ele, o que faria ela? A\textbackslash{}npergunta é ociosa; desde que ela o tivesse aos pés, trataria de conservá-lo aí\textbackslash{}nfazendo parelha ao velho Diogo. Era o melhor troféu que uma beleza altiva pode\textbackslash{}nambicionar.\textbackslash{}n\textbackslash{}nUma manhã, oito dias depois das cenas\textbackslash{}nreferidas no capítulo anterior, apareceu Diogo em casa de Azevedo. Tinham aí\textbackslash{}nacabado de almoçar; Azevedo subira para o gabinete, a fim de aviar alguma\textbackslash{}ncorrespondência para a corte; Adelaide achava-se na sala do pavimento térreo.\textbackslash{}n\textbackslash{}nDiogo entrou com uma cara contristada,\textbackslash{}ncomo nunca se lhe vira. Adelaide correu para ele.\textbackslash{}n\textbackslash{}n— Que é isso? perguntou ela.\textbackslash{}n\textbackslash{}n— Ah! minha senhora{\ldots} sou o mais infeliz\textbackslash{}ndos homens!\textbackslash{}n\textbackslash{}n— Por quê? Venha sentar-se{\ldots}\textbackslash{}n\textbackslash{}nDiogo sentou-se, ou antes deixou-se cair\textbackslash{}nna cadeira que Adelaide lhe ofereceu. Esta tomou lugar ao pé dele, animou-o a\textbackslash{}ncontar as suas mágoas.\textbackslash{}n\textbackslash{}n— Então que há?\textbackslash{}n\textbackslash{}n— Duas desgraças, respondeu ele. A\textbackslash{}nprimeira em forma de sentença. Perdi mais uma demanda. É uma desgraça isto, mas\textbackslash{}nnão é nada{\ldots}\textbackslash{}n\textbackslash{}n— Pois há maior?{\ldots}\textbackslash{}n\textbackslash{}n— Há. A segunda desgraça foi em forma de\textbackslash{}ncarta.\textbackslash{}n\textbackslash{}n— De carta? perguntou Adelaide.\textbackslash{}n\textbackslash{}n— De carta. Veja isto.\textbackslash{}n\textbackslash{}nDiogo tirou da carteira uma cartinha\textbackslash{}ncor-de-rosa, cheirando à essência de magnólia.\textbackslash{}n\textbackslash{}nAdelaide leu a carta para si.\textbackslash{}n\textbackslash{}nQuando ela acabou, perguntou-lhe o velho:\textbackslash{}n\textbackslash{}n— Que me diz a isto?\textbackslash{}n\textbackslash{}n— Não compreendo, respondeu Adelaide.\textbackslash{}n\textbackslash{}n— Esta carta é dela.\textbackslash{}n\textbackslash{}n— Sim, e depois?\textbackslash{}n\textbackslash{}n— É para ele.\textbackslash{}n\textbackslash{}n— Ele quem?\textbackslash{}n\textbackslash{}n— Ele! o diabo! o meu rival! o Tito!\textbackslash{}n\textbackslash{}n— Ah!\textbackslash{}n\textbackslash{}n— Dizer-lhe o que senti quando apanhei\textbackslash{}nesta carta, é impossível. Nunca tremi na minha vida! Mas quando li isto, não\textbackslash{}nsei que vertigem se apoderou de mim. Ando tonto! A cada passo como que\textbackslash{}ndesmaio{\ldots} Ah!\textbackslash{}n\textbackslash{}n— Ânimo! disse Adelaide.\textbackslash{}n\textbackslash{}n— É isto mesmo que eu vinha buscar{\ldots} é\textbackslash{}numa consolação, uma animação. Soube que estava aqui e estimei achá-la só{\ldots} Ah!\textbackslash{}nquanto sinto que o estimável seu marido esteja vivo{\ldots} porque a melhor\textbackslash{}nconsolação era aceitar Vossa Excelência um coração tão mal compreendido.\textbackslash{}n\textbackslash{}n— Felizmente ele está vivo.\textbackslash{}n\textbackslash{}nDiogo soltou um suspiro e disse:\textbackslash{}n\textbackslash{}n— Felizmente!\textbackslash{}n\textbackslash{}nE depois de um silêncio continuou:\textbackslash{}n\textbackslash{}n— Tive duas idéias: uma foi o desprezo;\textbackslash{}nmas desprezá-los é pô-los em maior liberdade e ralar-me de dor e de vergonha; a\textbackslash{}nsegunda foi o duelo{\ldots} é melhor{\ldots} eu mato{\ldots} ou{\ldots}\textbackslash{}n\textbackslash{}n— Deixe-se disso.\textbackslash{}n\textbackslash{}n— É indispensável que um de nós seja\textbackslash{}nriscado do número dos vivos.\textbackslash{}n\textbackslash{}n— Pode ser engano{\ldots}\textbackslash{}n\textbackslash{}n— Mas não é engano, é certeza.\textbackslash{}n\textbackslash{}n— Certeza de quê?\textbackslash{}n\textbackslash{}nDiogo abriu o bilhete e disse:\textbackslash{}n\textbackslash{}n— Ora, ouça:\textbackslash{}n\textbackslash{}nSe ainda não me compreendeu é bem curto\textbackslash{}nde penetração. Tire a máscara e eu me explicarei. Esta noite tomo chá sozinha.\textbackslash{}nO importuno Diogo não me incomodará com as suas tolices. Dê-me a felicidade de\textbackslash{}nvê-lo e admirá-lo.\textbackslash{}n\textbackslash{}nEMÍLIA\textbackslash{}n\textbackslash{}n— Mas que é isto?\textbackslash{}n\textbackslash{}n— Que é isto? Ah! se fosse mais do que\textbackslash{}nisto já eu estava morto! Pude pilhar a carta, e a tal entrevista não se deu{\ldots}\textbackslash{}n\textbackslash{}n— Quando foi escrita a carta?\textbackslash{}n\textbackslash{}n— Ontem.\textbackslash{}n\textbackslash{}n— Tranqüilize-se. É capaz de guardar um\textbackslash{}nsegredo? O que lhe vou dizer é grave. Mas só a sua aflição me faz falar. Posso\textbackslash{}nafirmar-lhe que esta carta é uma pura caçoada. Trata-se de vingar o nosso sexo\textbackslash{}nultrajado; trata-se de fazer com que Tito se apaixone{\ldots} nada mais.\textbackslash{}n\textbackslash{}nDiogo estremeceu de alegria.\textbackslash{}n\textbackslash{}n— Sim? perguntou ele.\textbackslash{}n\textbackslash{}n— É pura verdade. Mas veja lá, isto é\textbackslash{}nsegredo. Se lho descobri foi por vê-lo aflito. Não nos comprometa.\textbackslash{}n\textbackslash{}n— Isso é sério? insistiu Diogo.\textbackslash{}n\textbackslash{}n— Como quer que lho diga?\textbackslash{}n\textbackslash{}n— Ah! que peso me tirou! Pode estar certa\textbackslash{}nde que o segredo caiu num poço. Oh! muito me hei de rir{\ldots} muito me hei de\textbackslash{}nrir{\ldots} Que boa inspiração tive em vir falar-lhe! Diga-me, posso dizer a D.\textbackslash{}nEmília que sei tudo?\textbackslash{}n\textbackslash{}n— Não!\textbackslash{}n\textbackslash{}n— É então melhor que não me dê por\textbackslash{}nachado{\ldots}\textbackslash{}n\textbackslash{}n— Sim.\textbackslash{}n\textbackslash{}n— Muito bem!\textbackslash{}n\textbackslash{}nDizendo estas palavras o velho Diogo\textbackslash{}nesfregava as mãos e piscava os olhos. Estava radiante. Quê! ver o suposto rival\textbackslash{}nsendo vítima dos laços da viúva! Que glória! que felicidade!\textbackslash{}n\textbackslash{}nNisto estava quando à porta do interior\textbackslash{}napareceu Tito. Acabava de levantar-se da cama.\textbackslash{}n\textbackslash{}n— Bom dia, D. Adelaide, disse ele dirigindo-se\textbackslash{}npara a mulher de Azevedo.\textbackslash{}n\textbackslash{}nDepois sentando-se e voltando a cara para\textbackslash{}nDiogo:\textbackslash{}n\textbackslash{}n— Bom dia, disse. Está hoje alegre{\ldots}\textbackslash{}nTirou a sorte grande?\textbackslash{}n\textbackslash{}n— A sorte grande? perguntou Diogo.\textbackslash{}nTirei{\ldots} tirei{\ldots}\textbackslash{}n\textbackslash{}n— Dormiu bem? perguntou Adelaide a Tito.\textbackslash{}n\textbackslash{}n— Como um justo que sou. Tive sonhos\textbackslash{}ncor-de-rosa: sonhei com o Sr. Diogo.\textbackslash{}n\textbackslash{}n— Ah! sonhou comigo? murmurou entre\textbackslash{}ndentes o velho namorado. Coitado! tenho pena dele!\textbackslash{}n\textbackslash{}n— Mas onde está Azevedo? perguntou Tito a\textbackslash{}nAdelaide.\textbackslash{}n\textbackslash{}n— Anda de passeio.\textbackslash{}n\textbackslash{}n— Já?\textbackslash{}n\textbackslash{}n— Pois então. Onze horas.\textbackslash{}n\textbackslash{}n— Onze horas! É verdade, acordei muito\textbackslash{}ntarde. Tinha duas visitas para fazer: uma a D. Emília{\ldots}\textbackslash{}n\textbackslash{}n— Ah! disse Diogo.\textbackslash{}n\textbackslash{}n— De que se espanta, meu caro?\textbackslash{}n\textbackslash{}n— De nada! de nada!\textbackslash{}n\textbackslash{}n— Bom; vou mandar pôr o seu almoço, disse\textbackslash{}nAdelaide.\textbackslash{}n\textbackslash{}nOs dois ficaram sós. Tito acendeu um\textbackslash{}ncigarro de palha; Diogo afetava grande distração, mas olhava sorrateiramente\textbackslash{}npara o moço. Este, apenas soltou duas fumaças, voltou-se para o velho e disse:\textbackslash{}n\textbackslash{}n— Como vão os seus amores?\textbackslash{}n\textbackslash{}n— Que amores?\textbackslash{}n\textbackslash{}n— Os seus, a Emília{\ldots} Já lhe fez\textbackslash{}ncompreender toda a imensidade da paixão que o devora?\textbackslash{}n\textbackslash{}n— Qual{\ldots} Preciso de algumas lições{\ldots} Se\textbackslash{}nmas quisesse dar?\textbackslash{}n\textbackslash{}n— Eu? Está sonhando!\textbackslash{}n\textbackslash{}n— Ah! eu sei que o senhor é forte{\ldots} É\textbackslash{}nmodesto, mas é forte{\ldots} e até fortíssimo! Ora, eu sou realmente um aprendiz{\ldots}\textbackslash{}nTive há pouco a idéia de desafiá-lo.\textbackslash{}n\textbackslash{}n— A mim?\textbackslash{}n\textbackslash{}n— É verdade, mas foi uma loucura de que\textbackslash{}nme arrependi{\ldots}\textbackslash{}n\textbackslash{}n— Além de que não é uso em nosso país{\ldots}\textbackslash{}n\textbackslash{}n— Em toda a parte é uso vingar a honra.\textbackslash{}n\textbackslash{}n— Bravo, D. Quixote!\textbackslash{}n\textbackslash{}n— Ora, eu acreditava-me ofendido na\textbackslash{}nhonra.\textbackslash{}n\textbackslash{}n— Por mim?\textbackslash{}n\textbackslash{}n— Mas emendei a mão; reparei que era\textbackslash{}nantes eu quem ofendia pretendendo lutar com um mestre, eu simples aprendiz?{\ldots}\textbackslash{}n\textbackslash{}n— Mestre de quê?\textbackslash{}n\textbackslash{}n— Dos amores! Oh! eu sei que é mestre{\ldots}\textbackslash{}n\textbackslash{}n— Deixe-se disso{\ldots} eu não sou nada{\ldots} o\textbackslash{}nSr. Diogo, sim; o senhor vale um urso, vale mesmo dois. Como havia de eu{\ldots}\textbackslash{}nOra!{\ldots} Aposto que teve ciúmes?\textbackslash{}n\textbackslash{}n— Exatamente.\textbackslash{}n\textbackslash{}n— Mas era preciso não me conhecer; não\textbackslash{}nsabe das minhas idéias?\textbackslash{}n\textbackslash{}n— Homem, às vezes é pior.\textbackslash{}n\textbackslash{}n— Pior, como?\textbackslash{}n\textbackslash{}n— As mulheres não deixam uma afronta sem\textbackslash{}ncastigo{\ldots} As suas idéias são afrontosas{\ldots} Qual será o castigo? Paro aqui{\ldots}\textbackslash{}nparo aqui{\ldots}\textbackslash{}n\textbackslash{}n— Onde vai?\textbackslash{}n\textbackslash{}n— Vou sair. Adeus. Não se lembre mais da\textbackslash{}nminha desastrada idéia do duelo{\ldots}\textbackslash{}n\textbackslash{}n— Que está acabado{\ldots} Ah! o senhor\textbackslash{}nescapou de boa!\textbackslash{}n\textbackslash{}n— De quê?\textbackslash{}n\textbackslash{}n— De morrer. Eu enfiava-lhe a espada por\textbackslash{}nesse abdômen{\ldots} com um gosto{\ldots} com um gosto só comparável ao que tenho de\textbackslash{}nabraçá-lo vivo e são!\textbackslash{}n\textbackslash{}nDiogo riu-se com um riso amarelo.\textbackslash{}n\textbackslash{}n— Obrigado, obrigado. Até logo!\textbackslash{}n\textbackslash{}n— Venha cá, onde vai? Não se despede de\textbackslash{}nD. Adelaide?\textbackslash{}n\textbackslash{}n— Eu já volto, disse Diogo travando do\textbackslash{}nchapéu e saindo precipitadamente.\textbackslash{}n\textbackslash{}nTito ainda o acompanhou com os olhos.\textbackslash{}n\textbackslash{}n'Este sujeito', disse o moço\textbackslash{}nconsigo quando se viu só, 'não tem nada de original. Aquela opinião a\textbackslash{}nrespeito das mulheres não é dele{\ldots} Melhor{\ldots} já se conspira; é o que me\textbackslash{}nconvém. Hás de vir! hás de vir!”\textbackslash{}n\textbackslash{}nUm criado alemão veio anunciar a Tito que\textbackslash{}no almoço estava preparado. Tito ia entrando quando assomou à porta a figura de\textbackslash{}nAzevedo.\textbackslash{}n\textbackslash{}n— Ora, graças a Deus! O meu amigo não se\textbackslash{}nlevanta com o sol. Estás com olhos de quem acaba de dormir.\textbackslash{}n\textbackslash{}n— É verdade, e vou almoçar.\textbackslash{}n\textbackslash{}nDirigiram-se os dois para dentro, onde a\textbackslash{}nmesa estava posta à espera de Tito.\textbackslash{}n\textbackslash{}n— Almoças outra vez? perguntou Tito.\textbackslash{}n\textbackslash{}n— Não.\textbackslash{}n\textbackslash{}n— Pois então vais ver como se come.\textbackslash{}n\textbackslash{}nTito sentou-se à mesa; Azevedo estirou-se\textbackslash{}nnum sofá.\textbackslash{}n\textbackslash{}n— Onde foste? perguntou Tito.\textbackslash{}n\textbackslash{}n— Fui passear{\ldots} Compreendi que é preciso\textbackslash{}nver e admirar o que é indiferente, para apreciar e ver aquilo que faz a\textbackslash{}nfelicidade íntima do coração.\textbackslash{}n\textbackslash{}n— Ah! sim? Bem vês que até a felicidade\textbackslash{}npor igual fatiga! Afinal sempre a razão do meu lado.\textbackslash{}n\textbackslash{}n— Talvez. Apesar de tudo, quer-me parecer\textbackslash{}nque já intentas entrar na família dos casados.\textbackslash{}n\textbackslash{}n— Eu?\textbackslash{}n\textbackslash{}n— Tu, sim.\textbackslash{}n\textbackslash{}n— Por quê?\textbackslash{}n\textbackslash{}n— Mas, dize, é ou não verdade?\textbackslash{}n\textbackslash{}n— Qual, verdade!\textbackslash{}n\textbackslash{}n— O que sei, é que uma destas tardes em\textbackslash{}nque adormeceste lendo, não sei que livro, ouvi-te pronunciar em sonhos, com a\textbackslash{}nmaior ternura, o nome de Emília.\textbackslash{}n\textbackslash{}n— Deveras? perguntou Tito mastigando.\textbackslash{}n\textbackslash{}n— É exato. Concluí que se sonhavas com\textbackslash{}nela é que a tinhas no pensamento, e se a tinhas no pensamento é que a amavas.\textbackslash{}n\textbackslash{}n— Concluíste mal.\textbackslash{}n\textbackslash{}n— Mal?\textbackslash{}n\textbackslash{}n— Concluíste como um marido de cinco\textbackslash{}nmeses. Que prova um sonho? Não prova nada! Pareces velha supersticiosa{\ldots}\textbackslash{}n\textbackslash{}n— Mas enfim, alguma coisa há por força{\ldots}\textbackslash{}nSerás capaz de me dizeres o que é?\textbackslash{}n\textbackslash{}n— Homem, podia dizer-te alguma coisa se\textbackslash{}nnão fosses casado{\ldots}\textbackslash{}n\textbackslash{}n— Que tem que eu seja casado?\textbackslash{}n\textbackslash{}n— Tem tudo. Seria indiscreto sem querer e\textbackslash{}naté sem saber. À noite, entre um beijo e um bocejo, o marido e a mulher abrem\textbackslash{}num para o outro a bolsa das confidências. Sem pensares, podes deitar tudo a\textbackslash{}nperder.\textbackslash{}n\textbackslash{}n— Não digas isso. Vamos lá. Há novidade?\textbackslash{}n\textbackslash{}n— Não há nada.\textbackslash{}n\textbackslash{}n— Confirmas as minhas suspeitas. Gostas\textbackslash{}nda Emília.\textbackslash{}n\textbackslash{}n— Ódio não lhe tenho, é verdade.\textbackslash{}n\textbackslash{}n— Gostas. E ela merece. É uma boa\textbackslash{}nsenhora, de não vulgar beleza, possuindo as melhores qualidades. Talvez\textbackslash{}npreferisses que não fosse viúva?{\ldots}\textbackslash{}n\textbackslash{}n— Sim; é natural que se embale dez vezes\textbackslash{}npor dia na lembrança dos dois maridos que já exportou para o outro mundo{\ldots} à\textbackslash{}nespera de exportar o terceiro{\ldots}\textbackslash{}n\textbackslash{}n— Não é dessas{\ldots}\textbackslash{}n\textbackslash{}n— Afianças?\textbackslash{}n\textbackslash{}n— Quase que posso afiançar.\textbackslash{}n\textbackslash{}n— Ah! meu amigo, disse Tito levantando-se\textbackslash{}nda mesa e indo acender um charuto, toma o conselho de um tolo: nunca afiances\textbackslash{}nnada, principalmente em tais assuntos. Entre a prudência discreta e a cega\textbackslash{}nconfiança não é lícito duvidar, a escolha está decidida nos próprios termos da\textbackslash{}nprimeira. O que podes tu afiançar a respeito de Emília? Não a conheces melhor\textbackslash{}ndo que eu.\textbackslash{}n\textbackslash{}nHá quinze dias que nos conhecemos, e eu\textbackslash{}njá lhe leio no interior; estou longe de atribuir-lhe maus sentimentos, mas\textbackslash{}ntenho a certeza de que não possui as raríssimas qualidades que são necessárias\textbackslash{}nà exceção. Que sabes tu?\textbackslash{}n\textbackslash{}n— Realmente, eu não sei nada.\textbackslash{}n\textbackslash{}n'Não sabes nada!' disse Tito\textbackslash{}nconsigo.\textbackslash{}n\textbackslash{}n— Falo pelas minhas impressões.\textbackslash{}nParecia-me que um casamento entre vocês ambos não vinha fora de propósito.\textbackslash{}n\textbackslash{}n— Se me falas outra vez em casamento,\textbackslash{}nsaio.\textbackslash{}n\textbackslash{}n— Pois só a palavra?\textbackslash{}n\textbackslash{}n— A palavra, a idéia, tudo.\textbackslash{}n\textbackslash{}n— Entretanto, admiras e aplaudes o meu\textbackslash{}ncasamento{\ldots}\textbackslash{}n\textbackslash{}n— Ah! eu aplaudo nos outros muitas coisas\textbackslash{}nde que não sou capaz de usar. Depende da vocação{\ldots}\textbackslash{}n\textbackslash{}nAdelaide apareceu à porta da sala de\textbackslash{}njantar. A conversa cessou entre os dois rapazes.\textbackslash{}n\textbackslash{}n— Trago-lhe uma notícia.\textbackslash{}n\textbackslash{}n— Que notícia? perguntaram-lhe os dois.\textbackslash{}n\textbackslash{}n— Recebi um bilhete de Emília{\ldots} Pede-nos\textbackslash{}nque vamos lá amanhã, porque{\ldots}\textbackslash{}n\textbackslash{}n— Por quê? perguntou Azevedo.\textbackslash{}n\textbackslash{}n— Talvez dentro de oito dias se retire\textbackslash{}npara a cidade.\textbackslash{}n\textbackslash{}n— Ah! disse Tito com a maior indiferença\textbackslash{}ndeste mundo.\textbackslash{}n\textbackslash{}n— Apronta as tuas malas, disse Azevedo a\textbackslash{}nTito.\textbackslash{}n\textbackslash{}n— Por quê?\textbackslash{}n\textbackslash{}n— Não segues os passos da deusa?\textbackslash{}n\textbackslash{}n— Não zombes, cruel amigo! Quando não{\ldots}\textbackslash{}n\textbackslash{}n— Anda lá{\ldots}\textbackslash{}n\textbackslash{}nAdelaide sorriu ouvindo estas palavras.\textbackslash{}n\textbackslash{}nDaí a meia hora Tito subiu para o\textbackslash{}ngabinete em que Azevedo tinha os livros. Ia, dizia, ler as Confissões de\textbackslash{}nSanto Agostinho.\textbackslash{}n\textbackslash{}n— Que repentina viagem é esta? perguntou\textbackslash{}nAzevedo à sua mulher.\textbackslash{}n\textbackslash{}n— Tens muito empenho em saber?\textbackslash{}n\textbackslash{}n— Tenho.\textbackslash{}n\textbackslash{}n— Pois bem. Olha que é segredo. Eu não\textbackslash{}nsei positivamente, mas creio que é uma estratégia.\textbackslash{}n\textbackslash{}n— Estratégia? Não entendo.\textbackslash{}n\textbackslash{}n— Eu te digo. Trata-se de prender o Tito.\textbackslash{}n\textbackslash{}n— Prender?\textbackslash{}n\textbackslash{}n— Estás hoje tão bronco! Prender pelos\textbackslash{}nlaços do amor{\ldots}\textbackslash{}n\textbackslash{}n— Ah!\textbackslash{}n\textbackslash{}n— Emília julgou que deve fazê-lo. É só\textbackslash{}npara brincar. No dia em que ele se declarar vencido fica ela vingada do que ele\textbackslash{}ndisse contra o sexo.\textbackslash{}n\textbackslash{}n— Não está mau{\ldots} E tu entras nesta\textbackslash{}nestratégia{\ldots}\textbackslash{}n\textbackslash{}n— Como conselheira.\textbackslash{}n\textbackslash{}n— Trama-se então contra um amigo, um alter\textbackslash{}nego.\textbackslash{}n\textbackslash{}n— Tá, tá, tá. Cala a boca. Não vás fazer\textbackslash{}nabortar o plano.\textbackslash{}n\textbackslash{}nAzevedo riu-se a bandeiras despregadas.\textbackslash{}nNo fundo achava engraçada a punição premeditada ao pobre Tito.\textbackslash{}n\textbackslash{}nA visita que Tito disse ter de fazer à viúva\textbackslash{}nnaquele dia, não se realizou.\textbackslash{}n\textbackslash{}nDiogo, que apenas saíra da casa de\textbackslash{}nAzevedo, ciente das intenções da viúva, fora para casa desta esperar o rapaz,\textbackslash{}nembalde lá esteve durante o dia, embalde jantou, embalde aborreceu a tarde\textbackslash{}ninteira tanto a Emília como à tia; Tito não apareceu.\textbackslash{}n\textbackslash{}nMas, à noite, à hora em que Diogo, já\textbackslash{}nvexado de tanta demora na casa da moça, tratava de sair, anunciou-se a chegada\textbackslash{}nde Tito.\textbackslash{}n\textbackslash{}nEmília estremeceu; mas esse movimento\textbackslash{}nescapou a Diogo.\textbackslash{}n\textbackslash{}nTito entrou na sala onde se achavam\textbackslash{}nEmília, a tia, e Diogo.\textbackslash{}n\textbackslash{}n— Não contava com a sua visita, disse a\textbackslash{}nviúva.\textbackslash{}n\textbackslash{}n— Eu sou assim; apareço quando não me\textbackslash{}nesperam. Sou como a morte e a sorte grande.\textbackslash{}n\textbackslash{}n— Agora é a sorte grande, disse Emília.\textbackslash{}n\textbackslash{}n— Que número é o seu bilhete, minha\textbackslash{}nsenhora?\textbackslash{}n\textbackslash{}n— Número doze, isto é, doze horas que\textbackslash{}ntenho tido o prazer de ter hoje aqui o Sr. Diogo{\ldots}\textbackslash{}n\textbackslash{}n— Doze horas! exclamou Tito voltando-se\textbackslash{}npara o velho.\textbackslash{}n\textbackslash{}n— Sem que ainda o nosso bom amigo nos\textbackslash{}ncontasse uma história{\ldots}\textbackslash{}n\textbackslash{}n— Doze horas! repetiu Tito.\textbackslash{}n\textbackslash{}n— Que admira, meu caro senhor? perguntou\textbackslash{}nDiogo.\textbackslash{}n\textbackslash{}n— Acho um pouco estirado{\ldots}\textbackslash{}n\textbackslash{}n— As horas contam-se quando são\textbackslash{}naborrecidas{\ldots} Peço para me retirar{\ldots}\textbackslash{}n\textbackslash{}nE dizendo isto, Diogo travou do chapéu\textbackslash{}npara sair lançando um olhar de despeito e ciúme para a viúva.\textbackslash{}n\textbackslash{}n— Que é isso? perguntou esta. Onde vai?\textbackslash{}n\textbackslash{}n— Dou asas às horas, respondeu Diogo ao\textbackslash{}nouvido de Emília; vão correr depressa agora.\textbackslash{}n\textbackslash{}n— Perdôo-lhe e peço que se sente.\textbackslash{}n\textbackslash{}nDiogo sentou-se.\textbackslash{}n\textbackslash{}nA tia de Emília pediu licença para\textbackslash{}nretirar-se alguns minutos.\textbackslash{}n\textbackslash{}nFicaram os três.\textbackslash{}n\textbackslash{}n— Mas então, disse Tito, nem ao menos uma\textbackslash{}nhistória contou?\textbackslash{}n\textbackslash{}n— Nenhuma.\textbackslash{}n\textbackslash{}nEmília lançou um olhar a Diogo como para\textbackslash{}ntranqüilizá-lo. Este, mais calmo então, lembrou-se do que Adelaide lhe havia\textbackslash{}ndito, e voltou às boas.\textbackslash{}n\textbackslash{}n— Afinal de contas, disse ele consigo, o\textbackslash{}ncaçoado é ele. Eu sou apenas o meio de prendê-lo{\ldots} Contribuamos para que se\textbackslash{}nlhe tire a proa.\textbackslash{}n\textbackslash{}n— Nenhuma história, continuou Emília.\textbackslash{}n\textbackslash{}n— Pois olhe, eu sei muitas, disse Diogo\textbackslash{}ncom intenção.\textbackslash{}n\textbackslash{}n— Conte uma de tantas que sabe, disse\textbackslash{}nTito.\textbackslash{}n\textbackslash{}n— Nada! Por que não conta o senhor?\textbackslash{}n\textbackslash{}n— Se faz empenho{\ldots}\textbackslash{}n\textbackslash{}n— Muito{\ldots} muito, disse Diogo piscando os\textbackslash{}nolhos. Conte lá, por exemplo, a história do taboqueado, a história das\textbackslash{}nimposturas do amor, a história dos viajantes encouraçados; vá, vá.\textbackslash{}n\textbackslash{}n— Não, vou contar a história de um homem\textbackslash{}ne de um macaco.\textbackslash{}n\textbackslash{}n— Oh! disse a viúva.\textbackslash{}n\textbackslash{}n— É muito interessante, disse Tito. Ora,\textbackslash{}nouçam{\ldots}\textbackslash{}n\textbackslash{}n— Perdão, interrompeu Emília, será depois\textbackslash{}ndo chá.\textbackslash{}n\textbackslash{}n— Pois sim.\textbackslash{}n\textbackslash{}nDaí a pouco servia-se o chá aos três.\textbackslash{}nFindo ele, Tito tomou a palavra e começou a história:\textbackslash{}n\textbackslash{}nHISTÓRIA DE UM HOMEM E DE UM MACACO\textbackslash{}n\textbackslash{}nNão longe da vila ***, no interior do\textbackslash{}nBrasil, morava há uns vinte anos um homem de trinta e cinco anos, cuja vida\textbackslash{}nmisteriosa era o objeto das conversas das vilas próximas e o objeto do terror\textbackslash{}nque experimentavam os viajantes que passavam na estrada a dois passos da casa.\textbackslash{}n\textbackslash{}nA própria casa era já de causar\textbackslash{}napreensões ao espírito menos timorato. Vista de longe nem parecia casa, tão\textbackslash{}nbaixinha era. Mas quem se aproximasse conheceria aquela construção singular.\textbackslash{}nMetade do edifício estava ao nível do chão e metade abaixo da terra. Era\textbackslash{}nentretanto uma casa solidamente construída. Não tinha porta nem janelas. Tinha\textbackslash{}num vão quadrado que servia ao mesmo tempo de janela e de porta. Era por ali que\textbackslash{}no misterioso morador entrava e saía.\textbackslash{}n\textbackslash{}nPouca gente o via sair, não só porque ele\textbackslash{}nraras vezes o fazia, como porque o fazia em horas impróprias. Era nas horas da\textbackslash{}nlua cheia que o solitário deixava a residência para ir passear nos arredores.\textbackslash{}nLevava sempre consigo um grande macaco, que acudia pelo nome de Calígula.\textbackslash{}n\textbackslash{}nO macaco e o homem, o homem e o macaco\textbackslash{}neram dois amigos inseparáveis, dentro e fora de casa, na lua nova.\textbackslash{}n\textbackslash{}nMil versões corriam a respeito deste\textbackslash{}nmisterioso solitário.\textbackslash{}n\textbackslash{}nA mais geral é que era um feiticeiro.\textbackslash{}nHavia uma que o dava por doido; outra por simplesmente atacado de misantropia.\textbackslash{}n\textbackslash{}nEsta última versão tinha por si duas\textbackslash{}ncircunstâncias: a primeira era não constar nada de positivo que fizesse\textbackslash{}nreconhecer no homem hábitos de feiticeiro ou alienado; a segunda era a amizade\textbackslash{}nque ele parecia votar ao macaco e o horror com que fugia ao olhar dos homens.\textbackslash{}nQuando a gente se aborrece dos homens toma sempre a afeição dos animais, que\textbackslash{}ntêm a vantagem de não discorrer, nem intrigar.\textbackslash{}n\textbackslash{}nO misterioso{\ldots} É preciso dar-lhe um\textbackslash{}nnome: chamemo-lo Daniel. Daniel preferia o macaco, e não falava a mais homem\textbackslash{}nalgum. Algumas vezes os viajantes que passavam pela estrada ouviam partir de\textbackslash{}ndentro da casa gritos do macaco e do homem; era o homem que afagava o macaco.\textbackslash{}n\textbackslash{}nComo se alimentavam aquelas duas\textbackslash{}ncriaturas? Houve quem visse um dia de manhã abrir-se a porta, sair o macaco e\textbackslash{}nvoltar pouco depois com um embrulho na boca. O tropeiro que presenciava esta\textbackslash{}ncena quis descobrir onde ia o macaco buscar aquele embrulho que levava sem\textbackslash{}ndúvida os alimentos dos dois solitários. Na manhã seguinte introduziu-se no\textbackslash{}nmato; o macaco chegou à hora do costume, e dirigiu-se para um tronco de árvore;\textbackslash{}nhavia sobre esse tronco um grande galho, que o bicho atirou ao chão. Depois,\textbackslash{}nintroduzindo as mãos no interior do velho tronco, tirou um embrulho igual ao da\textbackslash{}nvéspera e partiu.\textbackslash{}n\textbackslash{}nO tropeiro persignou-se, e tão apreensivo\textbackslash{}nficou com a cena que acabava de presenciar que não a contou a ninguém.\textbackslash{}n\textbackslash{}nDurava esta existência três anos.\textbackslash{}n\textbackslash{}nDurante esse tempo o homem não\textbackslash{}nenvelhecera. Era o mesmo que no primeiro dia. Longas barbas ruivas e cabelos\textbackslash{}ngrandes caídos para trás. Usava um grande casaco de baeta, tanto no inverno,\textbackslash{}ncomo no verão. Calçava botas e não usava chapéu.\textbackslash{}n\textbackslash{}nEra impossível aos passageiros e aos\textbackslash{}nmoradores das vizinhanças penetrar na casa do solitário. Não o será decerto\textbackslash{}npara nós, minha bela senhora, e meu caro amigo.\textbackslash{}n\textbackslash{}nA casa divide-se em duas salas e um\textbackslash{}nquarto. Uma sala é para jantar; a outra é{\ldots} a de visitas. O quarto é ocupado\textbackslash{}npelos dois moradores, Daniel e Calígula.\textbackslash{}n\textbackslash{}nAs duas salas são de iguais dimensões; o\textbackslash{}nquarto é uma metade da sala. A mobília da primeira sala compõe-se de dois sujos\textbackslash{}nbancos encostados à parede, uma mesa baixa no centro. O chão é assoalhado.\textbackslash{}nPendem das paredes dois retratos: um de moça, outro de velho. A moça é uma\textbackslash{}nfigura angélica e deliciosa. O velho inspirava respeito e admiração. Das outras\textbackslash{}nduas paredes pendem, de um lado uma faca de cabo de marfim, e do outro uma mão\textbackslash{}nde defunto, amarela e seca.\textbackslash{}n\textbackslash{}nA sala de jantar tem apenas uma mesa e\textbackslash{}ndois bancos.\textbackslash{}n\textbackslash{}nA mobília do quarto resume-se num grabato\textbackslash{}nem que dorme Daniel. Calígula estende-se no chão, junto à cabeceira do\textbackslash{}ndono.\textbackslash{}n\textbackslash{}nTal é a mobília da casa.\textbackslash{}n\textbackslash{}nA casa, que de fora parece não ter\textbackslash{}ncapacidade suficiente para conter um homem em pé, é contudo suficiente, visto\textbackslash{}nestar, como disse, entranhada no chão.\textbackslash{}n\textbackslash{}nQue vida terão passado aí dentro o macaco\textbackslash{}ne o homem, no espaço de três anos? Não saberei dizê-lo.\textbackslash{}n\textbackslash{}nQuando Calígula traz de manhã o\textbackslash{}nembrulho, Daniel divide a comida em duas porções, uma para o almoço, outra para\textbackslash{}no jantar. Depois homem e macaco sentam-se em face um do outro na sala de jantar\textbackslash{}ne comem irmãmente as duas refeições.\textbackslash{}n\textbackslash{}nQuando chega a lua cheia saem os dois\textbackslash{}nsolitários, como já disse, todas as noites, até a época em que a lua passa a\textbackslash{}nser minguante. Saem às dez horas, pouco mais ou menos, e voltam pouco mais ou\textbackslash{}nmenos às duas horas da madrugada. Quando entram, Daniel tira a mão do finado\textbackslash{}nque pende da parede e dá com ela duas bofetadas em si próprio. Feito isto, vai\textbackslash{}ndeitar-se; Calígula acompanha-o.\textbackslash{}n\textbackslash{}nUma noite, era no mês de junho, época de\textbackslash{}nlua cheia, Daniel preparou-se para sair. Calígula deu um pulo e saltou à\textbackslash{}nestrada. Daniel fechou a porta, e lá se foi com o macaco estrada acima.\textbackslash{}n\textbackslash{}nA lua, inteiramente cheia, projetava os\textbackslash{}nseus reflexos pálidos e melancólicos na vasta floresta que cobria as colinas\textbackslash{}npróximas, e clareava toda a vasta campina que rodeava a casa.\textbackslash{}n\textbackslash{}nSó se ouvia ao longe o murmúrio de uma\textbackslash{}ncachoeira, e ao perto o piar de algumas corujas, e o chilrar de uma infinidade\textbackslash{}nde grilos espalhados na planície.\textbackslash{}n\textbackslash{}nDaniel caminhava pausadamente levando um\textbackslash{}npau debaixo do braço, e acompanhado do macaco, que saltava do chão aos ombros\textbackslash{}nde Daniel e dos ombros de Daniel para o chão.\textbackslash{}n\textbackslash{}nMesmo sem a forma lúgubre que tinha\textbackslash{}naquele lugar por causa da residência do solitário, qualquer pessoa que\textbackslash{}nencontrasse àquela hora Daniel e o macaco corria risco de morrer de medo.\textbackslash{}nDaniel, extremamente magro e alto, tinha em si um ar lúgubre. Os cabelos da\textbackslash{}nbarba e da cabeça, crescidos em abundância, faziam a sua cabeça ainda maior do\textbackslash{}nque era. Sem chapéu era uma cabeça verdadeiramente satânica.\textbackslash{}n\textbackslash{}nCalígula, que nos outros\textbackslash{}ndias era um macaco ordinário, tomava, naquelas horas de passeio noturno, um ar\textbackslash{}ntão lúgubre e tão misterioso como o de Daniel.\textbackslash{}n\textbackslash{}nHavia já uma hora que os dois solitários\textbackslash{}ntinham saído de casa. A casa ficara já um pouco longe. Nada mais natural do que\textbackslash{}nchegar a polícia nessa ocasião, tomar a entrada da casa e reconhecer o\textbackslash{}nmistério. Mas a polícia, apesar dos meios que tinha à sua disposição, não se\textbackslash{}nanimava a investigar no mistério que o povo reputava diabólico. Também a\textbackslash{}npolícia é humana, e nada do que é humano lhe é desconhecido.\textbackslash{}n\textbackslash{}nHavia uma hora, disse eu, que os dois\textbackslash{}npasseadores tinham saído de casa. Começavam então a subir uma pequena colina{\ldots}\textbackslash{}n\textbackslash{}nTito foi interrompido por um bocejo do\textbackslash{}nvelho Diogo.\textbackslash{}n\textbackslash{}n— Quer dormir? perguntou o rapaz.\textbackslash{}n\textbackslash{}n— É o que vou fazer.\textbackslash{}n\textbackslash{}n— Mas a história?\textbackslash{}n\textbackslash{}n— A história é muito divertida. Até aqui\textbackslash{}nsó temos visto duas coisas, um homem e um macaco; perdão{\ldots} temos mais dois, um\textbackslash{}nmacaco e um homem. É muito divertida! Mas, para variar, o homem vai sair e fica\textbackslash{}no macaco.\textbackslash{}n\textbackslash{}nDizendo estas palavras com uma raiva\textbackslash{}ncômica, Diogo travou do chapéu e saiu.\textbackslash{}n\textbackslash{}nTito soltou uma gargalhada.\textbackslash{}n\textbackslash{}n— Mas vamos ao fim da história{\ldots}\textbackslash{}n\textbackslash{}n— Que fim, minha senhora? Eu já estava em\textbackslash{}ntalas por não saber como continuar{\ldots} Era um meio de servi-la. Vejo que é um\textbackslash{}nvelho aborrecido{\ldots}\textbackslash{}n\textbackslash{}n— Não é, está enganado.\textbackslash{}n\textbackslash{}n— Ah! não?\textbackslash{}n\textbackslash{}n— Divirto-me com ele. O que não impede\textbackslash{}nque a presença do senhor me dê infinito prazer{\ldots}\textbackslash{}n\textbackslash{}n— Vossa Excelência disse agora uma\textbackslash{}nfalsidade.\textbackslash{}n\textbackslash{}n— Qual foi?\textbackslash{}n\textbackslash{}n— Disse que lhe era agradável a minha\textbackslash{}nconversa. Ora, isso é falso como tudo quanto é falso{\ldots}\textbackslash{}n\textbackslash{}n— Quer um elogio?\textbackslash{}n\textbackslash{}n— Não, falo franco. Eu nem sei como Vossa\textbackslash{}nExcelência me atura; desabrido, maçante, chocarreiro, sem fé em coisa alguma,\textbackslash{}nsou um conversador muito pouco digno de ser desejado. É preciso ter uma grande\textbackslash{}nsoma de bondade para ter expressões tão benévolas{\ldots} tão amigas{\ldots}\textbackslash{}n\textbackslash{}n— Deixe esse ar de mofa, e{\ldots}\textbackslash{}n\textbackslash{}n— Mofa, minha senhora?\textbackslash{}n\textbackslash{}n— Ontem eu e minha tia tomamos chá\textbackslash{}nsozinhas! sozinhas!{\ldots}\textbackslash{}n\textbackslash{}n— Ah!\textbackslash{}n\textbackslash{}n— Contava que o senhor viesse\textbackslash{}naborrecer-se uma hora conosco{\ldots}\textbackslash{}n\textbackslash{}n— Qual aborrecer{\ldots} Eu lhe digo: o\textbackslash{}nculpado foi o Ernesto.\textbackslash{}n\textbackslash{}n— Ah! foi ele?\textbackslash{}n\textbackslash{}n— É verdade; deu comigo aí em casa de uns\textbackslash{}namigos, éramos quatro ao todo, rolou a conversa sobre o voltarete e acabamos\textbackslash{}npor formar mesa. Ah! mas foi uma noite completa! Aconteceu-me o que me acontece\textbackslash{}nsempre: ganhei!\textbackslash{}n\textbackslash{}n— Está bom.\textbackslash{}n\textbackslash{}n— Pois, olhe, ainda assim eu não jogava\textbackslash{}ncom pexotes; eram mestres de primeira força: um principalmente; até às onze\textbackslash{}nhoras a fortuna pareceu desfavorecer-me, mas dessa hora em diante desandou a\textbackslash{}nroda para eles e eu comecei a assombrar{\ldots} pode ficar certa de que os\textbackslash{}nassombrei. Ah! é que eu tenho diploma{\ldots} mas que é isso, está chorando?\textbackslash{}n\textbackslash{}nEmília tinha com efeito o lenço nos\textbackslash{}nolhos. Chorava? É certo que quando tirou o lenço dos olhos, tinha-os úmidos.\textbackslash{}nVoltou-se contra a luz e disse ao moço:\textbackslash{}n\textbackslash{}n— Qual{\ldots} pode continuar.\textbackslash{}n\textbackslash{}n— Não há mais nada; foi só isto, disse\textbackslash{}nTito.\textbackslash{}n\textbackslash{}n— Estimo que a noite lhe corresse\textbackslash{}nfeliz{\ldots}\textbackslash{}n\textbackslash{}n— Alguma coisa{\ldots}\textbackslash{}n\textbackslash{}n— Mas a uma carta responde-se; por que\textbackslash{}nnão respondeu à minha? disse a viúva.\textbackslash{}n\textbackslash{}n— À sua qual?\textbackslash{}n\textbackslash{}n— A carta que lhe escrevi pedindo que\textbackslash{}nviesse tomar chá conosco?\textbackslash{}n\textbackslash{}n— Não me lembro.\textbackslash{}n\textbackslash{}n— Não se lembra?\textbackslash{}n\textbackslash{}n— Ou, se recebi essa carta, foi em\textbackslash{}nocasião que a não pude ler, e então esqueci, esqueci-a em algum lugar{\ldots}\textbackslash{}n\textbackslash{}n— É possível: mas é a última vez{\ldots}\textbackslash{}n\textbackslash{}n— Não me convida mais para tomar chá?\textbackslash{}n\textbackslash{}n— Não. Pode arriscar-se a perder\textbackslash{}ndistrações melhores.\textbackslash{}n\textbackslash{}n— Isso não digo: a senhora trata bem a\textbackslash{}ngente, e em sua casa passam-se bem as horas{\ldots} Isto é com franqueza. Mas então\textbackslash{}ntomou chá sozinha? E o Diogo?\textbackslash{}n\textbackslash{}n— Descartei-me dele. Acha que ele seja\textbackslash{}ndivertido?\textbackslash{}n\textbackslash{}n— Parece que sim{\ldots} É um homem delicado;\textbackslash{}num tanto dado às paixões, é verdade, mas sendo esse um defeito comum, acho que\textbackslash{}nnele não é muito digno de censura.\textbackslash{}n\textbackslash{}n— O Diogo está vingado.\textbackslash{}n\textbackslash{}n— De que, minha senhora?\textbackslash{}n\textbackslash{}nEmília olhou fixamente para Tito e disse:\textbackslash{}n\textbackslash{}n— De nada!\textbackslash{}n\textbackslash{}nE levantando-se dirigiu-se para o piano.\textbackslash{}n\textbackslash{}n— Vou tocar, disse ela; não o aborrece?\textbackslash{}n\textbackslash{}n— De modo nenhum.\textbackslash{}n\textbackslash{}nEmília começou a tocar; mas era uma música\textbackslash{}ntão triste que infundia certa melancolia no espírito do moço. Este, depois de\textbackslash{}nalgum tempo, interrompeu com estas palavras:\textbackslash{}n\textbackslash{}n— Que música triste!\textbackslash{}n\textbackslash{}n— Traduzo a minha alma, disse a viúva.\textbackslash{}n\textbackslash{}n— Anda triste?\textbackslash{}n\textbackslash{}n— Que lhe importam as minhas tristezas?\textbackslash{}n\textbackslash{}n— Tem razão, não me importam nada. Em\textbackslash{}ntodo o caso não é comigo?\textbackslash{}n\textbackslash{}nEmília levantou-se e foi para ele.\textbackslash{}n\textbackslash{}n— Acha que lhe hei de perdoar a desfeita\textbackslash{}nque me fez? disse ela.\textbackslash{}n\textbackslash{}n— Que desfeita, minha senhora?\textbackslash{}n\textbackslash{}n— A desfeita de não vir ao meu convite?\textbackslash{}n\textbackslash{}n— Mas eu já lhe expliquei{\ldots}\textbackslash{}n\textbackslash{}n— Paciência! O que sinto é que também\textbackslash{}nnesse voltarete estivesse o marido de Adelaide.\textbackslash{}n\textbackslash{}n— Ele retirou-se às dez horas, e entrou\textbackslash{}num parceiro novo, que não era de todo mau.\textbackslash{}n\textbackslash{}n— Pobre Adelaide!\textbackslash{}n\textbackslash{}n— Mas se eu lhe digo que ele se retirou\textbackslash{}nàs dez horas{\ldots}\textbackslash{}n\textbackslash{}n— Não devia ter ido. Devia pertencer\textbackslash{}nsempre à sua mulher. Sei que estou falando a um descrido; não pode calcular a\textbackslash{}nfelicidade e os deveres do lar doméstico. Viverem duas criaturas uma para\textbackslash{}noutra, confundidas, unificadas; pensar, aspirar, sonhar a mesma coisa; limitar\textbackslash{}no horizonte nos olhos de cada uma, sem outra ambição, sem inveja de mais nada.\textbackslash{}nSabe o que é isto?\textbackslash{}n\textbackslash{}n— Sei{\ldots} É o casamento por fora.\textbackslash{}n\textbackslash{}n— Conheço alguém que lhe provava aquilo\textbackslash{}ntudo{\ldots}\textbackslash{}n\textbackslash{}n— Deveras? Quem é essa fênix?\textbackslash{}n\textbackslash{}n— Se lho disser, há de mofar; não digo.\textbackslash{}n\textbackslash{}n— Qual mofar! Diga lá, eu sou curioso.\textbackslash{}n\textbackslash{}n— Não acredita que haja alguém que possa\textbackslash{}namá-lo?\textbackslash{}n\textbackslash{}n— Pode ser{\ldots}\textbackslash{}n\textbackslash{}n— Não acredita que alguém, por despeito, por\textbackslash{}noutra coisa que seja, tire da originalidade do seu espírito os influxos de um\textbackslash{}namor verdadeiro, mui diverso do amor ordinário dos salões; um amor capaz de\textbackslash{}nsacrifício, capaz de tudo? Não acredita!\textbackslash{}n\textbackslash{}n— Se me afirma, acredito; mas{\ldots}\textbackslash{}n\textbackslash{}n— Existe a pessoa e o amor.\textbackslash{}n\textbackslash{}n— São então duas fênix.\textbackslash{}n\textbackslash{}n— Não zombe. Existem{\ldots} Procure{\ldots}\textbackslash{}n\textbackslash{}n— Ah! isso há de ser mais difícil: não\textbackslash{}ntenho tempo. E suponha que achasse, de que me servia? Para mim é perfeitamente\textbackslash{}ninútil. Isso é bom para outros; para o Diogo, por exemplo{\ldots}\textbackslash{}n\textbackslash{}n— Para o Diogo?\textbackslash{}n\textbackslash{}nA bela viúva pareceu ter um assomo de\textbackslash{}ncólera. Depois de um silêncio disse:\textbackslash{}n\textbackslash{}n— Adeus! Desculpe, estou incomodada.\textbackslash{}n\textbackslash{}n— Então, até amanhã!\textbackslash{}n\textbackslash{}nDizendo o que, Tito apertou a mão de\textbackslash{}nEmília e saiu tão alegre e descuidoso como se saísse de um jantar de anos.\textbackslash{}n\textbackslash{}nEmília, apenas ficou só, caiu numa\textbackslash{}ncadeira e cobriu o rosto.\textbackslash{}n\textbackslash{}nEstava nessa posição havia cinco minutos,\textbackslash{}nquando assomou à porta a figura do velho Diogo.\textbackslash{}n\textbackslash{}nO rumor que o velho fez entrando\textbackslash{}ndespertou a viúva.\textbackslash{}n\textbackslash{}n— Ainda aqui!\textbackslash{}n\textbackslash{}n— É verdade, minha senhora, disse Diogo\textbackslash{}naproximando-se, é verdade. Ainda aqui, por minha infelicidade{\ldots}\textbackslash{}n\textbackslash{}n— Não entendo{\ldots}\textbackslash{}n\textbackslash{}n— Não saí para casa. Um demônio oculto me\textbackslash{}nimpeliu para cometer um ato infame. Cometi-o, mas tirei dele um proveito; estou\textbackslash{}nsalvo. Sei que me não ama.\textbackslash{}n\textbackslash{}n— Ouviu?\textbackslash{}n\textbackslash{}n— Tudo. E percebi.\textbackslash{}n\textbackslash{}n— Que percebeu, meu caro senhor?\textbackslash{}n\textbackslash{}n— Percebi que a senhora ama o Tito.\textbackslash{}n\textbackslash{}n— Ah!\textbackslash{}n\textbackslash{}n— Retiro-me, portanto, mas não quero\textbackslash{}nfazê-lo sem que ao menos fique sabendo de que saio com ciência de que não sou amado;\textbackslash{}ne que saio antes de me mandarem embora.\textbackslash{}n\textbackslash{}nEmília ouviu as palavras de Diogo com a\textbackslash{}nmaior tranqüilidade. Enquanto ele falava teve tempo de refletir no que devia\textbackslash{}ndizer.\textbackslash{}n\textbackslash{}nDiogo estava já a fazer o seu último cumprimento,\textbackslash{}nquando a viúva lhe dirigiu a palavra.\textbackslash{}n\textbackslash{}n— Ouça-me, Sr. Diogo. Ouviu bem, mas\textbackslash{}npercebeu mal. Já que pretende ter sabido{\ldots}\textbackslash{}n\textbackslash{}n— Já sei; vem dizer que há um plano\textbackslash{}nassentado de zombar com aquele moço{\ldots}\textbackslash{}n\textbackslash{}n— Como sabe?\textbackslash{}n\textbackslash{}n— Disse-mo D. Adelaide.\textbackslash{}n\textbackslash{}n— É verdade.\textbackslash{}n\textbackslash{}n— Não creio.\textbackslash{}n\textbackslash{}n— Por quê?\textbackslash{}n\textbackslash{}n— Havia lágrimas nas suas palavras.\textbackslash{}nOuvi-as com a dor n’alma. Se soubesse como eu sofria!\textbackslash{}n\textbackslash{}nA bela viúva não pôde deixar de sorrir ao\textbackslash{}ngesto cômico de Diogo. Depois, como ele parecesse mergulhado em meditação\textbackslash{}nsombria, disse:\textbackslash{}n\textbackslash{}n— Engana-se, tanto que volto para a\textbackslash{}ncidade.\textbackslash{}n\textbackslash{}n— Deveras?\textbackslash{}n\textbackslash{}n— Pois acredita que um homem como aquele\textbackslash{}npossa inspirar qualquer sentimento sério? Nem por sombras!\textbackslash{}n\textbackslash{}nEstas palavras foram ditas no tom com que\textbackslash{}nEmília costumava persuadir aquele eterno namorado. Isso e mais um sorriso, foi\textbackslash{}nquanto bastou para acalmar o ânimo de Diogo. Daí a alguns minutos estava ele\textbackslash{}nradiante.\textbackslash{}n\textbackslash{}n— Olhe, e para desenganá-lo de uma vez\textbackslash{}nvou escrever um bilhete ao Tito{\ldots}\textbackslash{}n\textbackslash{}n— Eu mesmo o levarei, disse Diogo louco\textbackslash{}nde contente.\textbackslash{}n\textbackslash{}n— Pois sim!\textbackslash{}n\textbackslash{}n— Adeus, até amanhã. Tenha sonhos\textbackslash{}ncor-de-rosa, e desculpe os meus maus modos. Até amanhã.\textbackslash{}n\textbackslash{}nO velho beijou graciosamente a mão de\textbackslash{}nEmília e saiu.\textbackslash{}n\textbackslash{}nCAPÍTULO IV\textbackslash{}n\textbackslash{}nNo dia seguinte, ao meio-dia, Diogo\textbackslash{}napresentou-se ao Tito, e depois de falar sobre diferentes coisas, tirou do\textbackslash{}nbolso uma cartinha, que fingira ter esquecido até então, e a qual mostrava não\textbackslash{}ndar grande apreço.\textbackslash{}n\textbackslash{}n'Que bomba!' disse ele consigo,\textbackslash{}nna ocasião em que Tito rasgou a sobrecarta.\textbackslash{}n\textbackslash{}nEis o que dizia a carta:\textbackslash{}n\textbackslash{}nDei-lhe o meu coração. Não quis\textbackslash{}naceitá-lo, desprezou-o mesmo. A sua bota magoou-o demais para que ele possa\textbackslash{}npalpitar ainda. Está morto. Não o censuro; não se deve falar de luz aos cegos;\textbackslash{}na culpada fui eu. Supus que pudesse dar-lhe uma felicidade, recebendo outra.\textbackslash{}nEnganei-me.\textbackslash{}n\textbackslash{}nTem a glória de retirar-se com todas as\textbackslash{}nhonras de guerra. Eu é que fico vencida. Paciência! Pode zombar de mim; não lhe\textbackslash{}ncontesto o direito que tem para isso.\textbackslash{}n\textbackslash{}nEntretanto, devo dizer-lhe que eu bem o\textbackslash{}nconhecia; nunca lho disse, mas conhecia-o; desde o dia em que o vi pela\textbackslash{}nprimeira vez em casa de Adelaide, reconheci na sua pessoa o mesmo homem que um\textbackslash{}ndia veio atirar-se aos meus pés{\ldots} Era zombaria então, como hoje. Eu já devia\textbackslash{}nconhecê-lo. Caro pago o meu engano. Adeus, adeus para sempre.\textbackslash{}n\textbackslash{}nLendo esta carta, Tito olhava repetidas\textbackslash{}nvezes para Diogo. Como é que o velho se prestara àquilo? Era autêntica ou\textbackslash{}napócrifa a tal carta? Sobre não trazer assinatura, tinha a letra disfarçada.\textbackslash{}nSeria uma arma de que o velho usara para descartar-se do rapaz? Mas, se fosse\textbackslash{}nassim, era preciso que ele soubesse do que se passara na véspera.\textbackslash{}n\textbackslash{}nTito releu a carta muitas vezes; e,\textbackslash{}ndespedindo-se do velho, disse-lhe que a resposta iria depois.\textbackslash{}n\textbackslash{}nDiogo retirou-se esfregando as mãos de\textbackslash{}ncontente.\textbackslash{}n\textbackslash{}nÉ que a carta cuja leitura os leitores\textbackslash{}nfizeram ao mesmo tempo que o nosso herói, não era a que Emília lera a Diogo. Na\textbackslash{}nminuta apresentada ao velho a viúva declarava simplesmente que se retirava para\textbackslash{}na Corte, e acrescentava que entre as recordações que levava de Petrópolis\textbackslash{}nfigurava Tito, pela figura que ela havia representado diante dele. Mas essa\textbackslash{}nminuta, por uma destreza puramente feminina, não foi a que Emília mandou a\textbackslash{}nTito, como viram os leitores.\textbackslash{}n\textbackslash{}nÀ carta de Emília respondeu Tito nos\textbackslash{}nseguintes termos:\textbackslash{}n\textbackslash{}nMinha senhora,\textbackslash{}n\textbackslash{}nLi e reli a sua carta; e não lhe\textbackslash{}nocultarei o sentimento de pesar que ela me inspirou. Realmente, minha senhora,\textbackslash{}né esse o estado do seu coração? Está assim tão perdido por mim?\textbackslash{}n\textbackslash{}nDiz Vossa Excelência que eu com a minha\textbackslash{}nbota machuquei o seu coração. Penaliza-me o fato, sem que eu entretanto o\textbackslash{}nconfirme. Não me lembra até hoje que tivesse feito estrago algum desta\textbackslash{}nnatureza. Mas, enfim, Vossa Excelência o diz, e eu devo crê-lo.\textbackslash{}n\textbackslash{}nLendo esta carta Vossa Excelência dirá\textbackslash{}nconsigo que eu sou o mais audaz cavalheiro que ainda pisou a terra de Santa\textbackslash{}nCruz. Será um engano de observação. Isto em mim não é audácia, é franqueza.\textbackslash{}nLastimo que as coisas chegassem a este ponto, mas não posso dizer-lhe nada mais\textbackslash{}nque a verdade.\textbackslash{}n\textbackslash{}nDevo confessar que não sei se a carta a\textbackslash{}nque respondo é de Vossa Excelência. A sua letra, de que eu já vi uma amostra no\textbackslash{}nálbum de D. Adelaide, não se parece com a da carta; está evidentemente\textbackslash{}ndisfarçada; é de qualquer mão. Demais, não traz assinatura.\textbackslash{}n\textbackslash{}nDigo isto porque a primeira dúvida que\textbackslash{}nnasceu em meu espírito proveio do portador escolhido. Pois quê? Vossa\textbackslash{}nExcelência não achou outro senão o próprio Diogo? Confesso que de tudo o que tenho\textbackslash{}nvisto em minha vida, é isto o que mais me faz rir.\textbackslash{}n\textbackslash{}nMas eu não devo rir, minha senhora. Vossa\textbackslash{}nExcelência abriu-me o seu coração de um modo que inspira antes compaixão. Esta\textbackslash{}ncompaixão não lhe é desairosa, porque não vem por sentido irônico. É pura e sincera.\textbackslash{}nSinto não poder dar-lhe essa felicidade que me pede; mas é assim.\textbackslash{}n\textbackslash{}nNão devo estender-me, contudo custa-me\textbackslash{}narrancar a pena de cima do papel. É que poucos terão a posição que eu ocupo\textbackslash{}nagora, a posição de requestado. Mas devo acabar e acabo aqui, mandando-lhe os\textbackslash{}nmeus pêsames e rogando a Deus para que encontre um coração menos frio que o\textbackslash{}nmeu.\textbackslash{}n\textbackslash{}nA letra vai disfarçada como a sua, e,\textbackslash{}ncomo na sua carta, deixo a assinatura em branco.\textbackslash{}n\textbackslash{}nEsta carta foi entregue à viúva na mesma\textbackslash{}ntarde. À noite, Azevedo e Adelaide foram visitá-la. Não puderam dissuadi-la da\textbackslash{}nidéia da viagem para a corte. Emília usou mesmo de uma certa reserva para com\textbackslash{}nAdelaide, que não pôde descobrir os motivos de semelhante procedimento, e\textbackslash{}nretirou-se um tanto triste.\textbackslash{}n\textbackslash{}nNo dia seguinte, com efeito, Emília e a\textbackslash{}ntia aprontaram-se e saíram para voltar para a corte.\textbackslash{}n\textbackslash{}nDiogo ficou em Petrópolis ainda, cuidando\textbackslash{}nem aprontar as malas{\ldots} Não queria, dizia ele, que o público, vendo-o partir em\textbackslash{}ncompanhia das duas senhoras, supusesse coisas desairosas à viúva.\textbackslash{}n\textbackslash{}nTodos estes passos admiravam Adelaide,\textbackslash{}nque, como disse, via na insistência de Emília e nos seus modos reservados um\textbackslash{}nsegredo que não compreendia. Quereria ela por aquele meio de viagem atrair\textbackslash{}nTito? Nesse caso era cálculo errado; visto que o rapaz, naquele dia como nos\textbackslash{}noutros, acordou tarde e almoçou alegremente.\textbackslash{}n\textbackslash{}n— Sabe, disse Adelaide, que a esta hora\textbackslash{}ndeve ter partido para a cidade nossa amiga Emília?\textbackslash{}n\textbackslash{}n— Já tinha ouvido dizer.\textbackslash{}n\textbackslash{}n— Por que será?\textbackslash{}n\textbackslash{}n— Ah! isso é que eu não sei. Altos\textbackslash{}nsegredos do espírito de mulher! Por que sopra hoje a brisa deste lado e não\textbackslash{}ndaquele? Interessa-me tanto saber uma coisa como outra.\textbackslash{}n\textbackslash{}nNo fim do almoço Tito, como quase sempre,\textbackslash{}nretirou-se para ler durante duas horas.\textbackslash{}n\textbackslash{}nAdelaide ia dar algumas ordens quando viu\textbackslash{}ncom pasmo entrar-lhe em casa a viúva, acompanhada de um criado.\textbackslash{}n\textbackslash{}n— Ah! não partiste! disse Adelaide\textbackslash{}ncorrendo a abraçá-la.\textbackslash{}n\textbackslash{}n— Não me vês aqui?\textbackslash{}n\textbackslash{}nO criado saiu a um sinal de Emília.\textbackslash{}n\textbackslash{}n— Mas que há? perguntou a mulher de\textbackslash{}nAzevedo, vendo os modos estranhos da viúva.\textbackslash{}n\textbackslash{}n— Que há? disse esta. Há o que não\textbackslash{}nprevíamos{\ldots} És quase minha irmã{\ldots} posso falar francamente. Ninguém nos ouve?\textbackslash{}n\textbackslash{}n— Ernesto está fora e o Tito lá em cima.\textbackslash{}nMas que ar é esse?\textbackslash{}n\textbackslash{}n— Adelaide! disse Emília com os olhos\textbackslash{}nrasos de lágrimas, eu o amo!\textbackslash{}n\textbackslash{}n— Que me dizes?\textbackslash{}n\textbackslash{}n— Isto mesmo. Amo-o doidamente,\textbackslash{}nperdidamente, completamente. Procurei até agora vencer esta paixão, mas não\textbackslash{}npude; e quando, por vãos preconceitos, tratava de ocultar-lhe o estado do meu\textbackslash{}ncoração, não pude, as palavras saíram-me dos lábios insensivelmente{\ldots}\textbackslash{}n\textbackslash{}n— Mas como se deu isto?\textbackslash{}n\textbackslash{}n— Eu sei! Parece que foi castigo; quis\textbackslash{}nfazer fogo e queimei-me nas mesmas chamas. Ah! não é de hoje que me sinto\textbackslash{}nassim. Desde que os seus desdéns em nada cederam, comecei a sentir não sei o\textbackslash{}nquê; ao princípio despeito, depois um desejo de triunfar, depois uma ambição de\textbackslash{}nceder tudo, contanto que tudo ganhasse; afinal não fui senhora de mim. Era eu\textbackslash{}nquem me sentia doidamente apaixonada e lho manifestava, por gestos, por\textbackslash{}npalavras, por tudo; e mais crescia nele a indiferença, mais crescia o amor em\textbackslash{}nmim.\textbackslash{}n\textbackslash{}n— Mas estás falando sério?\textbackslash{}n\textbackslash{}n— Olha antes para mim.\textbackslash{}n\textbackslash{}n— Quem pensara?{\ldots}\textbackslash{}n\textbackslash{}n— A mim própria parece impossível; porém\textbackslash{}né mais que verdade{\ldots}\textbackslash{}n\textbackslash{}n— E ele?{\ldots}\textbackslash{}n\textbackslash{}n— Ele disse-me quatro palavras\textbackslash{}nindiferentes, nem sei o que foi, e retirou-se.\textbackslash{}n\textbackslash{}n— Resistirá?\textbackslash{}n\textbackslash{}n— Não sei.\textbackslash{}n\textbackslash{}n— Se eu adivinhara isto não te insinuaria\textbackslash{}nnaquela malfadada idéia.\textbackslash{}n\textbackslash{}n— Não me compreendeste. Cuidas que eu deploro\textbackslash{}no que acontece? Oh! não! sinto-me feliz, sinto-me orgulhosa{\ldots} É um destes\textbackslash{}namores que brotam por si para encher a alma de satisfação: devo antes\textbackslash{}nabençoar-te{\ldots}\textbackslash{}n\textbackslash{}n— É uma verdadeira paixão{\ldots} Mas\textbackslash{}nacreditas impossível a conversão dele?\textbackslash{}n\textbackslash{}n— Não sei; mas seja ou não impossível,\textbackslash{}nnão é a conversão que eu peço; basta-me que seja menos indiferente e mais\textbackslash{}ncompassivo.\textbackslash{}n\textbackslash{}n— Mas que pretendes fazer? perguntou\textbackslash{}nAdelaide sentindo que as lágrimas também lhe rebentavam dos olhos.\textbackslash{}n\textbackslash{}nHouve alguns instantes de silêncio.\textbackslash{}n\textbackslash{}n— Mas o que tu não sabes, continuou\textbackslash{}nEmília, é que ele não é para mim um simples estranho. Já o conhecia antes de\textbackslash{}ncasada. Foi ele quem me pediu em casamento antes de Rafael{\ldots}\textbackslash{}n\textbackslash{}n— Ah!\textbackslash{}n\textbackslash{}n— Sabias?\textbackslash{}n\textbackslash{}n— Ele já me havia contado a história, mas\textbackslash{}nnão nomeara a santa. Eras tu?\textbackslash{}n\textbackslash{}n— Era eu. Ambos nos conhecíamos, sem\textbackslash{}ndizermos nada um ao outro{\ldots}\textbackslash{}n\textbackslash{}n— Por quê?\textbackslash{}n\textbackslash{}nA resposta a esta pergunta foi dada pelo\textbackslash{}npróprio Tito, que assomara à porta do interior. Tendo visto entrar a viúva de\textbackslash{}numa das janelas, Tito desceu abaixo a ouvir a conversa dela com Adelaide. A\textbackslash{}nestranheza que lhe causava a volta inesperada de Emília podia desculpar a\textbackslash{}nindiscrição do rapaz.\textbackslash{}n\textbackslash{}n— Por quê? repetiu ele. É o que lhes vou\textbackslash{}ndizer.\textbackslash{}n\textbackslash{}n— Mas antes de tudo, disse Adelaide, não\textbackslash{}nsei se sabe que uma indiferença, tão completa, como a sua, pode ser fatal a\textbackslash{}nquem lhe é menos indiferente?\textbackslash{}n\textbackslash{}n— Refere-se à sua amiga? perguntou Tito.\textbackslash{}nEu corto tudo com uma palavra.\textbackslash{}n\textbackslash{}nE voltando-se para Emília, disse,\textbackslash{}nestendendo-lhe a mão:\textbackslash{}n\textbackslash{}n— Aceita a minha mão de esposo?\textbackslash{}n\textbackslash{}nUm grito de alegria suprema ia saindo do\textbackslash{}npeito de Emília; mas não sei se um resto de orgulho, ou qualquer outro\textbackslash{}nsentimento, converteu essa manifestação em uma simples palavra, que aliás foi\textbackslash{}npronunciada com lágrimas na voz:\textbackslash{}n\textbackslash{}n— Sim! disse ela.\textbackslash{}n\textbackslash{}nTito beijou amorosamente a mão da viúva.\textbackslash{}nDepois acrescentou:\textbackslash{}n\textbackslash{}n— Mas é preciso medir toda a minha\textbackslash{}ngenerosidade; eu devia dizer: aceito a sua mão. Devia ou não devia? Sou um\textbackslash{}ntanto original e gosto de fazer inversão em tudo.\textbackslash{}n\textbackslash{}n— Pois sim; mas de um ou de outro modo\textbackslash{}nsou feliz. Contudo um remorso me surge na consciência. Dou-lhe uma felicidade\textbackslash{}ntão completa como a que recebo?\textbackslash{}n\textbackslash{}n— Remorso? Se é sujeita aos remorsos deve\textbackslash{}nter um, mas por motivo diverso. A senhora está passando neste momento pelas\textbackslash{}nforças caudinas. Fi-la sofrer, não? Ouvindo o que vou dizer concordará que eu\textbackslash{}njá antes sofria, e muito mais.\textbackslash{}n\textbackslash{}n— Temos romance? perguntou Adelaide a\textbackslash{}nTito.\textbackslash{}n\textbackslash{}n— Realidade, minha senhora, respondeu\textbackslash{}nTito, e realidade em prosa. Um dia, há já alguns anos, tive eu a felicidade de\textbackslash{}nver uma senhora, e amei-a. O amor foi tanto mais indomável quanto que me nasceu\textbackslash{}nde súbito. Era então mais ardente que hoje, não conhecia muito os usos do\textbackslash{}nmundo. Resolvi declarar-lhe a minha paixão e pedi-la em casamento. Tive em\textbackslash{}nresposta este bilhete{\ldots}\textbackslash{}n\textbackslash{}n— Já sei, disse Emília. Essa senhora fui\textbackslash{}neu. Estou humilhada; perdão!\textbackslash{}n\textbackslash{}n— Meu amor lhe perdoa; nunca deixei de\textbackslash{}namá-la. Eu estava certo de encontrá-la um dia e procedi de modo a fazer-me o\textbackslash{}ndesejado.\textbackslash{}n\textbackslash{}n— Escreva isto e dirão que é um romance,\textbackslash{}ndisse alegremente Adelaide.\textbackslash{}n\textbackslash{}n— A vida não é outra coisa{\ldots} acrescentou\textbackslash{}nTito,\textbackslash{}n\textbackslash{}nDaí a meia hora entrava Azevedo. Admirado\textbackslash{}nda presença de Emília quando a supunha a rodar no trem de ferro, e mais\textbackslash{}nadmirado ainda das maneiras cordiais por que se tratavam Tito e Emília, o\textbackslash{}nmarido de Adelaide inquiriu a causa disso.\textbackslash{}n\textbackslash{}n— A causa é simples, respondeu Adelaide;\textbackslash{}nEmília voltou porque vai casar-se com Tito.\textbackslash{}n\textbackslash{}nAzevedo não se deu por satisfeito;\textbackslash{}nexplicaram-lhe tudo.\textbackslash{}n\textbackslash{}n— Percebo, disse ele; Tito, não tendo alcançado\textbackslash{}nnada caminhando em linha reta, procurou ver se alcançava caminhando por linha\textbackslash{}ncurva. Às vezes é o caminho mais curto.\textbackslash{}n\textbackslash{}n— Como agora, acrescentou Tito.\textbackslash{}n\textbackslash{}nEmília jantou em casa de Adelaide. À\textbackslash{}ntarde apareceu ali o velho Diogo, que ia despedir-se porque devia partir para a\textbackslash{}ncorte no dia seguinte de manhã. Grande foi a sua admiração quando viu a viúva.\textbackslash{}n\textbackslash{}n— Voltou?\textbackslash{}n\textbackslash{}n— É verdade, respondeu Emília rindo.\textbackslash{}n\textbackslash{}n— Pois eu ia partir, mas já não parto.\textbackslash{}nAh! recebi uma carta da Europa: foi o capitão da galera Macedônia quem a\textbackslash{}ntrouxe! Chegou o urso!\textbackslash{}n\textbackslash{}n— Pois vá fazer-lhe companhia, respondeu\textbackslash{}nTito.\textbackslash{}n\textbackslash{}nDiogo fez uma careta. Depois, como\textbackslash{}ndesejasse saber o motivo da súbita volta da viúva, esta explicou-lhe que se ia\textbackslash{}ncasar com Tito.\textbackslash{}n\textbackslash{}nDiogo não acreditou.\textbackslash{}n\textbackslash{}n— É ainda um laço, não? disse ele\textbackslash{}npiscando os olhos.\textbackslash{}n\textbackslash{}nE não só não acreditou então, como não\textbackslash{}nacreditou daí em diante, apesar de tudo. Daí a alguns dias partiram todos para\textbackslash{}na corte. Diogo ainda se não convencia de nada. Mas, quando entrando um dia em\textbackslash{}ncasa de Emília viu a festa do noivado, o pobre velho não pôde negar a realidade\textbackslash{}ne sofreu um forte abalo. Todavia, teve ainda coração para assistir às festas do\textbackslash{}nnoivado. Azevedo e a mulher serviram de testemunhas.\textbackslash{}n\textbackslash{}nÉ preciso confessar, escrevia dois meses\textbackslash{}ndepois o feliz noivo ao esposo de Adelaide; - é preciso confessar que eu entrei\textbackslash{}nnum jogo arriscado. Podia perder; felizmente ganhei.\textbackslash{}n\textbackslash{}nFREI SIMÃO\textbackslash{}n\textbackslash{}nÍNDICE\textbackslash{}n\textbackslash{}nCAPÍTULO PRIMEIRO\textbackslash{}n\textbackslash{}nCAPÍTULO II\textbackslash{}n\textbackslash{}nCAPÍTULO III\textbackslash{}n\textbackslash{}nCAPÍTULO IV\textbackslash{}n\textbackslash{}nCAPÍTULO V\textbackslash{}n\textbackslash{}nCAPÍTULO PRIMEIRO\textbackslash{}n\textbackslash{}nFrei Simão era um frade da ordem dos\textbackslash{}nBeneditinos. Tinha, quando morreu, cinqüenta anos em aparência, mas na\textbackslash{}nrealidade trinta e oito. A causa desta velhice prematura derivava da que o\textbackslash{}nlevou ao claustro na idade de trinta anos, e, tanto quanto se pode saber por\textbackslash{}nuns fragmentos de memórias que ele deixou, a causa era justa.\textbackslash{}n\textbackslash{}nEra frei Simão de caráter taciturno e\textbackslash{}ndesconfiado. Passava dias inteiros na sua cela, donde apenas saía na hora do\textbackslash{}nrefeitório e dos ofícios divinos. Não contava amizade alguma no convento,\textbackslash{}nporque não era possível entreter com ele os preliminares que fundam e\textbackslash{}nconsolidam as afeições.\textbackslash{}n\textbackslash{}nEm um convento, onde a comunhão das almas\textbackslash{}ndeve ser mais pronta e mais profunda, frei Simão parecia fugir à regra geral.\textbackslash{}nUm dos noviços pôs-lhe alcunha de urso, que lhe ficou, mas só entre os\textbackslash{}nnoviços, bem entendido. Os frades professos, esses, apesar do desgosto que o\textbackslash{}ngênio solitário de frei Simão lhes inspirava, sentiam por ele certo respeito e\textbackslash{}nveneração.\textbackslash{}n\textbackslash{}nUm dia anuncia-se que frei Simão adoecera\textbackslash{}ngravemente. Chamaram-se os socorros e prestaram ao enfermo todos os cuidados\textbackslash{}nnecessários. A moléstia era mortal; depois de cinco dias frei Simão expirou.\textbackslash{}n\textbackslash{}nDurante estes cinco dias de moléstia, a\textbackslash{}ncela de frei Simão esteve cheia de frades. Frei Simão não disse uma palavra\textbackslash{}ndurante esses cinco dias; só no último, quando se aproximava o minuto fatal,\textbackslash{}nsentou-se no leito, fez chamar para mais perto o abade, e disse-lhe ao ouvido\textbackslash{}ncom voz sufocada e em tom estranho:\textbackslash{}n\textbackslash{}n— Morro odiando a humanidade!\textbackslash{}n\textbackslash{}nO abade recuou até a parede ao ouvir\textbackslash{}nestas palavras, e no tom em que foram ditas. Quanto a frei Simão, caiu sobre o\textbackslash{}ntravesseiro e passou à eternidade.\textbackslash{}n\textbackslash{}nDepois de feitas ao irmão finado as\textbackslash{}nhonras que se lhe deviam, a comunidade perguntou ao seu chefe que palavras\textbackslash{}nouvira tão sinistras que o assustaram. O abade referiu-as, persignando-se. Mas\textbackslash{}nos frades não viram nessas palavras senão um segredo do passado, sem dúvida\textbackslash{}nimportante, mas não tal que pudesse lançar o terror no espírito do abade. Este\textbackslash{}nexplicou-lhes a idéia que tivera quando ouviu as palavras de frei Simão, no tom\textbackslash{}nem que foram ditas, e acompanhadas do olhar com que o fulminou: acreditara que\textbackslash{}nfrei Simão estivesse doido; mais ainda, que tivesse entrado já doido para a\textbackslash{}nordem. Os hábitos da solidão e taciturnidade a que se votara o frade pareciam\textbackslash{}nsintomas de uma alienação mental de caráter brando e pacífico; mas durante oito\textbackslash{}nanos parecia impossível aos frades que frei Simão não tivesse um dia revelado\textbackslash{}nde modo positivo a sua loucura; objetaram isso ao abade; mas este persistia na\textbackslash{}nsua crença.\textbackslash{}n\textbackslash{}nEntretanto procedeu-se ao inventário dos\textbackslash{}nobjetos que pertenciam ao finado, e entre eles achou-se um rolo de papéis\textbackslash{}nconvenientemente enlaçados, com este rótulo:\textbackslash{}n\textbackslash{}nMemórias que há de escrever frei Simão de\textbackslash{}nSanta Águeda, frade beneditino.\textbackslash{}n\textbackslash{}nEste rolo de papéis foi um grande achado\textbackslash{}npara a comunidade curiosa. Iam finalmente penetrar alguma coisa no véu\textbackslash{}nmisterioso que envolvia o passado de frei Simão, e talvez confirmar as\textbackslash{}nsuspeitas do abade. O rolo foi aberto e lido para todos.\textbackslash{}n\textbackslash{}nEram, pela maior parte, fragmentos\textbackslash{}nincompletos, apontamentos truncados e notas insuficientes; mas de tudo junto\textbackslash{}npôde-se colher que realmente frei Simão estivera louco durante certo tempo.\textbackslash{}n\textbackslash{}nO autor desta narrativa despreza aquela\textbackslash{}nparte das Memórias que não tiver absolutamente importância; mas procura\textbackslash{}naproveitar a que for menos inútil ou menos obscura.\textbackslash{}n\textbackslash{}nCAPÍTULO II\textbackslash{}n\textbackslash{}nAs notas de frei Simão nada dizem do\textbackslash{}nlugar do seu nascimento nem do nome de seus pais. O que se pôde saber dos seus\textbackslash{}nprincípios é que, tendo concluído os estudos preparatórios, não pôde seguir a\textbackslash{}ncarreira das letras, como desejava, e foi obrigado a entrar como guarda-livros\textbackslash{}nna casa comercial de seu pai.\textbackslash{}n\textbackslash{}nMorava então em casa de seu pai uma prima\textbackslash{}nde Simão, órfã de pai e mãe, que haviam por morte deixado ao pai de Simão o\textbackslash{}ncuidado de a educarem e manterem. Parece que os cabedais deste deram para isto.\textbackslash{}nQuanto ao pai da prima órfã, tendo sido rico, perdera tudo ao jogo e nos azares\textbackslash{}ndo comércio, ficando reduzido à última miséria.\textbackslash{}n\textbackslash{}nA órfã chamava-se Helena; era bela, meiga\textbackslash{}ne extremamente boa. Simão, que se educara com ela, e juntamente vivia debaixo\textbackslash{}ndo mesmo teto, não pôde resistir às elevadas qualidades e à beleza de sua\textbackslash{}nprima. Amaram-se. Em seus sonhos de futuro contavam ambos o casamento, coisa\textbackslash{}nque parece mais natural do mundo para corações amantes.\textbackslash{}n\textbackslash{}nNão tardou muito que os pais de Simão\textbackslash{}ndescobrissem o amor dos dois. Ora é preciso dizer, apesar de não haver\textbackslash{}ndeclaração formal disto nos apontamentos do frade, é preciso dizer que os\textbackslash{}nreferidos pais eram de um egoísmo descomunal. Davam de boa vontade o pão da\textbackslash{}nsubsistência a Helena; mas lá casar o filho com a pobre órfã é que não podiam\textbackslash{}nconsentir. Tinham posto a mira em uma herdeira rica, e dispunham de si para si\textbackslash{}nque o rapaz se casaria com ela.\textbackslash{}n\textbackslash{}nUma tarde, como estivesse o rapaz a\textbackslash{}nadiantar a escrituração do livro mestre, entrou no escritório o pai com ar\textbackslash{}ngrave e risonho ao mesmo tempo, e disse ao filho que largasse o trabalho e o ouvisse.\textbackslash{}nO rapaz obedeceu. O pai falou assim:\textbackslash{}n\textbackslash{}n— Vais partir para a província de ***.\textbackslash{}nPreciso mandar umas cartas ao meu correspondente Amaral, e como sejam elas de\textbackslash{}ngrande importância, não quero confiá-las ao nosso desleixado correio. Queres ir\textbackslash{}nno vapor ou preferes o nosso brigue?\textbackslash{}n\textbackslash{}nEsta pergunta era feita com grande tino.\textbackslash{}n\textbackslash{}nObrigado a responder-lhe, o velho\textbackslash{}ncomerciante não dera lugar que seu filho apresentasse objeções.\textbackslash{}n\textbackslash{}nO rapaz enfiou, abaixou os olhos e\textbackslash{}nrespondeu:\textbackslash{}n\textbackslash{}n— Vou onde meu pai quiser.\textbackslash{}n\textbackslash{}nO pai agradeceu mentalmente a submissão\textbackslash{}ndo filho, que lhe poupava o dinheiro da passagem no vapor, e foi muito contente\textbackslash{}ndar parte à mulher de que o rapaz não fizera objeção alguma.\textbackslash{}n\textbackslash{}nNessa noite os dois amantes tiveram\textbackslash{}nocasião de encontrar-se sós na sala de jantar.\textbackslash{}n\textbackslash{}nSimão contou a Helena o que se passara.\textbackslash{}nChoraram ambos algumas lágrimas furtivas, e ficaram na esperança de que a\textbackslash{}nviagem fosse de um mês, quando muito.\textbackslash{}n\textbackslash{}nÀ mesa do chá, o pai de Simão conversou\textbackslash{}nsobre a viagem do rapaz, que devia ser de poucos dias. Isto reanimou as\textbackslash{}nesperanças dos dois amantes. O resto da noite passou-se em conselhos da parte\textbackslash{}ndo velho ao filho sobre a maneira de portar-se na casa do correspondente. Às\textbackslash{}ndez horas, como de costume, todos se recolheram aos aposentos.\textbackslash{}n\textbackslash{}nOs dias passaram-se depressa. Finalmente\textbackslash{}nraiou aquele em que devia partir o brigue. Helena saiu de seu quarto com os\textbackslash{}nolhos vermelhos de chorar. Interrogada bruscamente pela tia, disse que era uma\textbackslash{}ninflamação adquirida pelo muito que lera na noite anterior. A tia prescreveu-lhe\textbackslash{}nabstenção da leitura e banhos de água de malvas.\textbackslash{}n\textbackslash{}nQuanto ao tio, tendo chamado Simão,\textbackslash{}nentregou-lhe uma carta para o correspondente, e abraçou-o. A mala e um criado\textbackslash{}nestavam prontos. A despedida foi triste. Os dois pais sempre choraram alguma\textbackslash{}ncoisa, a rapariga muito.\textbackslash{}n\textbackslash{}nQuanto a Simão, levava os olhos secos e\textbackslash{}nardentes. Era refratário às lágrimas, por isso mesmo padecia mais.\textbackslash{}n\textbackslash{}nO brigue partiu. Simão, enquanto pôde ver\textbackslash{}nterra, não se retirou de cima; quando finalmente se fecharam de todo as paredes\textbackslash{}ndo cárcere que anda, na frase pitoresca de Ribeyrolles, Simão desceu ao seu\textbackslash{}ncamarote, triste e com o coração apertado. Havia como um pressentimento que lhe\textbackslash{}ndizia interiormente ser impossível tornar a ver sua prima. Parecia que ia para\textbackslash{}num degredo.\textbackslash{}n\textbackslash{}nChegando ao lugar do seu destino,\textbackslash{}nprocurou Simão o correspondente de seu pai e entregou-lhe a carta. O Sr. Amaral\textbackslash{}nleu a carta, fitou o rapaz e, depois de algum silêncio, disse-lhe, volvendo a\textbackslash{}ncarta:\textbackslash{}n\textbackslash{}n— Bem, agora é preciso esperar que eu\textbackslash{}ncumpra esta ordem de seu pai. Entretanto venha morar para a minha casa.\textbackslash{}n\textbackslash{}n— Quando poderei voltar? perguntou Simão.\textbackslash{}n\textbackslash{}n— Em poucos dias, salvo se as coisas se\textbackslash{}ncomplicarem.\textbackslash{}n\textbackslash{}nEste salvo, posto na boca de\textbackslash{}nAmaral como incidente, era a oração principal. A carta do pai de Simão versava\textbackslash{}nassim:\textbackslash{}n\textbackslash{}nMeu caro Amaral,\textbackslash{}n\textbackslash{}nMotivos ponderosos me obrigam a mandar\textbackslash{}nmeu filho desta cidade. Retenha-o por lá como puder. O pretexto da viagem é ter\textbackslash{}neu necessidade de ultimar alguns negócios com você, o que dirá ao pequeno,\textbackslash{}nfazendo-lhe sempre crer que a demora é pouca ou nenhuma. Você, que teve na sua\textbackslash{}nadolescência a triste idéia de engendrar romances, vá inventando circunstâncias\textbackslash{}ne ocorrências imprevistas, de modo que o rapaz não me torne cá antes de segunda\textbackslash{}nordem. Sou, como sempre, etc.\textbackslash{}n\textbackslash{}nCAPÍTULO III\textbackslash{}n\textbackslash{}nPassaram-se dias e dias, e nada de chegar\textbackslash{}no momento de voltar à casa paterna. O ex-romancista era na verdade fértil, e\textbackslash{}nnão se cansava de inventar pretextos que deixavam convencido o rapaz.\textbackslash{}n\textbackslash{}nEntretanto, como o espírito dos amantes\textbackslash{}nnão é menos engenhoso que o dos romancistas, Simão e Helena acharam meio de se\textbackslash{}nescreverem, e deste modo podiam consolar-se da ausência, com presença das\textbackslash{}nletras e do papel. Bem diz Heloísa que a arte de escrever foi inventada por\textbackslash{}nalguma amante separada do seu amante. Nestas cartas juravam-se os dois sua\textbackslash{}neterna fidelidade.\textbackslash{}n\textbackslash{}nNo fim de dois meses de espera baldada e\textbackslash{}nde ativa correspondência, a tia de Helena surpreendeu uma carta de Simão. Era a\textbackslash{}nvigésima, creio eu. Houve grande temporal em casa. O tio, que estava no\textbackslash{}nescritório, saiu precipitadamente e tomou conhecimento do negócio. O resultado\textbackslash{}nfoi proscrever de casa tinta, penas e papel, e instituir vigilância rigorosa\textbackslash{}nsobre a infeliz rapariga.\textbackslash{}n\textbackslash{}nComeçaram pois a escassear as cartas ao\textbackslash{}npobre deportado. Inquiriu a causa disto em cartas choradas e compridas; mas\textbackslash{}ncomo o rigor fiscal da casa de seu pai adquiria proporções descomunais,\textbackslash{}nacontecia que todas as cartas de Simão iam parar às mãos do velho, que, depois\textbackslash{}nde apreciar o estilo amoroso de seu filho, fazia queimar as ardentes epístolas.\textbackslash{}n\textbackslash{}nPassaram-se dias e meses. Carta de\textbackslash{}nHelena, nenhuma. O correspondente ia esgotando a veia inventadora, e já não\textbackslash{}nsabia como reter finalmente o rapaz.\textbackslash{}n\textbackslash{}nChega uma carta a Simão. Era letra do\textbackslash{}npai. Só diferençava das outras que recebia do velho em ser esta mais longa,\textbackslash{}nmuito mais longa. O rapaz abriu a carta, e leu trêmulo e pálido. Contava nesta\textbackslash{}ncarta o honrado comerciante que a Helena, a boa rapariga que ele destinava a\textbackslash{}nser sua filha casando-se com Simão, a boa Helena tinha morrido. O velho copiara\textbackslash{}nalgum dos últimos necrológios que vira nos jornais, e ajuntara algumas\textbackslash{}nconsolações de casa. A última consolação foi dizer-lhe que embarcasse e fosse\textbackslash{}nter com ele.\textbackslash{}n\textbackslash{}nO período final da carta dizia:\textbackslash{}n\textbackslash{}nAssim como assim, não se realizam os meus\textbackslash{}nnegócios; não te pude casar com Helena, visto que Deus a levou. Mas volta,\textbackslash{}nfilho, vem; poderás consolar-te casando com outra, a filha do conselheiro ***.\textbackslash{}nEstá moça feita e é um bom partido. Não te desalentes; lembra-te de mim.\textbackslash{}n\textbackslash{}nO pai de Simão não conhecia bem o amor do\textbackslash{}nfilho, nem era grande águia para avaliá-lo, ainda que o conhecesse. Dores tais\textbackslash{}nnão se consolam com uma carta nem com um casamento. Era melhor mandá-lo chamar,\textbackslash{}ne depois preparar- lhe a notícia; mas dada assim friamente em uma carta, era\textbackslash{}nexpor o rapaz a uma morte certa.\textbackslash{}n\textbackslash{}nFicou Simão vivo em corpo e morto\textbackslash{}nmoralmente, tão morto que por sua própria idéia foi dali procurar uma\textbackslash{}nsepultura. Era melhor dar aqui alguns dos papéis escritos por Simão\textbackslash{}nrelativamente ao que sofreu depois da carta; mas há muitas falhas, e eu não\textbackslash{}nquero corrigir a exposição ingênua e sincera do frade.\textbackslash{}n\textbackslash{}nA sepultura que Simão escolheu foi um\textbackslash{}nconvento. Respondeu ao pai que agradecia a filha do conselheiro, mas que daquele\textbackslash{}ndia em diante pertencia ao serviço de Deus.\textbackslash{}n\textbackslash{}nO pai ficou maravilhado. Nunca suspeitou\textbackslash{}nque o filho pudesse vir a ter semelhante resolução. Escreveu às pressas para\textbackslash{}nver se o desviava da idéia; mas não pôde conseguir.\textbackslash{}n\textbackslash{}nQuanto ao correspondente, para quem tudo\textbackslash{}nse embrulhava cada vez mais, deixou o rapaz seguir para o claustro, disposto a\textbackslash{}nnão figurar em um negócio do qual nada realmente sabia.\textbackslash{}n\textbackslash{}nCAPÍTULO IV\textbackslash{}n\textbackslash{}nFrei Simão de Santa Águeda foi obrigado a\textbackslash{}nir à província natal em missão religiosa, tempos depois dos fatos que acabo de\textbackslash{}nnarrar.\textbackslash{}n\textbackslash{}nPreparou-se e embarcou.\textbackslash{}n\textbackslash{}nA missão não era na capital, mas no\textbackslash{}ninterior. Entrando na capital, pareceu-lhe dever ir visitar seus pais. Estavam\textbackslash{}nmudados física e moralmente. Era com certeza a dor e o remorso de terem\textbackslash{}nprecipitado seu filho à resolução que tomou. Tinham vendido a casa comercial e\textbackslash{}nviviam de suas rendas.\textbackslash{}n\textbackslash{}nReceberam o filho com alvoroço e\textbackslash{}nverdadeiro amor. Depois das lágrimas e das consolações, vieram ao fim da viagem\textbackslash{}nde Simão.\textbackslash{}n\textbackslash{}n— A que vens tu, meu filho?\textbackslash{}n\textbackslash{}n— Venho cumprir uma missão do sacerdócio\textbackslash{}nque abracei. Venho pregar, para que o rebanho do Senhor não se arrede nunca do\textbackslash{}nbom caminho.\textbackslash{}n\textbackslash{}n— Aqui na capital?\textbackslash{}n\textbackslash{}n— Não, no interior. Começo pela vila de\textbackslash{}n***.\textbackslash{}n\textbackslash{}nOs dois velhos estremeceram; mas Simão\textbackslash{}nnada viu. No dia seguinte partiu Simão, não sem algumas instâncias de seus pais\textbackslash{}npara que ficasse. Notaram eles que seu filho nem de leve tocara em Helena.\textbackslash{}nTambém eles não quiseram magoá-lo falando em tal assunto.\textbackslash{}n\textbackslash{}nDaí a dias, na vila de que falara frei\textbackslash{}nSimão, era um alvoroço para ouvir as prédicas do missionário.\textbackslash{}n\textbackslash{}nA velha igreja do lugar estava atopetada\textbackslash{}nde povo.\textbackslash{}n\textbackslash{}nÀ hora anunciada, frei Simão subiu ao\textbackslash{}npúlpito e começou o discurso religioso. Metade do povo saiu aborrecido no meio\textbackslash{}ndo sermão. A razão era simples. Avezado à pintura viva dos caldeirões de Pedro\textbackslash{}nBotelho e outros pedacinhos de ouro da maioria dos pregadores, o povo não podia\textbackslash{}nouvir com prazer a linguagem simples, branda, persuasiva, a que serviam de\textbackslash{}nmodelo as conferências do fundador da nossa religião.\textbackslash{}n\textbackslash{}nO pregador estava a terminar, quando\textbackslash{}nentrou apressadamente na igreja um par, marido e mulher: ele, honrado lavrador,\textbackslash{}nmeio remediado com o sítio que possuía e a boa vontade de trabalhar; ela,\textbackslash{}nsenhora estimada por suas virtudes, mas de uma melancolia invencível.\textbackslash{}n\textbackslash{}nDepois de tomarem água benta, colocam-se\textbackslash{}nambos em lugar donde pudessem ver facilmente o pregador.\textbackslash{}n\textbackslash{}nOuviu-se então um grito, e todos correram\textbackslash{}npara a recém-chegada, que acabava de desmaiar. Frei Simão teve de parar o seu\textbackslash{}ndiscurso, enquanto se punha termo ao incidente. Mas, por uma aberta que a turba\textbackslash{}ndeixava, pôde ele ver o rosto da desmaiada.\textbackslash{}n\textbackslash{}nEra Helena.\textbackslash{}n\textbackslash{}nNo manuscrito do frade há uma série de\textbackslash{}nreticências dispostas em oito linhas. Ele próprio não sabe o que se passou. Mas\textbackslash{}no que se passou foi que, mal conhecera Helena, continuou o frade o discurso.\textbackslash{}nEra então outra coisa: era um discurso sem nexo, sem assunto, um verdadeiro\textbackslash{}ndelírio. A consternação foi geral.\textbackslash{}n\textbackslash{}nCAPÍTULO V\textbackslash{}n\textbackslash{}nO delírio de frei Simão durou alguns\textbackslash{}ndias. Graças aos cuidados, pôde melhorar, e pareceu a todos que estava bom,\textbackslash{}nmenos ao médico, que queria continuar a cura. Mas o frade disse positivamente\textbackslash{}nque se retirava ao convento, e não houve forças humanas que o detivessem.\textbackslash{}n\textbackslash{}nO leitor compreende naturalmente que o\textbackslash{}ncasamento de Helena fora obrigado pelos tios.\textbackslash{}n\textbackslash{}nA pobre senhora não resistiu à comoção.\textbackslash{}nDois meses depois morreu, deixando inconsolável o marido, que a amava com\textbackslash{}nveras.\textbackslash{}n\textbackslash{}nFrei Simão, recolhido ao convento,\textbackslash{}ntornou-se mais solitário e taciturno. Restava-lhe ainda um pouco da alienação.\textbackslash{}n\textbackslash{}nJá conhecemos o acontecimento de sua\textbackslash{}nmorte e a impressão que ela causara ao abade.\textbackslash{}n\textbackslash{}nA cela de frei Simão de Santa Águeda\textbackslash{}nesteve muito tempo religiosamente fechada. Só se abriu, algum tempo depois,\textbackslash{}npara dar entrada a um velho secular, que por esmola alcançou do abade acabar os\textbackslash{}nseus dias na convivência dos médicos da alma. Era o pai de Simão. A mãe tinha\textbackslash{}nmorrido.\textbackslash{}n\textbackslash{}nFoi crença, nos últimos anos de vida\textbackslash{}ndeste velho, que ele não estava menos doido que frei Simão de Santa Águeda.\textbackslash{}n\textbackslash{}nFIM\textbackslash{}n"
\end{Verbatim}
        
    Lendo o texto \emph{puro} dos livros de Machado:

    \begin{Verbatim}[commandchars=\\\{\}]
{\color{incolor}In [{\color{incolor}23}]:} \PY{c+c1}{\PYZsh{}textos = [machado.raw(id) for id in machado.fileids()]}
         \PY{c+c1}{\PYZsh{}len(textos)}
\end{Verbatim}

    \begin{Verbatim}[commandchars=\\\{\}]
{\color{incolor}In [{\color{incolor} }]:} 
\end{Verbatim}

    Carregando a lista de stopwords em lingua portuguesa para limpeza dos
textos. Note que é preciso trazer as palavras para \emph{UTF-8} antes de
usá-las.

    \begin{Verbatim}[commandchars=\\\{\}]
{\color{incolor}In [{\color{incolor}24}]:} \PY{n}{swu} \PY{o}{=} \PY{n}{stopwords}\PY{o}{.}\PY{n}{words}\PY{p}{(}\PY{l+s+s1}{\PYZsq{}}\PY{l+s+s1}{portuguese}\PY{l+s+s1}{\PYZsq{}}\PY{p}{)} \PY{o}{+} \PY{n+nb}{list} \PY{p}{(}\PY{n}{string}\PY{o}{.}\PY{n}{punctuation}\PY{p}{)}
         \PY{c+c1}{\PYZsh{}swu = [word.decode(\PYZsq{}utf8\PYZsq{}) for word in sw]}
\end{Verbatim}

    \begin{Verbatim}[commandchars=\\\{\}]
{\color{incolor}In [{\color{incolor}25}]:} \PY{n}{stopwords}\PY{o}{.}\PY{n}{words}\PY{p}{(}\PY{l+s+s1}{\PYZsq{}}\PY{l+s+s1}{portuguese}\PY{l+s+s1}{\PYZsq{}}\PY{p}{)}
         \PY{c+c1}{\PYZsh{}list (string.punctuation)}
\end{Verbatim}

            \begin{Verbatim}[commandchars=\\\{\}]
{\color{outcolor}Out[{\color{outcolor}25}]:} ['de',
          'a',
          'o',
          'que',
          'e',
          'do',
          'da',
          'em',
          'um',
          'para',
          'com',
          'não',
          'uma',
          'os',
          'no',
          'se',
          'na',
          'por',
          'mais',
          'as',
          'dos',
          'como',
          'mas',
          'ao',
          'ele',
          'das',
          'à',
          'seu',
          'sua',
          'ou',
          'quando',
          'muito',
          'nos',
          'já',
          'eu',
          'também',
          'só',
          'pelo',
          'pela',
          'até',
          'isso',
          'ela',
          'entre',
          'depois',
          'sem',
          'mesmo',
          'aos',
          'seus',
          'quem',
          'nas',
          'me',
          'esse',
          'eles',
          'você',
          'essa',
          'num',
          'nem',
          'suas',
          'meu',
          'às',
          'minha',
          'numa',
          'pelos',
          'elas',
          'qual',
          'nós',
          'lhe',
          'deles',
          'essas',
          'esses',
          'pelas',
          'este',
          'dele',
          'tu',
          'te',
          'vocês',
          'vos',
          'lhes',
          'meus',
          'minhas',
          'teu',
          'tua',
          'teus',
          'tuas',
          'nosso',
          'nossa',
          'nossos',
          'nossas',
          'dela',
          'delas',
          'esta',
          'estes',
          'estas',
          'aquele',
          'aquela',
          'aqueles',
          'aquelas',
          'isto',
          'aquilo',
          'estou',
          'está',
          'estamos',
          'estão',
          'estive',
          'esteve',
          'estivemos',
          'estiveram',
          'estava',
          'estávamos',
          'estavam',
          'estivera',
          'estivéramos',
          'esteja',
          'estejamos',
          'estejam',
          'estivesse',
          'estivéssemos',
          'estivessem',
          'estiver',
          'estivermos',
          'estiverem',
          'hei',
          'há',
          'havemos',
          'hão',
          'houve',
          'houvemos',
          'houveram',
          'houvera',
          'houvéramos',
          'haja',
          'hajamos',
          'hajam',
          'houvesse',
          'houvéssemos',
          'houvessem',
          'houver',
          'houvermos',
          'houverem',
          'houverei',
          'houverá',
          'houveremos',
          'houverão',
          'houveria',
          'houveríamos',
          'houveriam',
          'sou',
          'somos',
          'são',
          'era',
          'éramos',
          'eram',
          'fui',
          'foi',
          'fomos',
          'foram',
          'fora',
          'fôramos',
          'seja',
          'sejamos',
          'sejam',
          'fosse',
          'fôssemos',
          'fossem',
          'for',
          'formos',
          'forem',
          'serei',
          'será',
          'seremos',
          'serão',
          'seria',
          'seríamos',
          'seriam',
          'tenho',
          'tem',
          'temos',
          'tém',
          'tinha',
          'tínhamos',
          'tinham',
          'tive',
          'teve',
          'tivemos',
          'tiveram',
          'tivera',
          'tivéramos',
          'tenha',
          'tenhamos',
          'tenham',
          'tivesse',
          'tivéssemos',
          'tivessem',
          'tiver',
          'tivermos',
          'tiverem',
          'terei',
          'terá',
          'teremos',
          'terão',
          'teria',
          'teríamos',
          'teriam']
\end{Verbatim}
        
    Um outro ingrediente essencial é um stemmer para a nossa língua. O
Stemmer reduz as palavras a uma abreviação que se aproxima da ``raiz''
da palavra.

    \begin{Verbatim}[commandchars=\\\{\}]
{\color{incolor}In [{\color{incolor}26}]:} \PY{n}{stemmer} \PY{o}{=} \PY{n}{PortugueseStemmer}\PY{p}{(}\PY{p}{)}
\end{Verbatim}

    \begin{Verbatim}[commandchars=\\\{\}]
{\color{incolor}In [{\color{incolor}27}]:} \PY{n}{WordPunctTokenizer}\PY{p}{(}\PY{p}{)}\PY{o}{.}\PY{n}{tokenize}\PY{p}{(}\PY{n}{textos}\PY{p}{[}\PY{l+m+mi}{0}\PY{p}{]}\PY{p}{)}
\end{Verbatim}

            \begin{Verbatim}[commandchars=\\\{\}]
{\color{outcolor}Out[{\color{outcolor}27}]:} ['Conto',
          ',',
          'Contos',
          'Fluminenses',
          ',',
          '1870',
          'Contos',
          'Fluminenses',
          'Texto',
          '-',
          'fonte',
          ':',
          'Obra',
          'Completa',
          ',',
          'Machado',
          'de',
          'Assis',
          ',',
          'vol',
          '.',
          'II',
          ',',
          'Rio',
          'de',
          'Janeiro',
          ':',
          'Nova',
          'Aguilar',
          ',',
          '1994',
          '.',
          'Publicado',
          'originalmente',
          'pela',
          'Editora',
          'Garnier',
          ',',
          'Rio',
          'de',
          'Janeiro',
          ',',
          'em',
          '1870',
          '.',
          'ÍNDICE',
          'MISS',
          'DOLLAR',
          'LUÍS',
          'SOARES',
          'A',
          'MULHER',
          'DE',
          'PRETO',
          'O',
          'SEGREDO',
          'DE',
          'AUGUSTA',
          'CONFISSÕES',
          'DE',
          'UMA',
          'VIÚVA',
          'MOÇA',
          'LINHA',
          'RETA',
          'E',
          'LINHA',
          'CURVA',
          'FREI',
          'SIMÃO',
          'MISS',
          'DOLLAR',
          'ÍNDICE',
          'Capítulo',
          'Primeiro',
          'Capítulo',
          'II',
          'Capítulo',
          'iii',
          'Capítulo',
          'iv',
          'Capítulo',
          'v',
          'Capítulo',
          'vI',
          'Capítulo',
          'vII',
          'CAPÍTULO',
          'VIII',
          'CAPÍTULO',
          'PRIMEIRO',
          'Era',
          'conveniente',
          'ao',
          'romance',
          'que',
          'o',
          'leitor',
          'ficasse',
          'muito',
          'tempo',
          'sem',
          'saber',
          'quem',
          'era',
          'Miss',
          'Dollar',
          '.',
          'Mas',
          'por',
          'outro',
          'lado',
          ',',
          'sem',
          'a',
          'apresentação',
          'de',
          'Miss',
          'Dollar',
          ',',
          'seria',
          'o',
          'autor',
          'obrigado',
          'a',
          'longas',
          'digressões',
          ',',
          'que',
          'encheriam',
          'o',
          'papel',
          'sem',
          'adiantar',
          'a',
          'ação',
          '.',
          'Não',
          'há',
          'hesitação',
          'possível',
          ':',
          'vou',
          'apresentar',
          '-',
          'lhes',
          'Miss',
          'Dollar',
          '.',
          'Se',
          'o',
          'leitor',
          'é',
          'rapaz',
          'e',
          'dado',
          'ao',
          'gênio',
          'melancólico',
          ',',
          'imagina',
          'que',
          'Miss',
          'Dollar',
          'é',
          'uma',
          'inglesa',
          'pálida',
          'e',
          'delgada',
          ',',
          'escassa',
          'de',
          'carnes',
          'e',
          'de',
          'sangue',
          ',',
          'abrindo',
          'à',
          'flor',
          'do',
          'rosto',
          'dois',
          'grandes',
          'olhos',
          'azuis',
          'e',
          'sacudindo',
          'ao',
          'vento',
          'umas',
          'longas',
          'tranças',
          'loiras',
          '.',
          'A',
          'moça',
          'em',
          'questão',
          'deve',
          'ser',
          'vaporosa',
          'e',
          'ideal',
          'como',
          'uma',
          'criação',
          'de',
          'Shakespeare',
          ';',
          'deve',
          'ser',
          'o',
          'contraste',
          'do',
          'roastbeef',
          'britânico',
          ',',
          'com',
          'que',
          'se',
          'alimenta',
          'a',
          'liberdade',
          'do',
          'Reino',
          'Unido',
          '.',
          'Uma',
          'tal',
          'Miss',
          'Dollar',
          'deve',
          'ter',
          'o',
          'poeta',
          'Tennyson',
          'de',
          'cor',
          'e',
          'ler',
          'Lamartine',
          'no',
          'original',
          ';',
          'se',
          'souber',
          'o',
          'português',
          'deve',
          'deliciar',
          '-',
          'se',
          'com',
          'a',
          'leitura',
          'dos',
          'sonetos',
          'de',
          'Camões',
          'ou',
          'os',
          'Cantos',
          'de',
          'Gonçalves',
          'Dias',
          '.',
          'O',
          'chá',
          'e',
          'o',
          'leite',
          'devem',
          'ser',
          'a',
          'alimentação',
          'de',
          'semelhante',
          'criatura',
          ',',
          'adicionando',
          '-',
          'se',
          '-',
          'lhe',
          'alguns',
          'confeitos',
          'e',
          'biscoitos',
          'para',
          'acudir',
          'às',
          'urgências',
          'do',
          'estômago',
          '.',
          'A',
          'sua',
          'fala',
          'deve',
          'ser',
          'um',
          'murmúrio',
          'de',
          'harpa',
          'eólia',
          ';',
          'o',
          'seu',
          'amor',
          'um',
          'desmaio',
          ',',
          'a',
          'sua',
          'vida',
          'uma',
          'contemplação',
          ',',
          'a',
          'sua',
          'morte',
          'um',
          'suspiro',
          '.',
          'A',
          'figura',
          'é',
          'poética',
          ',',
          'mas',
          'não',
          'é',
          'a',
          'da',
          'heroína',
          'do',
          'romance',
          '.',
          'Suponhamos',
          'que',
          'o',
          'leitor',
          'não',
          'é',
          'dado',
          'a',
          'estes',
          'devaneios',
          'e',
          'melancolias',
          ';',
          'nesse',
          'caso',
          'imagina',
          'uma',
          'Miss',
          'Dollar',
          'totalmente',
          'diferente',
          'da',
          'outra',
          '.',
          'Desta',
          'vez',
          'será',
          'uma',
          'robusta',
          'americana',
          ',',
          'vertendo',
          'sangue',
          'pelas',
          'faces',
          ',',
          'formas',
          'arredondadas',
          ',',
          'olhos',
          'vivos',
          'e',
          'ardentes',
          ',',
          'mulher',
          'feita',
          ',',
          'refeita',
          'e',
          'perfeita',
          '.',
          'Amiga',
          'da',
          'boa',
          'mesa',
          'e',
          'do',
          'bom',
          'copo',
          ',',
          'esta',
          'Miss',
          'Dollar',
          'preferirá',
          'um',
          'quarto',
          'de',
          'carneiro',
          'a',
          'uma',
          'página',
          'de',
          'Longfellow',
          ',',
          'coisa',
          'naturalíssima',
          'quando',
          'o',
          'estômago',
          'reclama',
          ',',
          'e',
          'nunca',
          'chegará',
          'a',
          'compreender',
          'a',
          'poesia',
          'do',
          'pôr',
          '-',
          'do',
          '-',
          'sol',
          '.',
          'Será',
          'uma',
          'boa',
          'mãe',
          'de',
          'família',
          'segundo',
          'a',
          'doutrina',
          'de',
          'alguns',
          'padres',
          '-',
          'mestres',
          'da',
          'civilização',
          ',',
          'isto',
          'é',
          ',',
          'fecunda',
          'e',
          'ignorante',
          '.',
          'Já',
          'não',
          'será',
          'do',
          'mesmo',
          'sentir',
          'o',
          'leitor',
          'que',
          'tiver',
          'passado',
          'a',
          'segunda',
          'mocidade',
          'e',
          'vir',
          'diante',
          'de',
          'si',
          'uma',
          'velhice',
          'sem',
          'recurso',
          '.',
          'Para',
          'esse',
          ',',
          'a',
          'Miss',
          'Dollar',
          'verdadeiramente',
          'digna',
          'de',
          'ser',
          'contada',
          'em',
          'algumas',
          'páginas',
          ',',
          'seria',
          'uma',
          'boa',
          'inglesa',
          'de',
          'cinqüenta',
          'anos',
          ',',
          'dotada',
          'com',
          'algumas',
          'mil',
          'libras',
          'esterlinas',
          ',',
          'e',
          'que',
          ',',
          'aportando',
          'ao',
          'Brasil',
          'em',
          'procura',
          'de',
          'assunto',
          'para',
          'escrever',
          'um',
          'romance',
          ',',
          'realizasse',
          'um',
          'romance',
          'verdadeiro',
          ',',
          'casando',
          'com',
          'o',
          'leitor',
          'aludido',
          '.',
          'Uma',
          'tal',
          'Miss',
          'Dollar',
          'seria',
          'incompleta',
          'se',
          'não',
          'tivesse',
          'óculos',
          'verdes',
          'e',
          'um',
          'grande',
          'cacho',
          'de',
          'cabelo',
          'grisalho',
          'em',
          'cada',
          'fonte',
          '.',
          'Luvas',
          'de',
          'renda',
          'branca',
          'e',
          'chapéu',
          'de',
          'linho',
          'em',
          'forma',
          'de',
          'cuia',
          ',',
          'seriam',
          'a',
          'última',
          'demão',
          'deste',
          'magnífico',
          'tipo',
          'de',
          'ultramar',
          '.',
          'Mais',
          'esperto',
          'que',
          'os',
          'outros',
          ',',
          'acode',
          'um',
          'leitor',
          'dizendo',
          'que',
          'a',
          'heroína',
          'do',
          'romance',
          'não',
          'é',
          'nem',
          'foi',
          'inglesa',
          ',',
          'mas',
          'brasileira',
          'dos',
          'quatro',
          'costados',
          ',',
          'e',
          'que',
          'o',
          'nome',
          'de',
          'Miss',
          'Dollar',
          'quer',
          'dizer',
          'simplesmente',
          'que',
          'a',
          'rapariga',
          'é',
          'rica',
          '.',
          'A',
          'descoberta',
          'seria',
          'excelente',
          ',',
          'se',
          'fosse',
          'exata',
          ';',
          'infelizmente',
          'nem',
          'esta',
          'nem',
          'as',
          'outras',
          'são',
          'exatas',
          '.',
          'A',
          'Miss',
          'Dollar',
          'do',
          'romance',
          'não',
          'é',
          'a',
          'menina',
          'romântica',
          ',',
          'nem',
          'a',
          'mulher',
          'robusta',
          ',',
          'nem',
          'a',
          'velha',
          'literata',
          ',',
          'nem',
          'a',
          'brasileira',
          'rica',
          '.',
          'Falha',
          'desta',
          'vez',
          'a',
          'proverbial',
          'perspicácia',
          'dos',
          'leitores',
          ';',
          'Miss',
          'Dollar',
          'é',
          'uma',
          'cadelinha',
          'galga',
          '.',
          'Para',
          'algumas',
          'pessoas',
          'a',
          'qualidade',
          'da',
          'heroína',
          'fará',
          'perder',
          'o',
          'interesse',
          'do',
          'romance',
          '.',
          'Erro',
          'manifesto',
          '.',
          'Miss',
          'Dollar',
          ',',
          'apesar',
          'de',
          'não',
          'ser',
          'mais',
          'que',
          'uma',
          'cadelinha',
          'galga',
          ',',
          'teve',
          'as',
          'honras',
          'de',
          'ver',
          'o',
          'seu',
          'nome',
          'nos',
          'papéis',
          'públicos',
          ',',
          'antes',
          'de',
          'entrar',
          'para',
          'este',
          'livro',
          '.',
          'O',
          'Jornal',
          'do',
          'Comércio',
          'e',
          'o',
          'Correio',
          'Mercantil',
          'publicaram',
          'nas',
          'colunas',
          'dos',
          'anúncios',
          'as',
          'seguintes',
          'linhas',
          'reverberantes',
          'de',
          'promessa',
          ':',
          'Desencaminhou',
          '-',
          'se',
          'uma',
          'cadelinha',
          'galga',
          ',',
          'na',
          'noite',
          'de',
          'ontem',
          ',',
          '30',
          '.',
          'Acode',
          'ao',
          'nome',
          'de',
          'Miss',
          'Dollar',
          '.',
          'Quem',
          'a',
          'achou',
          'e',
          'quiser',
          'levar',
          'à',
          'Rua',
          'de',
          'Mata',
          '-',
          'cavalos',
          'no',
          '{\ldots},',
          'receberá',
          'duzentos',
          'mil',
          '-',
          'réis',
          'de',
          'recompensa',
          '.',
          'Miss',
          'Dollar',
          'tem',
          'uma',
          'coleira',
          'ao',
          'pescoço',
          'fechada',
          'por',
          'um',
          'cadeado',
          'em',
          'que',
          'se',
          'lêem',
          'as',
          'seguintes',
          'palavras',
          ':',
          'De',
          'tout',
          'mon',
          'coeur',
          '.',
          'Todas',
          'as',
          'pessoas',
          'que',
          'sentiam',
          'necessidade',
          'urgente',
          'de',
          'duzentos',
          'mil',
          '-',
          'réis',
          ',',
          'e',
          'tiveram',
          'a',
          'felicidade',
          'de',
          'ler',
          'aquele',
          'anúncio',
          ',',
          'andaram',
          'nesse',
          'dia',
          'com',
          'extremo',
          'cuidado',
          'nas',
          'ruas',
          'do',
          'Rio',
          'de',
          'Janeiro',
          ',',
          'a',
          'ver',
          'se',
          'davam',
          'com',
          'a',
          'fugitiva',
          'Miss',
          'Dollar',
          '.',
          'Galgo',
          'que',
          'aparecesse',
          'ao',
          'longe',
          'era',
          'perseguido',
          'com',
          'tenacidade',
          'até',
          'verificar',
          '-',
          'se',
          'que',
          'não',
          'era',
          'o',
          'animal',
          'procurado',
          '.',
          'Mas',
          'toda',
          'esta',
          'caçada',
          'dos',
          'duzentos',
          'mil',
          '-',
          'réis',
          'era',
          'completamente',
          'inútil',
          ',',
          'visto',
          'que',
          ',',
          'no',
          'dia',
          'em',
          'que',
          'apareceu',
          'o',
          'anúncio',
          ',',
          'já',
          'Miss',
          'Dollar',
          'estava',
          'aboletada',
          'na',
          'casa',
          'de',
          'um',
          'sujeito',
          'morador',
          'nos',
          'Cajueiros',
          'que',
          'fazia',
          'coleção',
          'de',
          'cães',
          '.',
          'CAPÍTULO',
          'II',
          'Quais',
          'as',
          'razões',
          'que',
          'induziram',
          'o',
          'Dr',
          '.',
          'Mendonça',
          'a',
          'fazer',
          'coleção',
          'de',
          'cães',
          ',',
          'é',
          'coisa',
          'que',
          'ninguém',
          'podia',
          'dizer',
          ';',
          'uns',
          'queriam',
          'que',
          'fosse',
          'simplesmente',
          'paixão',
          'por',
          'esse',
          'símbolo',
          'da',
          'fidelidade',
          'ou',
          'do',
          'servilismo',
          ';',
          'outros',
          'pensavam',
          'antes',
          'que',
          ',',
          'cheio',
          'de',
          'profundo',
          'desgosto',
          'pelos',
          'homens',
          ',',
          'Mendonça',
          'achou',
          'que',
          'era',
          'de',
          'boa',
          'guerra',
          'adorar',
          'os',
          'cães',
          '.',
          'Fossem',
          'quais',
          'fossem',
          'as',
          'razões',
          ',',
          'o',
          {\ldots}]
\end{Verbatim}
        
    \subsection{Preparando o Texto}\label{preparando-o-texto}

Na célula abaixo, vamos normalizar os nossos textos trazendo todas as
palavras para caixa baixa e abreviando-as de forma a deixar apenas as
suas raízes. Neste passo, removeremos também as \emph{stopwords}. Tenha
paciência, esta análise vai levar algum tempo\ldots{}

    \begin{Verbatim}[commandchars=\\\{\}]
{\color{incolor}In [{\color{incolor}28}]:} \PY{n}{textos\PYZus{}limpos} \PY{o}{=} \PY{p}{[}\PY{p}{]}
         \PY{k}{for} \PY{n}{texto} \PY{o+ow}{in} \PY{n}{textos}\PY{p}{:}
             \PY{n}{tlimpo} \PY{o}{=} \PY{p}{[}\PY{n}{stemmer}\PY{o}{.}\PY{n}{stem}\PY{p}{(}\PY{n}{token}\PY{o}{.}\PY{n}{lower}\PY{p}{(}\PY{p}{)}\PY{p}{)} \PY{k}{for} \PY{n}{token} \PY{o+ow}{in} \PY{n}{WordPunctTokenizer}\PY{p}{(}\PY{p}{)}\PY{o}{.}\PY{n}{tokenize}\PY{p}{(}\PY{n}{texto}\PY{p}{)} \PY{k}{if} \PY{n}{token} \PY{o+ow}{not} \PY{o+ow}{in} \PY{n}{swu}\PY{p}{]}
             \PY{n}{textos\PYZus{}limpos}\PY{o}{.}\PY{n}{append}\PY{p}{(}\PY{n}{tlimpo}\PY{p}{)}
\end{Verbatim}

    Vejamos uma amostra do resultado:

    \begin{Verbatim}[commandchars=\\\{\}]
{\color{incolor}In [{\color{incolor}29}]:} \PY{n}{textos\PYZus{}limpos}\PY{p}{[}\PY{l+m+mi}{0}\PY{p}{]}\PY{p}{[}\PY{l+m+mi}{150}\PY{p}{:}\PY{l+m+mi}{160}\PY{p}{]}
\end{Verbatim}

            \begin{Verbatim}[commandchars=\\\{\}]
{\color{outcolor}Out[{\color{outcolor}29}]:} ['uma', 'tal', 'miss', 'doll', 'dev', 'ter', 'poet', 'tennyson', 'cor', 'ler']
\end{Verbatim}
        
    \subsection{Construindo um Índice
Invertido}\label{construindo-um-uxedndice-invertido}

De posse da nossa lista de termos \emph{normalizados}, podemos agora
construir o nosso índice invertido.

    \begin{Verbatim}[commandchars=\\\{\}]
{\color{incolor}In [{\color{incolor}30}]:} \PY{n}{indice} \PY{o}{=} \PY{n}{defaultdict}\PY{p}{(}\PY{k}{lambda}\PY{p}{:}\PY{n+nb}{set}\PY{p}{(}\PY{p}{[}\PY{p}{]}\PY{p}{)}\PY{p}{)}
         \PY{k}{for} \PY{n}{tid}\PY{p}{,}\PY{n}{t} \PY{o+ow}{in} \PY{n+nb}{enumerate}\PY{p}{(}\PY{n}{textos\PYZus{}limpos}\PY{p}{)}\PY{p}{:}
             \PY{k}{for} \PY{n}{term} \PY{o+ow}{in} \PY{n}{t}\PY{p}{:}
                 \PY{n}{indice}\PY{p}{[}\PY{n}{term}\PY{p}{]}\PY{o}{.}\PY{n}{add}\PY{p}{(}\PY{n}{tid}\PY{p}{)}
\end{Verbatim}

    Podemos verificar a estrutura interna do nosso índice, procurando por
uma palavra:

    \begin{Verbatim}[commandchars=\\\{\}]
{\color{incolor}In [{\color{incolor}50}]:} \PY{n}{indice}\PY{p}{[}\PY{n}{stemmer}\PY{o}{.}\PY{n}{stem}\PY{p}{(}\PY{l+s+s2}{\PYZdq{}}\PY{l+s+s2}{Dollar}\PY{l+s+s2}{\PYZdq{}}\PY{p}{)}\PY{p}{]}
\end{Verbatim}

            \begin{Verbatim}[commandchars=\\\{\}]
{\color{outcolor}Out[{\color{outcolor}50}]:} \{0, 3\}
\end{Verbatim}
        
    \begin{Verbatim}[commandchars=\\\{\}]
{\color{incolor}In [{\color{incolor}35}]:} \PY{k}{print}\PY{p}{(}\PY{n}{stemmer}\PY{o}{.}\PY{n}{stem}\PY{p}{(}\PY{l+s+s1}{\PYZsq{}}\PY{l+s+s1}{Salarial}\PY{l+s+s1}{\PYZsq{}}\PY{p}{)}\PY{p}{)}
\end{Verbatim}

    \begin{Verbatim}[commandchars=\\\{\}]
salarial

    \end{Verbatim}

    Vamos ver o contexto em que a palavra \emph{Salário} ocorre em um dos
textos

    \begin{Verbatim}[commandchars=\\\{\}]
{\color{incolor}In [{\color{incolor}42}]:} \PY{n}{nltk}\PY{o}{.}\PY{n}{Text}\PY{p}{(}\PY{n}{WordPunctTokenizer}\PY{p}{(}\PY{p}{)}\PY{o}{.}\PY{n}{tokenize}\PY{p}{(}\PY{n}{textos}\PY{p}{[}\PY{l+m+mi}{182}\PY{p}{]}\PY{p}{)}\PY{p}{)}\PY{o}{.}\PY{n}{concordance}\PY{p}{(}\PY{l+s+s2}{\PYZdq{}}\PY{l+s+s2}{Salário}\PY{l+s+s2}{\PYZdq{}}\PY{p}{)}
\end{Verbatim}

    \begin{Verbatim}[commandchars=\\\{\}]
Displaying 2 of 2 matches:
operários que com esse acréscimo de salário proporcionariam às suas famílias ma
s 2 horas da sesta é equivalente ao salário de meio dia , em tais casos abonado

    \end{Verbatim}

    \begin{Verbatim}[commandchars=\\\{\}]
{\color{incolor}In [{\color{incolor}96}]:} \PY{k}{def} \PY{n+nf}{busca}\PY{p}{(}\PY{n}{consulta}\PY{p}{)}\PY{p}{:}
             \PY{l+s+sd}{\PYZdq{}\PYZdq{}\PYZdq{}}
         \PY{l+s+sd}{    A construção de uma função de busca é deixada com exercício ao leitor}
         \PY{l+s+sd}{    \PYZdq{}\PYZdq{}\PYZdq{}}
             \PY{k}{pass}
\end{Verbatim}

    Mas já podemos utilizar nosso índice diretamente com alguns termos e
verificar como o mesmo é eficiente.

    \begin{Verbatim}[commandchars=\\\{\}]
{\color{incolor}In [{\color{incolor}44}]:} \PY{o}{\PYZpc{}}\PY{n}{time}
         \PY{n}{results} \PY{o}{=} \PY{n}{indice}\PY{p}{[}\PY{n}{stemmer}\PY{o}{.}\PY{n}{stem}\PY{p}{(}\PY{l+s+s1}{\PYZsq{}}\PY{l+s+s1}{nacional}\PY{l+s+s1}{\PYZsq{}}\PY{p}{)}\PY{p}{]}\PY{o}{\PYZam{}}\PY{n}{indice}\PY{p}{[}\PY{n}{stemmer}\PY{o}{.}\PY{n}{stem}\PY{p}{(}\PY{l+s+s1}{\PYZsq{}}\PY{l+s+s1}{perdi}\PY{l+s+s1}{\PYZsq{}}\PY{p}{)}\PY{p}{]} \PY{o}{\PYZhy{}} \PY{n}{indice}\PY{p}{[}\PY{n}{stemmer}\PY{o}{.}\PY{n}{stem}\PY{p}{(}\PY{l+s+s1}{\PYZsq{}}\PY{l+s+s1}{campo}\PY{l+s+s1}{\PYZsq{}}\PY{p}{)}\PY{p}{]}
         \PY{n}{results}
\end{Verbatim}

    \begin{Verbatim}[commandchars=\\\{\}]
Wall time: 0 ns

    \end{Verbatim}

            \begin{Verbatim}[commandchars=\\\{\}]
{\color{outcolor}Out[{\color{outcolor}44}]:} \{27,
          49,
          61,
          69,
          73,
          84,
          87,
          95,
          122,
          137,
          138,
          141,
          144,
          154,
          155,
          164,
          171,
          189,
          219,
          235\}
\end{Verbatim}
        
    Para um exame mais preciso do tempo de execução da nossa consulta,
podemos usar a mágica do \emph{\%\%timeit}

    \begin{Verbatim}[commandchars=\\\{\}]
{\color{incolor}In [{\color{incolor}45}]:} \PY{o}{\PYZpc{}}\PY{o}{\PYZpc{}}\PY{n}{timeit}
         \PY{n}{results} \PY{o}{=} \PY{n}{indice}\PY{p}{[}\PY{n}{stemmer}\PY{o}{.}\PY{n}{stem}\PY{p}{(}\PY{l+s+s1}{\PYZsq{}}\PY{l+s+s1}{nacional}\PY{l+s+s1}{\PYZsq{}}\PY{p}{)}\PY{p}{]}\PY{o}{\PYZam{}}\PY{n}{indice}\PY{p}{[}\PY{n}{stemmer}\PY{o}{.}\PY{n}{stem}\PY{p}{(}\PY{l+s+s1}{\PYZsq{}}\PY{l+s+s1}{perdi}\PY{l+s+s1}{\PYZsq{}}\PY{p}{)}\PY{p}{]} \PY{o}{\PYZhy{}} \PY{n}{indice}\PY{p}{[}\PY{n}{stemmer}\PY{o}{.}\PY{n}{stem}\PY{p}{(}\PY{l+s+s1}{\PYZsq{}}\PY{l+s+s1}{campo}\PY{l+s+s1}{\PYZsq{}}\PY{p}{)}\PY{p}{]}
         \PY{c+c1}{\PYZsh{}results}
\end{Verbatim}

    \begin{Verbatim}[commandchars=\\\{\}]
10000 loops, best of 3: 121 µs per loop

    \end{Verbatim}

    \begin{Verbatim}[commandchars=\\\{\}]
{\color{incolor}In [{\color{incolor} }]:} 
\end{Verbatim}


    % Add a bibliography block to the postdoc
    
    
    
    \end{document}
